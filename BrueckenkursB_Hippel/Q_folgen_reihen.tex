\question{{\it Eigenschaften von Funktionen}}

Bestimmen Sie für die folgenden Funktionen jeweils, ob diese injektiv, surjektiv und/oder bijektiv sind, und geben Sie gegebenenfalls die Umkehrfunktion an.\\
\parbox{0.4\textwidth}{\begin{enumerate}
\item $f:\Rset\to\Rset$, $x\mapsto x^4$
\item $g:\Rset\to[0;\infty)$, $x\mapsto x^4$
\item $h:\Rset\to\Rset$, $x\mapsto x^3$
\item $q:\Rset\to\Rset$, $y\mapsto \rme^y$
\item $r:\Rset\to(0;\infty)$, $t\mapsto \rme^t$
\item $s:[0;\infty)\to[0;\infty)$, $s(\omega)=\sqrt{\omega}$
\item $b:\Cset\backslash\{0\}\to\Rset^+$, $z\mapsto |z|$
\end{enumerate}}\parbox{0.6\textwidth}{\begin{enumerate}\setcounter{enumi}{7}
\item $U:\Nset\to\Nset\backslash\{0\}$, $U(n)=n+1$
\item $H:\Nset\to\{x|\exists n\in\Nset~x=2n\}$, $H(m)=2m$
\item $f:\Rset\to\Cset$, $x\mapsto\rme^{ix}$
\item $f:[0;1]\to\Cset$, $x\mapsto\rme^{ix}$
\item $M: {\textrm{Menge der JGU-Studierenden}}\to \Nset$,\\ $S\mapsto{\textrm{Matrikelnummer von }S}$
\item $p: \Nset\to\Nset$, $n\mapsto n$te Nachkommastelle von $\pi$
\end{enumerate}}



\question{{\it Eigenschaften von Zahlenfolgen}}

Geben Sie jeweils wenigstens die ersten fünf Glieder der folgenden Zahlenfolgen an. Bestimmen Sie ferner jeweils, ob die Zahlenfolge nach oben beschränkt, nach unten beschränkt, beschränkt, monoton wachsend, monoton fallend, streng monoton wachsend, streng monoton fallend, alternierend und/oder konvergent ist.\\
\parbox{0.5\textwidth}{\begin{enumerate}
\item $c_i=2$
\item $a_n=4\left(-\frac{1}{5}\right)^5$
\item $a_m=\sqrt{m+1}$
\item $b_k=\frac{2}{k+3}\sin\left(\frac{k\pi}{2}\right)$
\item $a_n=\left(2-\frac{1}{n+1}\right)\cos(n\pi)$
\end{enumerate}}\parbox{0.5\textwidth}{\begin{enumerate}\setcounter{enumi}{5}
\item $r_n=\frac{3}{2}n+\cos\left(\frac{2 n \pi}{3}\right)$
\item $a_n=2^{-n}-5^{-n}$
\item $a_n=(n+4)^{-1}\rme^{\frac{\pi n}{2}i}$
\item $b_n=\log_{n+2}(2n+3)$ 
\item $a_n=\frac{n^2}{n!}$
\end{enumerate}}

\pagebreak

\question{{\it Grenzwerte von Folgen}}

Bestimmen Sie jeweils, ob folgende Folgen konvergieren, bestimmt divergieren oder unbestimmt divergieren, und geben Sie gegebenenfalls den Grenzwert $\lim\limits_{n\to\infty} a_n$ an:\\
\parbox{0.5\textwidth}{\begin{enumerate}
\item $a_n=\frac{n^3+3n}{n+1}$
\item $a_n=\frac{36(n^2+1)^2}{4n^4+2n^3+3}$
\item $a_n=\frac{4n^4+2n^2+1}{4n^4+2n^3+3}$
\item $a_n=\frac{n(n+2)}{n+1}-\frac{n^3}{n^2+1}$
\item $a_n=2-\left(-\frac{3}{5}\right)^n$
\item $a_n=\rme^{\frac{1}{n!}}$
\item $a_n=\sin(n\pi)$
\item $a_n=\sin\left(\frac{n\pi}{2}\right)$
\end{enumerate}}\parbox{0.5\textwidth}{\begin{enumerate}\setcounter{enumi}{9}
\item $a_n=-2^n$
\item $a_n=(-2)^n$
\item $a_n=2^{-n}$
\item $a_n=3^{-n}-3^{-(n+1)}$
\item $a_n=\sqrt{n+1}-\sqrt{n}$
\item $a_n=2^{n+1}-2^n$
\item $a_n=\left(1+\frac{1}{n+1}\right)^{n+1}$
\item $a_n=n^{-n}$
\item $a_n=\frac{1+n}{3+n^n}$
\end{enumerate}}



\question{{\it Rekursiv definierte Folgen -- I}}

Bestimmen Sie für folgende rekursiv definierten Folgen jeweils die Fixpunkte, sowie ob die Iteration vom angegebenen Startwert gegen einen Fixpunkt konvergiert oder nicht:\\
\parbox{0.5\textwidth}{\begin{enumerate}
\item $x_{n+1}=\sqrt{\alpha^2+x_n}$, $x_0=0$
\item $x_{n+1}=x_n^2$, $x_0=\frac{1}{2}$
\item $x_{n+1}=x_n^2$, $x_0=2$
\end{enumerate}}\parbox{0.5\textwidth}{\begin{enumerate}\setcounter{enumi}{3}
\item $x_{n+1}=x_n-\frac{x_n^2-2}{2x_n}$, $x_0=1$
\item $x_{n+1}=x_n-\frac{x_n^3-8}{3x_n^2}$, $x_0=1$
\item $x_{n+1}=\cos x_n$, $x_0=0$
\end{enumerate}}




\question{{\it Rekursiv definierte Folgen -- II}}

Berechnen Sie für folgende rekursiv definierte Folgen jeweils die ersten zehn Glieder ausgehend von den angegebenen Startwerten. Bestimmen Sie ferner die allgemeine Form des $n$-ten Folgengliedes und benutzen Sie diese, um Ihre explizite Rechnung zu überprüfen:
\begin{enumerate}
\item $a_{n+2}=a_{n+1}+a_{n}$, $a_0=2$, $a_1=1$
\item $a_{n+2}=a_{n+1}-a_{n}$, $a_0=1$, $a_1=0$
\item $a_{n+2}=a_{n+1}+2a_{n}$, $a_0=0$, $a_1=1$
\item $a_{n+2}=2a_{n+1}-2a_{n}$, $a_0=0$, $a_1=2$
\end{enumerate}




\question{{\it Unendliche Reihen}}

Bestimmen Sie für folgende unendliche Reihen jeweils, ob diese konvergieren oder divergieren, und geben Sie gegebenenfalls den Wert der Reihe an:\\
\parbox{0.5\textwidth}{\begin{enumerate}
\item $\sum_{n=0}^\infty n$
\item $\sum_{n=1}^\infty \frac{1}{n}$
\item $\sum_{n=0}^\infty \frac{1}{2^n}$
\end{enumerate}}\parbox{0.5\textwidth}{\begin{enumerate}\setcounter{enumi}{3}
\item $\sum_{n=0}^\infty (-1)^n$
\item $\sum_{n=0}^\infty (-1)^n 13^{-n}$
\item $\sum_{n=0}^\infty \frac{1}{\sqrt{n+1}}$
\end{enumerate}}
