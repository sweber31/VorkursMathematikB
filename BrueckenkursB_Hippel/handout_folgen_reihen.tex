\documentclass[a4paper,10pt]{article}
\usepackage[utf8x]{inputenc}
\usepackage[german]{babel}
\usepackage{a4wide}
\usepackage{amsmath,amssymb,amsfonts}

\input{definitions}

\title{Zusammenfassung zum Thema Folgen und Reihen}
\author{Mathematischer Brückenkurs (B)\\für Naturwissenschaftler:innen}
\date{WS 2023/2024}

%%%%%%%%%%%%%%%%%%%%%%%%%%%%%%%%%%%%%%%%%%%%%%%%%%%%%%%%%%%%%%%%%%%%%%%%%%

\begin{document}
\parindent0pt
\maketitle

{\bf Der Begriff der Funktion}

Eine Funktion $f:D\to W$ ist eine Vorschrift, die jedem Element $x$ der Definitionsmenge $D$ genau ein Element $f(x)$ der Zielmenge $W$ eindeutig zuordnet, $x\mapsto f(x)$.

Diese Zuordnung kann, muss aber nicht, in Form einer Rechenvorschrift (wie z.B. $f(x)=x^2$) erfolgen; auch die Vorschrift, jeder/jedem Studierenden an der JGU ihre/seine Matrikelnummer zuzuordnen, stellt eine Funktion (von der Menge der Studierenden in die Menge der natürlichen Zahlen) dar. Entscheidend ist, dass der Funktionswert für jedes zulässige Argument eindeutig festgelegt ist.

Für Funktionen $f:A\to B$ und $g:B\to C$ definieren wir die Verkettung
$g\circ f:A\to C$ durch $(g\circ f)(x)=g(f(x))$.

Die Bildmenge von $f$, $\textrm{Im}~f=\{f(x)|x\in D\}\subseteq W$, ist die Menge
aller auftretenden Funktionswerte.

Eine Funktion $f$ kann durch ihren Graphen,
$\textrm{Graph}~f=\{(x,f(x))|x\in D\}\subset D\times W$,
repräsentiert werden.

Eine Funktion $f:D\to W$ heißt
\begin{itemize}
\item {injektiv}, wenn $x_1\not=x_2 \Rightarrow f(x_1)\not=f(x_2)$,
\item {surjektiv}, wenn $\forall y\in W\exists x\in D~y=f(x)$,
\item {bijektiv}, wenn sie injektiv und surjektiv ist.
\end{itemize}
Für bijektives $f:D\to W$ existiert eine Umkehrfunktion $f^{-1}:W\to D$,
die $\forall x\in D~f^{-1}(f(x))=x$ und $\forall y\in W~f(f^{-1}(y))=y$ erfüllt. \\

%%%%%%%%%%%%%%%%%%%%%%%%%%%%%%%%%%%%%%%%%%%%%%%%%%%%%%%%%%%%%%%%%%%%%%%%%%

{\bf Zahlenfolgen und Grenzwerte}

Unter einer Zahlenfolge verstehen wir eine Funktion $\Nset\to\Rset$, $n\mapsto a_n$ (oder $\Nset\to\Cset$, $n\mapsto a_n$). Die $a_n$ heißen Glieder der Folge.

Eine reellwertige Zahlenfolge $a_n$ heißt
\begin{itemize}\setlength\itemsep{0pt}\parskip0pt\setlength\parsep{0pt}
\item {nach oben beschränkt}, wenn $\exists S\in\Rset~\forall n\in\Nset~a_n\le S$,
\item {nach unten beschränkt}, wenn $\exists S\in\Rset~\forall n\in\Nset~a_n\ge S$,
\item {beschränkt}, wenn $\exists S\in\Rset~\forall n\in\Nset~|a_n|\le S$,
\item {monoton wachsend}, wenn $\forall n\in\Nset~a_{n+1}\ge a_n$,
\item {monoton fallend}, wenn $\forall n\in\Nset~a_{n+1}\le a_n$,
\item {streng monoton wachsend}, wenn $\forall n\in\Nset~a_{n+1}> a_n$,
\item {streng monoton fallend}, wenn $\forall n\in\Nset~a_{n+1}< a_n$,
\item {alternierend}, wenn $\forall n\in\Nset~a_{n+1}a_n<0$,
\item {Nullfolge}, wenn $\forall \epsilon>0~\exists N\in\Nset~\forall n>N~|a_n|<\epsilon$.
\end{itemize}
Für eine komplexwertige Zahlenfolge machen hiervon nur die Begriffe ``beschränkt'' und ``Nullfolge'' (mit der jeweils gleichen Definition wie für reellwertige Folgen) Sinn.

Eine Zahl $a$ heißt Grenzwert (Limes) der Folge $a_n$,
$a=\lim\limits_{n\to\infty} a_n$, wenn $(a_n-a)$ Nullfolge ist.

Insbesondere hat eine Nullfolge den Limes Null.

Eine Folge heißt {konvergent}, wenn sie einen Grenzwert hat. Eine Folge, die keinen Grenzwert hat, heißt {divergent}.
Von einer reellwertigen Folge, die keinen Grenzwert hat, sagen wir
sie divergiere gegen $+\infty$, wenn $\forall S\in\Rset~\exists N\in\Nset~\forall n>N~a_n>S$, und
sie divergiere gegen $-\infty$, wenn $\forall S\in\Rset~\exists N\in\Nset~\forall n>N~a_n<S$.
Von sonstigen reellwertigen Folgen sagen wir auch, sie seien unbestimmt
divergent.\\

%%%%%%%%%%%%%%%%%%%%%%%%%%%%%%%%%%%%%%%%%%%%%%%%%%%%%%%%%%%%%%%%%%%%%%%%%%

{\bf Rekursiv definierte Folgen}

Aufgrund des Prinzips der vollständigen Induktion kann eine Folge über
eine Rekursionsrelation $a_{n+1}=f(a_n,\ldots,a_{n-k})$
zusammen mit $k+1$ Startwerten $a_0,\ldots,a_k$ definiert werden.

Ein wichtiger Spezialfall ist die Fixpunktiteration:
Für Funktionen $f:D\to D$, $D\subseteq\Rset$, die für alle $a$, $b\in D$
die Ungleichung
$|f(a)-f(b)|\le q|a-b|$ mit
einer Konstanten $q\in(0;1)$ erfüllen,
konvergiert die Folge $a_{n+1}=f(a_n)$ gegen die (eindeutige) Lösung
von $x=f(x)$.

Ein weiterer wichtiger Spezialfall sind Rekursionen der Form
$a_{n+1}=\alpha_0 a_n +\ldots+\alpha_k a_{n-k}$, die sich mit
dem Ansatz $a_n=\lambda^n$ lösen lassen:
Man löst $\lambda^{k+1} = \alpha_0 \lambda^k+\ldots+\alpha_{k-1}\lambda+\alpha_k$, um
$k+1$ Lösungen $\lambda_i\in\Cset$ zu finden.
Da mit den $\lambda_i^n$ auch jede Kombination der Form
$a_n=C_1\lambda_1^n+\ldots+C_k\lambda_{k+1}^n$ (mit Konstanten $C_i\in\Cset$)
die Rekursion löst, können die $k+1$ Konstanten $C_i$ aus den $k+1$
Startwerten durch Lösen des linearen Gleichungssystems
($a_n=C_1\lambda_1^n+\ldots+C_k\lambda_{k+1}^n$, $n=0,\ldots,k$)
bestimmt werden.

Ein bekanntes Beispiel sind die Fibonacci-Zahlen, die durch
$a_0=1$, $a_1=1$, $a_{n+1}=a_n+a_{n-1}$ definiert sind,
mit der Lösung
$a_n=(\lambda_+^{n+1}-\lambda_-^{n+1})/(\lambda_+-\lambda_-)$, wobei
$\lambda_\pm = \frac{1}{2}\pm\frac{1}{2}\sqrt{5}$.\\

%%%%%%%%%%%%%%%%%%%%%%%%%%%%%%%%%%%%%%%%%%%%%%%%%%%%%%%%%%%%%%%%%%%%%%%%%%

{\bf Unendliche Reihen}

Für eine Folge $a_n$ definieren wir die Folge der Partialsummen
\[
S_N=\sum_{n=0}^N a_n
\]
und die unendliche Reihe
\[
\sum_{n=0}^\infty a_n \equiv \lim_{N\to\infty} S_N \,,
\]
falls dieser Grenzwert existiert. Falls der Grenzwert nicht existiert,
heißt die unendliche Reihe divergent.

Die geometrische Reihe ist
\[
\sum_{n=0}^\infty q^n = \frac{1}{1-q}
\]
für $|q|<1$ (sonst divergent) wegen der ``Teleskopsumme''
\[
(1-q)\sum_{n=0}^N q^n = \sum_{n=0}^N \left(q^n-q^{n+1}\right) = 1-q^{N+1} \,,
\]
während die harmonische Reihe
\[
\sum_{n=1}^\infty \frac{1}{n}
\]
wegen
\[
\sum_{n=1}^{2^N} \frac{1}{n} \ge 1 + \frac{N}{2}
\]
divergiert.

\end{document}

