\documentclass[a4paper,10pt]{article}
\usepackage{a4wide}
\usepackage[utf8x]{inputenc}

\begin{document}

\title{\vskip-12ex Kommentierte Literaturliste}
\author{zum Mathematischen Brückenkurs (B) \\ für Naturwissenschaftler:innen}
\date{WS 2023/2024}
\maketitle

\parindent0pt
\pagestyle{empty}\thispagestyle{empty}

%\textbf{A. Literatur speziell zum Brückenkurs} %% if A/B/C

\begin{enumerate}
\item J. Erven, M. Erven, J. Hörwick, {\em Vorkurs Mathematik}, Oldenbourg Verlag (München, 2010)
\item H. J. Korsch, {\em Mathematik-Vorkurs}, Binomi Verlag (Barsinghausen, 2008)
\item P. van Dongen, {\em Einführungskurs Mathematik und Rechenmethoden}, Springer Spektrum (Wiesbaden, 2015) %% if not A/B/C as for physicists
\item G. Walz, F. Zeilfelder, T. Rießinger, {\em Brückenkurs Mathematik}, Springer Spektrum (Berlin und Heidelberg, 2014)
\item F. Ayres, E. Mendelson, {\em Schaum's Outline of Calculus} (Fourth Ed.), McGraw-Hill\\ (New~York,~1999)
\item I.N. Bronstein, K.A. Semendjajew, G. Musiol, H. Mühlig, {\em Taschenbuch der Mathematik} (7.~Aufl.), Harri Deutsch Verlag (Frankfurt/Main, 2008)
\end{enumerate}

Der Brückenkurs folgt in seiner ersten, rekapitulierenden, Hälfte annähernd dem Buch von Erven, Erven und Hörwick, wobei sich die Diskussion der Vektorrechnung und linearen Algebra stärker an Elemente des Buches von Korsch anlehnt.
Das Buch von van Dongen enthält viele weiterführende Themen, die weit über den Brückenkurs hinausgehen.
Das Buch von Walz {\em et al.} hingegen ist einfacher gehalten und deckt nicht den gesamten Stoff des Brückenkurses ab.
Details zur ein- und mehrdimensionalen Differential- und Integralrechnung finden sich im englischsprachigen Buch von Ayres und Mendelson. %
Der ``Bronstein'' ist ein unverzichtbarer Begleiter für das gesamte Studium der Physik (und anderer quantitativ orientierter Naturwissenschaften); dies gilt insbesondere für die Integraltabellen im hinteren Teil des Buches.
\\[1ex]

\end{document} %% if not A/B/C

\textbf{B. Literatur zum Brückenkurs und fürs Studium}

\begin{enumerate}
\item H.J. Korsch, {\em Mathematische Ergänzungen zur Einführung in die Physik}, Binomi Verlag (Barsinghausen, 2008)
\item M. Kallenrode, {\em Rechenmethoden der Physik: Mathematischer Begleiter zur Experimentalphysik}, Springer (Berlin, 2005)
\item P. van Dongen, {\em Einführungskurs Mathematik und Rechenmethoden}, Springer Spektrum (Wiesbaden, 2015)
\item K.-H. Goldhorn, H.-P. Heinz, {\em Mathematik für Physiker 1-3}, Springer (Berlin, 2007)
\item K. Jänich, {\em Mathematik 1+2: geschrieben für Physiker}, Springer (Berlin, 2005)
\end{enumerate}

Das Buch von Korsch enthält vieles, was im zweiten, vorausblickenden, Teil des Brückenkurses vorkommt, geht jedoch im Stoff weit darüber hinaus und deckt einen Großteil der physikalischen Rechenmethoden ab. Ähnliches gilt für die Bücher von Kallenrode und von van Dongen. Die Bücherserien von Goldhorn und Heinz sowie von Jänich kommen von der mathematischen Seite her und decken sowohl vertiefende Fragen als auch Themen, die später im Zusammenhang mit dem Studium der Theoretischen Physik von Relevanz werden, ab.\\[1ex]

\textbf{C. Nachschlagewerke zum häufigen Gebrauch im Studium}

\begin{enumerate}
\item I.N. Bronstein, K.A. Semendjajew, G. Musiol, H. Mühlig, {\em Taschenbuch der Mathematik} (7.~Aufl.), Harri Deutsch Verlag (Frankfurt/Main, 2008)
\item I.S. Gradshteyn, I.M. Ryzhik, A. Jeffrey, D. Zwillinger, {\em Table of Integrals, Series, and Products} (Seventh Ed.), Academic Press (Amsterdam, 2007)
\end{enumerate}

Der ``Bronstein'' ist ein unverzichtbarer Begleiter für das gesamte Studium der Physik; dies gilt insbesondere für die Integraltabellen im hinteren Teil des Buches. Der sehr umfangreiche Gradshteyn-Ryzhik enthält so gut wie jedes bekannte Integral; was sich im Bronstein nicht finden lässt, findet sich zumindest hier.

\end{document}

