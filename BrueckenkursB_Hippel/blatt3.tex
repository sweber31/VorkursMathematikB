\documentclass[11pt]{exam}
\usepackage[german]{babel}
\usepackage[utf8x]{inputenc}
\usepackage{graphicx}
\usepackage{latexsym,ifthen,amssymb,amsfonts,amsmath}

\begin{document}

\include{definitions}

\pagestyle{empty}

\def\loesungen{0}
\newcommand{\lansel}[2]{#1}

\ifthenelse{\equal{\loesungen}{1}}{\printanswers}{\relax}
\renewcommand{\solutiontitle}{\noindent\textbf{L\"osung:}\enspace}
\newcommand{\loesungname}{\ifthenelse{\equal{\loesungen}{1}}{L\"osungen zu }{\relax}}

\begin{center}
\textbf{\LARGE \loesungname \lansel{\"Ubungsblatt}{Examples Sheet} 4} \\ \vspace{1ex}
\textbf{\large \lansel{zum Mathematischen Brückenkurs \\ für Naturwissenschaftler:innen}{for the preparatory mathematics course for bio- and geoscientists}} \\ \vspace{1ex}
\textbf{\large \lansel{im Wintersemester}{Winter} 2023/24} \\ \vspace{0.5cm}
\textrm{\normalsize \hfill \lansel{Dozent}{Lecturer}: Apl.Prof. Dr. G. von Hippel\hfill${}$}
\end{center}
\normalsize\vspace{0.5cm}

\ifthenelse{\equal{\loesungen}{1}}{
\begin{center}
\textbf{Die direkte Weitergabe der Musterl\"osungen an Studierende ist nicht gestattet!}
\end{center}
}{
\vspace{3ex}
}

\begin{questions}
\pointname{ P.}

%%%%%%%%%%%%%%%%%%%%%%%%%%%%%%%%%%%%%%%%%%%%%%%%%%%%%%%%%%%%%%%%%%%%%%%%%%%%%%%

\question{{\it Eigenschaften der Hyperbelfunktionen}}

Zeigen Sie, dass die Hyperbelfunktionen $\sinh$ und $\cosh$ folgenden Beziehungen genügen:\\
\parbox{0.5\textwidth}{\begin{enumerate}
\item $\cosh^2 x-\sinh^2 x=1$
\item $\sinh (ix)=i\sin x$
\item $\cosh (ix)=\cos x$
\end{enumerate}}\parbox{0.5\textwidth}{\begin{enumerate}\setcounter{enumi}{3}
\item $\cosh^2 x+\sinh^2 x=\cosh (2x)$
\item $\sinh x+\cosh x=\rme^x$
\item $\cosh x-\sinh x=\rme^{-x}$
\end{enumerate}}



\question{{\it Ableitungen von Umkehrfunktionen}}

Benutzen Sie jeweils die Regel für die Ableitung der Umkehrfunktion, um die Ableitungen folgender Funktionen zu bestimmen:\\
\parbox{0.5\textwidth}{\begin{enumerate}
\item $x\mapsto\arcsin x$
\item $x\mapsto\arctan x$
\end{enumerate}}\parbox{0.5\textwidth}{\begin{enumerate}\setcounter{enumi}{2}
\item $x\mapsto \mathrm{arsinh}~x$
\item $x\mapsto \mathrm{artanh}~x$
\end{enumerate}}



\question{{\it Stammfunktionen}}

Bestimmen Sie für die folgenden Funktionen $f:D\to\Rset$ jeweils die maximale Definitionsmenge $D$ sowie eine Stammfunktion auf $D$:\\
\parbox{0.4\textwidth}{\begin{enumerate}
\item $f(x)=x^2$
\item $f(x)=\frac{1}{x^4}$
\item $f(x)=x^5+x^3-x$
\item $f(x)=(x^2-1)^2$
\item $f(x)=\rme^x$
\item $f(x)=\rme^{-x}$
\item $f(x)=\sin x$
\item $f(x)=\cos (x+a)$
\item $f(x)=\sinh x$
\item $f(x)=\cosh x$
\item $f(x)=\log x$
\item $f(x)=a x^2+\rme^{-bx}+\log (c x+d)$
\item $f(x)=x\log x$
\end{enumerate}}\parbox{0.6\textwidth}{\begin{enumerate}\setcounter{enumi}{13}
\item $f(x)=x^n\log x$
\item $f(x)=\frac{x^2}{1+x}$
\item $f(x)=x\rme^x$
\item $f(x)=\rme^{ax}\sin (\omega x)$
\item $f(x)=\sin x\cos x$
\item $f(x)=\frac{1}{\sqrt{1-x^2}}$
\item $f(x)=\sqrt{1+x^2}$
\item $f(x)=\rme^{\sin(\lambda x)}\cos(\lambda x)$
\item $f(x)=\frac{x}{\sqrt{1+x^2}}$
\item $f(x)=\frac{\sqrt{1+x}}{\sqrt{1-x^2}}$
\item $f(x)=\rme^{-\sin^2 x}\cos x\sin x$
\item $f(x)=\frac{2 x^3}{(x^2+1)^2}$
\item $f(x)=\frac{7 x^3-5 x^2-6}{x^4-x^3-x^2-x-2}$
\end{enumerate}}



\question{{\it Bestimmte Integrale}}

Bestimmen Sie jeweils den Wert der folgenden bestimmten Integrale:\\
\parbox{0.5\textwidth}{\begin{enumerate}
\item $\int_0^1 x~\rmd x$  
\item $\int_a^b x^n~\rmd x$
\item $\int_\alpha^\beta (3x^2-2\beta x+\alpha\beta)~\rmd x$  
\item $\int_0^1 \rme^x~\rmd x$ 
\item $\int_0^\pi \sin\alpha~\rmd\alpha$  
\item $\int_0^\pi \cos\beta~\rmd\beta$  
\item $\int_{-1}^1\sqrt{1-x^2}~\rmd x$  
\item $\int_{-1}^1\sqrt{1+x^2}~\rmd x$  
\item $\int_0^2 \frac{2x}{1+x^2}\rmd x$  
\item $\int_{\frac{1}{2}}^2\log x~\rmd x$  
\item $\int_{\frac{1}{2}}^2\frac{\log x}{x}\rmd x$  
\item $\int_0^{2\pi}\sin^2\omega~\rmd\omega$  
\item $\int_0^{2\pi}\sin^2\omega\cos\omega~\rmd\omega$  
\end{enumerate}}\parbox{0.5\textwidth}{\begin{enumerate}\setcounter{enumi}{13}
\item $\int_{-\pi}^{\pi/3}\sin x\cos x~\rmd x$  
\item $\int_1^{\rme^n} x^n\log x~\rmd x$  
\item $\int_0^1 \frac{7 x^3-5 x^2-6}{x^4-x^3-x^2-x-2}\rmd x$  
\item $\int_2^3 \frac{2 x^3}{(x^2+1)^2}\rmd x$  
\item $\int_0^y \frac{\rmd x}{1-xy}$,~$y<1$
\item $\int_0^y \frac{\rmd x}{1+xy}$,~$y>0$
\item $\int_0^{\pi/2}\rme^{-\sin^2 x}\cos x\sin x~\rmd x$  
\item $\int_{-1}^1\tanh t~\rmd t$  
\item $\int_\rme^{\rme^2}\frac{\log(\log \xi)}{\xi}\rmd\xi$  
\item $\int_\rme^{\rme^2}\frac{\log\xi \log(\log \xi)}{\xi}\rmd\xi$  
\item $\int_0^\omega \sinh(\cosh u)\sinh u~\rmd u$  
\item $\int_0^{\frac{\pi }{2}} \frac{(\sin x+9) \cos x}{\cos^2 x +8}~\rmd x$  
\item $\int_0^2 x^5\rme^{-x^2}~\rmd x$  
\end{enumerate}}



\question{{\it Uneigentliche Integrale}}

Bestimmen Sie jeweils, ob folgende uneigentliche Integrale existieren und bestimmen Sie gegebenenfalls deren Wert:\\
\parbox{0.5\textwidth}{\begin{enumerate}
\item $\int_0^\infty \frac{\rmd x}{x}$ 
\item $\int_1^\infty \frac{\rmd y}{y}$
\item $\int_0^\infty \frac{\rmd x}{x^2}$
\item $\int_1^\infty \frac{\rmd x}{x^2}$
\item $\int_0^\infty \frac{\rmd z}{\sqrt{z}}$
\item $\int_1^\infty \frac{\rmd u}{\sqrt{u}}$
\item $\int_0^1 \frac{\rmd u}{\sqrt{u}}$
\end{enumerate}}\parbox{0.5\textwidth}{\begin{enumerate}\setcounter{enumi}{7}
\item $\int_0^\infty \rme^{x}~\rmd x$
\item $\int_{-\infty}^0 \rme^{x}~\rmd x$
\item $\int_0^\infty \rme^{-x}~\rmd x$
\item $\int_{-\infty}^0 \rme^{-x}~\rmd x$
\item $\int_0^\infty x\rme^{-x}~\rmd x$
\item $\int_{-\infty}^\infty x\rme^{-x^2/2}~\rmd x$
\item $\int_{-\infty}^\infty \frac{\rmd \omega}{1+\omega^2}$ 
\end{enumerate}}




\end{questions}

\end{document}
