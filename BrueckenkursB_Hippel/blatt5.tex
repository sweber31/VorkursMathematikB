\documentclass[11pt]{exam}
\usepackage[german]{babel}
\usepackage[utf8x]{inputenc}
\usepackage{graphicx}
\usepackage{latexsym,ifthen,amssymb,amsfonts,amsmath}

\begin{document}

\include{definitions}

\pagestyle{empty}

\def\loesungen{0}
\newcommand{\lansel}[2]{#1}

\ifthenelse{\equal{\loesungen}{1}}{\printanswers}{\relax}
\renewcommand{\solutiontitle}{\noindent\textbf{L\"osung:}\enspace}
\newcommand{\loesungname}{\ifthenelse{\equal{\loesungen}{1}}{L\"osungen zu }{\relax}}

\begin{center}
\textbf{\LARGE \loesungname \lansel{\"Ubungsblatt}{Examples Sheet} 6} \\ \vspace{1ex}
\textbf{\large \lansel{zum Mathematischen Brückenkurs \\ für Naturwissenschaftler:innen}{for the preparatory mathematics course for bio- and geoscientists}} \\ \vspace{1ex}
\textbf{\large \lansel{im Wintersemester}{Winter} 2023/24} \\ \vspace{0.5cm}
\textrm{\normalsize \hfill \lansel{Dozent}{Lecturer}: Apl.Prof. Dr. G. von Hippel\hfill${}$}
\end{center}
\normalsize\vspace{0.5cm}

\ifthenelse{\equal{\loesungen}{1}}{
\begin{center}
\textbf{Die direkte Weitergabe der Musterl\"osungen an Studierende ist nicht gestattet!}
\end{center}
}{
\vspace{3ex}
}

\begin{questions}
\pointname{ P.}

%%%%%%%%%%%%%%%%%%%%%%%%%%%%%%%%%%%%%%%%%%%%%%%%%%%%%%%%%%%%%%%%%%%%%%%%%%%%%%%

\question{{\it Bewegung entlang gekrümmter Kurven}}

Berechnen Sie für folgende Bahnkurven $x:[0;T]\to\Rset^3$,
$t\mapsto\vec{x}(t)$ jeweils die Geschwindigkeit $\dot{\vec{x}}$ und
Beschleunigung $\ddot{\vec{x}}$ als Funktion der Zeit $t$.
Bestimmen Sie (soweit möglich) ferner die gesamte zurückgelegte Strecke
$s=\int_0^T |\dot{\vec{x}}|~\rmd t$.\\
\parbox{0.5\textwidth}{\begin{enumerate}
\item $\vec{x}(t)=(vt,0,0)$
\item $\vec{x}(t)=(0,0,ut)$
\item $\vec{x}(t)=(v_1t,v_2t,v_3t)$
\item $\vec{x}(t)=(d,d,h-kt^2)$
\item $\vec{x}(t)=(d,d+wt,h-\frac{1}{2}gt^2)$
\item $\vec{x}(t)=(s_1+w_1t,s_2+w_2t,s_3+w_3t-\frac{a}{2}t^2)$
\item $\vec{x}(t)=(r\sin(2\pi t/T),r\cos(2\pi t/T),0)$
\item $\vec{x}(t)=(r\sin(2\pi t/T),r\cos(2\pi t/T),ut)$
\item $\vec{x}(t)=(r\sin(2\pi t/T),r\cos(2\pi t/T),t(u-kt))$
\end{enumerate}}\parbox{0.5\textwidth}{\begin{enumerate}\setcounter{enumi}{9}
\item $\vec{x}(t)=(a\sin(2\pi t/T),b\cos(2\pi t/T),0)$
\item $\vec{x}(t)=(a\cos(2\pi t/T),b\sin(2\pi t/T),0)$
\item $\vec{x}(t)=(a\sin(2\pi t/T),b\cos(2\pi t/T),c\sin(8\pi t/T))$
\item $\vec{x}(t)=(v(t-T/2),0,\sqrt{d^2+u^2(t-T/2)^2})$
\item $\vec{x}(t)=(aT/(t+T),bT/(t+T),ct/(t+T))$
\item $\vec{x}(t)=(d+vt,d-vt,R+a\sin(2\pi t/T))$
\item $\vec{x}(t)=(\lambda\rme^{-\alpha t/T},\mu\rme^{\beta t/T},\xi t/T)$
\item $\vec{x}(t)=\rme^{-\gamma t/T}(R\cos(2\pi t/T),R\sin(2\pi t/T),R)$
\item $\vec{x}(t)=(\frac{RT}{t+T}\cos(2\pi t/T),\frac{RT}{t+T}\sin(2\pi t/T),0)$
\end{enumerate}}



\question{{\it Partielle Ableitungen}}

Bestimmen Sie jeweils die ersten und zweiten (einschließlich aller gemischten)
partiellen Ableitungen der folgenden Funktionen nach den angegebenen Variablen.
Verifizieren Sie jeweils, dass die partiellen Ableitungen vertauschen.\\
\parbox{0.5\textwidth}{\begin{enumerate}
\item $f(x,y)=x+y^2$
\item $g(x,y,z)=x+y+z-xy-yz-zx+xyz$
\item $h(\alpha,\beta,\gamma,\delta)=\frac{\alpha^2-\beta^2}{\gamma^2+\delta^2}$
\item $f(x_1,x_2,x_3)=\sqrt{x_1^2+x_2^2+x_3^2}$
\item $V(x,y,z)=\frac{1}{\sqrt{x^2+y^2+z^2}}$
\item $G(\phi_1,\phi_2)=(\phi_1^2+\phi_2^2)$
\item $H(\phi_1,\phi_2)=(\phi_1^2+\phi_2^2)^2$
\item $V(\phi_1,\phi_2)=\left(\phi_1^2+\phi_2^2-v^2\right)^2$
\end{enumerate}}\parbox{0.5\textwidth}{\begin{enumerate}\setcounter{enumi}{8}
\item $S(\alpha,\omega)=\frac{\alpha\omega}{\alpha^2+\omega^2+1}$
\item $P(r,s)=\rme^{-(r^2+s^2)}$
\item $p(x,y,z)=\rme^{-\alpha x^2-\beta y^2-\gamma z^2}$
\item $f(x,y)=\sin(xy)\cos(x+y)$
\item $\zeta(s_1,s_2)=(s_1+s_2)\rme^{-s_1}\sin^2(s_2)$
\item $a(\omega,\gamma,\phi,t)=\frac{A}{\omega^2-\gamma^2}\sin(\omega t+\phi)$
\item $A(s,\Gamma)=\frac{1}{(s^2-\Gamma^2)^2+4s^2\Gamma^2}$
\item $f(x_1,x_2,x_3)=\sum_{i=1}^3\sin\left(\sum_{j,k=1}^3\epsilon_{ijk}x_k\right)$
\end{enumerate}}


\end{document}
\pagebreak


\question{{\it Jacobi-Matrix}}% und -Determinante}}

Geben Sie für die folgenden vektorwertigen Funktionen $\Rset^n\to\Rset^m$ jeweils die Jacobi-Matrix an. %Wenn $n=m$, bestimmen Sie ferner die Jacobi-Determinante.
\begin{enumerate}
\item $f:\Rset^2\to\Rset^4,~\vec{f}(\vec{x})=(x_1,x_1,x_2,x_2)$
\item $g:\Rset^3\to\Rset^2,~\vec{g}(\vec{y})=(y_2y_3,y_1/y_3)$
\item $\Sigma:\Rset^2\to\Rset^3,~\vec{\Sigma}(\vec{\lambda})=(a\sin\lambda_1\sin\lambda_2,b\cos\lambda_1\sin\lambda_2,c\cos\lambda_2)$
\item $Z:\Rset^3\to\Rset^3,~\vec{Z}(\vec{c})=(c_1\sin c_2,c_1\cos c_2,c_3)$
\item $\Theta:\Rset^2\to\Rset^2,~\vec{\Theta}(\vec{x})=(x_1+\xi x_2,x_2)$
\item $H:\Rset^2\to\Rset^2,~\vec{H}(\vec{x})=(x_1+\xi x_2^3,x_2)$
\item $V:\Rset^3\to\Rset^4,~\vec{V}(\vec{z})=(\sqrt{z_1^2+\eta z_3^2},z_1,z_2,z_3+\eta z_1)$
\item $K:\Rset^3\to\Rset^3,~K_i(\vec{x})=\cos\left(\sum_{j,k=1}^3\epsilon_{ijk}x_k\right)$
\end{enumerate}




%\question{{\it Divergenz und Rotation}}
%
%Berechnen Sie jeweils die Divergenz und Rotation der folgenden Vektorfelder im $\Rset^3$. Verifizieren Sie jeweils explizit, dass die Divergenz der Rotation verschwindet. $\vec{a}$ und $\vec{b}$ seien jeweils konstante Vektoren.\\
%\parbox{0.5\textwidth}{\begin{enumerate}
%\item $\vec{f}(\vec{x})=\vec{a}$
%\item $\vec{g}(\vec{x})=\vec{x}$
%\item $\vec{h}(\vec{x})=(x_2,-x_1,x_3)$
%\item $\vec{f}(\vec{x})=\nabla(\vec{x}\cdot\vec{x})$
%\item $\vec{B}(\vec{x})=\vec{a}\times\vec{x}$
%\item $\vec{f}(\vec{x})=\vec{a}\times(\vec{x}-\vec{b})+\vec{x}+\vec{b}$
%\end{enumerate}}\parbox{0.5\textwidth}{\begin{enumerate}\setcounter{enumi}{6}
%\item $\vec{f}(\vec{x})=(\vec{a}\cdot\vec{x})^\alpha\vec{a}$
%\item $\vec{f}(\vec{x})=(\vec{a}\cdot\vec{x})^\alpha\vec{x}$
%\item $\vec{E}(\vec{x})=|\vec{x}|^\alpha\vec{a}$
%\item $\vec{F}(\vec{x})=|\vec{x}|^\alpha\vec{x}$
%\item $\vec{G}(\vec{x})=(\vec{a}\cdot\vec{x})\vec{b}-(\vec{b}\cdot\vec{x})\vec{a}$
%\item $f_i(\vec{x})=\sin\left(\sum_{j,k=1}^3\epsilon_{ijk}x_k\right)$
%\end{enumerate}}


\question{{\it Kurvenintegrale}}

Berechnen Sie jeweils die Kurvenintegrale der Vektorfelder
%aus der vorstehenden Frage
\\
\parbox{0.5\textwidth}{\begin{enumerate}
\item $\vec{f}(\vec{x})=\vec{a}$
\item $\vec{g}(\vec{x})=\vec{x}$
\item $\vec{h}(\vec{x})=(x_2,-x_1,x_3)$
\item $\vec{f}(\vec{x})=\nabla(\vec{x}\cdot\vec{x})$
\item $\vec{B}(\vec{x})=\vec{a}\times\vec{x}$
\item $\vec{f}(\vec{x})=\vec{a}\times(\vec{x}-\vec{b})+\vec{x}+\vec{b}$
\end{enumerate}}\parbox{0.5\textwidth}{\begin{enumerate}\setcounter{enumi}{6}
\item $\vec{f}(\vec{x})=(\vec{a}\cdot\vec{x})^\alpha\vec{a}$
\item $\vec{f}(\vec{x})=(\vec{a}\cdot\vec{x})^\alpha\vec{x}$
\item $\vec{E}(\vec{x})=|\vec{x}|^\alpha\vec{a}$
\item $\vec{F}(\vec{x})=|\vec{x}|^\alpha\vec{x}$
\item $\vec{G}(\vec{x})=(\vec{a}\cdot\vec{x})\vec{b}-(\vec{b}\cdot\vec{x})\vec{a}$
\item $f_i(\vec{x})=\sin\left(\sum_{j,k=1}^3\epsilon_{ijk}x_k\right)$
\end{enumerate}}
längs der beiden Kurven
\begin{align*}
\mathcal{C}_1 &= \{(0,0,t)|t\in[0;1]\}, \\
\mathcal{C}_2 &= \{(\cos (t),\sin(t),0)|t\in[0;2\pi]\}.
\end{align*}



%\question{{\it Volumenintegrale}}
%
%Integrieren Sie die folgenden Funktionen jeweils über das angegebene Volumen im $\Rset^3$. Falls sinnvoll, führen Sie hierzu jeweils eine Transformation in geeignete krummlinige Koordinaten durch.\\
%\parbox{0.47\textwidth}{\begin{enumerate}
%\item $f(\vec{x})=1$,~$\vec{x}\in[0;1]\times[1;3]\times[2;5]$
%\item $s(\vec{\omega})=\sin(\omega_1)\sin(\omega_2)\sin(\omega_3)$,~$\vec{\omega}\in[0;\pi]^3$
%\item $g(\vec{x})=|\vec{x}|$,~$r<|\vec{x}|<R$
%\item $h(\vec{x})=1/|\vec{x}|$,~$r<|\vec{x}|<R$
%\end{enumerate}}\parbox{0.53\textwidth}{\begin{enumerate}\setcounter{enumi}{4}
%\item $f(\vec{x})=A(1-\vec{x}\cdot\vec{x}/R^2),~|\vec{x}|<R$
%\item $f(\vec{x})=\rme^{-\alpha (\vec{x}\cdot\vec{x}-x_3^2)}$,~$x_3\in[-1;1]$
%\item $g(\vec{y})=1/y_3^4,~y_3>1\wedge (y_1^2+y_2^2)\in [1;4]$
%\item $\phi(\vec{x})=\arctan\left(\frac{x_2}{x_1}\right),~x_3>0\wedge x_2>0\wedge |\vec{x}|<R$
%\end{enumerate}}



\end{questions}

\end{document}
