\documentclass[a4paper,10pt]{article}
\usepackage[utf8x]{inputenc}
\usepackage[german]{babel}
\usepackage{a4wide}
\usepackage{amsmath,amssymb,amsfonts}

\input{definitions}

\title{Zusammenfassung zum Thema \\ Gewöhnliche Differentialgleichungen}
\author{Mathematischer Brückenkurs (B)\\für Naturwissenschaftler:innen}
\date{WS 2023/2024}

%%%%%%%%%%%%%%%%%%%%%%%%%%%%%%%%%%%%%%%%%%%%%%%%%%%%%%%%%%%%%%%%%%%%%%%%%%

\begin{document}
\parindent0pt
\maketitle


{\bf Gewöhnliche Differentialgleichungen}

Physikalische Gesetze bestimmen zukünftige Zustände durch die momentane
Veränderung physikalischer Größen in Abhängigkeit vom momentanen Zustand.
Mathematisch lässt sich eine solche Beziehung durch eine gewöhnliche
Differentialgleichung
\[
f(t,x(t),\dot{x}(t),\ddot{x}(t),\ldots,x^{(n)}(t))=0
\]
ausdrücken.
Physikalische Gesetze sind in der Regel Differentialgleichungen
2.~Ordnung ($n=2$), gelegentlich auch 1.~Ordnung ($n=1$).

Um eine eindeutige Lösung zu erhalten, müssen wir die
Differentialgleichung durch $n$ Anfangswertbedingungen
$x(0)=x_{0}$, $\ldots$, $x^{(n-1)}(0)=x^{(n-1)}_{0}$ ergänzen,
die den Ausgangszustand beschreiben.

%Zum Beispiel benötigt die Newtonsche Bewegungsgleichung
%$m\ddot{\vec{x}}=\vec{F}(t,\vec{x}(t),\dot{\vec{x}}(t))$
%die Angabe von Ausgangsort $\vec{x}(0)=\vec{x}_0$ und
%Ausgangsgeschwindigkeit $\dot{\vec{x}}(0)=\vec{v}_0$
%als Anfangswerten, um ein wohlgestelltes mechanisches Problem zu erhalten.

%Für Differentialgleichungen der Form
%\[
%\dot{x}(t)=f(x(t),t)
%\]
%definieren wir das Richtungsfeld $(1,\dot{x}(t))$ in der $(t,x(t))$-Ebene.
%Dieses Vektorfeld ist überall tangential
%zur Kurvenschar der Graphen der Lösungen.

Wir unterscheiden homogene lineare Differentialgleichungen
\[
\sum_{k=0}^n c_k\frac{\rmd^kx}{\rmd t^k}=0
\]
von inhomogenen linearen Differentialgleichungen
\[
\sum_{k=0}^n c_k\frac{\rmd^kx}{\rmd t^k}=f(t)
\]
Wenn $x_p(t)$ irgendeine Lösung einer inhomogenen linearen
Differentialgleichung (sog. partikuläre Lösung) ist
und $x_h(t)$ die zugehörige homogene Differentialgleichung löst, so
ist auch $x(t)=x_p(t)+ x_h(t)$ eine Lösung der inhomogenen Gleichung.\\

%\pagebreak
{\bf Einige lösbare Differentialgleichungen erster Ordnung}

Eine Differentialgleichung der trivialen Form
\[
\dot{x}(t)=f(t)
\]
hat die Lösung
\[
x(t)=\int f(t)\rmd t + c
\]
wobei die Integrationskonstante $c$ durch die Anfangswertbedingung
$x(0)=x_0$ bestimmt wird.
Entsprechend können Differentialgleichungen beliebiger Ordnung der Form
\[
x^{(n)}(t)=f(t)
\]
durch $n$-faches Integrieren gelöst werden. Die $n$ Integrationskonstanten
sind durch die $n$ Anfangswertbedingungen $x^{(k)}(0)=x_0^{(k)}$
eindeutig bestimmt.

Die einfachste nicht-triviale Differentialgleichung
\[
\dot{x}(t)=\lambda x(t)
\]
beschreibt einen Prozess, in dem die Veränderung einer Größe proportional
zu dieser Größe selbst ist.
Für $\lambda>0$ ist dies ein Wachstumsprozess (z.B. Bakterienkolonie), für $\lambda<0$ ein Zerfallsprozess (z.B. radioaktiver Zerfall). Die Lösung der Wachstums- oder Zerfallsgleichung lautet $x(t)=x_0 \rme^{\lambda t}$.

%Unbegrenztes Wachstum ist unrealistisch -- endliche Ressourcen
%beschränken Wachstumsprozesse.
Die logistische Differentialgleichung (frz. {\em logis} = Lebensraum)
\[
\dot{x}(t)=\lambda x(t) \left[ \kappa - x(t)\right]
\]
beschreibt einen durch eine endliche Kapazität $\kappa$ beschränkten
Wachstumsprozess.
%Das Richtungsfeld zeigt, dass die Lösungskurven für $t\to\infty$ gegen
%$x(t)=\kappa$ streben; d
Die Lösung lautet:
$x(t)=\frac{\kappa}{1-(1-\kappa/x_0)\rme^{-\kappa\lambda t}}$.\\%

%Eine Differentialgleichung der Form
%\[
%f(x(t))\frac{\rmd x}{\rmd t}=g(t)
%\]
%heißt separabel und kann durch Integration auf die Form
%\[
%F(x(t)) = G(t) + c
%\]
%mit $F$, $G$ Stammfunktionen von $f$, $g$ gebracht werden.
%Die Integrationskonstante $c$ wird durch die Anfangswertbedingung bestimmt.
%Die Wachstumsgleichung und die logistische Gleichung sind Beispiele
%für separable Differentialgleichungen.
%
%Eine weitere Klasse lösbarer Differentialgleichungen erster Ordnung sind
%Differentialgleichungen der Form
%\[
%u(x,t)\frac{\rmd x}{\rmd t}+v(x,t)=0
%~~~~~\textsf{mit}~~
%\frac{\partial u}{\partial t}=\frac{\partial v}{\partial x}
%\]
%(sog. exakte Differentialgleichungen), die sich aufgrund der Beziehungen
%\[
%\frac{\rmd f}{\rmd s} = \frac{\partial f}{\partial x}\frac{\rmd x}{\rmd s}
%+\frac{\partial f}{\partial t}\frac{\rmd t}{\rmd s}
%~~~~~~~~
%\frac{\partial^2 f}{\partial x\partial t} = \frac{\partial^2 f}{\partial t\partial x}
%\]
%in die Form
%\[
%\frac{\rmd f}{\rmd t}=0
%\]
%mit $u=\frac{\partial f}{\partial x}$ und $v=\frac{\partial f}{\partial t}$
%überführen und durch
%\[
%f(x(t),t)=c
%\]
%lösen lassen, wobei die Integrationskonstante $c$ durch die Anfangswertbedingung
%festgelegt wird.
%
%Wenn in einer Differentialgleichung der Form
%\[
%u(x,t)\frac{\rmd x}{\rmd t}+v(x,t)=0
%\]
%die Beziehung
%\[
%\frac{\partial u}{\partial t}=\frac{\partial v}{\partial x}
%\]
%nicht gilt, so kann dennoch eine Funktion $m(x,t)\not=0$ existieren,
%so dass
%\[
%\frac{\partial (mu)}{\partial t}=\frac{\partial (mv)}{\partial x}
%\]
%gilt. Da
%\[
%m(x,t)u(x,t)\frac{\rmd x}{\rmd t}+m(x,t)v(x,t)=0
%\]
%damit sowohl exakt als auch der ursprünglichen Gleichung äquivalent ist,
%können wir letztere mit Hilfe des sog. integrierenden Faktors $m$ lösen.\\


%\pagebreak
{\bf Die Schwingungsgleichung}

Wegen $\frac{\rmd^2}{\rmd t^2}\sin t=-\sin t$ und
$\frac{\rmd^2}{\rmd t^2}\cos t=-\cos t$ hat die lineare
Differentialgleichung 2. Ordnung
\[
\frac{\rmd^2 x}{\rmd t^2} = -\omega^2 x
\]
Lösungen der Form
\[
x(t) = C_1 \sin(\omega t) + C_2 \cos(\omega t)
\]
wobei $C_1$, $C_2$ durch die Anfangsbedingungen
$x(0)=x_0$, $\dot{x}(0)=v_0$ festgelegt werden.

Wenn wir der Schwingungsgleichung noch einen Reibungsterm hinzufügen,
erhalten wir die lineare
Differentialgleichung 2. Ordnung ($\gamma>0$)
\[
\frac{\rmd^2 x}{\rmd t^2} = -2\gamma \frac{\rmd x}{\rmd t}-\omega^2 x
\]
Wenn wir für die Lösung den Ansatz $x(t)=\rme^{\lambda t}$ machen,
so erhalten wir für $\lambda$ die Gleichung
\[
\lambda^2+2\gamma\lambda+\omega^2=0
\]
mit Lösungen
\[
\lambda_\pm = \left\{\begin{array}{ll}
                     -\gamma\pm\sqrt{\gamma^2-\omega^2},&~~\gamma\ge\omega\\
                     -\gamma\pm i\sqrt{\omega^2-\gamma^2}&~~\gamma<\omega
                     \end{array}\right.
\]
Wir unterscheiden drei Fälle:
\begin{enumerate}
\item $\gamma>\omega$: Kriechfall, $x(t)=C_1\rme^{-(\gamma+\sqrt{\gamma^2-\omega^2})t}+C_2\rme^{-(\gamma-\sqrt{\gamma^2-\omega^2})t}$
\item $\gamma=\omega$: aperiodischer Grenzfall, $x(t)=C_1\rme^{-\gamma t}+C_2 t\rme^{-\gamma t}$
\item $\gamma<\omega$: Schwingfall, $x(t)=\rme^{-\gamma t}\left(C_1\cos(\sqrt{\omega^2-\gamma^2}t)+C_2\sin(\sqrt{\omega^2-\gamma^2}t)\right)$
\end{enumerate}

Wir betrachten nun erzwungene Schwingungen:
\[
\frac{\rmd^2 x}{\rmd t^2} + 2\gamma \frac{\rmd x}{\rmd t}+\omega^2 x= A\rme^{i\omega_f t}
\]
wobei wir unsere Gleichung nun komplexifiziert haben: Real- und Imaginärteil
stellen jeweils eine Differentialgleichung dar.

Der Ansatz $x_p(t)=B\rme^{i\omega_f t}$ liefert
\[
(-\omega_f^2+2i\gamma\omega_f+\omega^2)B=A
\]
und für $\gamma>0$ erhalten wir
\[
B=\frac{A}{(\omega^2-\omega_f^2)+2i\gamma\omega_f}
\]
Die allgemeine Lösung lautet somit $x(t) = \frac{A}{(\omega^2-\omega_f^2)+2i\gamma\omega_f}\rme^{i\omega_f t} + x_h(t)$, wobei $x_h(t)$ für großes $t$ mit $\rme^{-\gamma t}$ abstirbt.
%
Wir folgern, dass für kleines $\gamma$ und $\omega_f=\omega$
die Amplitude der erzwungenen Schwingung groß wird -- es tritt Resonanz auf.

\end{document}

