\documentclass[11pt]{exam}
\usepackage[german]{babel}
\usepackage[utf8x]{inputenc}
\usepackage{graphicx}
\usepackage{latexsym,ifthen,amssymb,amsfonts,amsmath}

\begin{document}

\include{definitions}

\pagestyle{empty}

\def\loesungen{0}
\newcommand{\lansel}[2]{#1}

\ifthenelse{\equal{\loesungen}{1}}{\printanswers}{\relax}
\renewcommand{\solutiontitle}{\noindent\textbf{L\"osung:}\enspace}
\newcommand{\loesungname}{\ifthenelse{\equal{\loesungen}{1}}{L\"osungen zu }{\relax}}

\begin{center}
\textbf{\LARGE \loesungname \lansel{\"Ubungsblatt}{Examples Sheet} 7} \\ \vspace{1ex}
\textbf{\large \lansel{zum Mathematischen Brückenkurs \\ für Naturwissenschaftler:innen}{for the preparatory mathematics course for bio- and geoscientists}} \\ \vspace{1ex}
\textbf{\large \lansel{im Wintersemester}{Winter} 2023/24} \\ \vspace{0.5cm}
\textrm{\normalsize \hfill \lansel{Dozent}{Lecturer}: Apl.Prof. Dr. G. von Hippel\hfill${}$}
\end{center}
\normalsize\vspace{0.5cm}

\ifthenelse{\equal{\loesungen}{1}}{
\begin{center}
\textbf{Die direkte Weitergabe der Musterl\"osungen an Studierende ist nicht gestattet!}
\end{center}
}{
\vspace{3ex}
}

\begin{questions}
\pointname{ P.}

%%%%%%%%%%%%%%%%%%%%%%%%%%%%%%%%%%%%%%%%%%%%%%%%%%%%%%%%%%%%%%%%%%%%%%%%%%%%%%%

\question{{\it Wahrscheinlichkeitsverteilungen}}

Bestimmen Sie für folgende Funktionen $f:\Rset\to[0;\infty)$ jeweils
den Wert, den die Konstante $Z$ annehmen muss, damit
$p(x)=\frac{1}{Z}f(x)$ eine Wahrscheinlichkeitsdichte ist.
Bestimmen Sie ferner jeweils den Erwartungswert und die Varianz dieser
Wahrscheinlichkeitsdichte.\\
\parbox{0.5\textwidth}{\begin{enumerate}
\item $f(x)=\rme^{-|x|}$
\item $f(x)=\rme^{-x^2}$
\item $f(x)=\rme^{-(x-a)^2}$
\item $f(x)=x^2\rme^{-x^2}$
\item $f(x)=x^2\rme^{-(x-a)^2}$
\item $f(x)=\frac{1}{\left(x^2+a^2\right)^2}$
\end{enumerate}}\parbox{0.5\textwidth}{\begin{enumerate}\setcounter{enumi}{6}
\item $f(x)=\frac{1}{\left(2 a^2-2 a x+x^2\right)^2}$
\item $f(x)=|x|\rme^{-|x|}$
\item $f(x)=(x+|x|)\rme^{-x^2/(2\xi^2)}$
\item $f(x)=\frac{x+|x|}{2x}\rme^{-\lambda x}$
\item $f(x)=\rme^{-(\mu-\log |x|)^2/2}$
\item $f(x)=\frac{1}{1+\beta^2(x-\alpha)^2}$
\end{enumerate}}



\question{{\it Messreihen}}

Bestimmen Sie jeweils den Mittelwert und die Varianz der folgenden Messreihen.
Bestimmen Sie ferner den mittleren Fehler des Mittelwerts.
\begin{enumerate}
\item {3, 5, 3, 2, 2}
\item {4, 6, 4, 2, 4, 6, 1, 1, 2, 2}
\item {5, 3, 4, 2, 6, 1, 6, 2, 1, 3, 2, 5, 2, 2, 3, 2, 1, 5, 3, 5}
\item {2, 19, 15, 14, 19, 18, 1, 4, 11, 3}
\item {4, 4, 1, 1, 9, 9, 9, 9, 4, 4, 9, 9}
\item {0, 0, 0, 1, 0, 0, 0, 1, 1, 1, 0, 0, 0, 0, 1, 0, 1, 0, 0, 0}
\item {0.83, 0.25, 0.86, 0.67, 0.49, 0.01, 0.51, 0.61, 0.65, 0.09, 0.59, 0.87}
\item {0.614, 0.543, 0.654, 0.667, 0.114, 0.812, 
        0.359, 0.935, 0.696, 0.268, 0.430, 0.007, 0.630, 
        0.992, 0.329, 0.571, 0.145, 0.775, 0.467, 0.041}
\item {0.709, 0.554, 0.833, 2.066, 2.428, 0.412, 2.132, 0.956, 0.689, 1.372}
\item {0, 0, 9, 0, 9, 0, 0, 0, 0, 6, 6, 8, 0, 8, 0, 7, 0, 5, 0, 0, 6, 6, 9,
5, 0, 7, 5, 6, 5, 0, 0, 0, 0, 9, 6, 0, 0, 9, 0, 0, 0, 0, 8, 9, 0, 8,
0, 0, 0, 0} 
\end{enumerate}


\pagebreak


\question{{\it Fehlerfortpflanzung}}

Bestimmen Sie jeweils den Fehler folgender abhängiger Größen, wenn für die
unabhängigen Größen $x$, $y$, $z$ jeweils die Ergebnisse verschiedener
Messreihen aus der vorangehenden Aufgabe eingesetzt werden.\\
\parbox{0.5\textwidth}{\begin{enumerate}
\item $f(x,y)=x+y$
\item $f(x,y)=x-y$
\item $f(x,y)=xy$
\item $f(x,y)=x/y$
\item $f(x,y)=\sqrt{x^2+y^2}$
\item $f(x,y)=x^y$
\end{enumerate}}\parbox{0.5\textwidth}{\begin{enumerate}\setcounter{enumi}{6}
\item $f(x,y)=\rme^{x-y}$
\item $f(x,y)=\arctan\left(\frac{y}{x}\right)$
\item $f(x,y,z)=xy+yz-xz$
\item $f(x,y,z)=(y-x)^2/z^2$
\item $f(x,y,z)=\sin(xy+z)$
\item $f(x,y,z)=\frac{x^2-y^2+2xyz}{z^2+xy}$
\end{enumerate}}



\end{questions}

\end{document}
