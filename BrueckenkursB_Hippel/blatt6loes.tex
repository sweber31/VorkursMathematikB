\documentclass[11pt,answers]{exam}
\usepackage[german]{babel}
\usepackage[utf8x]{inputenc}
\usepackage{graphicx}
\usepackage{latexsym,ifthen,amssymb,amsfonts,amsmath}

\begin{document}

\include{definitions}

\pagestyle{empty}

\def\loesungen{1}
\newcommand{\lansel}[2]{#1}

\ifthenelse{\equal{\loesungen}{1}}{\printanswers}{\relax}
\renewcommand{\solutiontitle}{\noindent\textbf{L\"osung:}\enspace}
\newcommand{\loesungname}{\ifthenelse{\equal{\loesungen}{1}}{L\"osungen zu }{\relax}}

\begin{center}
\textbf{\LARGE \loesungname \lansel{\"Ubungsblatt}{Examples Sheet} 7} \\ \vspace{1ex}
\textbf{\large \lansel{zum Mathematischen Brückenkurs \\ für Naturwissenschaftler:innen}{for the preparatory mathematics course for bio- and geoscientists}} \\ \vspace{1ex}
\textbf{\large \lansel{im Wintersemester}{Winter} 2023/24} \\ \vspace{0.5cm}
\textrm{\normalsize \hfill \lansel{Dozent}{Lecturer}: Apl.Prof. Dr. G. von Hippel\hfill${}$}
\end{center}
\normalsize\vspace{0.5cm}

\ifthenelse{\equal{\loesungen}{1}}{
\begin{center}
\textbf{Die direkte Weitergabe der Musterl\"osungen an Studierende ist nicht gestattet!}
\end{center}
}{
\vspace{3ex}
}

\begin{questions}
\pointname{ P.}

%%%%%%%%%%%%%%%%%%%%%%%%%%%%%%%%%%%%%%%%%%%%%%%%%%%%%%%%%%%%%%%%%%%%%%%%%%%%%%%

% \question{{\it Wahrscheinlichkeitsverteilungen}}

Bestimmen Sie für folgende Funktionen $f:\Rset\to[0;\infty)$ jeweils
den Wert, den die Konstante $Z$ annehmen muss, damit
$p(x)=\frac{1}{Z}f(x)$ eine Wahrscheinlichkeitsdichte ist.
Bestimmen Sie ferner jeweils den Erwartungswert und die Varianz dieser
Wahrscheinlichkeitsdichte.\\
\parbox{0.5\textwidth}{\begin{enumerate}
\item $f(x)=\rme^{-|x|}$
\item $f(x)=\rme^{-x^2}$
\item $f(x)=\rme^{-(x-a)^2}$
\item $f(x)=x^2\rme^{-x^2}$
\item $f(x)=x^2\rme^{-(x-a)^2}$
\item $f(x)=\frac{1}{\left(x^2+a^2\right)^2}$
\end{enumerate}}\parbox{0.5\textwidth}{\begin{enumerate}\setcounter{enumi}{6}
\item $f(x)=\frac{1}{\left(2 a^2-2 a x+x^2\right)^2}$
\item $f(x)=|x|\rme^{-|x|}$
\item $f(x)=(x+|x|)\rme^{-x^2/(2\xi^2)}$
\item $f(x)=\frac{x+|x|}{2x}\rme^{-\lambda x}$
\item $f(x)=\rme^{-(\mu-\log |x|)^2/2}$
\item $f(x)=\frac{1}{1+\beta^2(x-\alpha)^2}$
\end{enumerate}}



\question{{\it Messreihen}}

Bestimmen Sie jeweils den Mittelwert und die Varianz der folgenden Messreihen.
Bestimmen Sie ferner den mittleren Fehler des Mittelwerts.
\begin{enumerate}
\item {3, 5, 3, 2, 2}
\item {4, 6, 4, 2, 4, 6, 1, 1, 2, 2}
\item {5, 3, 4, 2, 6, 1, 6, 2, 1, 3, 2, 5, 2, 2, 3, 2, 1, 5, 3, 5}
\item {2, 19, 15, 14, 19, 18, 1, 4, 11, 3}
\item {4, 4, 1, 1, 9, 9, 9, 9, 4, 4, 9, 9}
\item {0, 0, 0, 1, 0, 0, 0, 1, 1, 1, 0, 0, 0, 0, 1, 0, 1, 0, 0, 0}
\item {0.83, 0.25, 0.86, 0.67, 0.49, 0.01, 0.51, 0.61, 0.65, 0.09, 0.59, 0.87}
\item {0.614, 0.543, 0.654, 0.667, 0.114, 0.812, 
        0.359, 0.935, 0.696, 0.268, 0.430, 0.007, 0.630, 
        0.992, 0.329, 0.571, 0.145, 0.775, 0.467, 0.041}
\item {0.709, 0.554, 0.833, 2.066, 2.428, 0.412, 2.132, 0.956, 0.689, 1.372}
\item {0, 0, 9, 0, 9, 0, 0, 0, 0, 6, 6, 8, 0, 8, 0, 7, 0, 5, 0, 0, 6, 6, 9,
5, 0, 7, 5, 6, 5, 0, 0, 0, 0, 9, 6, 0, 0, 9, 0, 0, 0, 0, 8, 9, 0, 8,
0, 0, 0, 0} 
\end{enumerate}


\pagebreak


\question{{\it Fehlerfortpflanzung}}

Bestimmen Sie jeweils den Fehler folgender abhängiger Größen, wenn für die
unabhängigen Größen $x$, $y$, $z$ jeweils die Ergebnisse verschiedener
Messreihen aus der vorangehenden Aufgabe eingesetzt werden.\\
\parbox{0.5\textwidth}{\begin{enumerate}
\item $f(x,y)=x+y$
\item $f(x,y)=x-y$
\item $f(x,y)=xy$
\item $f(x,y)=x/y$
\item $f(x,y)=\sqrt{x^2+y^2}$
\item $f(x,y)=x^y$
\end{enumerate}}\parbox{0.5\textwidth}{\begin{enumerate}\setcounter{enumi}{6}
\item $f(x,y)=\rme^{x-y}$
\item $f(x,y)=\arctan\left(\frac{y}{x}\right)$
\item $f(x,y,z)=xy+yz-xz$
\item $f(x,y,z)=(y-x)^2/z^2$
\item $f(x,y,z)=\sin(xy+z)$
\item $f(x,y,z)=\frac{x^2-y^2+2xyz}{z^2+xy}$
\end{enumerate}}



\question{{\it Wahrscheinlichkeitsverteilungen}}

Bestimmen Sie für folgende Funktionen $f:\Rset\to[0;\infty)$ jeweils
den Wert, den die Konstante $Z$ annehmen muss, damit
$p(x)=\frac{1}{Z}f(x)$ eine Wahrscheinlichkeitsdichte ist.
Bestimmen Sie ferner jeweils den Erwartungswert und die Varianz dieser
Wahrscheinlichkeitsdichte.\\
\parbox{0.5\textwidth}{\begin{enumerate}
\item $f(x)=\rme^{-|x|}$
\item $f(x)=\rme^{-x^2}$
\item $f(x)=\rme^{-(x-a)^2}$
\item $f(x)=x^2\rme^{-x^2}$
\item $f(x)=x^2\rme^{-(x-a)^2}$
\item $f(x)=\frac{1}{\left(x^2+a^2\right)^2}$
\end{enumerate}}\parbox{0.5\textwidth}{\begin{enumerate}\setcounter{enumi}{6}
\item $f(x)=\frac{1}{\left(2 a^2-2 a x+x^2\right)^2}$
\item $f(x)=|x|\rme^{-|x|}$
\item $f(x)=(x+|x|)\rme^{-x^2/(2\xi^2)}$
\item $f(x)=\frac{x+|x|}{2x}\rme^{-\lambda x}$
\item $f(x)=\rme^{-(\mu-\log |x|)^2/2}$
\item $f(x)=\frac{1}{1+\beta^2(x-\alpha)^2}$
\end{enumerate}}
\begin{solution}Eine Funktion $p:\mathbb{R}\to[0,\infty)$ hei"st Wahrscheinilchkeitsdichte, falls sie integrierbar (im Riemann'schen Sinne) ist und 
\begin{align*}
\int_{-\infty}^{\infty}p(x)\,\mathrm dx=1.
\end{align*}
Sei $X:\Omega\to\mathbb{R}$ eine 
stetige Zufallsvariable mit Dichte $p$, dann gilt f"ur den 
Erwartungswert $\mathbb{E}[X]$ sowie f"ur die Varianz $\mathbb{V}[X]$
\begin{align*}
\mathbb{E}[X]&=\int_{-\infty}^{\infty}x\cdot p(x)\,\mathrm dx,
\\
\mathbb{V}[X]&=\mathbb{E}[|X-\mathbb{E}[X]|^2]=\mathbb{E}[X^2]-\mathbb{E}[X]^2.
\end{align*}
\begin{enumerate}
% 1
\item $f(x)=\rme^{-|x|}$
\begin{align*}
Z&=2
\\
\mathbb{E}[X]&=0
\\
\mathbb{V}[X]&=2
\end{align*}
% 2
\item $f(x)=\rme^{-x^2}$
\begin{align*}
Z&=\sqrt{\pi}
\\
\mathbb{E}[X]&=0
\\
\mathbb{V}[X]&=\frac{1}{2}
\end{align*}
% 3
\item $f(x)=\rme^{-(x-a)^2}$
\begin{align*}
Z&=\sqrt{\pi}
\\
\mathbb{E}[X]&=a
\\
\mathbb{V}[X]&=\frac{1}{2}
\end{align*}
% 4
\item $f(x)=x^2\rme^{-x^2}$
\begin{align*}
Z&=\frac{\sqrt{\pi}}{2}
\\
\mathbb{E}[X]&=0
\\
\mathbb{V}[X]&=\frac{3}{2}
\end{align*}
% 5
\item $f(x)=x^2\rme^{-(x-a)^2}$
\begin{align*}
Z&=\frac{1}{2}\left(1+2a^2\right)\sqrt{\pi}
\\
\mathbb{E}[X]&=a+\frac{2a}{1+2a^2}
\\
\mathbb{V}[X]&=\frac{5}{2}+a^2-\frac{1}{1+2a^2}-\left(a+\frac{2a}{1+2a^2}\right)^2
\end{align*}
% 6
\item $f(x)=\frac{1}{\left(x^2+a^2\right)^2}$
\begin{align*}
Z&=\frac{\pi}{2a^4}|a|
\\
\mathbb{E}[X]&=0
\\
\mathbb{V}[X]&=a|a|
\end{align*}
% 7
\item $f(x)=\frac{1}{\left(2 a^2-2 a x+x^2\right)^2}$
\begin{align*}
Z&=-\frac{\pi}{2a^3}
\\
\mathbb{E}[X]&=a
\\
\mathbb{V}[X]&=a^2
\end{align*}
% 8
\item $f(x)=|x|\rme^{-|x|}$
\begin{align*}
Z&=2
\\
\mathbb{E}[X]&=0
\\
\mathbb{V}[X]&=6
\end{align*}
% 9
\item $f(x)=(x+|x|)\rme^{-x^2/(2\xi^2)}$
\begin{align*}
Z&=2\xi^2
\\
\mathbb{E}[X]&=|\xi|\sqrt{\frac{\pi}{2}}
\\
\mathbb{V}[X]&=2\xi^2-\frac{\pi}{2}|\xi|^2
\end{align*}
% 10
\item $f(x)=\frac{x+|x|}{2x}\rme^{-\lambda x}$
\begin{align*}
Z&=\frac{1}{\lambda}
\\
\mathbb{E}[X]&=\frac{1}{\lambda}
\\
\mathbb{V}[X]&=\frac{1}{\lambda^2}
\end{align*}
% 11
\item $f(x)=\rme^{-(\mu-\log |x|)^2/2}$
\begin{align*}
Z&=2e^{1/2+\mu}\sqrt{2\pi}
\\
\mathbb{E}[X]&=0
\\
\mathbb{V}[X]&=e^{4+2\mu}
\end{align*}
\item $f(x)=\frac{1}{1+\beta^2(x-\alpha)^2}$
\begin{align*}
Z&=\frac{\pi}{|\beta|}
\\
\mathbb{E}[X] \textrm{ und } \mathbb{V}[X] & \textrm{ existieren nicht.}
%\\
%\mathbb{V}[X]&=\infty
\end{align*}
\end{enumerate}
\end{solution}


\question{{\it Messreihen}}

Bestimmen Sie jeweils den Mittelwert und die Varianz der folgenden Messreihen.
Bestimmen Sie ferner den mittleren Fehler des Mittelwerts.
\begin{enumerate}
\item {3, 5, 3, 2, 2}
\item {4, 6, 4, 2, 4, 6, 1, 1, 2, 2}
\item {5, 3, 4, 2, 6, 1, 6, 2, 1, 3, 2, 5, 2, 2, 3, 2, 1, 5, 3, 5}
\item {2, 19, 15, 14, 19, 18, 1, 4, 11, 3}
\item {4, 4, 1, 1, 9, 9, 9, 9, 4, 4, 9, 9}
\item {0, 0, 0, 1, 0, 0, 0, 1, 1, 1, 0, 0, 0, 0, 1, 0, 1, 0, 0, 0}
\item {0.83, 0.25, 0.86, 0.67, 0.49, 0.01, 0.51, 0.61, 0.65, 0.09, 0.59, 0.87}
\item {0.614, 0.543, 0.654, 0.667, 0.114, 0.812, 
        0.359, 0.935, 0.696, 0.268, 0.430, 0.007, 0.630, 
        0.992, 0.329, 0.571, 0.145, 0.775, 0.467, 0.041}
\item {0.709, 0.554, 0.833, 2.066, 2.428, 0.412, 2.132, 0.956, 0.689, 1.372}
\item {0, 0, 9, 0, 9, 0, 0, 0, 0, 6, 6, 8, 0, 8, 0, 7, 0, 5, 0, 0, 6, 6, 9,
5, 0, 7, 5, 6, 5, 0, 0, 0, 0, 9, 6, 0, 0, 9, 0, 0, 0, 0, 8, 9, 0, 8,
0, 0, 0, 0} 
\end{enumerate}
\begin{solution}Wir berechnen im Folgenden den Mittelwert $\bar x$, die 
Varianz $\sigma^2$ und den mittleren Fehler des Mittelwerts $\bar \sigma$
\begin{align*}
\bar{x}&=\frac{1}{N}\sum_{i=1}^Nx_i
\\
\sigma^2&=\frac{1}{N-1}\sum_{i=1}^N(x_i-\bar x)^2
\\
\bar \sigma&=\frac{\sigma}{\sqrt{N}}
\end{align*}
\begin{enumerate}

%1
\item {3, 5, 3, 2, 2}
\begin{align*}
\bar x&=3
\\
\sigma^2&=\frac{3}{2}=1,5
\\
\bar \sigma&=\sqrt{\frac{3}{10}}\approx 0,548
\end{align*}

%2
\item {4, 6, 4, 2, 4, 6, 1, 1, 2, 2}
\begin{align*}
\bar x&=\frac{16}{5}=3,2
\\
\sigma^2&=\frac{158}{45}\approx 3,511
\\
\bar \sigma&=\frac{\sqrt{79}}{15}\approx 0,593
\end{align*}

%3
\item {5, 3, 4, 2, 6, 1, 6, 2, 1, 3, 2, 5, 2, 2, 3, 2, 1, 5, 3, 5}
\begin{align*}
\bar x&=\frac{63}{20}=3,15
\\
\sigma^2&=\frac{1051}{380}\approx 2,766
\\
\bar \sigma&=\frac{\sqrt{1051/19}}{20}\approx 0,372
\end{align*}
\item {2, 19, 15, 14, 19, 18, 1, 4, 11, 3}
\begin{align*}
\bar x&=\frac{53}{3}=10,6
\\
\sigma^2&=\frac{824}{15}\approx 54,933
\\
\bar \sigma&=\frac{2\sqrt{103/3}}{5}\approx 2,344
\end{align*}
\item {4, 4, 1, 1, 9, 9, 9, 9, 4, 4, 9, 9}
\begin{align*}
\bar x&=6
\\
\sigma^2&=\frac{120}{11}\approx 10,909
\\
\bar \sigma&=\sqrt{\frac{10}{11}}\approx 0,953
\end{align*}
\item {0, 0, 0, 1, 0, 0, 0, 1, 1, 1, 0, 0, 0, 0, 1, 0, 1, 0, 0, 0}
\begin{align*}
\bar x&=\frac{3}{10}=0,3
\\
\sigma^2&=\frac{21}{95}\approx 0,221
\\
\bar \sigma&=\frac{\sqrt{21/19}}{10}
\end{align*}
\item {0.83, 0.25, 0.86, 0.67, 0.49, 0.01, 0.51, 0.61, 0.65, 0.09, 0.59, 0.87}
\begin{align*}
\bar x&=0.535833
\\
\sigma^2&=0.0820447
\\
\bar \sigma&=0.0826865
\end{align*}
\item {0.614, 0.543, 0.654, 0.667, 0.114, 0.812, 
        0.359, 0.935, 0.696, 0.268, 0.430, 0.007, 0.630, 
        0.992, 0.329, 0.571, 0.145, 0.775, 0.467, 0.041}
\begin{align*}
\bar x&=0,50245
\\
\sigma^2&=0,0825595
\\
\bar \sigma&=0,0642493
\end{align*}
\item {0.709, 0.554, 0.833, 2.066, 2.428, 0.412, 2.132, 0.956, 0.689, 1.372}
\begin{align*}
\bar x&=1,2151
\\
\sigma^2&=0,543173
\\
\bar \sigma&=0,233061
\end{align*}
\item {0, 0, 9, 0, 9, 0, 0, 0, 0, 6, 6, 8, 0, 8, 0, 7, 0, 5, 0, 0, 6, 6, 9,
5, 0, 7, 5, 6, 5, 0, 0, 0, 0, 9, 6, 0, 0, 9, 0, 0, 0, 0, 8, 9, 0, 8,
0, 0, 0, 0}
\begin{align*}
\bar x&=\frac{78}{25}=3,12
\\
\sigma^2&=\frac{16732}{1225}\approx 13,659
\\
\bar \sigma&=\frac{\sqrt{8366}}{175}\approx 0,523
\end{align*}
\end{enumerate}
\end{solution}

% \pagebreak


\question{{\it Fehlerfortpflanzung}}

Bestimmen Sie jeweils den Fehler folgender abhängiger Größen, wenn für die
unabhängigen Größen $x$, $y$, $z$ jeweils die Ergebnisse verschiedener
Messreihen aus der vorangehenden Aufgabe eingesetzt werden.\\
\parbox{0.5\textwidth}{\begin{enumerate}
\item $f(x,y)=x+y$
\item $f(x,y)=x-y$
\item $f(x,y)=xy$
\item $f(x,y)=x/y$
\item $f(x,y)=\sqrt{x^2+y^2}$
\item $f(x,y)=x^y$
\end{enumerate}}\parbox{0.5\textwidth}{\begin{enumerate}\setcounter{enumi}{6}
\item $f(x,y)=\rme^{x-y}$
\item $f(x,y)=\arctan\left(\frac{y}{x}\right)$
\item $f(x,y,z)=xy+yz-xz$
\item $f(x,y,z)=(y-x)^2/z^2$
\item $f(x,y,z)=\sin(xy+z)$
\item $f(x,y,z)=\frac{x^2-y^2+2xyz}{z^2+xy}$
\end{enumerate}}
\begin{solution}
Zur Fehlerfortpflanzung f"ur $\bar x$ und $\bar \sigma=\sigma/\sqrt{N}$ berechnen wir
\begin{align*}
\bar \sigma_{f(\bar x,\bar y)}=
\sqrt{
\left(\frac{\partial f}{\partial x}\right)^2\Bigg|_{\bar x,\bar y}\bar\sigma_{\bar x}^2
+
\left(\frac{\partial f}{\partial y}\right)^2\Bigg|_{\bar x,\bar y}\bar\sigma_{\bar y}^2}
\end{align*}
Wir zeigen im Folgenden einige Beispiele dazu, wobei $M_i$ die $i$-te Messreihe der vorherigen Aufgabe bezeichnet
\begin{enumerate}
% 1
\item $f(x,y)=x+y$, $X=M_1$, $Y=M_2$
\begin{align*}
&
\partial_xf(x,y)=1
,\quad
\partial_yf(x,y)=1
\\
&
\bar \sigma_{f}=\sqrt{\bar\sigma_x^2+\bar\sigma_y^2}\approx 0,807
\end{align*}
% 2
\item $f(x,y)=x-y$, $X=M_2$, $Y=M_3$
\begin{align*}
&
\partial_xf(x,y)=1
,\quad
\partial_yf(x,y)=-1
\\
&
\bar \sigma_{f}=\sqrt{\bar\sigma_x^2+\bar\sigma_y^2}\approx 0,7
\end{align*}
% 3
\item $f(x,y)=xy$, $X=M_3$, $Y=M_4$
\begin{align*}
&
\partial_xf(x,y)=y
,\quad
\partial_yf(x,y)=x
\\
&
\bar \sigma_{f}=\sqrt{\bar{y}^2\bar\sigma_x^2+\bar{x}^2\bar\sigma_y^2}\approx 9,883
\end{align*}
% 4
\item $f(x,y)=x/y$, $X=M_4$, $Y=M_5$
\begin{align*}
	&
	\partial_xf(x,y)=\frac{1}{y}
	,\quad
	\partial_yf(x,y)=-\frac{x}{y^2}
	\\
	&
	\bar \sigma_{f}=\sqrt{\frac{\bar\sigma_x^2}{\bar y^2}+\frac{\bar x^2\bar\sigma_y^2}{\bar y^4}}\approx 0.481
\end{align*}
% 5
\item $f(x,y)=\sqrt{x^2+y^2}$, $X=M_5$, $Y=M_6$
\begin{align*}
	&
	\partial_xf(x,y)=\frac{x}{\sqrt{x^2+y^2}}
	,\quad
	\partial_yf(x,y)=\frac{y}{\sqrt{x^2+y^2}}
	\\
	&
	\bar \sigma_{f}=\sqrt{\frac{\bar x^2\bar\sigma_x^2}{\bar x^2+\bar y^2}+\frac{\bar y^2\bar\sigma_y^2}{\bar x^2+\bar y^2}}\approx 0.952
\end{align*}
% 6
\item $f(x,y)=x^y$, $X=M_7$, $Y=M_8$
\begin{align*}
	&
	\partial_xf(x,y)=yx^{y-1}
	,\quad
	\partial_yf(x,y)=\ln(y)x^y
	\\
	&
	\bar \sigma_{f}=\sqrt{\bar y^2\bar x^{2\bar y-2}\bar\sigma_x^2+\ln^2(\bar y)\bar x^{2\bar y}\bar\sigma_y^2}\approx 0.0652
\end{align*}
% 7
\item $f(x,y)=\rme^{x-y}$, $X=M_9$, $Y=M_{10}$
\begin{align*}
	&
	\partial_xf(x,y)=\rme^{x-y}
	,\quad
	\partial_yf(x,y)=-\rme^{x-y}
	\\
	&
	\bar \sigma_{f}=\sqrt{\rme^{2\bar x-2\bar y}\bar\sigma_x^2+\rme^{2\bar x-2\bar y}\bar\sigma_y^2}\approx 0.0852
\end{align*}
% 8
\item $f(x,y)=\arctan\left(\frac{y}{x}\right)$, $X=M_5$, $Y=M_7$
\begin{align*}
&
\partial_xf(x,y)=-\frac{y}{x^2+y^2}
,\quad
\partial_yf(x,y)=\frac{x}{x^2+y^2}
\\
&
\bar \sigma_{f}=\sqrt{\frac{\bar y^2}{(\bar x^2+\bar y^2)^2}\bar\sigma_x^2
+\frac{\bar x^2}{(\bar x^2+\bar y^2)^2}\bar\sigma_y^2}\approx 0,012
\end{align*}
% 9
\item $f(x,y,z)=xy+yz-xz$, $X=M_1$, $Y=M_2$, $Z=M_3$
\begin{align*}
	&
	\partial_xf(x,y,z)=y-z
	,\quad
	\partial_yf(x,y,z)=x+z
	,\quad
	\partial_zf(x,y,z)=y-x
	\\
	&
	\bar \sigma_{f}=
	\sqrt{
		(\bar y - \bar z)^2\bar\sigma_x^2
		+
		(\bar x + \bar z)^2\bar\sigma_y^2
		+
		(\bar y - \bar x)^2\bar\sigma_z^2
	}
	\approx 3.65
\end{align*}
% 10
\item $f(x,y,z)=(y-x)^2/z^2$, $X=M_4$, $Y=M_5$, $Z=M_6$
\begin{align*}
	&
	\partial_xf(x,y,z)=\frac{-2(y-x)}{z^2}
	,\quad
	\partial_yf(x,y,z)=\frac{2(y-x)}{z^2}
	,\quad
	\partial_zf(x,y,z)=-\frac{2(y-x)^2}{z^3}
	\\
	&
	\bar \sigma_{f}=
	\sqrt{
		\frac{4(\bar y-\bar x)^2}{\bar z^4}\bar\sigma_x^2
		+
		\frac{4(\bar y-\bar x)^2}{\bar z^4}\bar\sigma_y^2
		+
		\frac{4(\bar y-\bar x)^4}{\bar z^6}\bar\sigma_z^2
	}
	\approx 307
\end{align*}
% 11
\item $f(x,y,z)=\sin(xy+z)$, $X=M_8$, $Y=M_9$, $Z=M_{10}$
\begin{align*}
&
\partial_xf(x,y,z)=y\cos(xy+z)
,\quad
\partial_yf(x,y,z)=x\cos(xy+z)
,\quad
\partial_zf(x,y,z)=\cos(xy+z)
\\
&
\bar \sigma_{f}=
\sqrt{
\bar y^2\cos^2(\bar x\bar y+\bar z)\bar\sigma_x^2
+
\bar x^2\cos^2(\bar x\bar y+\bar z)\bar\sigma_y^2
+
\cos^2(\bar x\bar y+\bar z)\bar\sigma_z^2
}
\approx 0,45
\end{align*}
% 12
\item $f(x,y,z)=\frac{x^2-y^2+2xyz}{z^2+xy}$, $X=M_3$, $Y=M_5$, $Z=M_7$
\begin{align*}
	&
	\partial_xf(x,y,z)=\frac{x^2y+2xz^2+y^3+2yz^3}{(z^2+xy)^2}
	,\quad
	\partial_yf(x,y,z)=\frac{-x^3-xy^2+2xz^3-2yz^2}{(z^2+xy)^2}
	,\\
	&\partial_zf(x,y,z)=\frac{2x^2y^2-2x^2z-2xyz^2+2y^2z}{(z^2+xy)^2}
	\\
	&
	\bar \sigma_{f}=
	\left(
		\frac{(\bar x^2\bar y+2\bar x\bar z^2+\bar y^3+2\bar y\bar z^3)^2}{(\bar z^2+\bar x\bar y)^4}\bar\sigma_x^2
		+
		\frac{(-\bar x^3-\bar x\bar y^2+2\bar x\bar z^3-2\bar y\bar z^2)^2}{(\bar z^2+\bar x\bar y)^4}\bar\sigma_y^2\right.\\&\left.
		+
		\frac{(2\bar x^2\bar y^2-2\bar x^2\bar z-2\bar x\bar y\bar z^2+2\bar y^2\bar z)^2}{(\bar z^2+\bar x\bar y)^4}\bar\sigma_z^2
	\right)^\frac{1}{2}
	\approx 0.502
\end{align*}

\end{enumerate}
\end{solution}


\end{questions}

\end{document}
