\documentclass[a4paper,10pt]{article}
\usepackage[utf8x]{inputenc}
\usepackage[german]{babel}
\usepackage{a4wide}
\usepackage{amsmath,amssymb,amsfonts}

\input{definitions}

\title{Zusammenfassung zum Thema \\ Differential- und Integralrechnung}
\author{Mathematischer Brückenkurs (B)\\für Naturwissenschaftler:innen}
\date{WS 2023/2024}

%%%%%%%%%%%%%%%%%%%%%%%%%%%%%%%%%%%%%%%%%%%%%%%%%%%%%%%%%%%%%%%%%%%%%%%%%%

\begin{document}
\parindent0pt
\maketitle

{\bf Funktionen einer reellen Variablen}

Wir betrachten im folgenden Funktionen $f:D\to\Rset$ mit $D\subseteq\Rset$.
Eine solche Funktion heißt
\begin{itemize}\setlength\itemsep{0pt}\parskip0pt\setlength\parsep{0pt}
\item {nach oben beschränkt}, wenn $\exists S\in\Rset~\forall x\in D~f(x)\le S$,
\item {nach unten beschränkt}, wenn $\exists S\in\Rset~\forall x\in D~f(x)\ge S$,
\item {beschränkt}, wenn $\exists S\in\Rset~\forall x\in D~|f(x)|\le S$,
\item {monoton wachsend}, wenn $x>y\Rightarrow f(x)\ge f(y)$,
\item {monoton fallend}, wenn $x>y\Rightarrow f(x)\le f(y)$,
\item {streng monoton wachsend}, wenn $x>y\Rightarrow f(x)>f(y)$, 
\item {streng monoton fallend}, wenn $x>y\Rightarrow f(x)<f(y)$.
\end{itemize}
Falls ferner $x\in D\Rightarrow -x\in D$ gilt, heißt eine Funktion
\begin{itemize}\setlength\itemsep{0pt}\parskip0pt\setlength\parsep{0pt}
\item {gerade}, wenn $f(-x)=f(x)$,
\item {ungerade}, wenn $f(-x)=-f(x)$.
\end{itemize}
Falls $x\in D\Rightarrow (x+p)\in D$ und $f(x+p)=f(x)$ gilt, heißt die
Funktion periodisch mit Periode $p$.

Einige besonders wichtige Funktionen sind die sog.~elementaren Funktionen:
\begin{center}
\begin{tabular}{cccccc}
$f(x)\equiv y$ & $D$ & $W$ & $f^{-1}(y)\equiv x$ & $f'(x)$ & $\int f(x)~\rmd x$ \\\hline
$x^\alpha$,~$\alpha>0$ & $[0;\infty)$ & $[0;\infty)$ & $y^{1/\alpha}$ & $\alpha x^{\alpha-1}$ & $\frac{1}{\alpha+1}x^{\alpha+1} $\\
$x^\alpha$,~$\alpha<0$ & $(0;\infty)$ & $(0;\infty)$ & $y^{1/\alpha}$ & $\alpha x^{\alpha-1}$ & $\left\{{\frac{1}{\alpha+1}x^{\alpha+1},~\alpha\not=-1 \atop \log|x|,~\alpha=-1}\right. $\\
$\rme^x$ & $\Rset$ & $(0;\infty)$ & $\log y$ & $\rme^x$ & $\rme^x$\\
$\sin x$ & $[-\frac{\pi}{2};\frac{\pi}{2}]$ & $[-1;1]$ & $\arcsin y$ & $\cos x$ & $-\cos x$\\
$\cos x$ & $[0;\pi]$ & $[-1;1]$ & $\arccos y$ & $-\sin x$ & $\sin x$\\
$\tan x$ & $(-\frac{\pi}{2};\frac{\pi}{2})$ & $\Rset$ & $\arctan y$ & $1/\cos^2 x$ & $-\log\cos x$\\
$\cot x$ & $(0;\pi)$ & $\Rset$ & $\mathrm{arccot}~y$ & $-1/\sin^2 x$ & $\log\sin x$\\
$\sinh x$ & $\Rset$ & $\Rset$ & $\mathrm{arsinh}~y$ & $\cosh x$ & $\cosh x$\\
$\cosh x$ & $[0;\infty)$ & $[1;\infty)$ & $\mathrm{arcosh}~y$ & $\sinh x$ & $\sinh x$
\end{tabular}
\end{center}
Hierbei ist jeweils der maximale Definitionsbereich angegeben, auf dem
eine Umkehrfunktion existiert. Natürlich lassen sich die trigonometrischen
Funktionen auf ganz $\Rset$ ausdehnen und sind dann periodisch,
$\sin(x+2\pi)=\sin x$ etc. Ebenso lassen sich ganzzahlige Potenzen und
der cosinus hyperbolicus auf negative Argumente ausdehnen.\\

{\bf Stetigkeit und Grenzwerte von Funktionen}

Eine Funktion $f:D\to\Rset$ heißt stetig an der Stelle $a\in D$,
wenn für {\bf alle} Folgen $a_n$ mit $a_n\in D$ und $\lim\limits_{n\to\infty} a_n=a$
gilt, dass $\lim\limits_{n\to\infty} f(a_n)=f(a)$.
Eine Funktion, die an allen $a\in D$ stetig ist, nennen wir auch stetig auf $D$.
Für stetige Funktionen $f,g:D\to\Rset$, $h:\textrm{Im}~f\to\Rset$, sind auch
$f+g:x\mapsto f(x)+g(x)$, $f\cdot g:x\mapsto f(x)g(x)$, $h\circ f:x\mapsto h(f(x))$, sowie für $g(x)\not= 0$ auch $f/g:x\mapsto f(x)/g(x)$ stetig.

Die elementaren Funktionen sind auf ihrem jeweiligen Definitionsbereich
stetig.
Ein Beispiel für eine nicht-stetige Funktion ist die Vorzeichenfunktion
\[
\textrm{sgn}~x=\left\{\begin{array}{l}+1,~x>0,\\0,~x=0,\\-1,~x<0,\end{array}\right.
\]
die bei Null unstetig ist.

Wenn für eine Funktion die Grenzwerte $\lim\limits_{n\to\infty} f(a_n)$
für {\bf alle} Folgen $a_n$ mit $a_n\in D$ und $\lim\limits_{n\to\infty} a_n=a$
übereinstimmen, definieren wir den Grenzwert der Funktion als
\[
\lim\limits_{x\to a} f(x)\equiv \lim\limits_{n\to\infty} f(a_n).
\]
Für stetiges $f$ gilt $\lim\limits_{x\to a} f(x)=f(a)$.

Ein rechtsseitiger Grenzwert $\lim\limits_{x\to a^+} f(x)$
lässt sich definieren, falls
die Grenzwerte $\lim\limits_{n\to\infty} f(a_n)$ für alle Folgen $a_n$
mit $a_n\in D$, $a_n>a$ und $\lim\limits_{n\to\infty} a_n=a$ übereinstimmen.
Entsprechend kann man einen linksseitigen Grenzwert
$\lim\limits_{x\to a^-} f(x)$ definieren.\\

{\bf Differenzierbarkeit}

Eine Funktion $f:D\to\Rset$ heißt differenzierbar an der Stelle $a\in D$,
wenn der Grenzwert $\lim\limits_{x\to a}\frac{f(x)-f(a)}{x-a}\equiv f'(a)$
existiert.
Dieser heißt die Ableitung von $f$ an der Stelle $a$.

Oft schreibt man auch $f'(x)=\frac{\rmd f}{\rmd x}$, wobei der
Differentialquotient den Grenzwert des Differenzenquotienten symbolisiert.
Eine differenzierbare Funktion ist auch stetig.

Die Ableitung von $f$ definiert wieder eine Funktion $f':D'\to \Rset$,
$x\mapsto f'(x)$ mit $D'\subseteq D$.
Die Ableitung $f''$ von $f'$ heißt die zweite Ableitung von $f$.
      Für die dritte und höhere Ableitungen schreiben wir auch
      $f^{(n)} = \frac{\rmd^nf}{\rmd x^n} = \frac{\rmd}{\rmd x}f^{(n-1)}$.

Die elementaren Funktionen
sind auf ihrem jeweiligen Definitionsbereich (z.T. mit Ausnahme der Randpunkte)
auch differenzierbar.
Ein Beispiel für eine stetige, nicht-differenzierbare Funktion ist
die Betragsfunktion $|x|$, die bei $x=0$ nicht differenzierbar ist.

Wenn $f,g:D\to\Rset$ an der Stelle $x$ differenzierbar sind, so gelten
die Summenregel
\[
\frac{\rmd}{\rmd x}(f+g)=f'(x)+g'(x)
\]
sowie die Produktregel
\[
\frac{\rmd}{\rmd x}(fg)=f'(x)g(x)+f(x)g'(x)\,.
\]
Wenn $f:D\to\Rset$ an der Stelle $x$ differenzierbar ist, und
$g:\textrm{Im}~f\to\Rset$ an der Stelle $f(x)$ differenzierbar ist,
so gilt die Kettenregel
\[
\frac{\rmd}{\rmd x}(g\circ f)=g'(f(x))f'(x)\,.
\]
Wenn $f:D\to W$, $W\subseteq\Rset$ bijektiv und an der Stelle $x$
differenzierbar mit $f'(x)\not=0$ ist, so ist die Umkehrfunktion
$f^{-1}:W\to D$ an der Stelle $y=f(x)$ differenzierbar, und es gilt
die Umkehrfunktionsregel
\[
\frac{\rmd}{\rmd y}f^{-1}(y) = \frac{1}{f'(f^{-1}(y))}\,.
\]
Aus der Produkt- und der Kettenregel folgt, wenn $f,g:D\to\Rset$
an der Stelle $x$ mit $g(x)\not=0$ differenzierbar sind, die Quotientenregel
\[
\frac{\rmd}{\rmd x}\frac{f}{g}=\frac{f'(x)g(x)-f(x)g'(x)}{g(x)^2}
\]
\pagebreak

{\bf Taylor-Entwicklung und Regel von de l'H\^opital}

Für eine differenzierbare Funktion $f$ ist die beste lineare Näherung
nahe $x=x_0$ durch die Tangente gegeben $f(x)= f(x_0)+f'(x_0)(x-x_0) +R(x)$,
wobei der Rest $R(x)$ für $x\to x_0$ schneller als $(x-x_0)$ verschwinden
muss.
Für Quotienten $f(x)/g(x)$ mit $f(x)=g(x)=0$ gilt daher die
Regel von de l'H\^opital,
\[
\lim\limits_{y\to x}\frac{f(y)}{g(y)} = \lim\limits_{y\to x}\frac{f'(y)}{g'(y)}\,.
\]
Eine bessere als die lineare Näherung an $f$ kann (für hinreichend oft
differenzierbares $f$) durch Polynome höheren Grades erzielt
werden, wenn wir verlangen, dass die ersten $n$ Ableitungen an der Stelle $x_0$
mit denen von $f$ übereinstimmen sollen:
\[
f(x) =\sum_{k=0}^n \frac{1}{k!}f^{(k)}(x_0)(x-x_0)^k + \underbrace{R_n(x)}_{\to 0,~x\to x_0}
\]
wobei der Rest für $x\to x_0$ schneller als $(x-x_0)^n$ verschwinden muss.
Wenn $\lim\limits_{n\to\infty}R_n(x)=0$, spricht man von einer reell
analytischen Funktion, die durch ihre Taylor-Reihe dargestellt wird:
\[
f(x)=\sum_{k=0}^\infty \frac{1}{k!}f^{(k)}(x_0)(x-x_0)^k
\]
Beispiele sind
\[
\rme^x = \sum\limits_{n=0}^\infty \frac{x^n}{n!}~~~~~~~\sin x = \sum\limits_{n=0}^\infty \frac{(-1)^n}{(2n+1)!}x^{2n+1}~~~~~~~\cos x = \sum\limits_{n=0}^\infty \frac{(-1)^n}{(2n)!}x^{2n}
\]

{\bf Extrema}

Die Funktion $f:D\to\Rset$ hat ein lokales
$\left\{{\textrm{Maximum} \atop \textrm{Minimum}}\right\}$ bei $x_0$, wenn
\[\exists \epsilon>0~\forall x\in D\cap(x_0-\epsilon;x_0+\epsilon)~
f(x)\left\{{\le \atop \ge}\right\}f(x_0)\] Gilt sogar
$\forall x\in D~f(x)\left\{{\le \atop \ge}\right\}f(x_0)$, heißt $x_0$
globales $\left\{{\textrm{Maximum} \atop \textrm{Minimum}}\right\}$.
Der Überbegriff für Maximum und Minimum ist Extremum.

Es gilt: Ein Extremum $x_0$ liegt entweder auf dem Rand von $D$, oder
$f$ ist an der Stelle $x_0$ nicht differenzierbar, oder $f'(x_0)=0$.

Wenn $f'(x_0)=0$, so ist $x_0$ für $f''(x_0)>0$ ein Minimum,
für $f''(x_0)<0$ ein Maximum, und für $f''(x_0)=0$ entweder ein Sattelpunkt,
$f^{(3)}(x_0)\not=0$, oder es müssen höhere Ableitungen betrachtet werden,
um das Verhalten von $f$ nahe $x_0$ zu entscheiden.\\

{\bf Asymptotisches Verhalten}

Wenn für alle Folgen $a_n$, die nach $+\infty$ bzw. nach $-\infty$ divergieren,
jeweils derselbe Grenzwert $\lim\limits_{n\to\infty} f(a_n)$ existiert,
definieren wir
\[
\lim\limits_{x\to\pm\infty} f(x)\equiv \lim\limits_{n\to\infty} f(a_n)
\]
als den Grenzwert von $f$ für $x\to\pm\infty$.

%Im Übrigen sind folgende Notationen zur Beschreibung des Verhaltens
%einer Funktion in der Umgebung einer Stelle $a$ üblich:
%\begin{align*}
%f(x)\sim g(x) & ~~~(x\to a)&&\Leftrightarrow~~~~\lim\limits_{x\to a}\frac{f(x)}{g(x)}=1\\
%f(x)=o(g(x)) & ~~~(x\to a)&&\Leftrightarrow~~~~\lim\limits_{x\to a}\frac{f(x)}{g(x)}=0\\
%f(x)=O(g(x)) & ~~~(x\to a)&&\Leftrightarrow~~~~\exists C>0~|f(x)|\le C|g(x)|
%\end{align*}
\pagebreak

{\bf Integration}

Eine Funktion $F:D\to\Rset$ heißt Stammfunktion von $f:D\to\Rset$, wenn
$F$ auf $D$ differenzierbar mit $F'(x)=f(x)$ ist.
Wenn $F$ Stammfunktion von $f$ ist, so auch $F+c$, $c\in\Rset$.
Man schreibt
\[
F(x) = \int f(x)~\rmd x
\]
für die Stammfunktion von $f$ (unbestimmtes Integral).

Wir definieren das bestimmte (Riemann-)Integral als Fläche unter dem
Graphen einer Funktion $f:[a;b]\to\Rset$ mit Hilfe von Stufensummen,
\[
\int_a^b f(x)~\rmd x = \lim\limits_{n\to\infty}\sum_{i=0}^{n-1}f\left(x_i\right)(x_{i+1}-x_i)
\]
wenn dieser Limes unabhängig von der gewählten Zerlegung
$x_i$ ($a=x_0<\ldots<x_n=b$) ist.


Für $f:[a;b]\to\Rset$, $F=\int f(x)~\rmd x$, gilt der Hauptsatz der
Differential- und Integralrechnung:
\[
\int_a^b f(x)~\rmd x = F(b)-F(a)\,.
\]
Intuitiv ist $f(x)=\frac{\rmd}{\rmd x}\int_a^x f(t)~\rmd t$ verständlich,
da $f(x)$ angibt, wie schnell die Fläche unter dem Graphen wächst
(welche für $x=a$ natürlich Null beträgt).
Algebraisch nutzt man $f(t)=F'(t)$ und nähert $F'(t)h=F(t+h)-F(h)$,
um die Stufensumme ($x_k=a+kh$, $h=\frac{b-a}{n}$)
in eine Teleskopsumme zu verwandeln, wobei die Näherung
im Limes $h\to 0$ exakt wird.

Für integrierbare Funktionen $f,g$ und Konstanten $c\in\Rset$ gilt
\begin{align*}
\int (f(x)+g(x))~\rmd x &= \int f(x)~\rmd x+\int g(x)~\rmd x \\
\int (c\cdot f(x))~\rmd x &= c \int f(x)~\rmd x
\end{align*}
Aus der Produktregel für die Ableitung folgt für differenzierbare
Funktionen $f,g$ die Regel zur partiellen Integration,
\[
\int f'(x)g(x)~\rmd x=f(x)g(x)-\int f(x)g'(x)~\rmd x \,.
\]
Aus der Kettenregel für die Ableitung folgt, dass für differenzierbare
Funktionen $f,g$ mit $f$ injektiv die Identität
\[
\int g(z)~\rmd z = \int g(f(x))f'(x)~\rmd x
\]
mit $z=f(x)$ und $\rmd z=f'(x)~\rmd x$ gilt.
Für bestimmte Integrale müssen auch die Grenzen transformiert werden:
\[
\int_a^b g(z)~\rmd z = \int_{f^{-1}(a)}^{f^{-1}(b)} g(f(x))f'(x)~\rmd x
~~~~~~bzw.~~~~~~
\int_a^b g(f(x))f'(x)~\rmd x = \int_{f(a)}^{f(b)} g(z)~\rmd z
\]
Rationale Funktionen lassen sich durch Partialbruchzerlegung
integrieren:
\[
\int \frac{P(x)}{Q(x)}~\rmd x = \sum_{i=1}^s\sum_{j=1}^{e_i}\int \frac{A_{ij}}{(x-y_i)^j}~\rmd x + \sum_{i=1}^t\sum_{j=1}^{d_i}\int \frac{B_{ij}x+C_{ij}}{(x^2+a_i x+b_i)^j}~\rmd x
\]
wobei Paare komplex konjugierter Wurzeln von $Q$ in der Form von (über $\Rset$
irreduziblen) quadratischen Polynomen zusammengefasst werden. Die Partialbrüche lassen sich alle elementar integrieren:
\[
\int \frac{\rmd x}{(x-y)^k} = \left\{\begin{array}{l}\log|x-y|,~k=1,\\-\frac{1}{(k-1)(x-y)^{k-1}},~k\not=1,\end{array}\right.
\]
und entsprechende Formeln für quadratische Nenner.

\end{document}

