\documentclass[11pt]{exam}
\usepackage[german]{babel}
\usepackage[utf8x]{inputenc}
\usepackage{graphicx}
\usepackage{latexsym,ifthen,amssymb,amsfonts,amsmath}

\begin{document}
	
\include{definitions}

\pagestyle{empty}

\def\loesungen{0}
\newcommand{\lansel}[2]{#1}

\ifthenelse{\equal{\loesungen}{1}}{\printanswers}{\relax}
\renewcommand{\solutiontitle}{\noindent\textbf{L\"osung:}\enspace}
\newcommand{\loesungname}{\ifthenelse{\equal{\loesungen}{1}}{L\"osungen zu }{\relax}}

\begin{center}
\textbf{\LARGE \loesungname \lansel{\"Ubungsblatt}{Examples Sheet} 3} \\ \vspace{1ex}
\textbf{\large \lansel{zum Mathematischen Brückenkurs \\ für Naturwissenschaftler:innen}{for the preparatory mathematics course for bio- and geoscientists}} \\ \vspace{1ex}
\textbf{\large \lansel{im Wintersemester}{Winter} 2023/24} \\ \vspace{0.5cm}
\textrm{\normalsize \hfill \lansel{Dozent}{Lecturer}: Apl.Prof. Dr. G. von Hippel\hfill${}$}
\end{center}
\normalsize\vspace{0.5cm}

\ifthenelse{\equal{\loesungen}{1}}{
\begin{center}
\textbf{Die direkte Weitergabe der Musterl\"osungen an Studierende ist nicht gestattet!}
\end{center}
}{
\vspace{3ex}
}

\begin{questions}
\pointname{ P.}

%%%%%%%%%%%%%%%%%%%%%%%%%%%%%%%%%%%%%%%%%%%%%%%%%%%%%%%%%%%%%%%%%%%%%%%%%%%%%%%

\question{{\it Eigenschaften von reellen Funktionen}}

Bestimmen Sie für die folgenden Funktionen $f:\Rset\to\Rset$ jeweils, ob diese nach oben bzw. nach unten beschränkt, beschränkt, monoton wachsend bzw. fallend, streng monoton wachsend bzw. fallend, und gerade bzw. ungerade sind.\\
\parbox{0.4\textwidth}{\begin{enumerate}
\item $f(x)=x^2$
\item $f(x)=\rme^x$
\item $f(x)=\rme^{-x^2}$
\item $f(x)=\sin x$
\item $f(x)=\sin (x^2)$
\end{enumerate}}\parbox{0.6\textwidth}{\begin{enumerate}\setcounter{enumi}{5}
\item $f(x)=x^5+x^3-x$
\item $f(x)=(x^2-1)^2$
\item $f(x)=\rme^x+\rme^{-x}$
\item $f(x)=\rme^x-\rme^{-x}$
\item $f(x)=x\log(x^2+1)$
\end{enumerate}}



\question{{\it Grenzwerte von Funktionen}}

Bestimmen Sie jeweils die folgenden Grenzwerte:\\
\parbox{0.5\textwidth}{\begin{enumerate}
\item $\lim\limits_{x\to 0} x^2$
\item $\lim\limits_{x\to 0} \rme^{-x^2}$
\item $\lim\limits_{x\to 0} \rme^{-\frac{1}{x^2}}$
\item $\lim\limits_{x\to 2} \frac{x-2}{x^2+1}$
\item $\lim\limits_{x\to 1^+} \left(x+\sqrt{x-1}\right)$
\end{enumerate}}\parbox{0.5\textwidth}{\begin{enumerate}\setcounter{enumi}{5}
\item $\lim\limits_{x\to 0} \frac{\rme^x-1}{2x}$
\item $\lim\limits_{x\to 1} \frac{(x^2-1)^2}{1-x}$
\item $\lim\limits_{x\to 0} x\cot x$
\item $\lim\limits_{x\to 0} \frac{x^3}{\sinh^3 x}$
\item $\lim\limits_{x\to 0^+} x\sin\left(\frac{1}{x}\right)$
\end{enumerate}}



\question{{\it Ableitung von Funktionen -- I}}

Bestimmen Sie jeweils die Ableitung folgender Funktionen ausgehend von der Definition der Ableitung:\\
\parbox{0.5\textwidth}{\begin{enumerate}
\item $f(x)=x^2$
\item $f(x)=x^3$
\end{enumerate}}\parbox{0.5\textwidth}{\begin{enumerate}\setcounter{enumi}{2}
\item $f(y)=\sin y$
\item $f(y)=\cos x$
\end{enumerate}}


\pagebreak

\question{{\it Ableitung von Funktionen -- II}}

Bestimmen Sie jeweils die Ableitung folgender Funktionen nach dem angegebenen Argument, und geben Sie die zugehörigen Definitionsbereiche an:\\
\parbox{0.5\textwidth}{\begin{enumerate}
\item $f(x)=x^2$
\item $f(x)=x^n+x^p-c$, $n\in\Nset$, $p>0$
\item $g(\omega)=\sin(\omega t+\varphi)$
\item $h(t)=\sin(\omega t+\varphi)$
\item $f(s)=\sqrt{s^2+1}$
\item $p(q)=-(q^2-a^2)^2$
\item $S(h)=\rme^{\alpha h^2+\beta h-\gamma}+\rme^{\omega h}$
\item $\lambda(x)=x\sin x+x^2\cos x$
\item $\rho(\sigma)=\sigma\log\sigma-\sigma$
\item $\theta(z)=\sum_{k=1}^8 k z^k$
\end{enumerate}}\parbox{0.5\textwidth}{\begin{enumerate}\setcounter{enumi}{10}
\item $f(y)=\frac{y^2-2y-1}{y^2+4}$
\item $f(y)=\frac{y^2-2y-1}{y^2-4}$
\item $f(x)=\frac{\rme^x+\rme^{-2x}}{1+\rme^{x^2}}$
\item $s(x)=x\sqrt{1+x^2}$
\item $w(t)=\sqrt{\left(1-x^2\right)^2}$
\item $\sigma(t)=|t+1|$
\item $r(t)=\sin^2(\omega t)+\cos^2(\omega t+\phi)$
\item $z(x)=\frac{\sin x-\cos x}{x^2+2}$
\item $k(x)=\log\sqrt{x^4+1}$
\item $f(x)=\sqrt{\log(x^4+1)}$
\end{enumerate}}



\question{{\it Höhere Ableitungen}}

Bestimmen Sie jeweils die ersten, zweiten und dritten Ableitungen folgender Funktionen nach dem angegebenen Argument:\\
\parbox{0.5\textwidth}{\begin{enumerate}
\item $f(x)=x^2-2x+1$
\item $f(x)=\rme^{-\frac{x^2}{2}}$
\item $g(y)=y\log y$
\item $h(r)=\frac{1}{1+r^2}$
\end{enumerate}}\parbox{0.5\textwidth}{\begin{enumerate}\setcounter{enumi}{4}
\item $u(\tau)=\sin^2(\tau^3) $
\item $g(x)=\rme^{\sin x}$
\item $f(x)=\sin(\alpha x)+\alpha\cos x$
\item $t(\alpha)=\tan\alpha$
\end{enumerate}}



\question{{\it Taylor-Reihen}}

Entwickeln Sie folgende Funktionen jeweils in eine Taylor-Reihe um den angegebenen Punkt:\\
\parbox{0.5\textwidth}{\begin{enumerate}
\item $P(x)=x^2+px+q$, $x=p$
\item $f(x)=\rme^{-x}$, $x=0$
\item $f(x)=\sin x$, $x=\pi$
\end{enumerate}}\parbox{0.5\textwidth}{\begin{enumerate}\setcounter{enumi}{3}
\item $f(y)=\log y$, $y=1$
\item $f(x)=\log(x+1)$, $x=0$
\item $g(\rho)=\frac{1}{1-\rho}$, $\rho=0$
\end{enumerate}}



\question{{\it Extrema von Funktionen}}

Bestimmen Sie jeweils alle lokalen sowie die globalen Extrema der folgenden Funktionen auf dem angegebenen Definitionsbereich:\\
\parbox{0.5\textwidth}{\begin{enumerate}
\item $f(x)=\lambda(x^2-v^2)^2$, $x\in\Rset$
\item $f(x)=\rme^{-x}$, $x\in[0;1]$
\item $f(x)=\sin x$, $x\in [-\pi/2;\pi]$
\end{enumerate}}\parbox{0.5\textwidth}{\begin{enumerate}\setcounter{enumi}{3}
\item $f(x)=|x^2-1|$, $x\in[-2;5]$
\item $f(x)=x^2\rme^{-x^2}$, $x\in[0;\infty)$
\item $g(t)=\frac{t+1}{t^2-4}$, $t\in\Rset\backslash\{\pm 2\}$
\end{enumerate}}



\end{questions}

\end{document}
