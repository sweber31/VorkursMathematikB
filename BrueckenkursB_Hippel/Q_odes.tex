%\question{{\it Differentialgleichungen erster Ordnung}}
%
%Finden Sie jeweils die Lösung für das Anfangswertproblem $u(0)=u_0$
%mit $u_0>0$ zu folgenden Differentialgleichungen erster Ordnung:\\
%\parbox{0.5\textwidth}{\begin{enumerate}
%\item $u'(x)+x^nu(x)=0$,~$n>-1$%separabel
%\item $u'(x)-u(x)^3=0$%separabel
%\item $u'(x)+x^2u(x)^2=u(x)$%separabel
%\item $(1+x)u'(x)=[xu(x)]^2$%separabel
%\item $(1+x)^2u'(x)-x^2u(x)=0$%separabel
%\item $u'(x)\cos x=-u(x)\sin x$%separabel
%\end{enumerate}}\parbox{0.5\textwidth}{\begin{enumerate}\setcounter{enumi}{6}
%\item $\left(x^2+1\right) u'(x)+2 x u(x)=0$%exakt
%\item $(2 x u(x)+1) u'(x)+u(x)^2=0$%exakt
%\item $(u(x)- x^3)u'(x)-3 x^2 u(x) = 3 x^5 $%exakt
%\item $2 x - u(x) - x u'(x) + 2 u(x)u'(x)=0$%exakt
%\item $u(x)u'(x)+\sqrt{x^2+u(x)^2}+x=0$%integrierender Faktor 1/\sqrt{x^2+u^2}
%\item $2 u(x) u'(x)+u(x)^2+e^{-x} = 0$%integrierender Faktor e^x
%\end{enumerate}}



\question{{\it Einige spezielle Differentialgleichungen zweiter Ordnung}}

\begin{parts}
\part Finden Sie für die folgenden Differentialgleichungen jeweils eine
Lösung in Form eines Polynoms vom Grad $n$ für $n\in\{2,3,4,5,6\}$,
dessen führender Koeffizient gleich Eins ist:
\begin{enumerate}
\item $p''(t)-2tp'(t)+2np(t)=0$
\item $tp''(t)+(1-t)p'(t)+np(t)=0$
\item $(1-t^2)p''(t)-2tp'(t)+n(n+1)p(t)=0$
\item $(1-t^2)p''(t)-3tp'(t)+n(n+2)p(t)=0$
\end{enumerate}

\part Verifizieren Sie, dass für $k\in\{0,1,2,3\}$ die Lösungen des
Anfangswertproblems
\[
-u_k''(x)+x^2u_k(x)=(2k+1)u_k(x),~~~~~u_k(0)=\frac{1+(-1)^k}{2},~~~u_k'(0)=\frac{1-(-1)^k}{2}
\]
durch
\[
u_0(x)=\rme^{-\frac{x^2}{2}},~~~
u_1(x)=x\rme^{-\frac{x^2}{2}},~~~
u_2(x)=(1-2x^2)\rme^{-\frac{x^2}{2}},~~~
u_3(x)=x(1-\frac{2}{3}x^2)\rme^{-\frac{x^2}{2}}
\]
gegeben sind.

\end{parts}



\question{{\it Erzwungene Schwingungen}}%$^\ast$}}

\begin{parts}
%\part Vollziehen Sie die in der Vorlesung dargestellte Herleitung der
%allgemeinen Lösung der homogenenen Differentialgleichung
%\[
%f''(t)+2\gamma f'(t)+\omega_0^2 f(t)=0
%\]
%nach.

\part Verifizieren Sie, dass für $\gamma>0$ oder $\omega\not=\omega_0$
eine partikuläre Lösung der inhomogenen Differentialgleichung
\[
f''(t)+2\gamma f'(t)+\omega_0^2 f(t)=A\sin(\omega t)
\]
durch
\[
f(t) = \frac{A \left(\omega_0^2-\omega ^2\right) \sin (t \omega )-2 A \gamma
   \omega  \cos (t \omega )}{4 \gamma^2 \omega ^2+\left(\omega ^2-\omega_0^2\right)^2}
\]
gegeben ist. Interpretieren Sie diese Lösung in Hinsicht auf die in der
Vorlesung angegebene komplexifizierte Lösung.
(Hinweis: Benutzen Sie die Euler-Formel!)

\part Verifizieren Sie ferner, dass für $\gamma=0$, $\omega=\omega_0$ eine
partikuläre Lösung der inhomogenen Differentialgleichung
\[
f''(t)+\omega_0^2 f(t)=A\sin(\omega_0 t)
\]
durch
\[
f(t)=-\frac{A t\cos(\omega_0 t)}{2\omega_0}
\]
gegeben ist. Interpretieren Sie diese Lösung im Hinblick auf das Verhalten
ungedämpft schwingender Systeme unter resonanter Anregung.

%\part Verifizieren Sie zu guter Letzt, dass eine partikuläre Lösung der
%inhomogenen Differentialgleichung
%\[
%f''(t)+\omega_0^2 f(t)=h(t)
%\]
%für stetiges $h$ durch
%\[
%f(t)=\frac{\sin(\omega_0 t)}{\omega_0}\int_0^t\cos(\omega_0\tau)h(\tau)~\rmd\tau-\frac{\cos(\omega_0 t)}{\omega_0}\int_0^t\sin(\omega_0\tau)h(\tau)~\rmd\tau
%\]
%gegeben ist.
\end{parts}

%\question{{\it Das Eulersche Polygonzugverfahren$^\ast$}}
%
%Sehr oft können Differentialgleichungen nicht analytisch gelöst werden.
%In diesem Fall sind numerische Näherungsverfahren von großer praktischer
%Bedeutung. Ein einfaches solches Verfahren ist das Eulersche
%Polygonzugverfahren, in dem die Lösung $y(t)$ des Anfangswertproblems
%\[
%y'(t)=f(y(t),t)~~~~~~~~y(0)=y_0
%\]
%%durch einen Polygonzug
%\[
%\tilde{y}(t)=y_n+(t-nh)f(y_n,nh)~~~\textrm{für}~t\in[nh;(n+1)h]
%\]
%mit der rekursiv definierten Folge von Vertices
%\[
%y_{n+1}=y_n+hf(y_n,nh)
%\]
%angenähert wird.
%
%\begin{parts}
%\part Welche Folge von Vertices $y_n$ ergibt sich für die Wachstumsgleichung?
%Was geschieht im Limes $h\to 0$?
%
%\part Wenden Sie das Eulersche Polygonzugverfahren auf die Probleme aus
%Aufgabe 1 an und vergleichen Sie die Folge der Vertices jeweils mit
%der exakten Lösung.
%\end{parts}
