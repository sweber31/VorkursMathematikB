\documentclass[12pt,answers]{exam}
\usepackage[utf8x]{inputenc}
\usepackage[german]{babel}
\usepackage{latexsym,ifthen,amsmath,amsfonts,bm}

\begin{document}

\input{definitions.tex}
\newcommand{\lansel}[2]{#1}

\renewcommand{\solutiontitle}{\noindent\textbf{L\"osung:}\enspace}

\setlength{\parindent}{0pt}
\pagestyle{empty}

%%%%%%%%%%%%%%%%%%%%%%%%%%%%%%%%%%%%%%%%%%%%%%%%%%%%%%%%%%%%%%%%%%%%%%%%%%%%%%

\begin{center}
\textbf{\LARGE Test \"uber Lernergebnisse} \\ \vspace{1ex}
\textbf{\huge LÖSUNG} \\ \vspace{1ex}
\textbf{\large im Mathematischen Brückenkurs (B)} \\ \vspace{1ex}
\textbf{\large f\"ur das Wintersemester 2023/24} \\ \vspace{0.5cm}
\textbf{\Large --- Deckblatt ---}
\end{center}
\normalsize\vspace{0.5cm}

Die folgenden Angaben bitte selbst eintragen:\vspace{0.5ex}

\fbox{
\begin{minipage}{\textwidth}
\vspace{2ex}
\hbox to \textwidth{Abiturjahrgang:\enspace\hrulefill\enspace
Leistungskurse:\enspace\hrulefill\hrulefill\enspace}
\vspace{2ex}
\hbox to \textwidth{Studienziel:\enspace$\Box$\enspace BSc
\enspace\enspace$\Box$\enspace BEd\enspace\enspace
Studienf\"acher:\enspace\hrulefill\hrulefill\enspace}
\vspace{2ex}
\hbox to \textwidth{Name/Mat.Nr. (falls R\"uckgabe gew\"unscht, Angabe freiwillig):\enspace\hrulefill\enspace}
\vspace{1ex}
\end{minipage}}

\vspace{0.5cm}

\begin{center} \textbf{Wichtige Hinweise -- bitte sorgf"altig lesen!} \end{center}
\begin{itemize}
\item Die Bearbeitungsdauer betr"agt \textbf{90 Minuten}.

\item Bitte schreiben Sie Ihre Antworten in die daf\"ur vorgesehenen
Leerstellen auf den Aufgabenbl\"attern.

\item Bitte geben Sie zu jeder Aufgabe durch Ankreuzen der entsprechenden
Box an, wie gut Sie \textbf{vor Beginn des Brückenkurses}
mit dem Thema der Aufgabe vertraut waren, auch dann,
wenn Sie die Aufgabe aufgrund Zeitmangels oder anderer Gründe
nicht weiter bearbeiten können.

\item Bitte schreiben Sie mit F"uller oder Kugelschreiber in schwarzer oder
blauer Tinte. Au"ser Ihrem Schreibwerkzeug 
d"urfen keine weiteren Hilfsmittel verwendet werden.

\item Die Angabe Ihres Namens (oder alternativ Ihrer Matrikelnummer) ist
freiwillig. Ohne diese Angabe ist eine Rückgabe des korrigierten Tests
nicht möglich. In keinem Falle werden Ihre persönlichen Daten gespeichert.

\end{itemize}

\clearpage
%%%%%%%%%%%%%%%%%%%%%%%%%%%%%%%%%%%%%%%%%%%%%%%%%%%%%%%%%%%%%%%%%%%%%%%%%%%%%%

\begin{questions}
\pointformat{\relax}
\pointsinmargin

%%%%%%%%%%%%%%%%%%%%%%%%%%%%%%%%%%%%%%%%%%%%%%%%%%%%%%%%%%%%%%%%%%%%%%%%%%%%%%

\question{{\it Algebraische Umformungen}}

Mit den natürlichen, rationalen und reellen Zahlen und ihren Rechenregeln war ich vor Beginn des Brückenkurses \\ $\Box$ gut vertraut \hfill $\Box$ etwas vertraut \hfill $\Box$ gar nicht vertraut. \\

\begin{parts}

\part Expandieren Sie die folgenden Ausdr\"ucke so, dass das Ergebnis keine
      Klammern mehr enth\"alt:\\
\begin{solution}
\begin{subparts}

\subpart $(a-b)^2 =a^2-2ab+b^2$\\

\subpart $(a-b)(a+b) = a^2-b^2$\\

\subpart $(d+c)^3 = d^3+3d^2c+3dc^2+c^3$\\

\end{subparts}
\end{solution}
\part Vereinfachen Sie die folgenden Ausdr\"ucke so weit wie m\"oglich:\\
\begin{solution}
\begin{subparts}

\subpart $ a^2+b^2+2ab =(a+b)^2$\\

\subpart $ ax^2+ay^2-2axy = a(x-y)^2$\\

\subpart $ (a+b)^2-(a-b)^2-4ab = 0$\\

\end{subparts}
\end{solution}
\part Geben Sie die Lösungsmengen folgender Gleichungen in einer reellen Variable $x$ an:\\
\begin{solution}
\begin{subparts}

\subpart $ (x+5)^4-1=0 ~~~\leadsto~~~ x\in\{-4,-6\}$\\

\subpart $ 2x^2-2x=6 ~~~\leadsto~~~ x\in\left\{\frac{1-\sqrt{13}}{2},\frac{1+\sqrt{13}}{2}\right\}$\\

\end{subparts}
\end{solution}
\part Vereinfachen Sie die folgenden Ausdr\"ucke so, dass noch h\"ochstens eine
      Exponential- oder Logarithmusfunktion auftritt ($a$ und $b$ sind reell):\\
\begin{solution}
\begin{subparts}

\subpart $ \rme^a\,\rme^b = e^{a+b}$\\

\subpart $ \rme^{2 \mathop{\rm ln} a} = a^2$\\

\subpart $ \mathop{\rm ln} a - \mathop{\rm ln} b = \ln\left(\frac{a}{b}\right)$\\

\end{subparts}
\end{solution}
\end{parts}

%%%%%%%%%%%%%%%%%%%%%%%%%%%%%%%%%%%%%%%%%%%%%%%%%%%%%%%%%%%%%%%%%%%%%%%%%%%%%%
\pagebreak
\question{{\it Rechnen mit komplexen Zahlen}}

Mit den komplexen Zahlen und ihren Rechenregeln war ich vor Beginn des Brückenkurses \\ $\Box$ gut vertraut \hfill $\Box$ etwas vertraut \hfill $\Box$ gar nicht vertraut. \\

\begin{parts}

\part Berechnen Sie jeweils:\\
\begin{solution}
\begin{subparts}

\subpart $ (3+2i)+(9-7i) = 12-5i$\\

\subpart $ (1+3i)(2-i) = 5+5i$\\

\subpart $ (\frac{1+i}{\sqrt{2}})^2 = i$\\

\end{subparts}
\end{solution}
\part Bestimmen Sie den Real- und Imagin\"arteil folgender Zahlen:\\
\begin{solution}
\begin{subparts}

\subpart $ z=(1+i)(1-i) ~~~~\leadsto~~~~ {\rm Re}(z) = 2~~~~~~~~~~~~~~ 
{\rm Im}(z) = 0~~~~~~~~~~~~~~$\\

\subpart $ z=\frac{1}{1+i} ~~~~\leadsto~~~~ {\rm Re}(z) =\frac{1}{2} ~~~~~~~~~~~~~~ 
{\rm Im}(z) = -\frac{1}{2}~~~~~~~~~~~~~~$\\

\subpart $ z=\frac{1+i}{i} ~~~~\leadsto~~~~ {\rm Re}(z) = 1~~~~~~~~~~~~~~ 
{\rm Im}(z) = -1~~~~~~~~~~~~~~$\\

\end{subparts}
\end{solution}
\end{parts}

%%%%%%%%%%%%%%%%%%%%%%%%%%%%%%%%%%%%%%%%%%%%%%%%%%%%%%%%%%%%%%%%%%%%%%%%%%%%%%
\pagebreak
\question{{\it Rechnen mit Vektoren}}

Mit Vektorrechnung und analytischer Geometrie war ich vor Beginn des Brückenkurses\\ $\Box$ gut vertraut \hfill $\Box$ etwas vertraut \hfill $\Box$ gar nicht vertraut. \\[2ex]
Gegeben seien die Vektoren $\vec{a}=(2,0,2)$ und $\vec{b}=(0,-2,2)$.

\begin{parts}

\part Berechnen Sie jeweils\\~~\\
\begin{solution}
\begin{subparts}

\subpart $ \vec{a}+\vec{b} = \begin{pmatrix}2\\-2\\4\end{pmatrix} $
~~\\~~\\~~\\
\subpart $ 3\vec{a}-2\vec{b} = \begin{pmatrix}6\\4\\2\end{pmatrix}$
~~\\~~\\~~\\
\subpart $ \vec{a}\cdot\vec{b} = 4$
~~\\~~\\~~\\
\subpart $ |\vec{a}| = \sqrt{8}=2\sqrt{2}$
~~\\~~\\~~\\
\subpart $ \vec{a}\times\vec{b} = \begin{pmatrix}4\\-4\\-4\end{pmatrix}$
~~\\~~\\~~\\
\end{subparts}
\end{solution}
\newpage
\part Bestimmen Sie
\begin{solution}
\begin{subparts}

\subpart die Fl\"ache des durch $\vec{a}$ und $\vec{b}$ aufgespannten Parallelogramms,
\\ \\
$A=|\vec{a}\times\vec{b}|=4\sqrt{3}$
~~\\~~\\~~\\~~\\
\subpart den Winkel zwischen $\vec{a}$ und $\vec{b}$.
\\ \\
$\cos(\alpha)=\frac{\vec{a}\cdot\vec{b}}{|\vec{a}|\cdot|\vec{b}|}
=\frac{4}{|\vec{a}|^2}=\frac{1}{2}\quad\Rightarrow\quad\alpha=\pi/3\ \hat{=}\ 60^\circ$
%~~\\~~\\~~\\~~\\~~\\~~\\

\end{subparts}
\end{solution}
\end{parts}
\newpage
%%%%%%%%%%%%%%%%%%%%%%%%%%%%%%%%%%%%%%%%%%%%%%%%%%%%%%%%%%%%%%%%%%%%%%%%%%%%%%
\question{{\it Rechnen mit Matrizen}}

Mit den Rechenregeln für Matrizen war ich vor Beginn des Brückenkurses\\ $\Box$ gut vertraut \hfill $\Box$ etwas vertraut \hfill $\Box$ gar nicht vertraut. \\[2ex]
Gegeben seien die Matrizen
\[
A=\left(\begin{array}{rr}1&1\\0&1\end{array}\right)
\textrm{~~~und~~~}
B=\left(\begin{array}{rr}1&2\\2&4\end{array}\right).
\]

Berechnen Sie jeweils\\
\begin{solution}
\begin{parts}


\part $ A+B = \begin{pmatrix}2&3\\2&5\end{pmatrix}$
~~\\~~\\~~\\~~\\~~\\

\part $ A-2B = \begin{pmatrix}-1&-3\\-4&-7\end{pmatrix}$
~~\\~~\\~~\\~~\\~~\\

\part $ AB = \begin{pmatrix}3&6\\2&4\end{pmatrix}$
~~\\~~\\~~\\~~\\~~\\

\part $ BA = \begin{pmatrix}1&3\\2&6\end{pmatrix}$
~~\\~~\\~~\\~~\\~~\\

\end{parts}
\end{solution}

%%%%%%%%%%%%%%%%%%%%%%%%%%%%%%%%%%%%%%%%%%%%%%%%%%%%%%%%%%%%%%%%%%%%%%%%%%%%%%
\pagebreak
\question{{\it Lineare Gleichungssysteme}}

Mit dem Lösen linearer Gleichungssysteme war ich vor Beginn des Brückenkurses\\ $\Box$ gut vertraut \hfill $\Box$ etwas vertraut \hfill $\Box$ gar nicht vertraut. \\[2ex]
L\"osen Sie folgendes Gleichungssystem nach den Variablen $x_1,\ldots,x_4$
auf:

\[\begin{array}{rrrrl}
7 x_1 &+ 4 x_2 &~      &-3 x_4 &=3 \\
7 x_1 &~       &+5 x_3 &-3 x_4 &=0 \\
~     &4   x_2 &-5 x_3 &-6 x_4 &=3 \\
7 x_1 &+ 7 x_2 &~      &~      &=0
\end{array}\]
%~~\\[20ex]\vfill
\begin{solution}
\begin{align*}
&
\left(\begin{array}{rrrr|r}
7 & 4 & 0 & -3 & 3\\
7 & 0 & 5 & -3 & 0\\
0 & 4 & -5 & -6 & 3\\
7 & 7 & 0 & 0 & 0
\end{array} \right)
\rightsquigarrow
\left(\begin{array}{rrrr|r}
7 & 4 & 0 & -3 & 3\\
0 & -4 & 5 & 0 & -3\\
0 & 4 & -5 & -6 & 3\\
0 & 3 & 0 & 3 & -3
\end{array} \right)
\rightsquigarrow
\\
&
\left(\begin{array}{rrrr|r}
7 & 4 & 0 & -3 & 3\\
0 & -4 & 5 & 0 & -3\\
0 & 0 & 0 & -6 & 0\\
0 & 3 & 0 & 3 & -3
\end{array} \right)
\\ \\
&(3)\ x_4=0
\\
&(4)\ 3x_2=-3\ \Rightarrow\ x_2=-1
\\
&(1)\ 7x_1+4x_2=3\ \Rightarrow\ x_1=1
\\
&(3)\ 4x_2-5x_3=3\ \Rightarrow\ x_3=-7/5
\\ \\
& \mathbb{L}=\left\{ \begin{pmatrix}1\\-1\\-7/5\\0\end{pmatrix} \right\}
\end{align*}
\end{solution}
%%%%%%%%%%%%%%%%%%%%%%%%%%%%%%%%%%%%%%%%%%%%%%%%%%%%%%%%%%%%%%%%%%%%%%%%%%%%%%
\pagebreak
\question{{\it Grenzwerte}}

Mit Grenzwerten war ich vor Beginn des Brückenkurses\\ $\Box$ gut vertraut \hfill $\Box$ etwas vertraut \hfill $\Box$ gar nicht vertraut. \\[2ex]
Bestimmen Sie jeweils die folgenden Grenzwerte:\\
\begin{solution}
\begin{parts}

\part $ \lim\limits_{n\to\infty}\frac{n+1}{n^2-1} = 0$\\

\part $ \lim\limits_{n\to\infty} 2^{\frac{1}{n}} = 1$\\

\part $ \lim\limits_{x\to 0} \rme^{\sin x} = 1$\\

\part $ \lim\limits_{x\to 0} \frac{\rme^x-1}{2x} = 
\lim\limits_{x\to 0} \frac{\rme^x}{2}=\frac{1}{2}$\\

\end{parts}
\end{solution}
\pagebreak
%%%%%%%%%%%%%%%%%%%%%%%%%%%%%%%%%%%%%%%%%%%%%%%%%%%%%%%%%%%%%%%%%%%%%%%%%%%%%%

\question{{\it Ableitung von Funktionen}}

Mit der Differentialrechnung (Ableiten von Funktionen) war ich vor Beginn des Brückenkurses\\ $\Box$ gut vertraut \hfill $\Box$ etwas vertraut \hfill $\Box$ gar nicht vertraut. \\[2ex]
Bestimmen Sie jeweils die erste und zweite Ableitung folgender
Funktionen einer reellen Variablen:\\
\begin{solution}
\begin{parts}

\part \[\begin{array}{ll}
 f(x)  &=x^5\\ ~~\\
 f'(x) &= 5x^4\\ ~~\\
 f''(x)&= 20x^3\\ ~~\\
\end{array}~~~~~~~~~~~~~~~~~~~~~~~~~~~~~~~~~~~~~~~~~~~~~~~~~~~~~~~~~~~~~~~~~~~~\]

\part \[\begin{array}{ll}
 f(t)  &={\rm e}^{t^2-tx+x}\\ ~~\\
 f'(t) &= (2t-x){\rm e}^{t^2-tx+x}\\ ~~\\
 f''(t)&= 2{\rm e}^{t^2-tx+x}+(2t-x)^2{\rm e}^{t^2-tx+x}\\ ~~\\
\end{array}~~~~~~~~~~~~~~~~~~~~~~~~~~~~~~~~~~~~~~~~~~~~~~~~~~~~~~~~~~~~~~~~~~~~\]


\part \[\begin{array}{ll}
 f(x)  &=a\cos(\omega x+\varphi)-b x\\ ~~\\
 f'(x) &=-\omega a\sin(\omega x+\varphi)-b \\ ~~\\
 f''(x)&= -a\omega^2\cos(\omega x+\varphi)\\ ~~\\
\end{array}~~~~~~~~~~~~~~~~~~~~~~~~~~~~~~~~~~~~~~~~~~~~~~~~~~~~~\]


\part \[\begin{array}{ll}
 g(\lambda)  &=\sqrt{\lambda^3+a^2}\\ ~~\\
 g'(\lambda) &= \frac{3\lambda^2}{2\sqrt{\lambda^3+a^2}}\\ ~~\\
 g''(\lambda)&= \frac{12\lambda(\lambda^3+a^2)-9\lambda^4}{4(\lambda^3+a^2)^{3/2}}\\ ~~\\
\end{array}~~~~~~~~~~~~~~~~~~~~~~~~~~~~~~~~~~~~~~~~~~~~~~~~~~~~~~~~~~~~~~~~~~~~\]

\end{parts}
\end{solution}
%%%%%%%%%%%%%%%%%%%%%%%%%%%%%%%%%%%%%%%%%%%%%%%%%%%%%%%%%%%%%%%%%%%%%%%%%%%%%%
\pagebreak
\question{{\it Partielle Ableitungen}}

Mit dem partiellen Ableiten von Funktionen mehrerer Variablen war ich vor Beginn des Brückenkurses\\ $\Box$ gut vertraut \hfill $\Box$ etwas vertraut \hfill $\Box$ gar nicht vertraut. \\[2ex]
Berechnen Sie alle ersten und zweiten partiellen Ableitungen der Funktion
\[
f(x,y)=\sin(x+xy)
\]
und verifizieren Sie, dass die gemischten zweiten Ableitungen vertauschen.
\begin{solution}
\[\begin{array}{ll} ~~\\ 
\displaystyle\frac{\partial f}{\partial x}= \cos(x+xy)\cdot (1+y)\\ ~~\\~~\\
\displaystyle\frac{\partial f}{\partial y}= x\cdot\cos(x+xy) \\ ~~\\~~\\
\displaystyle\frac{\partial^2 f}{\partial x^2}= -\sin(x+xy)\cdot(1+y)^2 \\ ~~\\~~\\
\displaystyle\frac{\partial^2 f}{\partial y\partial x}= \cos(x+xy)-(1+y)\cdot \sin(x+xy)\cdot x \\~~\\ ~~\\
\displaystyle\frac{\partial^2 f}{\partial x\partial y}= \cos(x+xy)-
x\cdot(1+y)\cdot\sin(x+xy)\\ ~~\\~~\\
\displaystyle\frac{\partial^2 f}{\partial y^2}= -x^2\sin(x+xy)\\ ~~\\~~\\
\end{array}~~~~~~~~~~~~~~~~~~~~~~~~~~~~~~~~~~~~~~~~~~~~~~~~~~~~~~~~~~~~~~~~~~~~
\]
\end{solution}
%%%%%%%%%%%%%%%%%%%%%%%%%%%%%%%%%%%%%%%%%%%%%%%%%%%%%%%%%%%%%%%%%%%%%%%%%%%%%%
\pagebreak
\question{{\it Bestimmte Integrale}}

Mit der Integralrechnung war ich vor Beginn des Brückenkurses\\ $\Box$ gut vertraut \hfill $\Box$ etwas vertraut \hfill $\Box$ gar nicht vertraut. \\[2ex]
Bestimmen Sie jeweils den Wert folgender bestimmter Integrale:\\
\begin{solution}
\begin{parts}

\part \[\int_a^b x^3 {\rm ~d}x = \left[\frac{1}{4}x^4\right]^b_a =\frac{b^4-a^4}{4}
~~~~~~~~~~~~~~~~~~~~~~~~~~~~~~~~~~~~~~~~~~~~~~~~~~~~~~~~~~~~~~~~~~~~~~~~~~~~\]
%~~\\~~\\

\part \[\int_0^x \sinh u {\rm ~d}u = \left[\cosh(u)\right]^x_0 =\cosh(x)-1
~~~~~~~~~~~~~~~~~~~~~~~~~~~~~~~~~~~~~~~~~~~~~~~~~~~~~~~~~~~~~~~~~~~~~~~~~~~~\]
%~~\\~~\\

\part \[\int_0^{\frac{\pi}{2}} {\rm e}^{\sin t}\cos t {\rm ~d}t = 
\left[e^{\sin(t)}\right]^{\pi/2}_0=e-1
~~~~~~~~~~~~~~~~~~~~~~~~~~~~~~~~~~~~~~~~~~~~~~~~~~~~~~~~~~~~~~~~~~~~~~~~~~~~\]
%~~\\~~\\

\part \[
\int_0^{\Lambda} \frac{w-1}{w^3+w^2-w-1} {\rm ~d}w = 
\int_0^{\Lambda} \frac{1}{(w+1)^2} {\rm ~d}w =\left[-\frac{1}{w+1}\right]^\Lambda_0 =\frac{\Lambda}{\Lambda+1},\quad \Lambda\neq -1
~~~~~~~~~~~~~~~~~~~~~~~~~~~~~~~~~~~~~~~~~~~~~~~~~~~~~~~~~~~~~~~~~~~~~~~~~~~~\]
%~~\\~~\\
\begin{align*}
&(w^3+w^2-w-1):(w-1)=(w^2+2w+1)
\\
\Rightarrow &\ \frac{w-1}{w^3+w^2-w-1}=\frac{1}{(w^2+2w+1)}=\frac{1}{(w+1)^2}
\end{align*}
\end{parts}
\end{solution}
%%%%%%%%%%%%%%%%%%%%%%%%%%%%%%%%%%%%%%%%%%%%%%%%%%%%%%%%%%%%%%%%%%%%%%%%%%%%%%
\end{questions}

\end{document}

