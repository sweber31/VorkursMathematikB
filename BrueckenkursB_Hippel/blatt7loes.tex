\documentclass[11pt,answers]{exam}
\usepackage[german]{babel}
\usepackage[utf8x]{inputenc}
\usepackage{graphicx}
\usepackage{latexsym,ifthen,amssymb,amsfonts,amsmath}

\begin{document}

\include{definitions}

\pagestyle{empty}

\def\loesungen{1}
\newcommand{\lansel}[2]{#1}

\ifthenelse{\equal{\loesungen}{1}}{\printanswers}{\relax}
\renewcommand{\solutiontitle}{\noindent\textbf{L\"osung:}\enspace}
\newcommand{\loesungname}{\ifthenelse{\equal{\loesungen}{1}}{L\"osungen zu }{\relax}}

\begin{center}
\textbf{\LARGE \loesungname \lansel{\"Ubungsblatt}{Examples Sheet} 8} \\ \vspace{1ex}
\textbf{\large \lansel{zum Mathematischen Brückenkurs \\ für Naturwissenschaftler:innen}{for the preparatory mathematics course for bio- and geoscientists}} \\ \vspace{1ex}
\textbf{\large \lansel{im Wintersemester}{Winter} 2023/24} \\ \vspace{0.5cm}
\textrm{\normalsize \hfill \lansel{Dozent}{Lecturer}: Apl.Prof. Dr. G. von Hippel\hfill${}$}
\end{center}
\normalsize\vspace{0.5cm}

%\begin{center}
%{Mit $^\ast$ versehene Aufgaben sind weiterführend und erfordern mehr Denkarbeit.}
%\end{center}

\begin{questions}
\pointname{ P.}

%%%%%%%%%%%%%%%%%%%%%%%%%%%%%%%%%%%%%%%%%%%%%%%%%%%%%%%%%%%%%%%%%%%%%%%%%%%%%%%

% %\question{{\it Differentialgleichungen erster Ordnung}}
%
%Finden Sie jeweils die Lösung für das Anfangswertproblem $u(0)=u_0$
%mit $u_0>0$ zu folgenden Differentialgleichungen erster Ordnung:\\
%\parbox{0.5\textwidth}{\begin{enumerate}
%\item $u'(x)+x^nu(x)=0$,~$n>-1$%separabel
%\item $u'(x)-u(x)^3=0$%separabel
%\item $u'(x)+x^2u(x)^2=u(x)$%separabel
%\item $(1+x)u'(x)=[xu(x)]^2$%separabel
%\item $(1+x)^2u'(x)-x^2u(x)=0$%separabel
%\item $u'(x)\cos x=-u(x)\sin x$%separabel
%\end{enumerate}}\parbox{0.5\textwidth}{\begin{enumerate}\setcounter{enumi}{6}
%\item $\left(x^2+1\right) u'(x)+2 x u(x)=0$%exakt
%\item $(2 x u(x)+1) u'(x)+u(x)^2=0$%exakt
%\item $(u(x)- x^3)u'(x)-3 x^2 u(x) = 3 x^5 $%exakt
%\item $2 x - u(x) - x u'(x) + 2 u(x)u'(x)=0$%exakt
%\item $u(x)u'(x)+\sqrt{x^2+u(x)^2}+x=0$%integrierender Faktor 1/\sqrt{x^2+u^2}
%\item $2 u(x) u'(x)+u(x)^2+e^{-x} = 0$%integrierender Faktor e^x
%\end{enumerate}}



\question{{\it Einige spezielle Differentialgleichungen zweiter Ordnung}}

\begin{parts}
\part Finden Sie für die folgenden Differentialgleichungen jeweils eine
Lösung in Form eines Polynoms vom Grad $n$ für $n\in\{2,3,4,5,6\}$,
dessen führender Koeffizient gleich Eins ist:
\begin{enumerate}
\item $p''(t)-2tp'(t)+2np(t)=0$
\item $tp''(t)+(1-t)p'(t)+np(t)=0$
\item $(1-t^2)p''(t)-2tp'(t)+n(n+1)p(t)=0$
\item $(1-t^2)p''(t)-3tp'(t)+n(n+2)p(t)=0$
\end{enumerate}

\part Verifizieren Sie, dass für $k\in\{0,1,2,3\}$ die Lösungen des
Anfangswertproblems
\[
-u_k''(x)+x^2u_k(x)=(2k+1)u_k(x),~~~~~u_k(0)=\frac{1+(-1)^k}{2},~~~u_k'(0)=\frac{1-(-1)^k}{2}
\]
durch
\[
u_0(x)=\rme^{-\frac{x^2}{2}},~~~
u_1(x)=x\rme^{-\frac{x^2}{2}},~~~
u_2(x)=(1-2x^2)\rme^{-\frac{x^2}{2}},~~~
u_3(x)=x(1-\frac{2}{3}x^2)\rme^{-\frac{x^2}{2}}
\]
gegeben sind.

\end{parts}



\question{{\it Erzwungene Schwingungen}}%$^\ast$}}

\begin{parts}
%\part Vollziehen Sie die in der Vorlesung dargestellte Herleitung der
%allgemeinen Lösung der homogenenen Differentialgleichung
%\[
%f''(t)+2\gamma f'(t)+\omega_0^2 f(t)=0
%\]
%nach.

\part Verifizieren Sie, dass für $\gamma>0$ oder $\omega\not=\omega_0$
eine partikuläre Lösung der inhomogenen Differentialgleichung
\[
f''(t)+2\gamma f'(t)+\omega_0^2 f(t)=A\sin(\omega t)
\]
durch
\[
f(t) = \frac{A \left(\omega_0^2-\omega ^2\right) \sin (t \omega )-2 A \gamma
   \omega  \cos (t \omega )}{4 \gamma^2 \omega ^2+\left(\omega ^2-\omega_0^2\right)^2}
\]
gegeben ist. Interpretieren Sie diese Lösung in Hinsicht auf die in der
Vorlesung angegebene komplexifizierte Lösung.
(Hinweis: Benutzen Sie die Euler-Formel!)

\part Verifizieren Sie ferner, dass für $\gamma=0$, $\omega=\omega_0$ eine
partikuläre Lösung der inhomogenen Differentialgleichung
\[
f''(t)+\omega_0^2 f(t)=A\sin(\omega_0 t)
\]
durch
\[
f(t)=-\frac{A t\cos(\omega_0 t)}{2\omega_0}
\]
gegeben ist. Interpretieren Sie diese Lösung im Hinblick auf das Verhalten
ungedämpft schwingender Systeme unter resonanter Anregung.

%\part Verifizieren Sie zu guter Letzt, dass eine partikuläre Lösung der
%inhomogenen Differentialgleichung
%\[
%f''(t)+\omega_0^2 f(t)=h(t)
%\]
%für stetiges $h$ durch
%\[
%f(t)=\frac{\sin(\omega_0 t)}{\omega_0}\int_0^t\cos(\omega_0\tau)h(\tau)~\rmd\tau-\frac{\cos(\omega_0 t)}{\omega_0}\int_0^t\sin(\omega_0\tau)h(\tau)~\rmd\tau
%\]
%gegeben ist.
\end{parts}

%\question{{\it Das Eulersche Polygonzugverfahren$^\ast$}}
%
%Sehr oft können Differentialgleichungen nicht analytisch gelöst werden.
%In diesem Fall sind numerische Näherungsverfahren von großer praktischer
%Bedeutung. Ein einfaches solches Verfahren ist das Eulersche
%Polygonzugverfahren, in dem die Lösung $y(t)$ des Anfangswertproblems
%\[
%y'(t)=f(y(t),t)~~~~~~~~y(0)=y_0
%\]
%%durch einen Polygonzug
%\[
%\tilde{y}(t)=y_n+(t-nh)f(y_n,nh)~~~\textrm{für}~t\in[nh;(n+1)h]
%\]
%mit der rekursiv definierten Folge von Vertices
%\[
%y_{n+1}=y_n+hf(y_n,nh)
%\]
%angenähert wird.
%
%\begin{parts}
%\part Welche Folge von Vertices $y_n$ ergibt sich für die Wachstumsgleichung?
%Was geschieht im Limes $h\to 0$?
%
%\part Wenden Sie das Eulersche Polygonzugverfahren auf die Probleme aus
%Aufgabe 1 an und vergleichen Sie die Folge der Vertices jeweils mit
%der exakten Lösung.
%\end{parts}


\question{{\it Differentialgleichungen erster Ordnung}}

Finden Sie jeweils die Lösung für das Anfangswertproblem $u(0)=u_0$
mit $u_0>0$ zu folgenden Differentialgleichungen erster Ordnung:\\
\parbox{0.5\textwidth}{\begin{enumerate}
\item $u'(x)+x^nu(x)=0$,~$n>-1$%separabel
\item $u'(x)-u(x)^3=0$%separabel
\item $u'(x)+x^2u(x)^2=u(x)$%separabel
\item $(1+x)u'(x)=[xu(x)]^2$%separabel
\item $(1+x)^2u'(x)-x^2u(x)=0$%separabel
\item $u'(x)\cos x=-u(x)\sin x$%separabel
\end{enumerate}}\parbox{0.5\textwidth}{\begin{enumerate}\setcounter{enumi}{6}
\item $\left(x^2+1\right) u'(x)+2 x u(x)=0$%exakt
\item $(2 x u(x)+1) u'(x)+u(x)^2=0$%exakt
\item $(u(x)- x^3)u'(x)-3 x^2 u(x) = 3 x^5 $%exakt
\item $2 x - u(x) - x u'(x) + 2 u(x)u'(x)=0$%exakt
\item $u(x)u'(x)+\sqrt{x^2+u(x)^2}+x=0$%integrierender Faktor 1/\sqrt{x^2+u^2}
\item $2 u(x) u'(x)+u(x)^2+e^{-x} = 0$%integrierender Faktor e^x
\end{enumerate}}
\begin{solution}Wir verwenden in den meisten F"allen die Standard-L"osungsmethode zur L"osung einer DGL mit getrennten Variablen f"ur $y'=f(x)g(y)$, wobei aus
\begin{align*}
\int_{y(x_0)}^{y(x)}\frac{1}{g(w)}\,\mathrm dw=\int_{x_0}^xf(t)\,\mathrm dt
\end{align*}
die eindeutige L"osung $y=\varphi(x)$ bestimmt werden kann.

\begin{enumerate}
\item $u'(x)+x^nu(x)=0$,~$n>-1$%separabel
\begin{align*}
u(x)&=u_0e^{-x^3/3}
\end{align*}
\item $u'(x)-u(x)^3=0$%separabel
\begin{align*}
u(x)&=\frac{u_0}{\sqrt{1-2xu_0^2}}
\end{align*}
\item $u'(x)+x^2u(x)^2=u(x)$%separabel
\begin{align*}
u(x)&=\frac{u_0e^x}{1-2u_0+2u_0e^x-2u_0e^x+u_0x^2e^x}
\end{align*}
\item $(1+x)u'(x)=[xu(x)]^2$%separabel
\begin{align*}
u(x)&=\frac{-2u_0}{u_0x^2+2u_0\ln(1+x)-2u_0x-2}
\end{align*}
\item $(1+x)^2u'(x)-x^2u(x)=0$%separabel
\begin{align*}
u(x)&=\frac{u_0e^{1+x-\frac{1}{1+x}}}{(1+x)^2}
\end{align*}
\item $u'(x)\cos x=-u(x)\sin x$%separabel
\begin{align*}
u(x)&=u_0\cos(x)
\end{align*}
\item $\left(x^2+1\right) u'(x)+2 x u(x)=0$%exakt
\begin{align*}
u(x)&=\frac{u_0}{1+x^2}
\end{align*}
\item $(2 x u(x)+1) u'(x)+u(x)^2=0$%exakt
\begin{align*}
u(x)&=\frac{\sqrt{1+4u_0x}-1}{2x}
\end{align*}
\item $(u(x)- x^3)u'(x)-3 x^2 u(x) = 3 x^5 $%exakt
\begin{align*}
u(x)&=x^3+\sqrt{u_0^2+2x^6}
\end{align*}
\item $2 x - u(x) - x u'(x) + 2 u(x)u'(x)=0$%exakt
\begin{align*}
u(x)&=\frac{1}{2}\left(x+\sqrt{4u_0^2-3x^2}\right)
\end{align*}
\item $u(x)u'(x)+\sqrt{x^2+u(x)^2}+x=0$%integrierender Faktor 1/\sqrt{x^2+u^2}
\begin{align*}
u(x)&=\sqrt{u_0^2-2xu_0}
\end{align*}
\item $2 u(x) u'(x)+u(x)^2+e^{-x} = 0$%integrierender Faktor e^x
\begin{align*}
u(x)&=\sqrt{e^{-x}(u_0^2-x)}
\end{align*}
\end{enumerate}
\end{solution}



\question{{\it Einige spezielle Differentialgleichungen zweiter Ordnung}}

\begin{parts}
\part Finden Sie für die folgenden Differentialgleichungen jeweils eine
Lösung in Form eines Polynoms vom Grad $n$ für $n\in\{2,3,4,5,6\}$,
dessen führender Koeffizient gleich Eins ist:
\begin{enumerate}
\item $p''(t)-2tp'(t)+2np(t)=0$
\item $tp''(t)+(1-t)p'(t)+np(t)=0$
\item $(1-t^2)p''(t)-2tp'(t)+n(n+1)p(t)=0$
\item $(1-t^2)p''(t)-3tp'(t)+n(n+2)p(t)=0$
\end{enumerate}
\begin{solution}Wir f"uhren im Folgenden nur den Fall $n=2$ vor.
\begin{enumerate}
\item $p''(t)-2tp'(t)+2np(t)=0$
\begin{align*}
p(t)&=-\frac{1}{2}+t^2
\end{align*}
\item $tp''(t)+(1-t)p'(t)+np(t)=0$
\begin{align*}
p(t)&=t^2-4t+2
\end{align*}
\item $(1-t^2)p''(t)-2tp'(t)+n(n+1)p(t)=0$
\begin{align*}
p(t)&=t^2-\frac{1}{3}
\end{align*}
\item $(1-t^2)p''(t)-3tp'(t)+n(n+2)p(t)=0$
\\
\\ Es gibt keine L"osung in Form eines Polynoms.
\end{enumerate}
\end{solution}
\part Verifizieren Sie, dass für $k\in\{0,1,2,3\}$ die Lösungen des
Anfangswertproblems
\[
-u_k''(x)+x^2u_k(x)=(2k+1)u_k(x),~~~~~u_k(0)=\frac{1+(-1)^k}{2},~~~u_k'(0)=\frac{1-(-1)^k}{2}
\]
durch
\[
u_0(x)=\rme^{-\frac{x^2}{2}},~~~
u_1(x)=x\rme^{-\frac{x^2}{2}},~~~
u_2(x)=(1-2x^2)\rme^{-\frac{x^2}{2}},~~~
u_3(x)=x(1-\frac{2}{3}x^2)\rme^{-\frac{x^2}{2}}
\]
gegeben sind.

\end{parts}
\begin{solution}Wir verifizieren hier nur die F"alle $k=0$ und $k=1$, 
die anderen F"alle ergeben sich in (komplizierterer) "ahnlicher Form.

1. $k=0$, $u_0(x)=e^{-x^2/2}$
\begin{align*}
u_0(0)&=1
\\
u_0'(x)&=-xe^{-\frac{x^2}{2}}=-xu_0(x),\quad u_0'(0)=0
\\
u_0''(x)&=-e^{-\frac{x^2}{2}}+x^2e^{-\frac{x^2}{2}}=-u_0(x)+x^2u_0(x)
\\
\Rightarrow\ -u_0''(x)+x^2u_0(x)&=u_0(x)
\end{align*}
1. $k=1$, $u_1(x)=xe^{-x^2/2}$
\begin{align*}
u_1(0)&=0
\\
u_1'(x)&=e^{-\frac{x^2}{2}}-x^2e^{-\frac{x^2}{2}},\quad u_1'(0)=1
\\
u_1''(x)&=-xe^{-\frac{x^2}{2}}-2xe^{-\frac{x^2}{2}}+x^3e^{-\frac{x^2}{2}}
\\
\Rightarrow\ -u_1''(x)+x^2u_1(x)&=3u_1(x)
\end{align*}
\end{solution}


\question{{\it Erzwungene Schwingungen}}%$^\ast$}}

\begin{parts}
%\part Vollziehen Sie die in der Vorlesung dargestellte Herleitung der
%allgemeinen Lösung der homogenenen Differentialgleichung
%\[
%f''(t)+2\gamma f'(t)+\omega_0^2 f(t)=0
%\]
%nach.

\part Verifizieren Sie, dass für $\gamma>0$ oder $\omega\not=\omega_0$
eine partikuläre Lösung der inhomogenen Differentialgleichung
\[
f''(t)+2\gamma f'(t)+\omega_0^2 f(t)=A\sin(\omega t)
\]
durch
\[
f(t) = \frac{A \left(\omega_0^2-\omega ^2\right) \sin (t \omega )-2 A \gamma
   \omega  \cos (t \omega )}{4 \gamma^2 \omega ^2+\left(\omega ^2-\omega_0^2\right)^2}
\]
gegeben ist. %Interpretieren Sie diese Lösung in Hinsicht auf die in der
%Vorlesung angegebene komplexifizierte Lösung.
%(Hinweis: Benutzen Sie die Euler-Formel!)
\begin{solution}Wir berechnen im Folgenden $f'(t)$ sowie $f''(t)$ gegeben durch
\begin{align*}
f'(t)
&=
\frac{A \omega  \left(2 \gamma  \omega  \sin (t \omega )+\left(\omega_0^2-\omega
   ^2\right) \cos (t \omega )\right)}{4 \gamma ^2 \omega ^2+
   \left(\omega ^2-\omega_0^2\right)^2}
\\
f''(t)
&=
\frac{A \omega ^2 (2 \gamma  \omega  \cos (t \omega )+
(\omega -\omega_0) 
(\omega
   +\omega_0) \sin (t \omega ))}{4 \gamma ^2 \omega ^2+
   \left(\omega ^2-\omega_0^2\right)^2}
\end{align*}
Einsetzen in die Differentialgleichung liefert dann
\begin{align*}
f''(t)+2\gamma f'(t)+\omega_0^2f(t)=A \sin (\omega t)
\end{align*}
\end{solution}

\part Verifizieren Sie ferner, dass für $\gamma=0$, $\omega=\omega_0$ eine
partikuläre Lösung der inhomogenen Differentialgleichung
\[
f''(t)+\omega_0^2 f(t)=A\sin(\omega_0 t)
\]
durch
\[
f(t)=-\frac{A t\cos(\omega_0 t)}{2\omega_0}
\]
gegeben ist. Interpretieren Sie diese Lösung im Hinblick auf das Verhalten
ungedämpft schwingender Systeme unter resonanter Anregung.
\begin{solution}Wir berechnen nun erneut $f'(t)$ und $f''(t)$ um die Beziehung zu zeigen
\begin{align*}
f'(t)&=
\frac{1}{2} A \left(t \sin (t \omega_0)-
\frac{\cos (t \omega_0)}{\omega_0}\right)
\\
f''(t)&=
A \sin (\omega_0t)+\frac{1}{2} A t \omega_0 \cos (\omega_0 t)
\end{align*}
Einsetzen in die obige DGL liefert dann
\begin{align*}
f''(t)+\omega_0^2f(t)=A\sin(\omega_0t)
\end{align*}
Somit ist gezeigt, dass $f(t)$ eine partikul"are L"osung ist.
\end{solution}
\part Verifizieren Sie zu guter Letzt, dass eine partikuläre Lösung der
inhomogenen Differentialgleichung
\[
f''(t)+\omega_0^2 f(t)=h(t)
\]
für stetiges $h$ durch
\[
f(t)=\frac{\sin(\omega_0 t)}{\omega_0}\int_0^t\cos(\omega_0\tau)h(\tau)~\rmd\tau-\frac{\cos(\omega_0 t)}{\omega_0}\int_0^t\sin(\omega_0\tau)h(\tau)~\rmd\tau
\]
gegeben ist.
\begin{solution}
Wir berechnen die Ableitungen unter Ausnutzung der Eigenschaften des bestimmten Integrals
\begin{align*}
F(x):=\int_a^xf(t)\,\mathrm dt,\quad
F'(x)=\frac{\mathrm d}{\mathrm dx}\int_a^xf(t)\,\mathrm dt=f(x)
\end{align*}
Demnach erhalten wir
\begin{align*}
f'(t)&=
\cos(\omega_0t)\int_0^t\cos(\omega_0\tau)h(\tau)\mathrm d\tau
+
\frac{\sin(\omega_0t)}{\omega_0}\cos(\omega_0t)h(t)+
\\
&\quad\, +
\sin(\omega_0t)\int_0^t\sin(\omega_0\tau)h(\tau)\mathrm d\tau
-
\frac{\cos(\omega_0t)}{\omega_0}\sin(\omega_0t)h(t)
\end{align*}
F"ur die zweite Ableitung ergibt sich
\begin{align*}
f''(t)&=
h(t)-\omega_0\sin(\omega_0t)\int_0^t\cos(\omega_0\tau)h(\tau)\mathrm d\tau
+\omega_0\cos(\omega_0t)\int_0^t\sin(\omega_0\tau)h(\tau)\mathrm d\tau
\end{align*}
wodurch man erkennt, dass $f''(t)+\omega_0^2f(t)=h(t)$ gilt.
\end{solution}
\end{parts}


\question{{\it Das Eulersche Polygonzugverfahren$^\ast$}}

Sehr oft können Differentialgleichungen nicht analytisch gelöst werden.
In diesem Fall sind numerische Näherungsverfahren von großer praktischer
Bedeutung. Ein einfaches solches Verfahren ist das Eulersche
Polygonzugverfahren, in dem die Lösung $y(t)$ des Anfangswertproblems
\[
y'(t)=f(y(t),t)~~~~~~~~y(0)=y_0
\]
durch einen Polygonzug
\[
\tilde{y}(t)=y_n+(t-nh)f(y_n,nh)~~~\textrm{für}~t\in[nh;(n+1)h]
\]
mit der rekursiv definierten Folge von Vertices
\[
y_{n+1}=y_n+hf(y_n,nh)
\]
angenähert wird.

\begin{parts}
\part Welche Folge von Vertices $y_n$ ergibt sich für die Wachstumsgleichung?
Was geschieht im Limes $h\to 0$?
\begin{solution}Die Wachstumsgleichung ist gegeben "uber $\dot{x}(t)=\lambda x(t)$, also ist
\begin{align*}
y'(t)=f(y(t),t)=\lambda y
\end{align*}
Somit ergeben sich die Vertices
\begin{align*}
y_1&=y_0+h\lambda y_0=y_0(1+h\lambda)
\\
y_2&=y_1+h\lambda y_1=y_0(1+h\lambda)+h\lambda y_0(1+h\lambda)
=
y_0(1+h\lambda)^2
\\
y_3&=y_2+h\lambda y_2=...=y_0(1+h\lambda)^3
\\
&\vdots
\\
y_n&=y_{n-1}+h\lambda y_{n-1}=y_0(1+h\lambda)^n
\end{align*}
F"ur den Limes $h\to 0$, setzen wir $h=k/n$ mit $k\in\mathbb{R}$ und betrachten dann $n\to\infty$. Also ist $t\in[k,k+k/n]$ und im Grenzwert $n\to\infty$ folgt dann $t=k$ sowie
\begin{align*}
\lim_{n\to\infty}y_n
&=
\lim_{n\to\infty}y_0\left(1+\frac{k}{n}\lambda\right)^n=y_0e^{k\lambda}
\end{align*}
Daher ist $y(k)=y_0e^{k\lambda}$ und f"ur allgemeine $t\in\mathbb{R}$ folgt die L"osung der Wachstumsgleichung
\begin{align*}
y(t)=y_0e^{\lambda t}.
\end{align*}
\end{solution}
\part Wenden Sie das Eulersche Polygonzugverfahren auf die Probleme aus
Aufgabe 1 an und vergleichen Sie die Folge der Vertices jeweils mit
der exakten Lösung.
\begin{solution}Wir betrachten als Beispiel die Aufgabe 1.6 mit
\begin{align*}
u'(x)&=f(u(x),x)=-u(x)\tan(x),\quad u(0)=u_0\\
\Rightarrow\, u(x)&=u_0\cos(x)
\\ \\
u_1&=u_0+hf(u_0,0)
\\
u_2&=u_1+hf(u_1,h)=u_0+hf(u_1,h)=u_0-hu_0\tan(h)=u_0(1-h\tan(h))
\\
u_3&=u_2+hf(u_2,2h)=u_0(1-h\tan(h))(1-h\tan(2h))
\\
\vdots
\\
u_n&=u_0\prod_{k=0}^{n-1}(1-h\tan(kh))
\end{align*}
Betrachtet man nun erneut f"ur den Limes $h\to 0$ die Folge $h_n=k/n$ 
mit $k\in\mathbb{R}$ f"ur $n\to\infty$, so ergibt sich im Grenzwert $x\in[k,k+k/n]\to[k,k]$, also $x=k$ sowie
\begin{align*}
\lim_{n\to\infty}u_n=
u_0\lim_{n\to\infty}\prod_{m=0}^{n-1}\left(1-\frac{k}{n}\tan\left(m\frac{k}{n}\right)\right)=u_0\cos(k).
\end{align*}
Somit folgt im Grenzfall $h\to 0$ die L"osung der DGL $u(x)=u_0\cos(x)$.
\end{solution}
\end{parts}

\end{questions}

\end{document}
