\question{{\it Polynomdivision}}

Führen Sie für die folgenden Paare von Polynomen jeweils die Polynomdivision durch.
\begin{enumerate}
\item $(x^3-x^2-5x-3)$, $(3-x)$
\item $(x^4+3x^3+4x^2+3x+1)$, $(x^2+x+1)$
\item $(6 x^4-12 x^3+37 x^2-48 x+45)$, $(2 x^2-4 x+4)$
\item $(x^5+4 x^4-9 x^3-40 x^2-4 x+48)$, $(x^2+4x+4)$
\item $(x^5+4 x^4-9 x^3-40 x^2-4 x+48)$, $(x^3-13 x+12)$
\item $(2 x^8+4 x^7+3 x^6-5 x^5-16 x^4-13 x^3+4 x^2-4 x+18)$, $(x^3+x-4)$
\item $(x^8-4 x^7+14 x^6-4 x^5+13 x^4+x^2-3)$, $(x^5-4 x^4+13 x^3)$
\item $(x^{10}-1)$, $(1 - x + x^2 - x^3 + x^4)$
\end{enumerate}



\question{{\it Faktorisierung von Polynomen}}

Bestimmen Sie für die folgenden Polynome jeweils alle reellen Nullstellen und überprüfen Sie, ob das Polynom über $\Rset$ in Linearfaktoren zerfällt. Falls nicht, geben Sie die verbleibenden quadratischen Faktoren an und bestimmen Sie die zugehörigen komplexen Nullstellen.\\
\parbox{0.5\textwidth}{\begin{enumerate}
\item $x^2-2x+1$
\item $x^2+2x+1$ 
\item $x^2+4$
\item $x^3+9x$
\end{enumerate}}\parbox{0.5\textwidth}{\begin{enumerate}\setcounter{enumi}{4}
\item $x^3-13 x+12$
\item $x^3-5 x^2+7 x-3$
\item $6 x^4-12 x^3+36 x^2-48 x+48$ 
\item $x^8 - 2 x^4 + 1$
\end{enumerate}}

\pagebreak


