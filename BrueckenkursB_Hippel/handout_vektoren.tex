\documentclass[a4paper,10pt]{article}
\usepackage[utf8x]{inputenc}
\usepackage[german]{babel}
\usepackage{a4wide}
\usepackage{amsmath,amssymb,amsfonts}

\input{definitions}

\title{Zusammenfassung zum Thema \\ Vektor- und Matrizenrechnung}
\author{Mathematischer Brückenkurs (A)\\für PhysikerInnen und ChemikerInnen}
\date{WS 2018/2019}

%%%%%%%%%%%%%%%%%%%%%%%%%%%%%%%%%%%%%%%%%%%%%%%%%%%%%%%%%%%%%%%%%%%%%%%%%%

\begin{document}
\parindent0pt
\maketitle

{\bf Grundbegriffe der Linearen Algebra}

Viele physikalische Größen (Geschwindigkeit, Kraft, $\ldots$) haben neben
einem Betrag auch eine räumliche Richtung
(Vektoren, von lat. {\em vector} ``Fahrer'')
und können als Pfeile dargestellt werden,
deren Länge dem Betrag der Größe entspricht.
Pfeile im Anschauungsraum können mit einem reellen Betrag skaliert, sowie
durch Aneinanderreihung zueinander addiert werden (Parallelogrammregel).
Durch Einführen von kartesischen Koordinaten identifizieren wir den
Anschauungsraum der zwei- bzw. dreidimensionalen Pfeile mit
$\Rset^2$ bzw. $\Rset^3$,
wobei das Skalieren bzw. die Addition in $\Rset^n$ komponentenweise erfolgen:
\begin{align*}
\lambda (x_1,\ldots,x_n) &= (\lambda x_1,\ldots,\lambda x_n) \\
(x_1,\ldots,x_n) + (y_1,\ldots,y_n) &= (x_1+y_1,\ldots,x_n+y_n)
\end{align*}
In einem weiteren Abstraktionsschritt definieren wir einen
einen rellen bzw. komplexen Vektorraum als
eine mit zwei Verknüpfungen $+:V\times V\to V$ und
$\cdot:{\mathbb{K}}\times V\to V$ (${\mathbb{K}}=\Rset$ bzw. $\Cset$)
ausgestattete Menge $V$, wenn die
Verknüpfungen den folgenden Axiomen genügen:
\begin{align*}
&\forall a,b,c\in V~(a+b)+c = a+(b+c)
&&\exists 0\in V~\forall a\in V~ a+0=a \\
&\forall a\in V~\exists (-a)\in V~ a+(-a)=0
&&\forall a,b\in V~ a+b=b+a\\
&\forall \lambda,\mu\in\mathbb{K}~\forall a\in V~ \lambda(\mu a)=(\lambda\mu)a
&&\forall \lambda,\mu\in\mathbb{K}~\forall a\in V~ (\lambda+\mu)a=\lambda a+\mu a \\
&\forall \lambda\in\mathbb{K}~\forall a,b\in V~ \lambda(a+b)=\lambda a+\lambda b
&&\forall a\in V~ 1a=a
\end{align*}
Aus einer endlichen Menge $\{\vec{v}_1,\ldots,\vec{v}_n\}$ von
Vektoren $\vec{v}_k\in V$
können weitere Vektoren durch Linearkombinationen
\[
\sum_{k=1}^n \lambda_k \vec{v}_k \in V
\]
erzeugt werden.
Die Frage, wann ein Vektor als Linearkombination anderer Vektoren
dargestellt werden kann, führt auf die Definition: Eine Menge $W\subseteq V$
von Vektoren heißt linear unabhängig, wenn aus
\[
\sum_{k=1}^n \lambda_k \vec{v}_k = 0
\]
mit $\vec{v}_k\in W$ immer $\forall k~\lambda_k=0$ folgt.

Eine linear unabhängige Menge $B\subset V$, zu der kein weiterer Vektor
hinzugefügt werden kann, ohne die lineare Unabhängigkeit zu zerstören,
nennen wir eine Basis von $V$.
Im $\Rset^n$ haben wir als besonders wichtige Basis die kanonische
Basis $\{\vec{e}_1,\ldots,\vec{e}_n\}$ mit
\[
\vec{e}_1=(1,0,\ldots,0),~~
\vec{e}_2=(0,1,\ldots,0),~
\vdots~~\vec{e}_n=(0,0,\ldots,1)
\]
bzw. kompakter $(\vec{e}_i)_j=\delta_{ij}$
(wobei das Kronecker-Symbol durch $\delta_{ii}=1$ und $\delta_{ij}=0$,
$j\not=i$ definiert ist).\\

{\bf Skalarprodukte}

Aus der Physik wissen wir, dass die verrichtete Arbeit das Produkt aus
dem zurückgelegten Weg und der längs dieses Weges wirkenden Komponente
der Kraft ist.
Definiert man dieses Produkt als Produkt der zugehörigen Vektoren, so
erhält man das skalare Produkt
\[
\vec{a}\cdot \vec{b}=|\vec{a}||\vec{b}|\cos\vartheta
\]
mit $|\vec{x}|$ dem Betrag des Vektors $\vec{x}$
und $\vartheta$ dem Winkel zwischen $\vec{a}$ und $\vec{b}$.
Man überprüft leicht, dass dieses Produkt die Eigenschaften
\begin{align*}
\vec{a}\cdot \vec{b} &= \vec{b}\cdot \vec{a} &
\vec{a}\cdot \vec{a} &> 0,~~~\vec{a}\not=0 \\
\vec{a}\cdot (\vec{b}+\vec{c}) &= \vec{a}\cdot \vec{b} + \vec{a}\cdot \vec{c} &
\vec{a}\cdot (\lambda \vec{b}) &= \lambda (\vec{a}\cdot \vec{b})
\end{align*}
besitzt, die wir zur Definition eines Skalarprodukts erheben.
Im $\Rset^n$ ist ein Skalarprodukt durch
\[
\vec{a}\cdot \vec{b} = \sum_{i=1}^n a_i b_i
\]
definiert (und man überzeugt sich leicht, dass dieses für $n=2$ und $n=3$
mit der anschaulichen Idee übereinstimmt).
Es gilt entsprechend
\[
|\vec{a}|=\sqrt{\vec{a}\cdot \vec{a}}
\]
und
\[
\cos\vartheta = \frac{\vec{a}\cdot \vec{b}}{|\vec{a}||\vec{b}|}
\]
und diese Gleichungen können als Definitionen auf beliebige Vektorräume,
auf denen ein Skalarprodukt definiert ist, erweitert werden.\\

{\bf Vektorprodukt im $\Rset^3$}

Speziell im $\Rset^3$ lässt sich ein weiteres Produkt zweier Vektoren
$\vec{a},\vec{b}\in \Rset^3$ definieren: Es gibt genau eine Richtung, die
senkrecht auf der von $\vec{a}$ und $\vec{b}$ aufgespannten Ebene steht
und mit $\vec{a}$ und $\vec{b}$ ein rechtshändiges System bildet. Ein
Vektor $\vec{a}\times\vec{b}$ mit dieser Richtung und dem Betrag
\[
|\vec{a}\times\vec{b}|=|\vec{a}||\vec{b}|\sin\vartheta
\]
(entspricht dem Flächeninhalt des von $\vec{a}$ und $\vec{b}$ aufgespannten
Parallelograms) definiert das Vektorprodukt $\vec{a}\times\vec{b}\in\Rset^3$.
Man überprüft leicht, dass dieses Produkt die Eigenschaften
\begin{align*}
\vec{a}\times \vec{b} &= -\vec{b}\times \vec{a} &
\vec{a}\cdot(\vec{a}\times\vec{b})&=\vec{b}\cdot(\vec{a}\times\vec{b})=0\\
\vec{a}\times (\vec{b}+\vec{c}) &= \vec{a}\times \vec{b} + \vec{a}\times \vec{c} &
\vec{a}\times (\lambda \vec{b}) &= \lambda (\vec{a}\times \vec{b})
\end{align*}
besitzt.
In Komponenten gilt
\[
\vec{a}\times \vec{b} = (a_2b_3-a_3b_2,a_3b_1-a_1b_3,a_1b_2-a_2b_1)
\]
oder kompakt
\[
[\vec{a}\times \vec{b}]_i = \sum_{j,k=1}^3 \epsilon_{ijk}a_jb_k
\]
mit dem Levi-Civita-Symbol
\[
\epsilon_{ijk} = \left\{\begin{array}{rl}1,&ijk\textrm{~zyklisch~(123,~231,~312),}\\-1,&ijk\textrm{~antizyklisch~(321,~132,~213),}\\0,&\textrm{sonst.}\end{array}\right.
\]

\pagebreak

{\bf Analytische Geometrie des Raumes}

\begin{itemize}
\item Gerade in Parameterform:
\[
\vec{x}=\vec{p}+\lambda\vec{u}
\]
mit Stützpunkt $\vec{p}$ und Richtungsvektor $\vec{u}$.

\item Ebene in Parameterform:
\[
\vec{x}=\vec{p}+\lambda_1\vec{u}_1+\lambda_2\vec{u}_2
\]
mit Stützpunkt $\vec{p}$ und linear unabhängigen Richtungsvektoren
$\vec{u}_1,\vec{u}_2$.

\item Ebene in Normalenform:
\[
\vec{n}\cdot\vec{x}=c
\]
mit Normalenvektor $\vec{n}\propto \vec{u}_1\times\vec{u}_2$ und
Abstandskonstante $c=\vec{n}\cdot\vec{p}$.
\end{itemize}

{\bf Lineare Gleichungssysteme}

Die Suche nach den Schnittmengen von Geraden und Ebenen untereinander
führt (wie bereits zuvor die Partialbruchzerlegung) auf lineare
Gleichungssysteme der Form
\[
\begin{array}{ccccccc}
a_{11}x_1&+&\ldots&+&a_{1n}x_n &=& b_1\\
\vdots &&\ddots&&\vdots &=&\vdots \\
a_{m1}x_1&+&\ldots&+&a_{mn}x_n&=&b_m
\end{array}
\]
Die geometrische Anschauung lehrt, dass diese Systeme entweder keine,
genau eine oder unendlich viele Lösungen haben.
Eine kurze Überlegung zeigt, dass die Reihenfolge der
Gleichungen keine
Rolle spielt, und dass wir jede Gleichung durch beidseitige Addition
einer anderen Gleichung umformen können, ohne hierdurch die Lösungsmenge
des Systems zu verändern.

Wir können daher lineare Gleichungssysteme mit dem Gaussschen
Eliminationsverfahren lösen:\\
Für jedes $j\in\{1,\ldots,m\}$
\begin{itemize}
\item stelle, falls nötig, durch Vertauschen von Gleichungen
$a_{jj}\not=0$ sicher,
\item eliminiere für jedes $i\in\{j+1,\ldots,m\}$ den Term mit
Koeffizienten $a_{ij}$ durch beidseitige Subtraktion des
$a_{ij}/a_{jj}$-fachen der $j$-ten Gleichung von der $i$-ten Gleichung.
\end{itemize}
Wenn hierbei
\begin{itemize}
\item eine Gleichung der Form $0=c$ mit $c\not=0$ auftritt,
hat das System keine Lösung,
\item am Ende weniger nichttriviale Gleichungen als Variablen
auftreten, hat das System unendlich viele Lösungen, die durch die
redundanten Variablen parametrisiert werden können.
\end{itemize}
Ansonsten ist das System durch Rückwärtseinsetzen ausgehend von der
letzten Gleichung (die nun auf die Form $ax_n=b$ reduziert ist)
eindeutig lösbar.\\

{\bf Lineare Abbildungen und Matrizen}

Seien $V$ und $W$ Vektorräume. Eine Funktion $f:V\to W$ heißt lineare
Abbildung, falls für alle $\vec{x},\vec{y}\in V$, $\lambda,\mu\in\mathbb{K}$
\[
f(\lambda \vec{x}+\mu \vec{y})=\lambda f(\vec{x})+\mu f(\vec{y})
\]
gilt.
Eine lineare Abbildung ist wegen
\[
f\left(\sum_k \lambda_k \vec{v}_k\right)=\sum_k \lambda_k f(\vec{v}_k)
\]
durch die Bilder der Vektoren einer Basis von $V$ eindeutig bestimmt.

Für lineare Abbildungen $f:\mathbb{K}^n\to\mathbb{K}^m$ können wir daher
$f$ mit der $m\times n$ Matrix
\[
A = \left(\begin{array}{ccc}a_{11}&\cdots&a_{1n}\\\vdots&\ddots&\vdots\\a_{m1}&\cdots&a_{mn}\end{array}\right)
\]
mit $f(\vec{e}_j)=\sum_{i=1}^m a_{ij}\vec{e}_i$ identifizieren.

Seien $f,g:V\to W$ lineare Abbildungen. Dann ist wegen
\[
\eta g(\lambda \vec{a}+\mu \vec{b}) + \xi f(\lambda \vec{a}+\mu \vec{b})
=\lambda (\eta g(\vec{a})+\xi f(\vec{a}))+\mu (\eta g(\vec{b})+\xi f(\vec{b}))
\]
auch ihre Linearkombination $(\eta g+\xi f):V\to W$ eine lineare Abbildung.
Für $f,g:\mathbb{K}^n\to \mathbb{K}^m$
mit zugehörigen Matrizen $A$ und $B$ bildet $(\eta g+\xi f):\mathbb{K}^n\to \mathbb{K}^m$
den Basisvektor $e_j\in \mathbb{K}^n$ auf
\[
(\eta g+\xi f)(e_j) = \sum_{i=1}^m \eta b_{ij} e_i + \sum_{i=1}^m \xi a_{ij} e_i
= \sum_{i=1}^m (\eta b_{ij}+\xi a_{ij})\vec{e}_i
\]
ab.
Entsprechend definieren wir Summe und skalare Multiplikation
von Matrizen elementweise als
\[
(A+B)_{ij} = a_{ij}+b_{ij}~~~~~~~(\lambda A)_{ij} = \lambda a_{ij}
\]
Seien $f:U\to V$, $g:V\to W$ lineare Abbildungen. Dann ist wegen
\[
g(f(\lambda \vec{a}+\mu\vec{b})) = g(\lambda f(\vec{a})+\mu f(\vec{b}))
=\lambda g(f(\vec{a}))+\mu g(f(\vec{b}))
\]
auch die Verkettung $g\circ f:U\to W$ eine lineare Abbildung.
Für $f:\mathbb{K}^n\to \mathbb{K}^r$ und $g:\mathbb{K}^r\to \mathbb{K}^m$
mit zugehörigen Matrizen $A$ und $B$ bildet
$g\circ f:\mathbb{K}^n\to \mathbb{K}^m$ den Basisvektor
$\vec{e}_j\in \mathbb{K}^n$ auf
\[
(g\circ f)(\vec{e}_j) = g(f(\vec{e}_j)) = \sum_{i=1}^r a_{ij}g(\vec{e}_i)
= \sum_{k=1}^m\sum_{i=1}^r a_{ij}b_{ki}\vec{e}_k
\]
ab.
Entsprechend definieren wir das Matrixprodukt als
\[
(BA)_{kj} = \sum_{i=1}^r b_{ki}a_{ij}
\]

Mit diesen Definitionen gelten die Rechengesetze
\begin{align*}
A+(B+C) &= (A+B)+C &
A+B &= B+A\\
(A+B)C &= AC+BC &
A(B+C) &= AB+AC\\
A(BC) &= (AB)C
\end{align*}
Die Summe $A+B$ ist nur definiert, wenn $A$ und $B$ dieselben Dimensionen haben.
Das Produkt $AB$ ist nur definiert, wenn $A$ soviele Spalten wie $B$ Zeilen hat.
Wenn zwei Matrizen $A$ und $B$ in beide Richtungen miteinander multipliziert werden können, so gilt im Allgemeinen $AB\not=BA$.

Unter der linearen Abbildung $f:\mathbb{K}^n\to \mathbb{K}^m$
mit zugehöriger Matrix $A$ wird $\vec{x}\in \mathbb{K}^n$ nach
\[
f(\vec{x}) = f\left(\sum_{j=1}^n x_j \vec{e}_j\right) = \sum_{j=1}^n x_j f(\vec{e}_j)
= \sum_{i=1}^n\sum_{j=1}^n a_{ij}x_j \vec{e}_i
\]
abgebildet.
Die Frage nach dem Urbild $\vec{x}\in \mathbb{K}^n$ eines Vektors
$\vec{b}\in \mathbb{K}^m$ unter $f$
entspricht also dem linearen Gleichungssystem mit Koeffizienten $a_{ij}$
und rechter Seite $b_i$, das wir als
\vskip-1ex\[
Ax=b
\]\enlargethispage{3ex}\nopagebreak
schreiben können, wenn wir Vektoren im $\mathbb{K}^r$ als einspaltige Matrizen
(Spaltenvektoren) auffassen.\\

\end{document}

{\bf Determinanten}

Für $m=n$ ist das lineare Gleichungssystem
\[
Ax=b
\]
genau dann für beliebiges $\vec{b}\in\mathbb{K}^n$ eindeutig lösbar,
wenn die zu $A$ gehörige lineare Abbildung bijektiv ist.
Dies ist genau der Fall, wenn das Bild der kanonischen Basis
$\{\vec{e}_i\}$
wieder eine Basis des $\mathbb{K}^n$ bildet.
Letzteres wird durch die $n$ Spalten von $A$ gegeben. Diese müssen also
linear unabhängig sein.
Das ist genau dann der Fall, wenn sie zusammen ein nichtverschwindendes
$n$-dimensionales Volumen aufspannen.
Das von den Spalten der $n\times n$ Matrix $A$ aufgespannte
$n$-dimensionale
(orientierte) Volumen bezeichnen wir als die Determinante $\det A$ von
$A$ (oft schreibt man auch $|A|$ hierfür).

Die Determinante hat folgende Eigenschaften:
\begin{itemize}
\item Vertauschung von zwei Zeilen oder von zwei Spalten führt zu $\det A\mapsto -\det A$,
\item Addition von Zeilen oder Spalten lässt $\det A$ unverändert,
\item lineare Abhängigkeit der Spalten oder Zeilen impliziert $\det A=0$,
\item es gilt $\det (AB)=\det A\cdot \det B$,
\item für $\det(A)\not=0$ ist $Ax=b$ für beliebiges $b\in\mathbb{K}^n$ lösbar.
\end{itemize}

Für $n=2$ ist die Determinante einfach
\[
\det A = a_{11}a_{22}-a_{12}a_{21}
\]
Für $n=3$ gilt die Regel von Sarrus:
\begin{align*}
\det A &= a_{11}a_{22}a_{33}+a_{12}a_{23}a_{31}+a_{13}a_{21}a_{32}\\
       &~~ -a_{13}a_{22}a_{31}-a_{11}a_{23}a_{32}-a_{12}a_{21}a_{33}
\end{align*}
Für $n\ge 4$ benötigt die Berechnung einige allgemeinere Ideen
(Permutationsgruppe), die Sie in den Mathematikvorlesungen kennenlernen
werden.\\

{\bf Eigenwerte und Eigenvektoren}

Eine lineare Abbildung $\mathbb{K}^n\to\mathbb{K}^n$ lässt die
Richtung eines Vektors $\vec{v}\in\mathbb{K}^n\backslash\{0\}$ unverändert, wenn
für die zugehörige $n\times n$ Matrix $A$
\[
Av=\lambda v
\]
oder äquivalent
\[
(A-\lambda\openone)v=0
\]
mit $\lambda\in\mathbb{K}$ gilt, wobei $\openone$ die Einheitsmatrix mit
Einsen auf der Diagonalen und Nullen überall sonst bedeutet.

In diesem Fall heißt $\lambda$ Eigenwert von $A$ und $\vec{v}$ Eigenvektor
zum Eigenwert $\lambda$ von $A$.

Offensichtlich muss
\[
\det(A-\lambda\openone)=0
\]
gelten, d.h. die Nullstellen $\lambda_i$ des charakteristischen Polynoms
\[
\chi_A(\lambda)=\det(A-\lambda\openone)
\]
von $A$ sind die Eigenwerte von $A$.

Es gilt: $\det(A) = \prod_{i=1}^n \lambda_i$, d.h. die Determinante von $A$ ist
das Produkt der (ggfs. komplexen) Eigenwerte von $A$.

\end{document}

