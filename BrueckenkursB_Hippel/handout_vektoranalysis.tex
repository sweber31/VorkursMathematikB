\documentclass[a4paper,10pt]{article}
\usepackage[utf8x]{inputenc}
\usepackage[german]{babel}
\usepackage{a4wide}
\usepackage{amsmath,amssymb,amsfonts}

\input{definitions}

\title{Zusammenfassung zum Thema \\ Vektoranalysis}
\author{Mathematischer Brückenkurs (A)\\für PhysikerInnen und ChemikerInnen}
\date{WS 2018/2019}

%%%%%%%%%%%%%%%%%%%%%%%%%%%%%%%%%%%%%%%%%%%%%%%%%%%%%%%%%%%%%%%%%%%%%%%%%%

\begin{document}
\parindent0pt
\maketitle


{\bf Vektorwertige Funktionen einer reellen Variablen}

Wir verallgemeinern die Parameterform der Geraden, um allgemeine
Bewegungen von Teilchen darstellen zu können: $x:\Rset\to\Rset^n$,
$t\mapsto \vec{x}(t)$. Die Ableitung erfolgt komponentenweise.
Physikalisch gesehen ist $\dot{\vec{x}}(t) =\frac{\rmd \vec{x}}{\rmd t}$
die Geschwindigkeit, und $\ddot{\vec{x}}(t) =\frac{\rmd^2 \vec{x}}{\rmd t^2}$
die Beschleunigung, wenn wir $t$ als Zeit interpretieren.
Geometrisch gesehen ist $\dot{\vec{x}}(t)$ ein Tangentialvektor zur
Kurve $\{\vec{x}(t)|t\in\Rset\}$ am Punkt $\vec{x}(t)$.\\


{\bf Funktionen mehrerer reeller Variablen}

In einem weiteren Verallgemeinerungschritt betrachten wir im Folgenden
allgemeine Funktionen $f:\Rset^n\to\Rset^m$, z.B. 
parametrisierte Flächen $x:\Rset^2\to\Rset^3$, oder
das Vektorfeld $E:\Rset^3\to\Rset^3$, das jedem Punkt im Raum die
an diesem Punkt herrschende elektrische Feldstärke zuweist.

Für eine Funktion $f:\Rset^n\to\Rset$ ist die
partielle Ableitung $\frac{\partial f}{\partial x_i}$ nach $x_i$
dadurch definiert, dass nach $x_i$ differenziert wird, während alle
anderen Komponenten von $x$ konstant gehalten werden, d.h. z.B.
\[
\frac{\partial f}{\partial x_1} = \lim\limits_{h\to 0} \frac{f(x_1+h,x_2,\ldots,x_n)-f(x_1,\ldots,x_n)}{h}\,.
\]
Höhere partielle Ableitungen können sowohl nach derselben als auch nach verschiedenen Variablen genommen werden, z.B.
\[
\frac{\partial^2 f}{\partial x_1^2} = \frac{\partial}{\partial x_1}\left(\frac{\partial f}{\partial x_1}\right)\,,~~~~~
\frac{\partial^2 f}{\partial x_1\partial x_2} = \frac{\partial}{\partial x_1}\left(\frac{\partial f}{\partial x_2}\right)\,.
\]
Es gilt
\[
\frac{\partial^2 f}{\partial x_i\partial x_j} = \frac{\partial^2 f}{\partial x_j\partial x_i}\,,
\]
wenn beide gemischten Ableitungen existieren und stetig sind.

Für die Verkettung $(f\circ x):\Rset\to\Rset$ von Funktionen
$x:\Rset\to\Rset^n$, $f:\Rset^n\to\Rset$ gilt die verallgemeinerte
Kettenregel
\[
\frac{\rmd f}{\rmd t}=\sum_{i=1}^n \frac{\partial f}{\partial x_i}\frac{\rmd x_i}{\rmd t}\,.
\]
Das totale Differential
\[
\rmd f = \sum_{i=1}^n \frac{\partial f}{\partial x_i} \rmd x_i
\]
gibt an, wie sich $f$ verändert, wenn sich die $x_i$ jeweils um einen
infinitesimalen Betrag $\rmd x_i$ verändern; für endliche Änderungen
gilt die Näherung
\[
\Delta f \approx \sum_{i=1}^n \frac{\partial f}{\partial x_i} \Delta x_i\,.
\]

Für eine Funktion $f:\Rset^n\to\Rset^m$ gilt komponentenweise
\[
\rmd f_i = \sum_{j=1}^n \frac{\partial f_i}{\partial x_j} \rmd x_j
\]
oder
in Matrixschreibweise
\[
\frac{\rmd f}{\rmd t}=J_f\frac{\rmd x}{\rmd t}
\]
mit der Jacobi-Matrix
\[
J_f=\left(\begin{array}{ccc}\frac{\partial f_1}{\partial x_1}&\cdots&\frac{\partial f_1}{\partial x_n}\\\vdots&\ddots&\vdots\\\frac{\partial f_m}{\partial x_1}&\cdots&\frac{\partial f_m}{\partial x_n}\end{array}\right)\,.
\]
Für $m=n$ gibt die Determinante der Jacobi-Matrix
(die Jacobi-Determinante)
an, wie sich ein infinitesimales Volumen unter der Abbildung $f$ jeweils
lokal verändert.
\\

{\bf Gradient, Divergenz und Rotation}

Für eine Funktion $f:\Rset^n\to\Rset$ definieren wir den Gradienten von $f$
als den Vektor
\[
{\mathrm{grad}}~f=\left(\begin{array}{c}\frac{\partial f}{\partial x_1}\\\vdots\\\frac{\partial f}{\partial x_n}\end{array}\right)
\]
der in die Richtung der maximalen Zunahme von $f$
unter Variation der $x_i$ zeigt.

Der Gradient steht senkrecht auf den Flächen $f={\mathrm{const.}}$.
Damit eine differenzierbare Funktion $f$ an einem Punkt $x\in\Rset^n$
ein Extremum hat, ist
\[
{\mathrm{grad}}~f(x)=0
\]
eine notwendige (aber nicht hinreichende) Bedingung.

Der Gradient lässt sich mit Hilfe des Nabla-Operators
\[
\nabla = \left(\begin{array}{c}\frac{\partial}{\partial x_1}\\\vdots\\\frac{\partial}{\partial x_n}\end{array}\right)
\]
als
\[
{\mathrm{grad}}~f=\nabla f
\]
schreiben.
Wir können den Nabla-Operator wie einen Vektor behandeln, wenn
wir folgende Rechenregeln beachten:
\begin{align*}
\nabla(f+g) &= \nabla f+\nabla g \\
\nabla(fg)  &= (\nabla f)g+f(\nabla g)
\end{align*}
Das totale Differential kann als
\[
\rmd f = \nabla f\cdot \rmd\vec{x}
\]
geschrieben werden.\\

Für eine Funktion $\vec{f}:\Rset^3\to\Rset^3$ können wir mit Hilfe des
Nabla-Operators die Divergenz
\[
{\mathrm{div}}~\vec{f}=\nabla\cdot\vec{f}
\]
und die Rotation
\[
{\mathrm{rot}}~\vec{f}=\nabla\times\vec{f}
\]
definieren.
Physikalisch gesehen ist die Divergenz ein Skalarfeld, das die Quellstärke
des Vektorfeldes $\vec{f}$ misst, und die Rotation ein Vektorfeld, das die
Wirbelstärke des Vektorfeldes $\vec{f}$ misst.

Es gelten die wichtigen Beziehungen
\begin{align*}
\nabla\cdot(\nabla\times\vec{f})&=0\\
\nabla\times(\nabla f)&=0
\end{align*}
d.h. ein Gradient ist wirbelfrei und ein Wirbelfeld ist quellenfrei.\\

{\bf Höherdimensionale Integrale}

Wir definieren das Kurvenintegral eines Vektorfeldes
$\vec{f}:\Rset^n\to\Rset^n$ entlang der Kurve\\
$\mathcal{C}=\{\vec{x}(t)|t\in[a;b]\wedge \vec{x}(a)=\vec{x}_a\wedge \vec{x}(b)=\vec{x}_b\}$ durch
\[
\int_\mathcal{C} \vec{f}\cdot\rmd\vec{x}
= \lim\limits_{n\to\infty}\sum_{i=0}^{n-1} \vec{f}(\vec{x}_i)\cdot(\vec{x}_{i+1}-\vec{x}_i)
= \lim\limits_{h\to 0}\sum_{i=0}^{n-1} \vec{f}(\vec{x}(a+ih))\cdot\frac{\rmd\vec{x}}{\rmd t}h
= \int_a^b \vec{f}(\vec{x}(t))\cdot\dot{\vec{x}}(t)~\rmd t
\]
Das Kurvenintegral bei umgekehrtem Durchlaufen der Kurve hat den gleichen Betrag und das entgegengesetzte Vorzeichen.
Bei Aneinandersetzen zweier Kurven addieren sich die Kurvenintegrale.

Wenn für ein Vektorfeld $\vec{f}:\Rset^n\to\Rset^n$ ein Skalarfeld
$\phi:\Rset^n\to\Rset$ existiert, so dass $\vec{f}=\nabla\phi$ gilt,
heißt $\vec{f}$ ein Gradientenfeld, und es gilt
\[
\int_\mathcal{C} \vec{f}\cdot\rmd\vec{x}
= \int_a^b \vec{f}(\vec{x}(t))\cdot\dot{\vec{x}}(t)\rmd t 
= \int_a^b \nabla\phi(\vec{x}(t))\cdot\dot{\vec{x}}(t)\rmd t
= \int_a^b \frac{\rmd}{\rmd t}\phi(\vec{x}(t)) \rmd t
= \phi(\vec{x}_b)-\phi(\vec{x}_a)
\]
so dass das Kurvenintegral nur von den Endpunkten von $\mathcal{C}$,
jedoch nicht vom gewählten Pfad abhängt.
Insbesondere verschwindet für ein Gradientenfeld das Kurvenintegral
über eine geschlossene Kurve mit $\vec{x}_a=\vec{x}_b$, da
\[
\oint_\mathcal{C} \vec{f}\cdot\rmd\vec{x} = \phi(\vec{x}_a)-\phi(\vec{x}_a)
 = 0
\]
Im $\Rset^3$ gilt: $\vec{f}$ ist ein Gradientenfeld g.d.w.
$\nabla\times\vec{f}=0$.\\

Wir definieren das Integral einer Funktion $g:\Rset^2\to\Rset$ über die
Fläche\\ $\mathcal{F}=\{(x,y)|x\in[a;b]\wedge y\in[y_1(x);y_2(x)]\}$ über
\[
\iint\limits_{\mathcal{F}} g~\rmd F = \int_a^b\int_{y_1(x)}^{y_2(x)}g(x,y)~\rmd y\rmd x
= \int_a^b\left(G(x,y_2(x))-G(x,y_1(x))\right)~\rmd x
\]
mit $G$ einer beliebigen Stammfunktion von $g$ bezüglich $y$ für alle $x$,
\[
\frac{\partial G}{\partial y}(x,y)=g(x,y)\,.
\]
Entsprechend definieren wir das Volumenintegral einer Funktion
$g:\Rset^3\to\Rset$ über das Volumen
$\mathcal{V}=\{(x,y,z)|x\in[a;b]\wedge y\in[y_1(x);y_2(x)]\wedge z\in[z_1(x,y);z_2(x,y)]\}$ als
\[
\iiint\limits_{\mathcal{V}} g~\rmd V = \int_a^b\int_{y_1(x)}^{y_2(x)}\int_{z_1(y,x)}^{z_2(y,x)}g(x,y,z)~\rmd z\rmd y\rmd x\,.
\]
${}$\\
Wir definieren das Oberflächen\-integral eines Vektorfeldes
$\vec{g}:\Rset^3\to\Rset^3$ über eine Oberfläche
$\mathcal{F}=\{\vec{x}(u,v)|u\in[a;b]\wedge v\in[\alpha;\beta]\}$
durch
\[
\iint\limits_\mathcal{F} \vec{g}\cdot\rmd\vec{F}
= \lim\limits_{m\to\infty}\lim\limits_{n\to\infty}\sum_{i=0}^m\sum_{j=0}^n\vec{g}\cdot\left(\frac{\partial \vec{x}}{\partial u}\times\frac{\partial \vec{x}}{\partial v}\right)_{u=\frac{i}{m}\atop v=\frac{j}{n}}\frac{(b-a)(\beta-\alpha)}{mn}
= \int_\alpha^\beta\int_a^b \vec{g}\cdot\left(\frac{\partial \vec{x}}{\partial u}\times\frac{\partial \vec{x}}{\partial v}\right)\rmd u\rmd v
\]
Das Flächenintegral über die entgegengesetzt orientierte Fläche hat den gleichen Betrag und das entgegengesetzte Vorzeichen.
Beim Aneinandersetzen von Flächen addieren sich die Flächenintegrale.
Bei Flächenintegralen über geschlossene Oberflächen ist es Konvention, die nach außen weisende Normale zu wählen.\\

Es gilt der Satz von Gauss: Das Oberflächenintegral eines Vektorfeldes über die Oberfläche eines Volumens
ist gleich dem Volumenintegral der Divergenz,
\[
\iiint\limits_{\mathcal{V}}\nabla\cdot\vec{f}\rmd V = \iint\limits_{\partial\mathcal{V}}\vec{f}\cdot\rmd\vec{F}\,.
\]
Ferner gilt der Satz von Satz von Stokes:
Das Kurvenintegral eines Vektorfeldes über den Rand einer Fläche
ist gleich dem Oberflächenintegral der Rotation,
\[
\iint\limits_{\mathcal{F}}(\nabla\times\vec{f})\cdot\rmd\vec{F} = \oint_{\partial\mathcal{V}}\vec{f}\cdot\rmd\vec{x}\,.
\]
${}$\\

{\bf Krummlinige Koordinaten}

Wir führen krummlinige Koordinaten auf dem $\Rset^2$ ein, indem wir
ihn als die Ebene $\{(x,y,0)|(x,y)\in\Rset^2\}$ in den $\Rset^3$
einbetten und durch $(u,v)\mapsto (x,y)$ parametrisieren.
Dann ist das Flächenelement
\[
\rmd \vec{F} = \left(\frac{\partial\vec{x}}{\partial u}\times\frac{\partial\vec{x}}{\partial v}\right)\rmd u\rmd v = \vec{e}_z \det(J)~\rmd u\rmd v
\]
mit $J$ der Jacobi-Matrix der Funktion $(u,v)\mapsto(x,y)$,
und wir können allgemein
\[
\rmd F = \rmd x\rmd y = |\det(J)|\rmd u\rmd v
\]
identifizieren.

Beispiel: Ebene Polarkoordinaten $(r,\varphi)$
\[
\begin{array}{rl}x &= r\cos\varphi\\y &= r\sin\varphi\end{array}~~~~~~~
J = \left(\begin{array}{cc}\cos\varphi&-r\sin\varphi\\\sin\varphi&r\cos\varphi\end{array}\right)~~~~~~~
\begin{array}{rl}
\det(J) &= r\cos^2\varphi+r\sin^2\varphi=r\\
        &\leadsto \rmd F = r\rmd r\rmd\varphi
\end{array}
\]

In gleicher Weise können wir auf dem $\Rset^3$ krummlinige Koordinaten
einführen, indem wir ihn durch $(u,v,w)\mapsto (x,y,z)$ parametrisieren.
Wir erhalten das Volumenelement
\[
\rmd V = |\det(J)|\rmd u\rmd v\rmd w
\]
mit $J$ der Jacobi-Matrix der Funktion $(u,v,w)\mapsto (x,y,z)$.

Beispiel:
Zylinderkoordinaten $(\varrho,\varphi,z)$
\[\begin{array}{rl}
x &= \varrho\cos\varphi\\
y &= \varrho\sin\varphi\\
z &= z
\end{array}~~~~~~~
J=\left(\begin{array}{ccc}
\cos\varphi&-\varrho\sin\varphi&0\\
\sin\varphi&\varrho\cos\varphi&0\\
0&0&1
\end{array}\right)~~~~~~~
\begin{array}{rl}
\det(J) &= \varrho\cos^2\varphi+\varrho\sin^2\varphi=\varrho\\
        &\leadsto \rmd V = \varrho\rmd \varrho\rmd\varphi\rmd z
\end{array}
\]

Beispiel: Kugelkoordinaten $(r,\vartheta,\varphi)$
\[\begin{array}{rl}
x &= r\sin\vartheta\cos\varphi\\
y &= r\sin\vartheta\sin\varphi\\
z &= r\cos\vartheta
\end{array}~~~~~
J=\left(\begin{array}{ccc}
\sin\vartheta\cos\varphi&r\cos\vartheta\cos\varphi&-r\sin\vartheta\sin\varphi\\
\sin\vartheta\sin\varphi&r\cos\vartheta\sin\varphi&r\sin\vartheta\cos\varphi\\
\cos\vartheta&-r\sin\vartheta&0
\end{array}\right)
\]\[
\begin{array}{rl}
\det(J) &= (\cos^2\varphi+\sin^2\varphi)(\cos^2\vartheta+\sin^2\vartheta)r^2\sin\vartheta=r^2\sin\vartheta\\
        &\leadsto \rmd V = r^2\sin\vartheta~\rmd r\rmd\vartheta\rmd\varphi
\end{array}
\]

\end{document}

