\documentclass[german]{beamer}

\mode<presentation>
{
 \usetheme{Madrid}

% \usecolortheme{crane}
 \usecolortheme{wolverine}
}

\usepackage{hyperref}

\usepackage[german]{babel}
\usepackage{times}
\usepackage[latin1,utf8]{inputenc}
\usepackage[OT2,T1]{fontenc}
\usepackage{shuffle}
\usepackage{axodraw}


% Stefan's abbreveations
\newcommand{\bq}{\begin{eqnarray*}}
\newcommand{\eq}{\end{eqnarray*}}
\newcommand{\eps}{\varepsilon}

% dedicated environments
\newtheorem*{myemptytheorem}{}
\newtheorem*{mytheorem6}{Die komplexen Zahlen $\mathbb C$ sind die Menge}
\newtheorem*{mytheorem6a}{Der K\"orper $\mathbb{Q}[\sqrt{3}]$ ist die Menge}
\newtheorem*{mytheorem7}{Definition der Addition und der Multiplikation:}
\newtheorem*{mytheorem7a}{Definition der Subtraktion und Division:}


%%%%%%%%%%%%%%%%%%%%%%%%%%%%%%%%%%%%%%%%%%%%%%%%%%%%%%%%%%%%%%%%%%%%%%%%%%%%%%%%%%%%%%%%%%%%%%%%%%%%%%%%%
%%%%%%%%%%%%%%%%%%%%%%%%%%%%%%%%%%%%%%%%%%%%%%%%%%%%%%%%%%%%%%%%%%%%%%%%%%%%%%%%%%%%%%%%%%%%%%%%%%%%%%%%%
%%%%%%%%%%%%%%%%%%%%%%%%%%%%%%%%%%%%%%%%%%%%%%%%%%%%%%%%%%%%%%%%%%%%%%%%%%%%%%%%%%%%%%%%%%%%%%%%%%%%%%%%%

\title{Komplexe Zahlen}

\subtitle{Mathematischer Br\"uckenkurs}

\author{Stefan Weinzierl}

\institute[Uni Mainz]{Institut f\"ur Physik, Universit\"at Mainz}%

\date[WiSe 2020/21]{Wintersemester 2020/21}

\begin{document}

%%%%%%%%%%%%%%%%%%%%%%%%%%%%%%%%%%%%%%%%%%%%%%%%%%%%%%%%%%%%%%%%%%%%%%%%%%%%%%%%%%%%%%%%%%%%%%%%%%%%%%%%%
%%%%%%%%%%%%%%%%%%%%%%%%%%%%%%%%%%%%%%%%%%%%%%%%%%%%%%%%%%%%%%%%%%%%%%%%%%%%%%%%%%%%%%%%%%%%%%%%%%%%%%%%%
%%%%%%%%%%%%%%%%%%%%%%%%%%%%%%%%%%%%%%%%%%%%%%%%%%%%%%%%%%%%%%%%%%%%%%%%%%%%%%%%%%%%%%%%%%%%%%%%%%%%%%%%%

\begin{frame}
  \titlepage
\end{frame}

%%%%%%%%%%%%%%%%%%%%%%%%%%%%%%%%%%%%%%%%%%%%%%%%%%%%%%%%%%%%%%%%%%%%%%%%%%%%%%%%%%%%%%%%%%%%%%%%%%%%%%%%%
%%%%%%%%%%%%%%%%%%%%%%%%%%%%%%%%%%%%%%%%%%%%%%%%%%%%%%%%%%%%%%%%%%%%%%%%%%%%%%%%%%%%%%%%%%%%%%%%%%%%%%%%%
%%%%%%%%%%%%%%%%%%%%%%%%%%%%%%%%%%%%%%%%%%%%%%%%%%%%%%%%%%%%%%%%%%%%%%%%%%%%%%%%%%%%%%%%%%%%%%%%%%%%%%%%%

\section{Motivation und Definition}

\frame{\sectionpage}

% page --------------------------------------------------------------------------------------------------
\begin{frame}{Motivation}

Die reellen Zahlen enthalten algebraische Zahlen, so zum Beispiel die L\"osungen $\sqrt{2}$ und $-\sqrt{2}$ der
Gleichung
\bq
 x^2 & = & 2.
\eq
Aber: Nicht jede algebraische Zahl ist eine reelle Zahl.
So hat zum Beispiel die Gleichung
\bq
 x^2 & = & -2
\eq
keine reellen L\"osungen.

\end{frame}

% page --------------------------------------------------------------------------------------------------
\begin{frame}{Definition der komplexen Zahlen}

\begin{definition}
Man definiert die {\bf imagin\"are Einheit} $i$ als eine L\"osung der Gleichung
\bq
 x^2 & = & -1.
\eq
\end{definition}

\begin{mytheorem6}
\bq
 {\mathbb C} & = & \left\{ x+iy \;|\; x,y \in {\mathbb R} \right\}.
\eq
\end{mytheorem6}

\end{frame}

% page --------------------------------------------------------------------------------------------------
\begin{frame}{Analogie mit $\mathbb{Q}[\sqrt{3}]$}

\begin{definition}
Setzen wir $w=\sqrt{3}$, so ist $w$ 
eine L\"osung der Gleichung
\bq
 w^2 & = & 3.
\eq
\end{definition}

\begin{mytheorem6a}
\bq
 \mathbb{Q}[\sqrt{3}] & = & \left\{ a+bw \;|\; a,b \in {\mathbb Q} \right\}.
\eq
\end{mytheorem6a}

\end{frame}

% page --------------------------------------------------------------------------------------------------
\begin{frame}{Addition und Multiplikation von komplexen Zahlen}

Sei $z_1=x_1+iy_1$ und $z_2=x_2+iy_2$.
\begin{mytheorem7}
\bq
 z_1 + z_2 & = & \left( x_1+iy_1 \right) + \left( x_2+iy_2 \right) = \left(x_1+x_2\right) + i \left(y_1+y_2\right),
 \nonumber \\
z_1 \cdot z_2 & = & \left( x_1+iy_1 \right) \cdot \left( x_2+iy_2 \right) 
 = \left(x_1 x_2 - y_1 y_2 \right) + i \left(x_1 y_2 + y_1 x_2 \right)
\eq
\end{mytheorem7}

\begin{example}
\bq
 \left(1+2i\right) + \left(3+4i\right) & = & 4 + 6 i
 \nonumber \\
 \left(1+2i\right) \cdot \left(3+4i\right) & = & -5 + 10 i
\eq
\end{example}

\end{frame}

% page --------------------------------------------------------------------------------------------------
\begin{frame}{Quiz}

Sei $z_1=7+13i$ und $z_2=2-5i$.
\bq
 z_1+z_2 & = & ?
\eq 
\begin{description}
\item{(A)} $17i$
\item{(B)} $9+8i$
\item{(C)} $9+18i$
\item{(D)} $5-18i$
\end{description}

\end{frame}

% page --------------------------------------------------------------------------------------------------
\begin{frame}{Quiz}

Sei $z_1=5+9i$ und $z_2=2i$.
\bq
 z_1 \cdot z_2 & = & ?
\eq 
\begin{description}
\item{(A)} $10+18i$
\item{(B)} $10-18i$
\item{(C)} $-18+10i$
\item{(D)} $18+10i$
\end{description}

\end{frame}

% page --------------------------------------------------------------------------------------------------
\begin{frame}{Subtraktion und Division von komplexen Zahlen}

Sei $z_1=x_1+iy_1$ und $z_2=x_2+iy_2$.
\begin{mytheorem7a}
\bq
 z_1 - z_2 & = & \left( x_1+iy_1 \right) - \left( x_2+iy_2 \right) = \left(x_1-x_2\right) + i \left(y_1-y_2\right),
 \nonumber \\
\frac{z_1}{z_2} & = & \frac{x_1+iy_1}{x_2+iy_2} 
 = \frac{\left( x_1+iy_1 \right) \cdot \left( x_2-iy_2 \right)}{\left( x_2+iy_2 \right) \cdot \left( x_2-iy_2 \right)} 
 \nonumber \\
 & = & 
 \frac{\left(x_1 x_2 + y_1 y_2 \right) + i \left(-x_1 y_2 + y_1 x_2 \right)}{x_2^2+y_2^2}
 \nonumber \\
 & = & 
 \frac{\left(x_1 x_2 + y_1 y_2 \right)}{x_2^2+y_2^2}
 + 
 i \frac{\left(y_1 x_2 - x_1 y_2\right)}{x_2^2+y_2^2}.
\eq
\end{mytheorem7a}

\end{frame}

% page --------------------------------------------------------------------------------------------------
\begin{frame}{Subtraktion und Division von komplexen Zahlen}

\begin{example}
\bq
 \left(1+2i\right) - \left(3+4i\right) & = & \left(1-3\right) + i \left(2-4\right)
 \nonumber \\
 & = & 
 -2 - 2i,
 \nonumber \\
 & & \nonumber \\
 \frac{1+2i}{3+4i} 
 & = & 
 \frac{\left(1+2i\right)\left(3-4i\right)}{\left(3+4i\right)\left(3-4i\right)}
 \nonumber \\
 & = &
 \frac{\left(3+8\right) + i\left(6-4\right)}{9+16}
 \nonumber \\
 & = &
 \frac{11}{25} + \frac{2}{25} i.
\eq
\end{example}

\end{frame}

% page --------------------------------------------------------------------------------------------------
\begin{frame}{Quiz}

Sei $z_1=6+8i$ und $z_2=2i$.
\bq
 \frac{z_1}{z_2} & = & ?
\eq 
\begin{description}
\item{(A)} $10$
\item{(B)} $3+4i$
\item{(C)} $4-3i$
\item{(D)} $4+3i$
\end{description}

\end{frame}

% page --------------------------------------------------------------------------------------------------
\begin{frame}{Der K\"orper der komplexen Zahlen}

\begin{itemize}
\item Mit dieser Addition und Multiplikation bilden die komplexen Zahlen einen K\"orper.
\item Dieser K\"orper ist algebraisch abgeschlossen, d.h. die Nullstellen eines jeden Polynoms liegen in dem K\"orper.
\item Der K\"orper ist allerdings nicht angeordnet.
\item Das Vollst\"andigkeitsaxiom gilt.
\end{itemize}

\end{frame}

% page --------------------------------------------------------------------------------------------------
\begin{frame}{Nullstellen eines Polynoms}

\begin{myemptytheorem}
Es seien $c_n, c_{n-1}, \dots c_1, c_0 \in \mathbb{C}$.
Wir betrachten die Gleichung
\bq
 c_n z^n + c_{n-1} z^{n-1} + \dots + c_1 z + c_0 & = & 0.
\eq
Diese Gleichung hat f\"ur die unbekannte Variable $z$ in $\mathbb{C}$ genau $n$ L\"osungen, wobei Vielfachheiten mitgez\"ahlt werden.
\end{myemptytheorem}

\vspace*{5mm}
\begin{myemptytheorem}
Anders ausgedr\"uckt: Ein Polynom $n$-ten Grades hat in $\mathbb{C}$ genau $n$ Nullstellen, wobei Vielfachheiten mitgez\"ahlt werden.
\end{myemptytheorem}

\end{frame}

% page --------------------------------------------------------------------------------------------------
\begin{frame}{Vielfachheiten}

\begin{example}
Betrachte das Polynom
\bq
 \left(z-4\right) \left(z-5\right)^2
\eq
Die Nullstelle $z=4$ hat die Vielfachtheit $1$, die Nullstelle $5$ hat die Vielfachtheit $2$.

Das Polynom hat den Grad $3$, es sollte also drei Nullstellen haben.
Eine einfache Nullstelle und eine doppelte Nullstelle ergibt
\bq
 1 + 2 & = & 3.
\eq
\end{example}

\end{frame}

% page --------------------------------------------------------------------------------------------------
\begin{frame}{Nullstellen eines Polynoms}

\begin{example}
Wir betrachten die quadratische Gleichung
\bq
 2 z^2 - 8 z + 26 & = & 0
\eq
Die Diskriminante ist
\bq
 D & = & b^2 - 4 a c \; = \; -144
\eq
Somit
\bq
 z_{1/2}
 & = &
 \frac{1}{4} \left( 8 \pm \sqrt{-144} \right)
 \; = \;
 \frac{1}{4} \left( 8 \pm \sqrt{\left(-1\right) \cdot \left(12\right)^2} \right)
 \nonumber \\
 & = &
 \frac{1}{4} \left( 8 \pm 12 i \right)
 \; = \; 
 2 \pm 3 i
\eq


\end{example}

\end{frame}

% page --------------------------------------------------------------------------------------------------
\begin{frame}{Real- und Imagin\"arteil}

\begin{definition}
Sei $z=x+iy$ eine komplexe Zahl. 

Man bezeichnet $x$ als \alert{Realteil} und $y$ als \alert{Imagin\"arteil}.
\bq
 \mathrm{Re}\;z & = & x,
 \nonumber \\
 \mathrm{Im}\;z & = & y.
\eq
\end{definition}

\begin{example}
\bq
 \mathrm{Re}\;\left(3+5i\right) & = & 3,
 \nonumber \\
 \mathrm{Im}\;\left(3+5i\right) & = & 5.
\eq
\end{example}

\end{frame}

% page --------------------------------------------------------------------------------------------------
\begin{frame}{Konjugation}

\begin{definition}
Die zu $z=x+iy$ \alert{konjugiert komplexe Zahl} ist
\bq
 z^\ast & = & x-iy.
\eq
\end{definition}

\begin{example}
\bq
 \left( 3 + 5 i \right)^\ast & = & 3 - 5 i
\eq
\end{example}

\end{frame}

% page --------------------------------------------------------------------------------------------------
\begin{frame}{Rechenregeln}

\begin{myemptytheorem}
\bq
 \left( z^\ast \right)^\ast & = & z,
\eq
\end{myemptytheorem}

\begin{myemptytheorem}
\bq
 \left( z_1 + z_2 \right)^\ast & = & z_1^\ast + z_2^\ast,
 \nonumber \\
 \left( z_1 - z_2 \right)^\ast & = & z_1^\ast - z_2^\ast,
 \nonumber \\
 \left( z_1 \cdot z_2 \right)^\ast & = & z_1^\ast \cdot z_2^\ast,
 \nonumber \\
 \left( \frac{z_1}{z_2} \right)^\ast & = & \frac{z_1^\ast}{z_2^\ast},
\eq
\end{myemptytheorem}

\begin{myemptytheorem}
\bq
 \mathrm{Re}\;z \; = \; \frac{1}{2} \left( z + z^\ast \right),
 & &
 \mathrm{Im}\;z \; = \; \frac{1}{2i} \left( z - z^\ast \right).
\eq
\end{myemptytheorem}

\end{frame}

% page --------------------------------------------------------------------------------------------------
\begin{frame}{Betrag einer komplexen Zahl}

Sei $z=x+iy$ eine komplexe Zahl. 
Es ist
\bq
 z \cdot z^\ast & = & (x+iy) \cdot (x-iy) = x^2 + y^2.
\eq
\begin{definition}
Als \alert{Betrag} der komplexen Zahl bezeichnet man
\bq
 \left| z \right| & = & \sqrt{ z z^\ast} = \sqrt{x^2+y^2}.
\eq
\end{definition}
\begin{example}
\bq
 \left| 3+5i \right| & = & \sqrt{\left(3+5i\right)\left(3-5i\right)}
 \; = \; \sqrt{9+25} \; = \; \sqrt{34}.
\eq
\end{example}

\end{frame}

% page --------------------------------------------------------------------------------------------------
\begin{frame}{Die komplexe Zahlenebene}

Eine komplexe Zahl $z=x+iy$ wird durch ein Paar $(x,y)$ zweier reeller Zahlen beschrieben.
\begin{center}
\begin{picture}(250,130)(0,0)
\LongArrow(0,20)(230,20)
\LongArrow(20,0)(20,110)
\Vertex(140,80){2}
\DashLine(140,20)(140,80){5}
\DashLine(20,80)(140,80){5}
\Text(240,20)[l]{$\mathrm{Re}(z)$}
\Text(20,120)[b]{$\mathrm{Im}(z)$}
\Text(143,83)[lb]{$z$}
\Text(10,80)[l]{$y$}
\Text(140,10)[t]{$x$}
\end{picture}
\end{center}
Die reellen Zahlen sind genau die Zahlen, f\"ur die $\mathrm{Im}(z)=0$ gilt.

\end{frame}

% page --------------------------------------------------------------------------------------------------
\begin{frame}{Die komplexe Zahlenebene}

\begin{center}
\begin{picture}(250,130)(0,0)
\LongArrow(0,20)(230,20)
\LongArrow(20,0)(20,110)
\Line(20,20)(140,100)
\Vertex(140,100){2}
\Line(140,20)(140,100)
\CArc(20,20)(20,0,33.7)
\Text(240,20)[l]{$\mathrm{Re}(z)$}
\Text(20,120)[b]{$\mathrm{Im}(z)$}
\Text(143,103)[lb]{$z$}
\Text(150,60)[l]{$y$}
\Text(80,10)[t]{$x$}
\Text(50,30)[l]{$\varphi$}
\Text(80,70)[b]{$|z|$}
\end{picture}
\end{center}
Polardarstellung einer komplexen Zahl:
\bq
 z & = & \left| z \right| \cdot \left( \cos \varphi + i \sin \varphi \right)
\eq
$\varphi$ nennt man das Argument oder die Phase der komplexen Zahl.

\end{frame}

% page --------------------------------------------------------------------------------------------------
\begin{frame}{Umrechnung: Normalform in Polarform}

{\small
\bq
 \left| z \right| & = & \sqrt{x^2+y^2}
 \nonumber \\
 \tan \varphi & = & \frac{y}{x}, \;\;\; x\neq0,
 \nonumber \\
 \varphi & = & \frac{\pi}{2} \;\;\;\mbox{f\"ur} \;\; x=0, \; y>0,
 \nonumber \\
 \varphi & = & \frac{3\pi}{2} \;\;\;\mbox{f\"ur} \;\; x=0, \; y<0.
\eq
}
Die Aufl\"osung der Gleichung $\tan \varphi = y/x$ nach $\varphi$ ergibt
{\small
\begin{align*}
 \varphi \;\; & = \;\; \arctan \frac{y}{x}, && \mbox{f\"ur} \; x>0, y \ge 0, 
 \nonumber \\
 \varphi \;\; & = \;\; \pi + \arctan \frac{y}{x}, && \mbox{f\"ur} \; x<0 
 \nonumber \\
 \varphi \;\; & = \;\; 2\pi + \arctan \frac{y}{x}, && \mbox{f\"ur} \; x>0, y < 0.
\end{align*}
}

\end{frame}

% page --------------------------------------------------------------------------------------------------
\begin{frame}{Umrechnung: Polarform in Normalform}

\bq
 x = |z| \cos \varphi,
 & &
 y = |z| \sin \varphi.
\eq

\end{frame}

% page --------------------------------------------------------------------------------------------------
\begin{frame}{Multiplikation und Division in Polarform}

\begin{myemptytheorem}
In der Normalform hatten wir:
\bq
 z_1 \cdot z_2 & = & \left(x_1 x_2 - y_1 y_2 \right) + i \left(x_1 y_2 + y_1 x_2 \right),
 \nonumber \\
\frac{z_1}{z_2} & = &  
 \frac{\left(x_1 x_2 + y_1 y_2 \right)}{x_2^2+y_2^2}
 + 
 i \frac{\left(y_1 x_2 - x_1 y_2\right)}{x_2^2+y_2^2}.
\eq
\end{myemptytheorem}
\begin{myemptytheorem}
In der Polarform sind Multiplikation und Division besonders einfach:
\bq
 z_1 \cdot z_2 & = & \left| z_1 \right| \left| z_2 \right|
   \left[ \cos\left(\varphi_1+\varphi_2\right) + i \sin\left(\varphi_1+\varphi_2\right) \right],
 \nonumber \\
 \frac{z_1}{z_2} & = & \frac{\left| z_1 \right|}{\left| z_2 \right|}
   \left[ \cos\left(\varphi_1-\varphi_2\right) + i \sin\left(\varphi_1-\varphi_2\right) \right].
\eq
\end{myemptytheorem}

\end{frame}

% page --------------------------------------------------------------------------------------------------
\begin{frame}{Die Formel von Moivre}

\begin{myemptytheorem}
Aus
\bq
 z_1 \cdot z_2 & = & \left| z_1 \right| \left| z_2 \right|
   \left[ \cos\left(\varphi_1+\varphi_2\right) + i \sin\left(\varphi_1+\varphi_2\right) \right]
\eq
folgt insbesondere
\bq
 z^n & = & \left| z \right|^n \left( \cos n \varphi + i \sin n \varphi \right).
\eq
Diese Gleichung wird auch als Formel von Moivre bezeichnet.
\end{myemptytheorem}


\end{frame}

% page --------------------------------------------------------------------------------------------------
\begin{frame}{Die Formel von Euler}

Polardarstellung einer komplexen Zahl:
\bq
 z & = & \left| z \right| \cdot \left( \cos \varphi + i \sin \varphi \right)
\eq
Wir werden sp\"ater komplexwertige Funktionen kennenlernen. Im Vorgriff soll allerdings hier schon die Formel von Euler erw\"ahnt werden
\begin{myemptytheorem}
\bq
 e^{i \varphi} & = & \cos \varphi + i \sin \varphi.
\eq
\end{myemptytheorem}
Diese Formel werden wir sp\"ater mit Hilfe der Reihendarstellung der Funktionen $\exp$, $\sin$ und $\cos$ relativ einfach beweisen k\"onnen.
Somit
\bq
 z & = & \left| z \right| \cdot e^{i \varphi}
\eq

\end{frame}

% page --------------------------------------------------------------------------------------------------
\begin{frame}{Multiplikation und Division mit der Formel von Euler}

Es sei $z_1=|z_1| e^{i \varphi_1}$ und $z_2=|z_2| e^{i \varphi_2}$.
Dann ist
\begin{myemptytheorem}
\bq
 z_1 \cdot z_2 & = & 
  \left| z_1 \right| \left| z_2 \right| e^{i \left(\varphi_1+\varphi_2\right)},
 \nonumber \\
 \frac{z_1}{z_2} & = & \frac{\left| z_1 \right|}{\left| z_2 \right|} e^{i \left(\varphi_1-\varphi_2\right)}.
\eq
\end{myemptytheorem}


\end{frame}

% page --------------------------------------------------------------------------------------------------
\begin{frame}{Quiz}

\bq
 i^9 & = & ?
\eq 
\begin{description}
\item{(A)} $-i$
\item{(B)} $i$
\item{(C)} $9 i$
\item{(D)} $9+i$
\end{description}

\end{frame}

% page --------------------------------------------------------------------------------------------------
\begin{frame}{Antwort}

\bq
 i^9 & = & 
 i \cdot i \cdot i \cdot i \cdot i \cdot i \cdot i \cdot i \cdot i 
 \nonumber \\
 & = &
 i^2 \cdot i^2 \cdot i^2 \cdot i^2 \cdot i
 \nonumber \\
 & = &  
 \left(-1\right) \cdot \left(-1\right) \cdot \left(-1\right) \cdot \left(-1\right) \cdot i
 \nonumber \\
 & = &
 \left(-1\right)^4 \cdot i
 \nonumber \\
 & = & 
 i 
\eq 

\end{frame}

% page --------------------------------------------------------------------------------------------------
\begin{frame}{Betrag und Argument von $i$}

Wir schreiben $i$ in Polarform: Es ist
\bq
 \left| i \right| 
 & = & 
 \sqrt{ i \cdot i^\ast}
 \; = \; \sqrt{ i \cdot \left(-i\right)}
 \; = \; \sqrt{1}
 \; = \; 1.
\eq
Da $i = 0 + 1 \cdot i$ und somit $x=0$ und $y=1$ gilt
\bq
 \varphi & = & \frac{\pi}{2}.
\eq
Somit
\bq
 i & = & \underbrace{\cos \frac{\pi}{2}}_{0} + i \underbrace{\sin \frac{\pi}{2}}_{1}.
\eq


\end{frame}

% page --------------------------------------------------------------------------------------------------
\begin{frame}{Potenzen von $i$}

Sei $n \in {\mathbb Z}$. Mit Formel von Moivre haben wir
\bq
 i^n
 & = &
 \cos\left( \frac{n \pi}{2} \right) + i \sin\left( \frac{n \pi}{2}\right)
\eq
Insbesondere:
\begin{align*}
 i^{-4} & = 1,
 &
 i^{-3} & = i,
 &
 i^{-2} & = -1,
 &
 i^{-1} & = -i,
 \nonumber \\
 \alert{i^0} & \alert{= 1},
 &
 \alert{i^1} & \alert{= i},
 &
 \alert{i^2} & \alert{= -1},
 &
 \alert{i^3} & = \alert{-i},
 \nonumber \\
 i^4 & = 1,
 &
 i^5 & = i,
 &
 i^6 & = -1,
 &
 i^7 & = -i,
 \nonumber \\
 i^8 & = 1,
 &
 i^9 & = i,
 &
 i^{10} & = -1,
 &
 i^{11} & = -i,
\end{align*}

\end{frame}

% page --------------------------------------------------------------------------------------------------
\begin{frame}{Quiz}

\bq
 \sqrt{i} & = & ?
\eq 
\begin{description}
\item{(A)} $-1$
\item{(B)} $i$
\item{(C)} $\frac{1}{\sqrt{2}}\left(1+i\right)$
\item{(D)} $-1+i$
\end{description}

\end{frame}

%%%%%%%%%%%%%%%%%%%%%%%%%%%%%%%%%%%%%%%%%%%%%%%%%%%%%%%%%%%%%%%%%%%%%%%%%%%%%%%%%%%%%%%%%%%%%%%%%%%%%%%%%
%%%%%%%%%%%%%%%%%%%%%%%%%%%%%%%%%%%%%%%%%%%%%%%%%%%%%%%%%%%%%%%%%%%%%%%%%%%%%%%%%%%%%%%%%%%%%%%%%%%%%%%%%
%%%%%%%%%%%%%%%%%%%%%%%%%%%%%%%%%%%%%%%%%%%%%%%%%%%%%%%%%%%%%%%%%%%%%%%%%%%%%%%%%%%%%%%%%%%%%%%%%%%%%%%%%

% page --------------------------------------------------------------------------------------------------
\begin{frame}

\end{frame}

\end{document}
