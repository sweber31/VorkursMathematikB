\documentclass[german]{beamer}

\mode<presentation>
{
 \usetheme{Madrid}

% \usecolortheme{crane}
 \usecolortheme{wolverine}
}

\usepackage{hyperref}

\usepackage[german]{babel}
\usepackage{times}
\usepackage[latin1,utf8]{inputenc}
\usepackage[OT2,T1]{fontenc}
\usepackage{shuffle}

\definecolor{MyYellowOrange}{cmyk}{0,0.5,1,0}
\newcommand{\superalert}[1]{{\color{MyYellowOrange}{#1}}}

% Stefan's abbreveations
\newcommand{\bq}{\begin{eqnarray*}}
\newcommand{\eq}{\end{eqnarray*}}
\newcommand{\eps}{\varepsilon}

% dedicated environments
\newtheorem*{myemptytheorem}{}
\newtheorem*{mytheorem8}{Definition eines Vektorraumes}
\newtheorem*{mytheorem9}{Ein euklidische Skalarprodukt eines reellen Vektorraumes ist eine positiv definite symmetrische Bilinearform $V \times V \rightarrow {\mathbb R}$:}
\newtheorem*{mytheorem10}{Ein unit\"ares Skalarprodukt eines komplexen Vektorraumes ist eine positiv definite Hermitische Form $V \times V \rightarrow {\mathbb C}$:}

%%%%%%%%%%%%%%%%%%%%%%%%%%%%%%%%%%%%%%%%%%%%%%%%%%%%%%%%%%%%%%%%%%%%%%%%%%%%%%%%%%%%%%%%%%%%%%%%%%%%%%%%%
%%%%%%%%%%%%%%%%%%%%%%%%%%%%%%%%%%%%%%%%%%%%%%%%%%%%%%%%%%%%%%%%%%%%%%%%%%%%%%%%%%%%%%%%%%%%%%%%%%%%%%%%%
%%%%%%%%%%%%%%%%%%%%%%%%%%%%%%%%%%%%%%%%%%%%%%%%%%%%%%%%%%%%%%%%%%%%%%%%%%%%%%%%%%%%%%%%%%%%%%%%%%%%%%%%%

\title{Vektoren}

\subtitle{Mathematischer Br\"uckenkurs}

\author{Stefan Weinzierl}

\institute[Uni Mainz]{Institut f\"ur Physik, Universit\"at Mainz}%

\date[WiSe 2020/21]{Wintersemester 2020/21}

\begin{document}

%%%%%%%%%%%%%%%%%%%%%%%%%%%%%%%%%%%%%%%%%%%%%%%%%%%%%%%%%%%%%%%%%%%%%%%%%%%%%%%%%%%%%%%%%%%%%%%%%%%%%%%%%
%%%%%%%%%%%%%%%%%%%%%%%%%%%%%%%%%%%%%%%%%%%%%%%%%%%%%%%%%%%%%%%%%%%%%%%%%%%%%%%%%%%%%%%%%%%%%%%%%%%%%%%%%
%%%%%%%%%%%%%%%%%%%%%%%%%%%%%%%%%%%%%%%%%%%%%%%%%%%%%%%%%%%%%%%%%%%%%%%%%%%%%%%%%%%%%%%%%%%%%%%%%%%%%%%%%

\begin{frame}
  \titlepage
\end{frame}

%%%%%%%%%%%%%%%%%%%%%%%%%%%%%%%%%%%%%%%%%%%%%%%%%%%%%%%%%%%%%%%%%%%%%%%%%%%%%%%%%%%%%%%%%%%%%%%%%%%%%%%%%
%%%%%%%%%%%%%%%%%%%%%%%%%%%%%%%%%%%%%%%%%%%%%%%%%%%%%%%%%%%%%%%%%%%%%%%%%%%%%%%%%%%%%%%%%%%%%%%%%%%%%%%%%
%%%%%%%%%%%%%%%%%%%%%%%%%%%%%%%%%%%%%%%%%%%%%%%%%%%%%%%%%%%%%%%%%%%%%%%%%%%%%%%%%%%%%%%%%%%%%%%%%%%%%%%%%

\section{Motivation und Definition}

\frame{\sectionpage}

% page --------------------------------------------------------------------------------------------------
\begin{frame}{Motivation}

Aus der Schulmathematik sind die Vektorr\"aume ${\mathbb R}^2$ und ${\mathbb R}^3$ bekannt.
Vektoren aus dem \alert{${\mathbb R}^2$} k\"onnen durch \superalert{zwei reelle Zahlen $x$ und $y$} beschrieben werden
\bq
 \left( \begin{array}{c} x \\ y \\ \end{array} \right)
 & \in & 
 {\mathbb R}^2,
\eq
Vektoren aus dem \alert{${\mathbb R}^3$} k\"onnen durch \superalert{drei reelle Zahlen $x$, $y$ und $z$} 
\bq
 \left( \begin{array}{c} x \\ y \\ z \\ \end{array} \right)
 & \in & 
 {\mathbb R}^3.
\eq

\end{frame}

% page --------------------------------------------------------------------------------------------------
\begin{frame}{Motivation}

Wir k\"onnen das Konzept in zwei Richtungen erweitern:

\begin{itemize}
\item Wir lassen \superalert{andere Dimensionen} zu und beschr\"anken uns nicht mehr auf Vektorr\"aume der Dimension $2$ und $3$.

Beispiel: ${\mathbb R}^n$
\item Wir lassen \superalert{andere Grundk\"orper} zu, z.B die komplexen Zahlen ${\mathbb C}$.

Beispiel: ${\mathbb C}^n$
\end{itemize}

\end{frame}

% page --------------------------------------------------------------------------------------------------
\begin{frame}{Vektorr\"aume}

Sei $K$ ein K\"orper und $(V,+)$ eine kommutative Gruppe. 
Weiter sei eine zus\"atzliche Verkn\"upfung gegeben, die man skalare Multiplikation nennt:
\bq
K \times V & \rightarrow & V \\
(k,\vec{v}) & \rightarrow & k \cdot \vec{v}
\eq

\begin{mytheorem8}
$V$ ist ein $K$-Vektorraum falls gilt:
\begin{itemize}
\item (V1) $(K,+,\cdot)$ ist ein K\"orper
\item (V2) $(V,+)$ ist eine kommutative Gruppe
\end{itemize}
{\footnotesize (Fortsetzung n\"achste Folie)}
\end{mytheorem8}

\end{frame}

% page --------------------------------------------------------------------------------------------------
\begin{frame}{Vektorr\"aume}

\begin{mytheorem8}[Fortsetzung]
\begin{itemize}
\item (V3) Es gelten die Distributivgesetze:
\bq
k \cdot ( \vec{v}_1 + \vec{v}_2 ) &=  & (k \cdot \vec{v}_1 ) + ( k \cdot \vec{v}_2 ) \\
(k_1 + k_2 ) \cdot \vec{v} & = & (k_1 \cdot \vec{v}) + ( k_2 \cdot \vec{v})
\eq
\item (V4) Es gilt das Assoziativgesetz:
\bq
k_1 \cdot ( k_2 \cdot \vec{v}) & = & ( k_1 \cdot k_2 ) \cdot \vec{v}
\eq
\item (V5) F\"ur die Eins gilt:
\bq
1 \cdot \vec{v} & = & \vec{v}
\eq
\end{itemize}
\end{mytheorem8}
Bemerkung: Bei $(k_1 \cdot k_2)$ ist die Multiplikation im K\"orper gemeint.

\end{frame}

% page --------------------------------------------------------------------------------------------------
\begin{frame}{Vektorr\"aume}

Als Grundk\"orper treten in den Naturwissenschaften fast immer $\mathbb R$ oder $\mathbb C$ auf.
Beispiele f\"ur Vektorr\"aume sind der ${\mathbb R}^n$  (mit Grundk\"orper $\mathbb R$)
und der ${\mathbb C}^n$ (mit Grundk\"orper $\mathbb C$).
\bq
 {\mathbb R}^n & = & \left\{ \left( \begin{array}{c} x_1 \\ x_2 \\ ... \\ x_n \\ \end{array} \right) | \; x_1,x_2,...,x_n \in {\mathbb R} \right\},
 \nonumber \\
 {\mathbb C}^n & = & \left\{ \left( \begin{array}{c} z_1 \\ z_2 \\ ... \\ z_n \\ \end{array} \right) | \; z_1,z_2,...,z_n \in {\mathbb C} \right\}.
\eq

\end{frame}

% page --------------------------------------------------------------------------------------------------
\begin{frame}{Dualer Vektorraum}

Man schreibt die Elemente aus dem Vektorraum als Spaltenvektoren, so zum Beispiel:
\bq
 \left( \begin{array}{c} x \\ y \\ z \\ \end{array} \right) \in {\mathbb R}^3.
\eq
Ebenso ist die Schreibweise als Zeilenvektor gebr\"auchlich:
\bq
 \left( x, y, z \right) \in \left.{\mathbb R}^3\right.^\ast.
\eq
$V^\ast$ bezeichnet den zu $V$ {\bf dualen Vektorraum} (falls $V$ alle Spaltenvektoren enth\"alt,
so enth\"alt $V^\ast$ die Zeilenvektoren).

\end{frame}

% page --------------------------------------------------------------------------------------------------
\begin{frame}{Transposition}

Man bezeichnet mit $\vec{v}^T$ den zu $\vec{v}$ transponierten Vektor
(d.h. aus einem Spaltenvektor wird ein Zeilenvektor, und aus einem Zeilenvektor
wird ein Spaltenvektor):
\bq
 \left( x_1, x_2, ..., x_n \right)^T 
 & = & 
 \left( \begin{array}{c} x_1 \\ x_2 \\ ... \\ x_n \\ \end{array} \right).
\eq

\end{frame}

% page --------------------------------------------------------------------------------------------------
\begin{frame}{Addition und skalare Multiplikation}

Bei der Summe zweier Vektoren werden die Vektoren komponentenweise addiert:
\bq
 \left( \begin{array}{c} 1 \\ 2 \\ 3 \\ \end{array} \right) 
 +
 \left( \begin{array}{c} 4 \\ 5 \\ 6 \\ \end{array} \right) 
 =
 \left( \begin{array}{c} 5 \\ 7 \\ 9 \\ \end{array} \right).
\eq
Bei der skalaren Multiplikation wird jede Komponente mit dem Skalar multipliziert:
\bq
 3 \cdot
 \left( \begin{array}{c} 4 \\ 5 \\ 6 \\ \end{array} \right) 
 & = & 
 \left( \begin{array}{c} 12 \\ 15 \\ 18 \\ \end{array} \right).
\eq

\end{frame}

% page --------------------------------------------------------------------------------------------------
\begin{frame}{Quiz}

\bq
 \vec{v}_1 \; = \; \left( \begin{array}{c} 3 \\ 5 \\ \end{array} \right),
 & &
 \vec{v}_2 \; = \; \left( \begin{array}{c} 1 \\ 1 \\ \end{array} \right).
\eq
\bq
 2 \vec{v}_1 + 3 \vec{v}_2 & = & ?
\eq 
\begin{columns}[b]
\begin{column}{5cm}
\begin{description}
\item{(A)} $\left( \begin{array}{c} 4 \\ 6 \\ \end{array} \right)$
\item{(C)} $\left( \begin{array}{c} 9 \\ 6 \\ \end{array} \right)$
\end{description}
\end{column}
\begin{column}{5cm}
\begin{description}
\item{(B)} $\left( \begin{array}{c} 6 \\ 8 \\ \end{array} \right)$
\item{(D)} $\left( \begin{array}{c} 9 \\ 13 \\ \end{array} \right)$
\end{description}
\end{column}
\end{columns}

\end{frame}

% page --------------------------------------------------------------------------------------------------
\begin{frame}{Quiz}

\bq
 \vec{v}_1 \; = \; \left( \begin{array}{c} 1 \\ 1 \\ \end{array} \right),
 & &
 \vec{v}_2 \; = \; \left( \begin{array}{c} 2 i \\ 2 i \\ \end{array} \right).
\eq
\bq
 i \vec{v}_1 - \frac{1}{2} \vec{v}_2 & = & ?
\eq 
\begin{columns}[b]
\begin{column}{5cm}
\begin{description}
\item{(A)} $\left( \begin{array}{c} 0 \\ 0 \\ \end{array} \right)$
\item{(C)} $\left( \begin{array}{c} 2 i \\ 2 i \\ \end{array} \right)$
\end{description}
\end{column}
\begin{column}{6cm}
\begin{description}
\item{(B)} $\frac{1}{\sqrt{2}} \left( \begin{array}{c} 1+i \\ 1+i \\ \end{array} \right)$
\item{(D)} $\left( \begin{array}{c} 0 \\ 1 \\ \end{array} \right)$
\end{description}
\end{column}
\end{columns}

\end{frame}

% page --------------------------------------------------------------------------------------------------
\begin{frame}{Einheitsvektoren}

\begin{definition}
Vektoren, die in fast allen Komponenten eine Null haben, bis auf eine Komponente, in
der sie eine Eins haben, spielen eine wichtige Rolle. Hat so ein Vektor 
in der $i$-ten Komponente eine Eins,
\bq
 \vec{e}_i & = & \left( \underbrace{0,0,...,0}_{i-1},1,0,...,0 \right)^T,
\eq
so bezeichnet man diesen Vektor als den {\bf $i$-ten Einheitsvektor}.
\end{definition}

\end{frame}

% page --------------------------------------------------------------------------------------------------
\begin{frame}{Lineare Unabh\"angigkeit}

\begin{definition}
Seien $n$ Vektoren $\vec{v}_1$, $\vec{v}_2$, ..., $\vec{v}_n$ gegeben.
Folgt aus
\bq
 a_1 \vec{v}_1 + a_2 \vec{v}_2 + ... + a_n \vec{v}_n & = & \vec{0}
\nonumber \\
& \Rightarrow & a_1 = a_2 = ... = a_n = 0,
\eq
so nennt man die Vektoren {\bf linear unabh\"angig}. Anderfalls nennt man sie linear abh\"angig.
\end{definition}

\end{frame}

% page --------------------------------------------------------------------------------------------------
\begin{frame}{Basis und Dimension}

\begin{definition}
Sei $V$ ein Vektorraum. Die maximale Anzahl linear unabh\"angiger Vektoren
in $V$ nennt man die {\bf Dimension des Vektorraumes}.

Eine Menge linearer unabh\"angiger Vektoren, die maximal ist, nennt man eine {\bf Basis} von $V$.
\end{definition}

\begin{example}
${\mathbb R}^n$ und ${\mathbb C}^n$ haben die Dimension $n$. Eine Basis
von ${\mathbb R}^n$ und ${\mathbb C}^n$ ist zum Beispiel
\bq
 \left\{ \vec{e}_1, \vec{e}_2, ..., \vec{e}_n \right\}.
\eq
Man nennt diese Basis die Standardbasis.
\end{example}

\end{frame}

% page --------------------------------------------------------------------------------------------------
\begin{frame}{Standardbasis}

\begin{example}[Standardbasis des $\mathbb{R}^4$]
\bq
 \left\{
  \left( \begin{array}{c} 1 \\ 0 \\ 0 \\ 0 \\ \end{array} \right),
  \left( \begin{array}{c} 0 \\ 1 \\ 0 \\ 0 \\ \end{array} \right),
  \left( \begin{array}{c} 0 \\ 0 \\ 1 \\ 0 \\ \end{array} \right),
  \left( \begin{array}{c} 0 \\ 0 \\ 0 \\ 1 \\ \end{array} \right)
 \right\}
\eq
\end{example}

\end{frame}

% page --------------------------------------------------------------------------------------------------
\begin{frame}{Quiz}

Die Standardbasis des $\mathbb{C}^2$ ist

\vspace*{5mm}

\begin{columns}[b]
\begin{column}{6cm}
\begin{description}
\item{(A)} $\left\{ \left( \begin{array}{c} 1 \\ 0 \\ \end{array} \right), \left( \begin{array}{c} 0 \\ 1 \\ \end{array} \right) \right\}$
\item{(C)} $\left\{ \left( \begin{array}{c} 1 \\ 0 \\ \end{array} \right), \left( \begin{array}{c} 0 \\ i \\ \end{array} \right) \right\}$
\end{description}
\end{column}
\begin{column}{7cm}
\begin{description}
\item{(B)} $\left\{ \left( \begin{array}{c} i \\ 0 \\ \end{array} \right), \left( \begin{array}{c} 0 \\ i \\ \end{array} \right) \right\}$
\item{(D)} $\left\{ \left( \begin{array}{c} i \\ 0 \\ \end{array} \right), \left( \begin{array}{c} 0 \\ -i \\ \end{array} \right) \right\}$
\end{description}
\end{column}
\end{columns}

\end{frame}

%%%%%%%%%%%%%%%%%%%%%%%%%%%%%%%%%%%%%%%%%%%%%%%%%%%%%%%%%%%%%%%%%%%%%%%%%%%%%%%%%%%%%%%%%%%%%%%%%%%%%%%%%
%%%%%%%%%%%%%%%%%%%%%%%%%%%%%%%%%%%%%%%%%%%%%%%%%%%%%%%%%%%%%%%%%%%%%%%%%%%%%%%%%%%%%%%%%%%%%%%%%%%%%%%%%
%%%%%%%%%%%%%%%%%%%%%%%%%%%%%%%%%%%%%%%%%%%%%%%%%%%%%%%%%%%%%%%%%%%%%%%%%%%%%%%%%%%%%%%%%%%%%%%%%%%%%%%%%

\section{Skalarprodukte}

\frame{\sectionpage}

% page --------------------------------------------------------------------------------------------------
\begin{frame}{Das euklidische Standardskalarprodukt im ${\mathbb R}^n$}

Wir betrachten zun\"achst den \alert{${\mathbb R}^n$}. Seien $\vec{x}, \vec{y} \in {\mathbb R}^n$.
Die Komponentendarstellung der beiden Vektoren bez\"uglich der Standardbasis sei
\bq
 \vec{x} =  \left( \begin{array}{c} x_1 \\ x_2 \\ ... \\ x_n \\ \end{array} \right),
 & &
 \vec{y} =  \left( \begin{array}{c} y_1 \\ y_2 \\ ... \\ y_n \\ \end{array} \right).
\eq
Wir definieren das \superalert{euklidische Standardskalarprodukt} zwischen zwei Vektoren als die Abbildung
\bq
 V \times V & \rightarrow & {\mathbb R},
 \nonumber \\
 \vec{x} \cdot \vec{y} & = & x_1 y_1 + x_2 y_2 + ... + x_n y_n.
\eq

\end{frame}

% page --------------------------------------------------------------------------------------------------
\begin{frame}{Euklidische Skalarprodukte}

\begin{mytheorem9}
\begin{itemize}
\item Linear in der ersten Komponente:
{\small
\bq
 \left( \vec{x} + \vec{y} \right) \cdot \vec{z} 
 \; = \; \vec{x} \cdot \vec{z} + \vec{y} \cdot \vec{z},
 & &
 \left( \lambda \cdot \vec{x} \right) \cdot \vec{y} \; = \; \lambda \left( \vec{x} \cdot \vec{y} \right).
\eq
}
\item Linear in der zweiten Komponente:
{\small
\bq
 \vec{x} \cdot \left( \vec{y} + \vec{z} \right)  
 \; = \; \vec{x} \cdot \vec{y} + \vec{x} \cdot \vec{z},
 & &
 \vec{x} \cdot \left( \lambda \cdot \vec{y} \right) \; = \; \lambda \left( \vec{x} \cdot \vec{y} \right).
\eq
}
\item Symmetrisch:
{\small
\bq
 \vec{x} \cdot \vec{y} & = & \vec{y} \cdot \vec{x}.
\eq
}
\item Positiv definit:
{\small
\bq
 \vec{x} \cdot \vec{x} & > & 0, \;\;\;\mbox{falls}\;\;\; \vec{x} \neq \vec{0}.
\eq
}
\end{itemize}
\end{mytheorem9}

\end{frame}

% page --------------------------------------------------------------------------------------------------
\begin{frame}{Euklidische Skalarprodukte}

\begin{itemize}

\item 
Ein reeller Vektorraum mit einem euklidischen Skalarprodukt bezeichnet man als einen euklidischen Vektorraum.

\item 
Die Bezeichnung ``euklidisch'' bezieht sich insbesondere auf Forderung nach positiver Definitheit.

\item
In der Physik treten auch Skalarprodukte auf, bei denen die Forderung nach positiv Definitheit aufgegeben wird.

Ein Beispiel hierf\"ur ist das Skalarprodukt im Minkowskiraum.

\end{itemize}

\end{frame}

% page --------------------------------------------------------------------------------------------------
\begin{frame}{Euklidische Skalarprodukte}

\begin{example}
\bq
 \left( \begin{array}{c} 1 \\ 2 \\ 3 \end{array} \right)
 \cdot
 \left( \begin{array}{c} 4 \\ 5 \\ 6 \end{array} \right)
 & = &
 1 \cdot 4 + 2 \cdot 5 + 3 \cdot 6
 \; = \;
 4 + 10 + 18 
 \; = \;
 32 
\eq
\end{example}

\end{frame}

% page --------------------------------------------------------------------------------------------------
\begin{frame}{Quiz}

\bq
 \left( \begin{array}{r} 7 \\ 3 \\ 1 \\ \end{array} \right)
 \cdot
 \left( \begin{array}{r} 2  \\ -5 \\ 3 \\ \end{array} \right)
 & = & ?
\eq
\begin{description}
\item{(A)} $1$
\item{(B)} $2$
\item{(C)} $32$
\item{(D)} $42$
\end{description}

\end{frame}

% page --------------------------------------------------------------------------------------------------
\begin{frame}{Das unit\"are Standardskalarprodukt im ${\mathbb C}^n$}

Wir betrachten nun \alert{${\mathbb C}^n$}. 

Seien $\vec{x}, \vec{y} \in {\mathbb C}^n$.

In diesem Fall definieren wir das \superalert{unit\"are Standardskalarprodukt} als
\bq
 V \times V & \rightarrow & {\mathbb C},
 \nonumber \\
 \vec{x} \cdot \vec{y} & = & x_1^\ast y_1 + x_2^\ast y_2 + ... + x_n^\ast y_n.
\eq

\end{frame}

% page --------------------------------------------------------------------------------------------------
\begin{frame}{Unit\"are Skalarprodukte}

\begin{mytheorem10}
\begin{itemize}
\item Semilinear in der ersten Komponente:
{\small
\bq
 \left( \vec{x} + \vec{y} \right) \cdot \vec{z} 
 \; = \; \vec{x} \cdot \vec{z} + \vec{y} \cdot \vec{z},
 & &
 \left( \lambda \cdot \vec{x} \right) \cdot \vec{y} \; = \; \lambda^\ast \left( \vec{x} \cdot \vec{y} \right).
\eq
}
\item Linear in der zweiten Komponente:
{\small
\bq
 \vec{x} \cdot \left( \vec{y} + \vec{z} \right)  
 \; = \; \vec{x} \cdot \vec{y} + \vec{x} \cdot \vec{z},
 & &
 \vec{x} \cdot \left( \lambda \cdot \vec{y} \right) \; = \; \lambda \left( \vec{x} \cdot \vec{y} \right).
\eq
}
\item Hermitisch:
{\small
\bq
 \vec{x} \cdot \vec{y} & = & \left( \vec{y} \cdot \vec{x} \right)^\ast.
\eq
}
\item Positiv definit:
{\small
\bq
 \vec{x} \cdot \vec{x} & > & 0, \;\;\;\mbox{falls}\;\;\; \vec{x} \neq \vec{0}.
\eq
}
\end{itemize}
\end{mytheorem10}

\end{frame}

% page --------------------------------------------------------------------------------------------------
\begin{frame}{Unit\"are Skalarprodukte}

\begin{example}
\bq
 \left( \begin{array}{c} i \\ 2 \\ 3i \end{array} \right)
 \cdot
 \left( \begin{array}{c} 4i \\ 5 \\ 6 \end{array} \right)
 & = &
 i^\ast \cdot \left(4i\right) + \left(2\right)^\ast \cdot 5 + \left(3i\right)^\ast \cdot 6
 \nonumber \\
 & = & 
 \left(-i\right) \cdot \left(4i\right) + 2 \cdot 5 + \left(-3i\right) \cdot 6
 \nonumber \\
 & = &
 4 + 10 - 18 i
 \nonumber \\
 & = &
 14 - 18 i.
\eq
\end{example}

\end{frame}

% page --------------------------------------------------------------------------------------------------
\begin{frame}{Quiz}

\bq
 \left( \begin{array}{c} 1+i \\ i \\ \end{array} \right)
 \cdot
 \left( \begin{array}{c} 1+i  \\ i \\ \end{array} \right)
 & = & ?
\eq
\begin{description}
\item{(A)} $1+3i$
\item{(B)} $2$
\item{(C)} $3$
\item{(D)} $-1+2i$
\end{description}

\end{frame}

% page --------------------------------------------------------------------------------------------------
\begin{frame}{Betrag eines Vektors}

\begin{definition}
Man bezeichnet mit
\bq
 \left| \vec{x} \right|
 & = & \sqrt{ \vec{x} \cdot \vec{x}}
\eq
die L\"ange oder den {\bf Betrag von $\vec{x}$}.
\end{definition}


\end{frame}

% page --------------------------------------------------------------------------------------------------
\begin{frame}{Winkel zweier Vektoren und Orthogonalit\"at}

Sei \alert{$\vec{x}, \vec{y} \in {\mathbb R}^n$}. Der Winkel zwischen den beiden Vektoren
ist gegeben durch
\bq
 \vec{x} \cdot \vec{y} & = & \left| \vec{x} \right| \left| \vec{y} \right| \cos \varphi,
\eq
also
\bq 
\varphi & = & \arccos \frac{\vec{x} \cdot \vec{y}}{\left| \vec{x} \right| \left| \vec{y} \right|}.
\eq
Zwei Vektoren stehen {\bf senkrecht} aufeinander ($\varphi=90^\circ$), falls
\bq
 \vec{x} \cdot \vec{y} & = & 0.
\eq

\end{frame}

%%%%%%%%%%%%%%%%%%%%%%%%%%%%%%%%%%%%%%%%%%%%%%%%%%%%%%%%%%%%%%%%%%%%%%%%%%%%%%%%%%%%%%%%%%%%%%%%%%%%%%%%%
%%%%%%%%%%%%%%%%%%%%%%%%%%%%%%%%%%%%%%%%%%%%%%%%%%%%%%%%%%%%%%%%%%%%%%%%%%%%%%%%%%%%%%%%%%%%%%%%%%%%%%%%%
%%%%%%%%%%%%%%%%%%%%%%%%%%%%%%%%%%%%%%%%%%%%%%%%%%%%%%%%%%%%%%%%%%%%%%%%%%%%%%%%%%%%%%%%%%%%%%%%%%%%%%%%%

\section{Das Kreuzprodukt}

\frame{\sectionpage}

% page --------------------------------------------------------------------------------------------------
\begin{frame}{Das Kreuzprodukt}

Sei $V$ der Vektorraum ${\mathbb R}^3$ oder ${\mathbb C}^3$. 

In einem \superalert{dreidimensionalen Vektorraum} ist zus\"atzlich das Kreuzprodukt als eine Abbildung
\bq
 V \times V & \rightarrow & V,
 \nonumber \\
 \left( \begin{array}{c} x_1 \\ x_2 \\ x_3 \\ \end{array} \right)
 \times
 \left( \begin{array}{c} y_1 \\ y_2 \\ y_3 \\ \end{array} \right)
  & = &
 \left( \begin{array}{c} x_2 y_3 -x_3 y_2 \\ x_3 y_1 - x_1 y_3 \\ x_1 y_2 - x_2 y_1 \\ \end{array} \right)
\eq
definiert.

\vspace*{5mm} 

\alert{Wichtig}: Das Kreuzprodukt gibt es nur in drei Dimensionen!

\end{frame}

% page --------------------------------------------------------------------------------------------------
\begin{frame}{Das Kreuzprodukt}

\begin{example}
\bq
 \left( \begin{array}{c} 1 \\ 2 \\ 3 \\ \end{array} \right)
 \times
 \left( \begin{array}{c} 4 \\ 5 \\ 6 \\ \end{array} \right)
  & = &
 \left( \begin{array}{c} 2 \cdot 6 - 3 \cdot 5 \\ 3 \cdot 4 - 1 \cdot 6 \\ 1 \cdot 5 - 2 \cdot 4 \\ \end{array} \right)
 \; = \;
 \left( \begin{array}{r} -3 \\ 6 \\ -3 \\ \end{array} \right)
\eq
\end{example}

\end{frame}

% page --------------------------------------------------------------------------------------------------
\begin{frame}{Eigenschaften des Kreuzproduktes}

Das Kreuzprodukt ist anti-symmetrisch:
\bq
 \vec{x} \times \vec{y} & = & - \vec{y} \times \vec{x}.
\eq
Der Vektor $\vec{x} \times \vec{y}$ steht senkrecht auf $\vec{x}$ und $\vec{y}$:
\bq
 \vec{x} \cdot \left( \vec{x} \times \vec{y} \right) & = & 0,
 \nonumber \\
 \vec{y} \cdot \left( \vec{x} \times \vec{y} \right) & = & 0,
\eq
F\"ur den Betrag von $\vec{x} \times \vec{y}$ gilt:
\bq
 \left| \vec{x} \times \vec{y} \right | & = &
 \left| \vec{x} \right| \left| \vec{y} \right| \sin \varphi,
\eq
wobei $\varphi$ der Winkel zwischen $\vec{x}$ und $\vec{y}$ ist.

\end{frame}

% page --------------------------------------------------------------------------------------------------
\begin{frame}{Antisymmetrischer Tensor}

Sei $\vec{z} = \vec{x} \times \vec{y}$.
F\"ur die Komponenten von $\vec{z}$ gilt:
\bq
 z_i & = & \sum\limits_{j=1}^3 \sum\limits_{k=1}^3 \eps_{ijk} x_j y_k
\eq
Hier wurde der {\bf antisymmetrische Tensor} (oder {\bf Levi-Civita-Tensor}) $\eps_{ijk}$ verwendet.
\begin{definition}[antisymmetrischer Tensor]
\bq
 \eps_{ijk} & = & \left\{
 \begin{array}{rl}
   +1 & \mbox{f\"ur $(i,j,k)$ eine gerade Permutation von $(1,2,3)$,} \\
   -1 & \mbox{f\"ur $(i,j,k)$ eine ungerade Permutation von $(1,2,3)$,} \\
   0 & \mbox{sonst}. \\
 \end{array}
\right.
\eq
\end{definition}

\end{frame}

% page --------------------------------------------------------------------------------------------------
\begin{frame}{Permutationen}

\begin{definition}
Eine Permutation $(\sigma_1,\sigma_2,...,\sigma_n)$ nennt man gerade, wenn man sie durch eine
gerade Anzahl von paarweisen Vertauschungen aus $(1,2,...,n)$ erzeugen kann.
Ben\"otigt man eine ungerade Anzahl von Vertauschungen, so nennt man die Permutation ungerade.
\end{definition}

\begin{example}
\bq
 (3,2,1,5,4) & & \mbox{ist eine gerade Permutation}
\nonumber \\
 & & \mbox{(vertausche $1\leftrightarrow 3$ und $4 \leftrightarrow 5$),}
\nonumber \\
 (1,5,3,4,2) & & \mbox{ist eine ungerade Permutation}
\nonumber \\
 & & \mbox{(vertausche $2\leftrightarrow 5$).}
\eq
\end{example}

\end{frame}

% page --------------------------------------------------------------------------------------------------
\begin{frame}{Kronecker-Delta-Symbol}

\begin{definition}[Kronecker-Delta-Symbol]
\bq
 \delta_{ij}
 & = & \left\{
 \begin{array}{rl}
   +1 & \mbox{f\"ur $i=j$,} \\
   0 & \mbox{f\"ur $i\neq j$}. \\
 \end{array}
\right.
\eq
\end{definition}

\end{frame}

% page --------------------------------------------------------------------------------------------------
\begin{frame}{Quiz}

\bq
 \eps_{132}
 & = & ?
\eq
\begin{description}
\item{(A)} $-1$
\item{(B)} $0$
\item{(C)} $1$
\item{(D)} $6$
\end{description}

\end{frame}

% page --------------------------------------------------------------------------------------------------
\begin{frame}{Bemerkungen}

\begin{myemptytheorem}
Unit\"ares Skalarprodukt: Sei $\vec{x}, \vec{y} \in \mathbb{C}^n$.
Im Allgemeinen
\bq
 \vec{y} \cdot \vec{x} & \neq & \vec{x} \cdot \vec{y}.
\eq
Es ist
\bq
 \vec{y} \cdot \vec{x} & = & \left( \vec{x} \cdot \vec{y} \right)^\ast.
\eq
\end{myemptytheorem}
\begin{myemptytheorem}
Kreuzprodukt: Sei $\vec{x}, \vec{y} \in \mathbb{R}^3$.
Im Allgemeinen
\bq
 \vec{y} \times \vec{x} & \neq & \vec{x} \times \vec{y}.
\eq
Es ist
\bq
 \vec{y} \times \vec{x} & = & - \vec{x} \times \vec{y}.
\eq
\end{myemptytheorem}

\end{frame}

%%%%%%%%%%%%%%%%%%%%%%%%%%%%%%%%%%%%%%%%%%%%%%%%%%%%%%%%%%%%%%%%%%%%%%%%%%%%%%%%%%%%%%%%%%%%%%%%%%%%%%%%%
%%%%%%%%%%%%%%%%%%%%%%%%%%%%%%%%%%%%%%%%%%%%%%%%%%%%%%%%%%%%%%%%%%%%%%%%%%%%%%%%%%%%%%%%%%%%%%%%%%%%%%%%%
%%%%%%%%%%%%%%%%%%%%%%%%%%%%%%%%%%%%%%%%%%%%%%%%%%%%%%%%%%%%%%%%%%%%%%%%%%%%%%%%%%%%%%%%%%%%%%%%%%%%%%%%%

% page --------------------------------------------------------------------------------------------------
\begin{frame}

\end{frame}

\end{document}

