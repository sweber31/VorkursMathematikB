\documentclass[german]{beamer}

\mode<presentation>
{
 \usetheme{Madrid}

% \usecolortheme{crane}
 \usecolortheme{wolverine}
}

\usepackage{hyperref}

\usepackage[german]{babel}
\usepackage{times}
\usepackage[latin1,utf8]{inputenc}
\usepackage[OT2,T1]{fontenc}
\usepackage{shuffle}


% Stefan's abbreveations
\newcommand{\bq}{\begin{eqnarray*}}
\newcommand{\eq}{\end{eqnarray*}}
\newcommand{\eps}{\varepsilon}

%%%%%%%%%%%%%%%%%%%%%%%%%%%%%%%%%%%%%%%%%%%%%%%%%%%%%%%%%%%%%%%%%%%%%%%%%%%%%%%%%%%%%%%%%%%%%%%%%%%%%%%%%
%%%%%%%%%%%%%%%%%%%%%%%%%%%%%%%%%%%%%%%%%%%%%%%%%%%%%%%%%%%%%%%%%%%%%%%%%%%%%%%%%%%%%%%%%%%%%%%%%%%%%%%%%
%%%%%%%%%%%%%%%%%%%%%%%%%%%%%%%%%%%%%%%%%%%%%%%%%%%%%%%%%%%%%%%%%%%%%%%%%%%%%%%%%%%%%%%%%%%%%%%%%%%%%%%%%

\title{Mathematischer Br\"uckenkurs}

\subtitle{Einf\"uhrung}

\author{Stefan Weinzierl}

\institute[Uni Mainz]{Institut f\"ur Physik, Universit\"at Mainz}%

\date[WiSe 2020/21]{Wintersemester 2020/21}

\begin{document}

%%%%%%%%%%%%%%%%%%%%%%%%%%%%%%%%%%%%%%%%%%%%%%%%%%%%%%%%%%%%%%%%%%%%%%%%%%%%%%%%%%%%%%%%%%%%%%%%%%%%%%%%%
%%%%%%%%%%%%%%%%%%%%%%%%%%%%%%%%%%%%%%%%%%%%%%%%%%%%%%%%%%%%%%%%%%%%%%%%%%%%%%%%%%%%%%%%%%%%%%%%%%%%%%%%%
%%%%%%%%%%%%%%%%%%%%%%%%%%%%%%%%%%%%%%%%%%%%%%%%%%%%%%%%%%%%%%%%%%%%%%%%%%%%%%%%%%%%%%%%%%%%%%%%%%%%%%%%%

\begin{frame}
  \titlepage
\end{frame}

%%%%%%%%%%%%%%%%%%%%%%%%%%%%%%%%%%%%%%%%%%%%%%%%%%%%%%%%%%%%%%%%%%%%%%%%%%%%%%%%%%%%%%%%%%%%%%%%%%%%%%%%%
%%%%%%%%%%%%%%%%%%%%%%%%%%%%%%%%%%%%%%%%%%%%%%%%%%%%%%%%%%%%%%%%%%%%%%%%%%%%%%%%%%%%%%%%%%%%%%%%%%%%%%%%%
%%%%%%%%%%%%%%%%%%%%%%%%%%%%%%%%%%%%%%%%%%%%%%%%%%%%%%%%%%%%%%%%%%%%%%%%%%%%%%%%%%%%%%%%%%%%%%%%%%%%%%%%%

\section{Begr\"u{\ss}ung}

% page --------------------------------------------------------------------------------------------------
\begin{frame}{Willkommen an der Universit\"at Mainz!}

\begin{itemize}
\item Mathematik ist die Grundlage aller Naturwissenschaften.

\item Dieser Br\"uckenkurs richtet sich an Studienanf\"anger 
in naturwissenschaftlichen F\"achern (Biologie, Geowissenschaften, Physik, Chemie, ...)

\item Zeitumfang: Ganzt\"atig drei Wochen vor Semesterbeginn.

\end{itemize}
\end{frame}

% page --------------------------------------------------------------------------------------------------
\begin{frame}{Ziele des Br\"uckenkurses}

\begin{itemize}
\item Sie haben Themen aus der Schulmathematik vergessen: 

\alert{Auffrischen der Kenntnisse}.

\item Sie kommen von unterschiedlichen Schulen, aus verschiedenen (Bundes-) L\"andern
und haben in der Schule unterschiedliche optionale Themen behandelt:

\alert{Angleichen des Kenntnisstandes}.

\item Sie sind neu an der Universit\"at:

\alert{Kn\"upfen neuer sozialer Kontakte}.

\end{itemize}
\end{frame}

% page --------------------------------------------------------------------------------------------------
\begin{frame}{Uni ist nicht gleich Schule!}

\begin{itemize}
\item Mit der Uni beginnt ein neuer Lebensabschnitt.

\item Sie sind erwachsen und werden als erwachsene Menschen behandelt.

\item Im Allgemeinen keine Anwesenheitspflicht!

\end{itemize}
\end{frame}

% page --------------------------------------------------------------------------------------------------
\begin{frame}{Uni ist nicht gleich Schule:}

Die Kehrseite der Freiheit:
\begin{itemize}
\item Sie sind selbst verantwortlich, wie Sie lernen.

\item Stoffmenge und Tempo einer Vorlesung liegt deutlich \"uber einer Schulstunde.

\item In der Vorlesung wird ein neues Thema \alert{einmal} diskutiert, es wird nicht gewartet, bis es auch der Letzte verstanden hat.

\end{itemize}
\end{frame}

%%%%%%%%%%%%%%%%%%%%%%%%%%%%%%%%%%%%%%%%%%%%%%%%%%%%%%%%%%%%%%%%%%%%%%%%%%%%%%%%%%%%%%%%%%%%%%%%%%%%%%%%%
%%%%%%%%%%%%%%%%%%%%%%%%%%%%%%%%%%%%%%%%%%%%%%%%%%%%%%%%%%%%%%%%%%%%%%%%%%%%%%%%%%%%%%%%%%%%%%%%%%%%%%%%%
%%%%%%%%%%%%%%%%%%%%%%%%%%%%%%%%%%%%%%%%%%%%%%%%%%%%%%%%%%%%%%%%%%%%%%%%%%%%%%%%%%%%%%%%%%%%%%%%%%%%%%%%%

\section{Organisatorisches}

\frame{\sectionpage}

%%%%%%%%%%%%%%%%%%%%%%%%%%%%%%%%%%%%%%%%%%%%%%%%%%%%%%%%%%%%%%%%%%%%%%%%%%%%%%%%%%%%%%%%%%%%%%%%%%%%%%%%%
%%%%%%%%%%%%%%%%%%%%%%%%%%%%%%%%%%%%%%%%%%%%%%%%%%%%%%%%%%%%%%%%%%%%%%%%%%%%%%%%%%%%%%%%%%%%%%%%%%%%%%%%%
%%%%%%%%%%%%%%%%%%%%%%%%%%%%%%%%%%%%%%%%%%%%%%%%%%%%%%%%%%%%%%%%%%%%%%%%%%%%%%%%%%%%%%%%%%%%%%%%%%%%%%%%%

% page --------------------------------------------------------------------------------------------------
\begin{frame}{Corona}

Im Wintersemester 2020/21:

\begin{itemize}
\item Mathematischer Br\"uckenkurs A (Prof. T. Hurth):

\alert{Soviel Pr\"asenz wie m\"oglich}.

\item Mathematischer Br\"uckenkurs B (Prof. S. Weinzierl):

\alert{Rein Online}.

\end{itemize}
\end{frame}

% page --------------------------------------------------------------------------------------------------
\begin{frame}{Organisation}

Mathematischer Br\"uckenkurs B:

\vspace*{5mm}

\begin{tabular}{rl}
 9:15 & Vorlesung (via BigBlueButton, wird aufgezeichnet) \\
 11:30 & Plenumsdiskussion (via BigBlueButton) \\
 14:00 & \"Ubungsgruppen (via BigBlueButton) \\
\end{tabular}

\end{frame}

% page --------------------------------------------------------------------------------------------------
\begin{frame}{Organisation}

Als \alert{Plattform} wird \alert{BigBlueButton} verwendet:

\begin{itemize}
\item Link f\"ur Vorlesung und Plenumsdiskussion:
\href{https://bbb.rlp.net/b/wei-hgp-axv-fqv}{\footnotesize \texttt{https://bbb.rlp.net/b/wei-hgp-axv-fqv}}

\item \"Ubungsgruppen:
\begin{tabular}{ll}
 Gruppe 1 & \href{https://bbb.rlp.net/b/sch-ki2-awt-bxf}{\footnotesize \texttt{https://bbb.rlp.net/b/sch-ki2-awt-bxf}} \\
 Gruppe 2 & \href{https://bbb.rlp.net/b/sau-pff-7xa-fi5}{\footnotesize \texttt{https://bbb.rlp.net/b/sau-pff-7xa-fi5}} \\
 Gruppe 3 & \href{https://bbb.rlp.net/b/koc-jeb-38l-rvo}{\footnotesize \texttt{https://bbb.rlp.net/b/koc-jeb-38l-rvo}} \\
 Gruppe 4 & \href{https://bbb.rlp.net/b/kre-k13-ljh-nbc}{\footnotesize \texttt{https://bbb.rlp.net/b/kre-k13-ljh-nbc}} \\
\end{tabular}

\end{itemize}

\vspace*{10mm}

Webseite des Br\"uckenkurses
(\alert{Aktuelle Informationen}, \alert{Folien} der Vorlesung und \alert{\"Ubungsbl\"atter} als pdf-Dateien):
\href{https://particlephysics.uni-mainz.de/weinzierl/vorkurs/}{\footnotesize \texttt{https://particlephysics.uni-mainz.de/weinzierl/vorkurs/}}

\end{frame}

% page --------------------------------------------------------------------------------------------------
\begin{frame}{Studienf\"acher}

Sie studieren:
\begin{description}
\item{(A)} Biologie
\item{(B)} Geowissenschaften
\item{(C)} Chemie
\item{(D)} Physik
\item{(E)} sonstige F\"acher
\end{description}

\end{frame}

% page --------------------------------------------------------------------------------------------------
\begin{frame}{Ort}

Sie befinden sich jetzt 
\begin{description}
\item{(A)} in Mainz
\item{(B)} im Umkreis von 10 km um Mainz
\item{(C)} im Umkreis von 50 km um Mainz
\item{(D)} im restlichen Universum
\end{description}

\end{frame}

% page --------------------------------------------------------------------------------------------------
\begin{frame}{Quiz}

\bq
 \frac{3}{5} + \frac{2}{3} & = & ?
\eq 
\begin{description}
\item{(A)} $\frac{5}{8}$
\item{(B)} $\frac{5}{15}$
\item{(C)} $\frac{19}{15}$
\item{(D)} $\frac{2}{5}$
\end{description}

\end{frame}


% page --------------------------------------------------------------------------------------------------
\begin{frame}{Quiz}

Bestimmen Sie $x$:
\bq
 \frac{2x-3}{x+3} & = & 5
\eq 
\begin{description}
\item{(A)} $x=\frac{3}{2}$
\item{(B)} $x=-3$
\item{(C)} $x=-6$
\item{(D)} $x=\frac{5}{2}$
\end{description}

\end{frame}

% page --------------------------------------------------------------------------------------------------
\begin{frame}{Quiz}

\bq
 \log_2\left(32^4\right) & = & ?
\eq 
\begin{description}
\item{(A)} $\frac{5}{4}$
\item{(B)} $9$
\item{(C)} $20$
\item{(D)} $32$
\end{description}

\end{frame}

% page --------------------------------------------------------------------------------------------------
\begin{frame}{Quiz}

\bq
 f\left(x\right) & = & x^3 - 2 x^2 + 7 x +1
\eq 
Die Ableitung $f'(1)$ ist 
\begin{description}
\item{(A)} $0$
\item{(B)} $5$
\item{(C)} $6$
\item{(D)} $7$
\end{description}

\end{frame}

% page --------------------------------------------------------------------------------------------------
\begin{frame}{Quiz}

\bq
 \int\limits_0^1 \left( 3 x^2 - 6 x + 1 \right) dx & = & ?
\eq 
\begin{description}
\item{(A)} $-42$
\item{(B)} $-2$
\item{(C)} $-1$
\item{(D)} $7$
\end{description}

\end{frame}

% page --------------------------------------------------------------------------------------------------
\begin{frame}{Einteilung der \"Ubungsgruppen}

Sei $N$ die Anzahl der \"Ubungsgruppen.
\begin{itemize}
\item Sie nehmen die Zahl Ihres Geburtsmonats \\
(Januar $=1$, ..., Dezember $=12$).
\item Sie teilen diese Zahl durch $N$ mit Rest.
\item Falls ein Rest \"ubrig bleibt, gibt der Rest Ihre \"Ubungsgruppe an.
\item Bleibt kein Rest \"ubrig, so sind Sie in Gruppe $N$.
\end{itemize}

\end{frame}

% page --------------------------------------------------------------------------------------------------
\begin{frame}{Einteilung der \"Ubungsgruppen}

\begin{example}[3 \"Ubungsgruppen]
\begin{itemize}
\item Gruppe 1: Januar, April, Juli, Oktober
\item Gruppe 2: Februar, Mai, August, November
\item Gruppe 3: M\"arz, Juni, September, Dezember
\end{itemize}
\end{example}

\begin{example}[4 \"Ubungsgruppen]
\begin{itemize}
\item Gruppe 1: Januar, Mai, September
\item Gruppe 2: Februar, Juni, Oktober
\item Gruppe 3: M\"arz, Juli, November
\item Gruppe 4: April, August, Dezember
\end{itemize}
\end{example}

\end{frame}

% page --------------------------------------------------------------------------------------------------
\begin{frame}{Organisation}

Erste \"Ubungsgruppen am Montag, 12.10.2020, 14h:

\begin{itemize}
\item Kennenlernen.

\item Noch keine mathematischen \"Ubungen.

\item Nutzen Sie diese Gelegenheit, um Kontaktdaten auszutauschen!

\end{itemize}

\end{frame}

% page --------------------------------------------------------------------------------------------------
\begin{frame}{Literatur}

  \begin{thebibliography}{10}
    
  \beamertemplatebookbibitems
  % Anfangen sollte man mit Übersichtswerken.

  \bibitem{Brauner}
    R. Brauner, F. Gei{\ss}
    \newblock {\em Abiturwissen Mathematik}.
    \newblock Fischer-Verlag, 2004.

  \bibitem{Pross}
    S. Pro{\ss}, Th. Imkamp
    \newblock {\em Br\"uckenkurs Mathematik}.
    \newblock Springer-Verlag, 2018.

  \bibitem{Walz}
    G. Walz, F. Zeilfelder, Th. Rie{\ss}inger
    \newblock {\em Br\"uckenkurs Mathematik}.
    \newblock Springer-Verlag, 2019.

  \end{thebibliography}


\end{frame}


%%%%%%%%%%%%%%%%%%%%%%%%%%%%%%%%%%%%%%%%%%%%%%%%%%%%%%%%%%%%%%%%%%%%%%%%%%%%%%%%%%%%%%%%%%%%%%%%%%%%%%%%%
%%%%%%%%%%%%%%%%%%%%%%%%%%%%%%%%%%%%%%%%%%%%%%%%%%%%%%%%%%%%%%%%%%%%%%%%%%%%%%%%%%%%%%%%%%%%%%%%%%%%%%%%%
%%%%%%%%%%%%%%%%%%%%%%%%%%%%%%%%%%%%%%%%%%%%%%%%%%%%%%%%%%%%%%%%%%%%%%%%%%%%%%%%%%%%%%%%%%%%%%%%%%%%%%%%%

\section{Schreibweisen und Notation}

\frame{\sectionpage}

%%%%%%%%%%%%%%%%%%%%%%%%%%%%%%%%%%%%%%%%%%%%%%%%%%%%%%%%%%%%%%%%%%%%%%%%%%%%%%%%%%%%%%%%%%%%%%%%%%%%%%%%%
%%%%%%%%%%%%%%%%%%%%%%%%%%%%%%%%%%%%%%%%%%%%%%%%%%%%%%%%%%%%%%%%%%%%%%%%%%%%%%%%%%%%%%%%%%%%%%%%%%%%%%%%%
%%%%%%%%%%%%%%%%%%%%%%%%%%%%%%%%%%%%%%%%%%%%%%%%%%%%%%%%%%%%%%%%%%%%%%%%%%%%%%%%%%%%%%%%%%%%%%%%%%%%%%%%%

% page --------------------------------------------------------------------------------------------------
\begin{frame}{Notation}

\begin{itemize}

\item 
\alert{$\{a,b,c\}$}: Menge der Elemente $a$, $b$, und $c$.
\\
Die Ordnung spielt keine Rolle: $\{a,b,c\} = \{b,a,c\}$
\pause

\item
\alert{$a\in A$}: $a$ ist ein Element der Menge $A$.
\pause

\item
\alert{$A \subset B$}: Die Menge $A$ ist eine Teilmenge der Menge $B$.
\pause

\item Vereinigung: \alert{$A \cup B$} enth\"alt alle Elemente sowohl aus $A$ als auch aus $B$.
\\
$\{a,b,c\} \cup \{c,d,e\} = \{a,b,c,d,e\}$.
\pause

\item
Durchschnitt: \alert{$A \cap B$} enh\"alt alle Elemente die sowohl in $A$ als auch in $B$ enthalten sind.
\\
$\{a,b,c\} \cap \{c,d,e\} = \{c\}$.

\end{itemize}

\end{frame}

% page --------------------------------------------------------------------------------------------------
\begin{frame}{Notation}

\begin{itemize}

\item
Differenzmenge: \alert{$A \backslash B$} enth\"alt alle Elemente, die in $A$ enthalten sind, die aber nicht in $B$ enthalten sind.
\\
$\{a,b,c,d\} \backslash \{b,c\} = \{a,d\}$
\pause

\item
\alert{$A \times B$}: Produktmenge, dies ist die Menge aller geordneten Paare $(a,b)$, wobei $a \in A$ und $b \in B$ gilt.
\\
$\{a\} \times \{b,c\} = \{ (a,b), (a,c) \}$
\pause

\item
\alert{$[a,b]$}: Intervall, die Grenzen sind im Intervall enthalten: \\
$a\in[a,b]$, $b\in[a,b]$ 
\pause

\item
\alert{$]a,b[$}: Intervall, die Grenzen sind im Intervall nicht enthalten: \\
$a\notin[a,b]$, $b\notin[a,b]$ 
\pause

\item
Analog: \alert{$[a,b[$} und \alert{$]a,b]$}.

\end{itemize}

\end{frame}

% page --------------------------------------------------------------------------------------------------
\begin{frame}{Notation}

\begin{itemize}

\item 
Logisch und: \alert{$\wedge$}
{\footnotesize
\bq
 0 \wedge 0 & = & 0
 \nonumber \\
 0 \wedge 1 & = & 0
 \nonumber \\
 1 \wedge 0 & = & 0
 \nonumber \\
 1 \wedge 1 & = & 1
\eq
}
\pause

\item
Logisch oder: \alert{$\vee$}
{\footnotesize
\bq
 0 \vee 0 & = & 0
 \nonumber \\
 0 \vee 1 & = & 1
 \nonumber \\
 1 \vee 0 & = & 1
 \nonumber \\
 1 \vee 1 & = & 1
\eq
}
\pause

\item
Negation: \alert{$\neg$}
{\footnotesize
\bq
 \neg 0 & = & 1
 \nonumber \\
 \neg 1 & = & 0
\eq
}

\end{itemize}

\end{frame}

% page --------------------------------------------------------------------------------------------------
\begin{frame}{Notation}

\begin{itemize}

\item
\alert{$\exists$}: Es existiert
\pause

\item
\alert{$\forall$}: F\"ur alle
\pause

\item
\alert{$\infty$}: Symbol f\"ur Unendlich.
\pause

\item
\alert{$\mathbb{N}$}: Die nat\"urlichen Zahlen $1,2,3,\dots$
\pause

\item
\alert{$\mathbb{Z}$}: Die ganzen Zahlen $\dots,-2,-1,0,1,2,\dots$
\pause

\item
\alert{$\mathbb{Q}$}: Die rationalen Zahlen, z.B. $\frac{2}{3}$
\pause

\item
\alert{$\mathbb{R}$}: Die reellen Zahlen, z.B. $\sqrt{2}$
\pause

\item
\alert{$\mathbb{C}$}: Die komplexen Zahlen, z.B. $\sqrt{-2}$

\end{itemize}

\end{frame}

% page --------------------------------------------------------------------------------------------------
\begin{frame}{Notation}

\begin{itemize}

\item 
\alert{$\sum$}: Summenzeichen 
{\small
\bq
 \sum\limits_{j=1}^n a_j & = & a_1 + a_2 + a_3 + ... + a_{n-1} + a_{n}.
\eq
}
\pause

\item 
\alert{$\prod$}: Produktzeichen
{\small
\bq
 \prod\limits_{j=1}^n a_j & = & a_1 \cdot a_2 \cdot a_3 \cdot ... \cdot a_{n-1} \cdot a_{n}.
\eq
}
\pause

\item
\alert{$n!$}: Fakult\"at. 
{\small
\bq
 n! \; = \; 1 \cdot 2 \cdot ... \cdot (n-1) \cdot n,
 & &
 0! \; = \; 1. 
\eq
}
\pause

\item
\alert{$\left( \begin{array}{c} n \\ k \\ \end{array} \right)$}:
Binomialkoeffizient:
{\small
\bq
 \left( \begin{array}{c} n \\ k \\ \end{array} \right)
 & = & \frac{n!}{k!(n-k)!}.
\eq
}

\end{itemize}

\end{frame}

% page --------------------------------------------------------------------------------------------------
\begin{frame}{Notation}

\begin{itemize}

\item
\alert{$\lim\limits_{x\rightarrow a}$}: Grenzwert f\"ur den Fall, da{\ss} sich $x$ dem Wert $a$ ann\"ahert.
\pause

\item 
Ableitung: Sei $f(x)$ eine Funktion von $x$.
{\small
\bq
 \alert{f'(x)} & = & \frac{df(x)}{dx} = \frac{d}{dx} f(x) = \lim\limits_{h\rightarrow 0} \frac{f(x+h)-f(x)}{(x+h)-x}.
\eq
}
\pause

\item
Integral:
{\small
\bq
 \alert{\int\limits_a^b f(x) dx} & = & \lim\limits_{n\rightarrow\infty} \sum\limits_{j=1}^n f(\xi_j) \Delta x_j,
 \nonumber \\
 & &
 \Delta x_j = x_j - x_{j-1}, \; x_0=a, \; x_n=b, \; \xi_j\in[x_{j-1},x_j].
\eq
}

\end{itemize}

\end{frame}

% page --------------------------------------------------------------------------------------------------
\begin{frame}{Notation}

{\small
Neben lateinischen Buchstaben
verwendet man auch oft griechische Buchstaben:
}
{\footnotesize
\begin{center}
\begin{tabular}{llllllll}
\alert{$\alpha$} & alpha 
&&
\alert{$\beta$} & beta
&&
\alert{$\gamma$} & gamma
\\
\alert{$\delta$} & delta
&&
\alert{$\epsilon$} oder \alert{$\varepsilon$} & epsilon
&&
\alert{$\zeta$} & zeta
\\
\alert{$\eta$} & eta
&&
\alert{$\theta$} oder \alert{$\vartheta$} & theta
&&
\alert{$\iota$} & iota
\\
\alert{$\kappa$} & kappa
&&
\alert{$\lambda$} & lambda
&&
\alert{$\mu$} & mu
\\
\alert{$\nu$} & nu
&&
\alert{$\xi$} & xi
&&
\alert{$o$} & o
\\
\alert{$\pi$} oder \alert{$\varpi$} & pi
&&
\alert{$\rho$} oder \alert{$\varrho$} & rho
&&
\alert{$\sigma$} oder \alert{$\varsigma$} & sigma
\\
\alert{$\tau$} & tau
&&
\alert{$\upsilon$} & upsilon
&&
\alert{$\phi$} oder \alert{$\varphi$} & phi
\\
\alert{$\chi$} & chi
&&
\alert{$\psi$} & psi
&&
\alert{$\omega$} & omega
\\
\end{tabular}
\end{center}
}

\end{frame}

% page --------------------------------------------------------------------------------------------------
\begin{frame}{Notation}

{\small
Griechische Gro{\ss}buchstaben:
}
{\footnotesize
\begin{center}
\begin{tabular}{llllllll}
\alert{$A$} & Alpha 
&&
\alert{$B$} & Beta
&&
\alert{$\Gamma$} & Gamma
\\
\alert{$\Delta$} & Delta
&&
\alert{$E$} & Epsilon
&&
\alert{$Z$} & Zeta
\\
\alert{$H$} & Eta
&&
\alert{$\Theta$} & Theta
&&
\alert{$I$} & Iota
\\
\alert{$K$} & Kappa
&&
\alert{$\Lambda$} & Lambda
&&
\alert{$M$} & Mu
\\
\alert{$N$} & Nu
&&
\alert{$\Xi$} & Xi
&&
\alert{$O$} & O
\\
\alert{$\Pi$} & Pi
&&
\alert{$P$} & Rho
&&
\alert{$\Sigma$} & Sigma
\\
\alert{$T$} & Tau
&&
\alert{$\Upsilon$} & Upsilon
&&
\alert{$\Phi$} & Phi
\\
\alert{$X$} & Chi
&&
\alert{$\Psi$} & Psi
&&
\alert{$\Omega$} & Omega
\\
\end{tabular}
\end{center}
}

\end{frame}

% page --------------------------------------------------------------------------------------------------
\begin{frame}{Notation}

{\small
\begin{itemize}

\item Aus dem hebr\"aischen Alphabet:
\bq
 \alert{\aleph} && \mbox{Aleph}.
\eq
\"Ublicherweise wird dieser Buchstabe zur Beschreibung der M\"achtigkeit der nat\"urlichen Zahlen verwendet.

\item 
Aus dem kyrillischen Alphabet:
\bq
 \alert{\shuffle} & & \mbox{Sha}.
\eq
\"Ublicherweise verwendet man dieses Zeichen zur Notation f\"ur das Shuffle-Produkt.

Dies ist ein spezielles Produkt, wird aber in dieser Vorlesung nicht vorkommen.

\end{itemize}
}

\end{frame}

\end{document}
