\documentclass[german]{beamer}

\mode<presentation>
{
 \usetheme{Madrid}

% \usecolortheme{crane}
 \usecolortheme{wolverine}
}

\usepackage{hyperref}

\usepackage[german]{babel}
\usepackage{times}
\usepackage[latin1,utf8]{inputenc}
\usepackage[OT2,T1]{fontenc}
\usepackage{shuffle}


% Stefan's abbreveations
\newcommand{\bq}{\begin{eqnarray*}}
\newcommand{\eq}{\end{eqnarray*}}
\newcommand{\eps}{\varepsilon}

\definecolor{MyYellowOrange}{cmyk}{0,0.5,1,0}
\newcommand{\superalert}[1]{{\color{MyYellowOrange}{#1}}}
\newtheorem*{myemptytheorem}{}

% dedicated environments
\newtheorem*{mytheorem12}{Die Multiplikation einer $n\times k$-Matrix $A$ mit einer $k \times m$-Matrix $B$ ist wie folgt definiert:}
\newtheorem*{mytheorem13}{Die Addition zweier Matrizen mit gleicher Spalten- und Zeilenanzahl ist definiert durch}
\newtheorem*{mytheorem14}{Eine Multiplikation mit Skalaren ist definiert durch}
\newtheorem*{mytheorem15}{Die Spur (engl. ``trace'') einer quadratischen Matrix ist die Summe der Diagonalelemente:}
\newtheorem*{mytheorem16}{Die Determinante einer quadratischen $n \times n$-Matrix ist definiert durch}
\newtheorem*{mytheorem17}{Laplace'sche Entwicklungssatz:}


%%%%%%%%%%%%%%%%%%%%%%%%%%%%%%%%%%%%%%%%%%%%%%%%%%%%%%%%%%%%%%%%%%%%%%%%%%%%%%%%%%%%%%%%%%%%%%%%%%%%%%%%%
%%%%%%%%%%%%%%%%%%%%%%%%%%%%%%%%%%%%%%%%%%%%%%%%%%%%%%%%%%%%%%%%%%%%%%%%%%%%%%%%%%%%%%%%%%%%%%%%%%%%%%%%%
%%%%%%%%%%%%%%%%%%%%%%%%%%%%%%%%%%%%%%%%%%%%%%%%%%%%%%%%%%%%%%%%%%%%%%%%%%%%%%%%%%%%%%%%%%%%%%%%%%%%%%%%%

\title{Matrizen}

\subtitle{Mathematischer Br\"uckenkurs}

\author{Stefan Weinzierl}

\institute[Uni Mainz]{Institut f\"ur Physik, Universit\"at Mainz}%

\date[WiSe 2020/21]{Wintersemester 2020/21}

\begin{document}

%%%%%%%%%%%%%%%%%%%%%%%%%%%%%%%%%%%%%%%%%%%%%%%%%%%%%%%%%%%%%%%%%%%%%%%%%%%%%%%%%%%%%%%%%%%%%%%%%%%%%%%%%
%%%%%%%%%%%%%%%%%%%%%%%%%%%%%%%%%%%%%%%%%%%%%%%%%%%%%%%%%%%%%%%%%%%%%%%%%%%%%%%%%%%%%%%%%%%%%%%%%%%%%%%%%
%%%%%%%%%%%%%%%%%%%%%%%%%%%%%%%%%%%%%%%%%%%%%%%%%%%%%%%%%%%%%%%%%%%%%%%%%%%%%%%%%%%%%%%%%%%%%%%%%%%%%%%%%

\begin{frame}
  \titlepage
\end{frame}

%%%%%%%%%%%%%%%%%%%%%%%%%%%%%%%%%%%%%%%%%%%%%%%%%%%%%%%%%%%%%%%%%%%%%%%%%%%%%%%%%%%%%%%%%%%%%%%%%%%%%%%%%
%%%%%%%%%%%%%%%%%%%%%%%%%%%%%%%%%%%%%%%%%%%%%%%%%%%%%%%%%%%%%%%%%%%%%%%%%%%%%%%%%%%%%%%%%%%%%%%%%%%%%%%%%
%%%%%%%%%%%%%%%%%%%%%%%%%%%%%%%%%%%%%%%%%%%%%%%%%%%%%%%%%%%%%%%%%%%%%%%%%%%%%%%%%%%%%%%%%%%%%%%%%%%%%%%%%

\section{Definition}

\frame{\sectionpage}

% page --------------------------------------------------------------------------------------------------
\begin{frame}{Matrizen}

\begin{definition}
Eine rechteckige Anordnung 
\bq
\left( \begin{array}{cccc}
 a_{11} & a_{12} & ... & a_{1m} \\
 a_{21} & a_{22} & ... & a_{2m} \\
 ...    & ...    & ... & ...    \\
 a_{n1} & a_{n2} & ... & a_{nm} \\
\end{array} \right)
\eq
von Elementen $a_{ij}$ aus einem K\"orper nennt man Matrix.

Die Elemente $a_{ij}$ nennt man die Komponenten der Matrix.

Eine Matrix mit $n$ Zeilen und $m$ Spalten bezeichnet man als $n \times m$-Matrix.
\end{definition}

\end{frame}

% page --------------------------------------------------------------------------------------------------
\begin{frame}{Matrizen}

\begin{itemize}
\item Eine Matrix bezeichnet man als \alert{quadratisch}, falls $n=m$.

\item Eine Matrix bezeichnet man als \alert{Einheitsmatrix}, falls sie quadratisch ist und $a_{ij}=\delta_{ij}$.

\item Eine Matrix bezeichnet man als \alert{Diagonalmatrix}, falls sie quadratisch ist und $a_{ij}=0$ f\"ur alle $i \neq j$.

\item Eine Matrix bezeichnet man als \alert{obere Dreiecksmatrix}, falls sie quadratisch ist und $a_{ij}=0$ f\"ur alle $i>j$.
\end{itemize}

\end{frame}

% page --------------------------------------------------------------------------------------------------
\begin{frame}{Matrizen}

\begin{columns}[b]
\begin{column}{5cm}
\alert{Quadratisch}:
\bq
\left( \begin{array}{cccc}
 a_{11} & a_{12} & ... & a_{1n} \\
 a_{21} & a_{22} & ... & a_{2n} \\
 ...    & ...    & ... & ...    \\
 a_{n1} & a_{n2} & ... & a_{nn} \\
\end{array} \right)
\eq
%
\alert{Diagonalmatrix}:
\bq
\left( \begin{array}{cccc}
 \lambda_1 & 0 & ... & 0 \\
 0 & \lambda_2 & ... & 0 \\
 ...    & ...    & ... & ...    \\
 0 & 0 & ... & \lambda_n \\
\end{array} \right)
\eq
\end{column}
%
\begin{column}{5cm}
\alert{Einheitsmatrix}:
\bq
\left( \begin{array}{cccc}
 1 & 0 & ... & 0 \\
 0 & 1 & ... & 0 \\
 ...    & ...    & ... & ...    \\
 0 & 0 & ... & 1 \\
\end{array} \right)
\eq
%
\alert{obere Dreiecksmatrix}:
\bq
\left( \begin{array}{cccc}
 a_{11} & a_{12} & ... & a_{1n} \\
 0 & a_{22} & ... & a_{2n} \\
 ...    & ...    & ... & ...    \\
 0 & 0 & ... & a_{nn} \\
\end{array} \right)
\eq
\end{column}
\end{columns}

\end{frame}

% page --------------------------------------------------------------------------------------------------
\begin{frame}{Addition von Matrizen}

Seien $A$ und $B$ zwei $n \times m$-Matrizen.

\begin{mytheorem13}
{\scriptsize
\bq
\lefteqn{
\left( \begin{array}{cccc}
 a_{11} & a_{12} & ... & a_{1m} \\
 a_{21} & a_{22} & ... & a_{2m} \\
 ...    & ...    & ... & ...    \\
 a_{n1} & a_{n2} & ... & a_{nm} \\
\end{array} \right)
 +
\left( \begin{array}{cccc}
 b_{11} & b_{12} & ... & b_{1m} \\
 b_{21} & b_{22} & ... & b_{2m} \\
 ...    & ...    & ... & ...    \\
 b_{n1} & b_{n2} & ... & b_{nm} \\
\end{array} \right)
 } & &
 \nonumber \\ 
 & = &
\left( \begin{array}{cccc}
 a_{11} + b_{11} & a_{12} + b_{12} & ... & a_{1m} + b_{1m} \\
 a_{21} + b_{21} & a_{22} + b_{22} & ... & a_{2m} + b_{2m} \\
 ...    & ...    & ... & ...    \\
 a_{n1} + b_{n1} & a_{n2} + b_{n2} & ... & a_{nm} + b_{nm} \\
\end{array} \right)
\eq
}
\end{mytheorem13}

\end{frame}

% page --------------------------------------------------------------------------------------------------
\begin{frame}{Addition von Matrizen}

\begin{example}
\bq
\left( \begin{array}{ccc}
 1 & 2 & 3 \\
 4 & 5 & 6 \\
\end{array} \right)
 +
\left( \begin{array}{ccc}
 7 & 8 & 9 \\
 10 & 11 & 12 \\
\end{array} \right)
 & = &
\left( \begin{array}{ccc}
 1+7 & 2+8 & 3+9 \\
 4+10 & 5+11 & 6+12 \\
\end{array} \right)
 \nonumber \\
 & = &
\left( \begin{array}{ccc}
 8 & 10 & 12 \\
 14 & 16 & 18 \\
\end{array} \right)
\eq
\end{example}

\end{frame}

% page --------------------------------------------------------------------------------------------------
\begin{frame}{Multiplikation mit Skalaren}

\begin{mytheorem14}
{\small
\bq
\lambda \cdot
\left( \begin{array}{cccc}
 a_{11} & a_{12} & ... & a_{1m} \\
 a_{21} & a_{22} & ... & a_{2m} \\
 ...    & ...    & ... & ...    \\
 a_{n1} & a_{n2} & ... & a_{nm} \\
\end{array} \right)
& = & 
\left( \begin{array}{cccc}
 \lambda a_{11} & \lambda a_{12} & ... & \lambda a_{1m} \\
 \lambda a_{21} & \lambda a_{22} & ... & \lambda a_{2m} \\
 ...    & ...    & ... & ...    \\
 \lambda a_{n1} & \lambda a_{n2} & ... & \lambda a_{nm} \\
\end{array} \right)
\eq
}
\end{mytheorem14}

\end{frame}

% page --------------------------------------------------------------------------------------------------
\begin{frame}{Multiplikation mit Skalaren}

\begin{example}
\bq
 3 \cdot
\left( \begin{array}{ccc}
 1 & 2 & 3 \\
 4 & 5 & 6 \\
\end{array} \right)
 & = &
\left( \begin{array}{ccc}
 3 \cdot 1 & 3 \cdot 2 & 3 \cdot 3 \\
 3 \cdot 4 & 3 \cdot 5 & 3 \cdot 6 \\
\end{array} \right)
 \nonumber \\
 & = &
\left( \begin{array}{ccc}
 3 & 6 & 9 \\
 12 & 15 & 18 \\
\end{array} \right)
\eq
\end{example}

\end{frame}

% page --------------------------------------------------------------------------------------------------
\begin{frame}{Der Vektorraum der $n \times m$-Matrizen}

\begin{itemize}
\item Mit dieser Addition und dieser skalaren Multiplikation bilden die $n\times m$-Matrizen einen Vektorraum.

\item Die Dimension dieses Vektorraumes ist $n \cdot m$.

\item Eine Basis ist gegegeben durch die Matrizen $e_{ij}$,
{\small
\bq
 e_{ij} & = &
\left( \begin{array}{ccccccc}
 0 & ... & ... & 0 & ... & ... & 0 \\
 ... & ... & ... & ... & ... & ... & ...\\
 0 & ... & ... & 0 & ... & ... & 0 \\
 0 & ... & 0 & 1 & 0 & ... & 0 \\
 0 & ... & ... & 0 & ... & ... & 0 \\
 ... & ... & ... & ... & ... & ... & ...\\
 0 & ... & ... & 0 & ... & ... & 0 \\
\end{array} \right),
\eq
}

die nur in dem Eintrag in der $i$-ten Zeile und $j$-ten Spalte eine Eins haben, ansonsten nur Nullen.

\end{itemize}

\end{frame}

% page --------------------------------------------------------------------------------------------------
\begin{frame}{Multiplikation}

\begin{mytheorem12}
Das Ergebnis ist eine $n \times m$-Matrix $C$
{\footnotesize
\bq
\left( \begin{array}{cccc}
 a_{11} & a_{12} & ... & a_{1k} \\
 a_{21} & a_{22} & ... & a_{2k} \\
 ...    & ...    & ... & ...    \\
 a_{n1} & a_{n2} & ... & a_{nk} \\
\end{array} \right)
 \cdot
\left( \begin{array}{cccc}
 b_{11} & b_{12} & ... & b_{1m} \\
 b_{21} & b_{22} & ... & b_{2m} \\
 ...    & ...    & ... & ...    \\
 b_{k1} & b_{k2} & ... & b_{km} \\
\end{array} \right)
& = &
\left( \begin{array}{cccc}
 c_{11} & c_{12} & ... & c_{1m} \\
 c_{21} & c_{22} & ... & c_{2m} \\
 ...    & ...    & ... & ...    \\
 c_{n1} & c_{n2} & ... & c_{nm} \\
\end{array} \right),
\eq
}
wobei
\bq
 c_{ij} & = & a_{i1} b_{1j} + a_{i2} b_{2j} + ... + a_{ik} b_{kj}.
\eq
\end{mytheorem12}

\vspace*{2mm}
Regel: \hspace*{17mm} Zeile $\times$ Spalte

\end{frame}

% page --------------------------------------------------------------------------------------------------
\begin{frame}{Multiplikation}

\begin{example}
{\footnotesize
\bq
\left( \begin{array}{ccc}
 1 & 2 & 3 \\
 4 & 5 & 6 \\
\end{array} \right)
 \cdot
\left( \begin{array}{cc}
 7 & 8 \\
 9 & 10 \\
 11 & 12 \\
\end{array} \right)
 & = &
\left( \begin{array}{cc}
 1 \cdot 7 + 2 \cdot 9 + 3 \cdot 11 & 1 \cdot 8 + 2 \cdot 10 + 3 \cdot 12 \\
 4 \cdot 7 + 5 \cdot 9 + 6 \cdot 11 & 4 \cdot 8 + 5 \cdot 10 + 6 \cdot 12 \\
\end{array} \right)
 \nonumber \\
 & = &
\left( \begin{array}{cc}
 7 + 18 + 33 & 8 + 20 + 36 \\
 28 + 45 + 66 & 32 + 50 + 72 \\
\end{array} \right)
 \nonumber \\
 & = &
\left( \begin{array}{cc}
 58 & 64 \\
 139 & 154 \\
\end{array} \right)
\eq
}
\end{example}

\end{frame}

% page --------------------------------------------------------------------------------------------------
\begin{frame}{Quiz}

\bq
\left( \begin{array}{cc}
 1 & 2 \\
 3 & 4 \\
\end{array} \right)
 \cdot
\left( \begin{array}{c}
 1 \\
 2 \\
\end{array} \right)
 & = & ?
\eq
\begin{columns}[b]
\begin{column}{5cm}
\begin{description}
\item{(A)} $\left( \begin{array}{cc} 1 & 2 \\ 6 & 8 \\ \end{array} \right)$
\item{(C)} $\left( \begin{array}{cc} 7 & 10 \end{array} \right)$
\end{description}
\end{column}
\begin{column}{5cm}
\begin{description}
\item{(B)} $\left( \begin{array}{c} 5 \\ 11 \\ \end{array} \right)$
\item{(D)} $\left( \begin{array}{c} 17 \\ \end{array} \right)$
\end{description}
\end{column}
\end{columns}

\end{frame}

% page --------------------------------------------------------------------------------------------------
\begin{frame}{Quiz}

\bq
\left( \begin{array}{cc}
 1 \\
 2 \\
\end{array} \right)
 \cdot
\left( \begin{array}{cc}
 3 & 4 \\
\end{array} \right)
 & = & ?
\eq
\begin{columns}[b]
\begin{column}{5cm}
\begin{description}
\item{(A)} $\left( \begin{array}{cc} 3 & 4 \\ 6 & 8 \\ \end{array} \right)$
\item{(C)} $\left( \begin{array}{cc} 3 & 8 \end{array} \right)$
\end{description}
\end{column}
\begin{column}{5cm}
\begin{description}
\item{(B)} $\left( \begin{array}{c} 3 \\ 8 \\ \end{array} \right)$
\item{(D)} $\left( \begin{array}{c} 11 \\ \end{array} \right)$
\end{description}
\end{column}
\end{columns}

\end{frame}

% page --------------------------------------------------------------------------------------------------
\begin{frame}{Spaltenvektoren und Zeilenvektoren als Matrizen}

\begin{itemize}
\item Ein $n$-dimensionaler Spaltenvektor 
\bq
 \left(\begin{array}{c} a_{11} \\ a_{21} \\ \dots \\ a_{n1} \\ \end{array} \right)
\eq
kann als eine $n \times 1$-Matrix aufgefasst werden.
\item Ebenso kann ein $n$-dimensionaler Zeilenvektor 
\bq
 \left( a_{11}, a_{12}, \dots, a_{1n} \right)
\eq
als eine $1 \times n$-Matrix betrachtet werden.
\end{itemize}

\end{frame}

% page --------------------------------------------------------------------------------------------------
\begin{frame}{Lineare Gleichungssysteme}

Setzt man
{\small
\bq
A =  
\left( \begin{array}{cccc}
 a_{11} & a_{12} & ... & a_{1m} \\
 a_{21} & a_{22} & ... & a_{2m} \\
 ...    & ...    & ... & ...    \\
 a_{n1} & a_{n2} & ... & a_{nm} \\
\end{array} \right),
 \;\;\;
\vec{x} = \left(\begin{array}{c} x_1 \\ x_2 \\ ... \\ x_m \\ \end{array} \right),
 \;\;\;
\vec{b} = \left(\begin{array}{c} b_1 \\ b_2 \\ ... \\ b_n \\ \end{array} \right),
\eq
}
so l\"a{\ss}t sich das lineare Gleichungssystem
{\small
\bq
 a_{11} x_1 + a_{12} x_2 + a_{13} x_3 + ... + a_{1m} x_m & = & b_1,
\nonumber \\
 a_{21} x_1 + a_{22} x_2 + a_{23} x_3 + ... + a_{2m} x_m & = & b_2,
\nonumber \\
 ... & & 
\nonumber \\
 a_{n1} x_1 + a_{n2} x_2 + a_{n3} x_3 + ... + a_{nm} x_m & = & b_n.
\eq
}
auch wie folgt schreiben:
\bq
 A \cdot \vec{x} & = & \vec{b}.
\eq

\end{frame}

%%%%%%%%%%%%%%%%%%%%%%%%%%%%%%%%%%%%%%%%%%%%%%%%%%%%%%%%%%%%%%%%%%%%%%%%%%%%%%%%%%%%%%%%%%%%%%%%%%%%%%%%%
%%%%%%%%%%%%%%%%%%%%%%%%%%%%%%%%%%%%%%%%%%%%%%%%%%%%%%%%%%%%%%%%%%%%%%%%%%%%%%%%%%%%%%%%%%%%%%%%%%%%%%%%%
%%%%%%%%%%%%%%%%%%%%%%%%%%%%%%%%%%%%%%%%%%%%%%%%%%%%%%%%%%%%%%%%%%%%%%%%%%%%%%%%%%%%%%%%%%%%%%%%%%%%%%%%%

\section{Spuren und Determinanten}

\frame{\sectionpage}

% page --------------------------------------------------------------------------------------------------
\begin{frame}{Quadratische Matrizen}

Wir betrachten im folgenden die \alert{quadratischen} $n\times n$-Matrizen
\bq
 A & = & 
\left( \begin{array}{cccc}
 a_{11} & a_{12} & ... & a_{1n} \\
 a_{21} & a_{22} & ... & a_{2n} \\
 ...    & ...    & ... & ...    \\
 a_{n1} & a_{n2} & ... & a_{nn} \\
\end{array} \right)
\eq
und f\"uhren die Begriffe {\bf Spur} und {\bf Determinante} ein.

\end{frame}

% page --------------------------------------------------------------------------------------------------
\begin{frame}{Spur}

Sei $A$ eine $n\times n$-Matrix:
\bq
 A & = & 
\left( \begin{array}{cccc}
 a_{11} & a_{12} & ... & a_{1n} \\
 a_{21} & a_{22} & ... & a_{2n} \\
 ...    & ...    & ... & ...    \\
 a_{n1} & a_{n2} & ... & a_{nn} \\
\end{array} \right).
\eq
\begin{mytheorem15}
\bq
\mbox{Tr}\; A & = & \sum\limits_{i=1}^n a_{ii}
 = a_{11} + a_{22} + ... + a_{nn}.
\eq
\end{mytheorem15}

\end{frame}

% page --------------------------------------------------------------------------------------------------
\begin{frame}{Rechenregeln f\"ur die Spur}

Es seien $A$ und $B$ $n\times n$-Matrizen, $\lambda$ ein Skalar:
\bq
 \mbox{Tr} \; \left( A + B \right) & = & \mbox{Tr}\; A + \mbox{Tr}\; B,
 \nonumber \\
 \mbox{Tr} \; \left( \lambda \cdot A \right) & = & \lambda \; \mbox{Tr} \; A.
\eq

\end{frame}

% page --------------------------------------------------------------------------------------------------
\begin{frame}{Spur}

\begin{example}
\bq
 \mbox{Tr}
\left( \begin{array}{rrrr}
 1 & 2 & 3 & 4 \\
 5 & 6 & 7 & 8 \\
 9 & 10 & 11 & 12 \\
 13 & 14 & 15 & 16 \\
\end{array} \right)
 & = & 
 1 + 6 + 11 +16
 \; = \;
 34
\eq
\end{example}

\end{frame}

% page --------------------------------------------------------------------------------------------------
\begin{frame}{Quiz}

\bq
 \mbox{Tr}
\left( \begin{array}{rrrr}
 1 & 0 & 0 & 42 \\
 0 & 1 & 0 & 0 \\
 0 & 0 & 1 & 0 \\
 3 & 0 & 0 & 0 \\
\end{array} \right)
 & = & ?
\eq
\begin{description}
\item{(A)} $\sqrt{3}$
\item{(B)} $3$
\item{(C)} $6$
\item{(D)} $45$
\end{description}

\end{frame}

% page --------------------------------------------------------------------------------------------------
\begin{frame}{Determinante}

\begin{mytheorem16}
\bq
 \det A & = & 
 \sum\limits_{i_1=1}^n \sum\limits_{i_2=1}^n ... \sum\limits_{i_n=1}^n
 \eps_{i_1 i_2 ... i_n} a_{1i_1} a_{2i_2} ... a_{ni_n},
\eq
wobei $\eps_{i_1 i_2 ... i_n}$ das total antisymmetrische Symbol in $n$ Dimensionen ist
{\footnotesize
\bq
 \eps_{i_1 i_2 ... i_n} & = & \left\{
 \begin{array}{rl}
   +1 & \mbox{f\"ur $(i_1 i_2 ... i_n)$ eine gerade Permutation von $(1,2,...,n)$,} \\
   -1 & \mbox{f\"ur $(i_1 i_2 ... i_n)$ eine ungerade Permutation von $(1,2,...,n)$,} \\
   0 & \mbox{sonst}. \\
 \end{array}
\right.
\eq
}
\end{mytheorem16}

\end{frame}

% page --------------------------------------------------------------------------------------------------
\begin{frame}{Determinante}

F\"ur die Determinante existiert auch die folgende Schreibweise
\bq
 \det A & = & 
\left| \begin{array}{cccc}
 a_{11} & a_{12} & ... & a_{1n} \\
 a_{21} & a_{22} & ... & a_{2n} \\
 ...    & ...    & ... & ...    \\
 a_{n1} & a_{n2} & ... & a_{nn} \\
\end{array} \right|.
\eq

\end{frame}

% page --------------------------------------------------------------------------------------------------
\begin{frame}{Determinante einer Diagonalmatrix}

Sei $D=\mbox{diag}(\lambda_1, \lambda_2, ..., \lambda_n)$ eine Diagonalmatrix.

\vspace*{5mm} 

Dann ist
\bq
 \det D & = & \lambda_1 \cdot \lambda_2 \cdot ... \cdot \lambda_n.
\eq

\vspace*{10mm}

Eine $1 \times 1$-Matrix ist immer eine Diagonalmatrix und somit
\bq
 \left| a_{11} \right| & = & a_{11}.
\eq

\end{frame}

% page --------------------------------------------------------------------------------------------------
\begin{frame}{Berechnung der Determinante}

Zu einer $n\times n$-Matrix $A$ definieren wir zun\"achst eine $(n-1)\times (n-1)$-Matrix $A_{ij}$, 
die dadurch ensteht, da{\ss} man die $i$-te Zeile und die $j$-te Spalte der Matrix $A$ entfernt.

\begin{mytheorem17}
Entwicklung nach der $i$-ten Zeile:
{\small
\bq
 \det A & = & \sum\limits_{j=1}^n (-1)^{i+j} a_{ij} \det A_{ij}.
\eq
}
\"Aquivalent kann auch nach der $j$-ten Spalte entwickelt werden:
{\small
\bq
 \det A & = & \sum\limits_{i=1}^n (-1)^{i+j} a_{ij} \det A_{ij}.
\eq
}
\end{mytheorem17}

Dies erlaubt die rekursive Berechnung einer Determinante. 

\end{frame}

% page --------------------------------------------------------------------------------------------------
\begin{frame}{Rechenregeln f\"ur die Determinante:}

Es seien $A$ und $B$ $n\times n$-Matrizen, $\lambda$ ein Skalar:
\bq
 \det \left( A \cdot B \right) & = & \left( \det A \right) \cdot \left( \det B \right),
 \nonumber \\
 \det \left( \lambda \cdot A \right) & = & \lambda ^n \cdot \det A.
\eq

\end{frame}

% page --------------------------------------------------------------------------------------------------
\begin{frame}{Determinante}

\begin{example}
\bq
\left| \begin{array}{cccc}
 1 & 0 & 0 & 0 \\
 2 & 0 & 3 & 0 \\
 4 & 5 & 6 & 7 \\
 8 & 9 & 10 & 11 \\
\end{array} \right|
 & = &
 1 \cdot
\left| \begin{array}{ccc}
 0 & 3 & 0 \\
 5 & 6 & 7 \\
 9 & 10 & 11 \\
\end{array} \right|
 \; = \;
 -3 \cdot
\left| \begin{array}{cc}
 5 & 7 \\
 9 & 11 \\
\end{array} \right|
 \nonumber \\
 & = & 
 -3 \cdot \left( 5 \cdot 11 - 7 \cdot 9 \right)
 \; = \;
 24 
\eq
\end{example}

\end{frame}

% page --------------------------------------------------------------------------------------------------
\begin{frame}{Quiz}

\bq
\left| \begin{array}{cccc}
 1 & 0 & 0 & 42 \\
 0 & 2 & 0 & 0 \\
 0 & 0 & 3 & 0 \\
 0 & 0 & 0 & 4 \\
\end{array} \right|
 & = & ?
\eq
\begin{description}
\item{(A)} $0$
\item{(B)} $10$
\item{(C)} $24$
\item{(D)} $-228$
\end{description}

\end{frame}

% page --------------------------------------------------------------------------------------------------
\begin{frame}{Quadratische Matrizen}

Seien $A$ und $B$ zwei $n \times n$-Matrizen. In diesem Fall ist das Matrixprodukt
\bq
 A \cdot B
\eq
wieder eine $n \times n$-Matrix. 

F\"ur $n \times n$-Matrizen ist die Matrizenmultiplikation also abgeschlossen.

\vspace*{5mm}

Das neutrale Element bez\"uglich der Matrizenmultiplikation ist offensichtlich die Einheitsmatrix
{\small
\bq
 {\bf 1}  & = & 
\left( \begin{array}{ccccc}
 1 & 0 & ... & ... & 0 \\
 0 & 1 & & & 0 \\
 ... & & ... & & ... \\
 0 & & & 1 & 0 \\
 0 & ... & ... & 0 & 1 \\
\end{array} \right).
\eq
}

\end{frame}

% page --------------------------------------------------------------------------------------------------
\begin{frame}{Die inverse Matrix}

Unter welchen Bedingungen existiert auch ein inverses Element?

Falls so ein Element existiert bezeichnen wir es mit $A^{-1}$. Es soll also gelten
\bq
 A \cdot A^{-1} & = & {\bf 1}.
\eq
Nehmen wir auf beiden Seiten die Determinante, so erhalten wir
\bq
 \det A \cdot \det A^{-1} & = & \det {\bf 1} = 1,
\eq
also falls $\det A \neq 0$
\bq
 \det A^{-1} & = & \frac{1}{\det A}.
\eq
$\det A \neq 0$ ist eine notwendige Bedingung f\"ur die Existenz eines Inversen. 

\end{frame}

% page --------------------------------------------------------------------------------------------------
\begin{frame}{Die inverse Matrix}

Es l\"a{\ss}t sich zeigen, da{\ss} $\det A \neq 0$ 
auch eine hinreichende Bedingung ist.

\begin{theorem}
$A^{-1}$ existiert genau dann, wenn $\det A \neq 0$.
\end{theorem}

\end{frame}

% page --------------------------------------------------------------------------------------------------
\begin{frame}{Die Gruppe der invertierbaren Matrizen}

Wir betrachten nun die Menge aller quadratischen $n \times n$-Matrizen mit der Eigenschaft $\det A \neq 0$.

Wegen $\det(A B) = \det A \det B$ ist diese Menge abgeschlossen bez\"uglich der Matrizenmultiplikation.

Wie gerade diskutiert wurde, existiert in dieser Menge zu jeder Matrix auch ein Inverses.

Diese Menge bildet daher bez\"uglich der Matrizenmultiplikation eine Gruppe, die man als
\bq
 GL\left(n,{\mathbb R}\right),
\;\;\;\mbox{bzw.}\;\;\;
 GL\left(n,{\mathbb C}\right)
\eq
bezeichnet.

\end{frame}

%%%%%%%%%%%%%%%%%%%%%%%%%%%%%%%%%%%%%%%%%%%%%%%%%%%%%%%%%%%%%%%%%%%%%%%%%%%%%%%%%%%%%%%%%%%%%%%%%%%%%%%%%
%%%%%%%%%%%%%%%%%%%%%%%%%%%%%%%%%%%%%%%%%%%%%%%%%%%%%%%%%%%%%%%%%%%%%%%%%%%%%%%%%%%%%%%%%%%%%%%%%%%%%%%%%
%%%%%%%%%%%%%%%%%%%%%%%%%%%%%%%%%%%%%%%%%%%%%%%%%%%%%%%%%%%%%%%%%%%%%%%%%%%%%%%%%%%%%%%%%%%%%%%%%%%%%%%%%

\section{Berechnung der inversen Matrix}

\frame{\sectionpage}

% page --------------------------------------------------------------------------------------------------
\begin{frame}{Die inverse Matrix}

Sei $A$ eine $n \times n$-Matrix mit $\det A \neq 0$. Gesucht ist eine $n \times n$-Matrix $X$
\bq
 X & = & 
\left( \begin{array}{cccc}
 x_{11} & x_{12} & ... & x_{1n} \\
 x_{21} & x_{22} & ... & x_{2n} \\
 ...    & ...    & ... & ...    \\
 x_{n1} & x_{n2} & ... & x_{nn} \\
\end{array} \right),
\eq
so da{\ss} gilt:
\bq
 A \cdot X & = & {\bf 1}
\eq

\end{frame}

% page --------------------------------------------------------------------------------------------------
\begin{frame}{Die inverse Matrix}

Wir multiplizieren die linke Seite aus und betrachten danach die $j$-te Spalte auf beiden Seiten:
\bq
a_{11} x_{1j} + a_{12} x_{2j} + ... + a_{1n} x_{nj} & = & 0, 
 \nonumber \\
 ... & = & 0
 \nonumber \\
a_{j1} x_{1j} + a_{j2} x_{2j} + ... + a_{jn} x_{nj} & = & 1, 
 \nonumber \\
 ... & = & 0
 \nonumber \\
a_{n1} x_{1j} + a_{n2} x_{2j} + ... + a_{nn} x_{nj} & = & 0.
\eq
Diese $n$ Gleichungen bilden eine lineares Gleichungssystem f\"ur die Unbekannten $x_{1j}$, $x_{2j}$, ..., $x_{nj}$,
welches mit Hilfe des Gau{\ss}schen Algorithmus gel\"ost werden kann.

\end{frame}

% page --------------------------------------------------------------------------------------------------
\begin{frame}{Die inverse Matrix}

Da dies f\"ur jede Spalte $j$ gilt, kann man so alle $n^2$ Unbekannten $x_{ij}$ bestimmen.

Da die Koeffizienten der linken Seite des linearen Gleichungssystems immer gleich sind, verf\"ahrt man in der Praxis wie
folgt:

Man schreibt die Gleichungen wie folgt an
{\small
\bq
\begin{array}{cccc|cccc}
 a_{11} & a_{12} & ... & a_{1n} & 1 & 0 & ... & 0 \\
 a_{21} & a_{22} & ... & a_{2n} & 0 & 1 & ... & 0 \\
 ...    & ...    & ... & ...    & ... & ... & ... & ... \\
 a_{n1} & a_{n2} & ... & a_{nn} & 0 & 0 & ... & 1 \\
\end{array}
\eq
}

und bringt dieses Gleichungssystem mit Hilfe des Gau{\ss}'schen Algorithmus auf die Form
{\small
\bq
\begin{array}{cccc|cccc}
 1 & 0 & ... & 0 & x_{11} & x_{12} & ... & x_{1n} \\
 0 & 1 & ... & 0 & x_{21} & x_{22} & ... & x_{2n} \\
 ... & ... & ... & ... & ...    & ...    & ... & ...    \\
 0 & 0 & ... & 1 & x_{n1} & x_{n2} & ... & x_{nn} \\
\end{array} 
\eq
}

\end{frame}

% page --------------------------------------------------------------------------------------------------
\begin{frame}{Die inverse Matrix}

Die inverse Matrix $A^{-1}$ ist dann gegeben durch
\bq
A^{-1} & = &
\left( \begin{array}{cccc}
 x_{11} & x_{12} & ... & x_{1n} \\
 x_{21} & x_{22} & ... & x_{2n} \\
 ...    & ...    & ... & ...    \\
 x_{n1} & x_{n2} & ... & x_{nn} \\
\end{array} \right).
\eq

\end{frame}

% page --------------------------------------------------------------------------------------------------
\begin{frame}{Beispiel}

Sei $A$ die Matrix
\bq
 A & = &
\left( \begin{array}{ccc}
 1 &  1 &  3 \\
 2 &  3 &  7 \\
 0 &  1 &  4 \\
\end{array} \right).
\eq
Wir beginnen mit
\begin{center}
\begin{tabular}{rrr|rrr}
 1 &  1 &  3 & 1 & 0 & 0 \\
 2 &  3 &  7 & 0 & 1 & 0 \\
 0 &  1 &  4 & 0 & 0 & 1 \\
\end{tabular}
\end{center}


\end{frame}

% page --------------------------------------------------------------------------------------------------
\begin{frame}{Berechnung der inversen Matrix}

{\footnotesize
\begin{center}
\begin{tabular}{rrr|rrrl}
 1 &  1 &  3 & 1 & 0 & 0 & \\
 2 &  3 &  7 & 0 & 1 & 0 & \mbox{Addiere das $(-2)$-fache der 1. Zeile} \\
 0 &  1 &  4 & 0 & 0 & 1 & \\
 & & & & \\
 1 &  1 &  3 & 1 & 0 & 0 & \mbox{Addiere das $(-1)$-fache der 2. Zeile} \\
 0 &  1 &  1 & -2 & 1 & 0 & \\
 0 &  1 &  4 & 0 & 0 & 1 & \mbox{Addiere das $(-1)$-fache der 2. Zeile} \\
 & & & & \\
 1 &  0 &  2 & 3 & -1 & 0 & \\
 0 &  1 &  1 & -2 & 1 & 0 & \\
 0 &  0 &  3 & 2 & -1 & 1 & \mbox{Multipliziere mit $\frac{1}{3}$} \\
 & & & & \\
 1 &  0 &  2 & 3 & -1 & 0 & \mbox{Addiere das $(-2)$-fache der 3. Zeile} \\
 0 &  1 &  1 & -2 & 1 & 0 & \mbox{Addiere das $(-1)$-fache der 3. Zeile} \\
 0 &  0 &  1 & $\frac{2}{3}$ & $-\frac{1}{3}$ & $\frac{1}{3}$ & \\
 & & & & \\
 1 &  0 &  0 & $\frac{5}{3}$ & $-\frac{1}{3}$ & $-\frac{2}{3}$ & \\
 0 &  1 &  0 & $-\frac{8}{3}$ & $\frac{4}{3}$ & $-\frac{1}{3}$ & \\
 0 &  0 &  1 & $\frac{2}{3}$ & $-\frac{1}{3}$ & $\frac{1}{3}$ & \\
\end{tabular}
\end{center}
}

\end{frame}

% page --------------------------------------------------------------------------------------------------
\begin{frame}{Berechnung der inversen Matrix}

\begin{center}
\begin{tabular}{rrr|rrrl}
 1 &  0 &  0 & $\frac{5}{3}$ & $-\frac{1}{3}$ & $-\frac{2}{3}$ & \\
 0 &  1 &  0 & $-\frac{8}{3}$ & $\frac{4}{3}$ & $-\frac{1}{3}$ & \\
 0 &  0 &  1 & $\frac{2}{3}$ & $-\frac{1}{3}$ & $\frac{1}{3}$ & \\
\end{tabular}
\end{center}
Somit ist $A^{-1}$ gegeben durch
\bq
 A^{-1}
 & = &
 \frac{1}{3}
\left( \begin{array}{rrr}
 5 &  -1 &  -2 \\
 -8 &  4 &  -1 \\
 2 &  -1 &  1 \\
\end{array} \right).
\eq

\end{frame}

%%%%%%%%%%%%%%%%%%%%%%%%%%%%%%%%%%%%%%%%%%%%%%%%%%%%%%%%%%%%%%%%%%%%%%%%%%%%%%%%%%%%%%%%%%%%%%%%%%%%%%%%%
%%%%%%%%%%%%%%%%%%%%%%%%%%%%%%%%%%%%%%%%%%%%%%%%%%%%%%%%%%%%%%%%%%%%%%%%%%%%%%%%%%%%%%%%%%%%%%%%%%%%%%%%%
%%%%%%%%%%%%%%%%%%%%%%%%%%%%%%%%%%%%%%%%%%%%%%%%%%%%%%%%%%%%%%%%%%%%%%%%%%%%%%%%%%%%%%%%%%%%%%%%%%%%%%%%%

% page --------------------------------------------------------------------------------------------------
\begin{frame}

\end{frame}

\end{document}



