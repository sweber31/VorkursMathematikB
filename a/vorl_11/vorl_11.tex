\documentclass[german]{beamer}

\mode<presentation>
{
 \usetheme{Madrid}

% \usecolortheme{crane}
 \usecolortheme{wolverine}
}

\usepackage{hyperref}

\usepackage[german]{babel}
\usepackage{times}
\usepackage[latin1,utf8]{inputenc}
\usepackage[OT2,T1]{fontenc}
\usepackage{shuffle}


% Stefan's abbreveations
\newcommand{\bq}{\begin{eqnarray*}}
\newcommand{\eq}{\end{eqnarray*}}
\newcommand{\eps}{\varepsilon}

\definecolor{MyYellowOrange}{cmyk}{0,0.5,1,0}
\newcommand{\superalert}[1]{{\color{MyYellowOrange}{#1}}}
\newtheorem*{myemptytheorem}{}

% dedicated environments
\newtheorem*{mytheorem28}{Zusammenfassung:} 

%%%%%%%%%%%%%%%%%%%%%%%%%%%%%%%%%%%%%%%%%%%%%%%%%%%%%%%%%%%%%%%%%%%%%%%%%%%%%%%%%%%%%%%%%%%%%%%%%%%%%%%%%
%%%%%%%%%%%%%%%%%%%%%%%%%%%%%%%%%%%%%%%%%%%%%%%%%%%%%%%%%%%%%%%%%%%%%%%%%%%%%%%%%%%%%%%%%%%%%%%%%%%%%%%%%
%%%%%%%%%%%%%%%%%%%%%%%%%%%%%%%%%%%%%%%%%%%%%%%%%%%%%%%%%%%%%%%%%%%%%%%%%%%%%%%%%%%%%%%%%%%%%%%%%%%%%%%%%

\title{Differentialgleichungen}

\subtitle{Mathematischer Br\"uckenkurs}

\author{Stefan Weinzierl}

\institute[Uni Mainz]{Institut f\"ur Physik, Universit\"at Mainz}%

\date[WiSe 2020/21]{Wintersemester 2020/21}

\begin{document}

%%%%%%%%%%%%%%%%%%%%%%%%%%%%%%%%%%%%%%%%%%%%%%%%%%%%%%%%%%%%%%%%%%%%%%%%%%%%%%%%%%%%%%%%%%%%%%%%%%%%%%%%%
%%%%%%%%%%%%%%%%%%%%%%%%%%%%%%%%%%%%%%%%%%%%%%%%%%%%%%%%%%%%%%%%%%%%%%%%%%%%%%%%%%%%%%%%%%%%%%%%%%%%%%%%%
%%%%%%%%%%%%%%%%%%%%%%%%%%%%%%%%%%%%%%%%%%%%%%%%%%%%%%%%%%%%%%%%%%%%%%%%%%%%%%%%%%%%%%%%%%%%%%%%%%%%%%%%%

\begin{frame}
  \titlepage
\end{frame}

%%%%%%%%%%%%%%%%%%%%%%%%%%%%%%%%%%%%%%%%%%%%%%%%%%%%%%%%%%%%%%%%%%%%%%%%%%%%%%%%%%%%%%%%%%%%%%%%%%%%%%%%%
%%%%%%%%%%%%%%%%%%%%%%%%%%%%%%%%%%%%%%%%%%%%%%%%%%%%%%%%%%%%%%%%%%%%%%%%%%%%%%%%%%%%%%%%%%%%%%%%%%%%%%%%%
%%%%%%%%%%%%%%%%%%%%%%%%%%%%%%%%%%%%%%%%%%%%%%%%%%%%%%%%%%%%%%%%%%%%%%%%%%%%%%%%%%%%%%%%%%%%%%%%%%%%%%%%%

%%%%%%%%%%%%%%%%%%%%%%%%%%%%%%%%%%%%%%%%%%%%%%%%%%%%%%%%%%%%%%%%%%%%%%%%%%%%%%%%%%%%%%%%%%%%%%%%%%%%%%%%%
%%%%%%%%%%%%%%%%%%%%%%%%%%%%%%%%%%%%%%%%%%%%%%%%%%%%%%%%%%%%%%%%%%%%%%%%%%%%%%%%%%%%%%%%%%%%%%%%%%%%%%%%%
%%%%%%%%%%%%%%%%%%%%%%%%%%%%%%%%%%%%%%%%%%%%%%%%%%%%%%%%%%%%%%%%%%%%%%%%%%%%%%%%%%%%%%%%%%%%%%%%%%%%%%%%%

\section{Allgemeines}

\frame{\sectionpage}

% page --------------------------------------------------------------------------------------------------
\begin{frame}{Einf\"uhrung}

Es sei $f(x)$ eine unbekannte Funktion der Variablen $x$.

Nehmen wir weiter an, es sei bekannt, da{\ss} $f(x)$ die Gleichung
\bq
 f(x)^2 - x \cdot f(x) - 1 & = & 0
\eq
erf\"ullt. 

Dies ist eine \alert{algebraische Gleichung} f\"ur $f(x)$.

Durch Aufl\"osen nach $f(x)$ finden wir
\bq
 f\left(x\right)
 & = &
 \frac{1}{2} \left( x \pm \sqrt{x^2+4} \right)
\eq

\end{frame}

% page --------------------------------------------------------------------------------------------------
\begin{frame}{Einf\"uhrung}

In den Naturwissenschaften tritt oft der Fall auf, da{\ss} wir eine Gleichung bestimmen k\"onnen,
die die \alert{unbekannte Funktion $f(x)$} \superalert{und deren Ableitung $f'(x)$} enth\"alt.

\begin{example}
\bq
 f'\left(x\right) - x \cdot f\left(x\right) & = & 0
\eq
\end{example}

Eine solche Gleichung nennt man eine \superalert{Differentialgleichung}.

\end{frame}

% page --------------------------------------------------------------------------------------------------
\begin{frame}{Einf\"uhrung}

\begin{itemize}
\item Die Theorie der Differentialgleichungen geht weit \"uber den Inhalt des 
mathematischen Br\"uckenkurses hinaus.
\item In den Naturwissenschaften treten einige wenige Differentialgleichungen relativ oft auf.
\item In dieser Vorlesung: Einstieg in die Differentialgleichungen mittels wichtiger Beispiele und 
elementarer L\"osungsmethoden.
\end{itemize}

\end{frame}

% page --------------------------------------------------------------------------------------------------
\begin{frame}{Klassifizierung}

\begin{itemize}
\item Tritt nur die Ableitung nach einer Variablen auf, spricht man von einer
\superalert{gew\"ohnlichen Differentialgleichung}. 

\item H\"angt dagegen die gesuchte Funktion von mehreren Variablen ab, und
treten Ableitungen nach verschiedenen Variablen auf, so spricht man von einer
\superalert{partiellen Differentialgleichung}.

\end{itemize}

\end{frame}

% page --------------------------------------------------------------------------------------------------
\begin{frame}{Gew\"ohnliche Differentialgleichungen}

\begin{itemize}
\item Tritt neben der unbekannten Funktion $f$ \alert{nur die erste Ableitung $f'$} auf, so spricht man von
einer \superalert{Differentialgleichung erster Ordnung}.

\item Ist die \alert{h\"ochste auftretende Ableitung $f^{(n)}$}, so spricht man von einer
\superalert{Differentialgleichung $n$-ter Ordnung}.
\end{itemize}

\end{frame}

% page --------------------------------------------------------------------------------------------------
\begin{frame}{Gew\"ohnliche Differentialgleichungen}

\begin{definition}
Sei $D$ eine Teilmenge von ${\mathbb R}^2$ und
\bq
 G & : & D \rightarrow {\mathbb R},
 \nonumber \\
 & & (x,y) \rightarrow G(x,y)
\eq
eine stetige Funktion.
Dann nennt man
\bq
 y' & = & G(x,y)
\eq
eine \superalert{Differentialgleichung erster Ordnung}. 
\end{definition}

\end{frame}

% page --------------------------------------------------------------------------------------------------
\begin{frame}{Gew\"ohnliche Differentialgleichungen}

\begin{myemptytheorem}
Unter einer L\"osung versteht man
eine auf einem Intervall $I \subset \mathbb R$ definierte differenzierbare Funktion
\bq
 f & : & I \rightarrow {\mathbb R}
\eq
mit folgenden Eigenschaften:
\begin{itemize}
\item Der Graph von $f$ ist in $D$ enthalten, d.h.
\bq
 \Gamma_f = \{(x,y) \in I \times {\mathbb R} : y=f(x) \} \subset D.
\eq
\item Es gilt
\bq
 f'(x) & = & G(x,f(x)).
\eq
\end{itemize}
\end{myemptytheorem}

\end{frame}

% page --------------------------------------------------------------------------------------------------
\begin{frame}{Gew\"ohnliche Differentialgleichungen}

\begin{example}
$G(x,y)=-\lambda y$ f\"uhrt auf die Differentialgleichung
\bq
 f'(x) & = & -\lambda f(x).
\eq
\end{example}

\end{frame}

% page --------------------------------------------------------------------------------------------------
\begin{frame}{Gew\"ohnliche Differentialgleichungen}

\begin{definition}
Sei $D$ eine Teilmenge von ${\mathbb R} \times {\mathbb R}^n$ und 
\bq
 G & : & D \rightarrow {\mathbb R},
 \nonumber \\
 & & \left(x,\vec{y}\right) \rightarrow G\left(x,\vec{y}\right)
\eq
eine stetige Funktion.
Dann nennt man
\bq
 y^{(n)} & = & G\left(x,y,y',...,y^{(n-1)}\right)
\eq
eine \superalert{Differentialgleichungen $n$-ter Ordnung}.
\end{definition}

\end{frame}

% page --------------------------------------------------------------------------------------------------
\begin{frame}{Gew\"ohnliche Differentialgleichungen}

\begin{myemptytheorem}
Unter einer L\"osung versteht man
eine auf einem Intervall $I \subset \mathbb R$ definierte differenzierbare Funktion
\bq
 f & : & I \rightarrow {\mathbb R}
\eq
mit folgenden Eigenschaften:
\begin{itemize}
\item Die Menge
\bq
 \left\{ (x,y_0,y_1,...,y_{n-1}) \in I \times {\mathbb R}^n : y_j = f^{(j)}(x), 0 \le j < n \right\}
\eq
ist in $D$ enthalten.
\item Es gilt
\bq
 f^{(n)}(x) & = & G\left(x,f(x),f'(x),...,f^{(n-1)}(x)\right).
\eq
\end{itemize}
\end{myemptytheorem}

\end{frame}

% page --------------------------------------------------------------------------------------------------
\begin{frame}{Gew\"ohnliche Differentialgleichungen}

\begin{example}
$G(x,y_0,y_1)=-\omega^2 y_0$ f\"uhrt auf die Differentialgleichung zweiter Ordnung
\bq
 f''(x) & = & - \omega^2 f(x).
\eq
\end{example}

\end{frame}

% page --------------------------------------------------------------------------------------------------
\begin{frame}{Existenz und Eindeutigkeit von L\"osungen}

\begin{itemize}
\item Die S\"atze \"uber die Existenz und die Eindeutigkeit von L\"osungen einer Differentialgleichung setzen voraus,
da{\ss} die Funktion $G$ lokal eine Lipschitz-Bedingung erf\"ullt.
\end{itemize}

\end{frame}

% page --------------------------------------------------------------------------------------------------
\begin{frame}{Lipschitz-Bedingung}

\begin{definition}[Lipschitz-Bedingung f\"ur eine Differentialgleichung erster Ordnung]
Sei $D$ eine Teilmenge von ${\mathbb R}^2$ und 
\bq
 G & : & D \rightarrow {\mathbb R},
 \nonumber \\
 & & \left(x,y\right) \rightarrow G\left(x,y\right)
\eq
eine Funktion.
Man sagt, $G$ gen\"ugt in $D$ einer {\bf Lipschitz-Bedingung} mit der
Lipschitz-Konstanten $L \ge 0$, falls f\"ur alle $(x,y)$, $(x,z) \in D$ gilt
\bq
 \left| G\left(x,y\right) - G\left(x,z\right) \right|
 & \le &
 L \left| y - z \right|.
\eq
\end{definition}

\end{frame}

% page --------------------------------------------------------------------------------------------------
\begin{frame}{Eindeutigkeit von L\"osungen}

\begin{theorem}
Wir setzen voraus, da{\ss} die Funktion
$G$ in $D$ lokal einer Lipschitz-Bedingung gen\"ugt.
Seien $f(x)$ und $g(x)$ zwei L\"osungen der Differentialgleichung
\bq
 y^{(n)} & = & G\left(x,\vec{y}\right)
\eq
\"uber einem Intervall $I \subset {\mathbb R}$. Gilt dann
\bq
 f^{(j)}\left(x_0\right) & = & g^{(j)}\left(x_0\right) 
 \;\;\;\;\;\; \forall \; 0 \le j < n 
\eq
f\"ur ein $x_0 \in I$, so folgt
\bq
 f\left(x\right) & = & g\left(x\right) 
\eq
f\"ur alle $x\in I$.
\end{theorem}

\end{frame}

% page --------------------------------------------------------------------------------------------------
\begin{frame}{Existenz von L\"osungen}

\begin{theorem}[Picard -- Lindel\"of]
Sei $D$ offen und $G : D \rightarrow {\mathbb R}$ eine stetige Funktion, die lokal
einer Lipschitz-Bedingung gen\"ugt.
Dann gibt es zu jedem $\left(x_0,\tilde{y}_0,\tilde{y}_1,\dots,\tilde{y}_{n-1}\right) \in D$ ein $\eps > 0$ und eine L\"osung
\bq
 f & : & \left[x_0-\eps, x_0+\eps \right] \rightarrow {\mathbb R}
\eq
der Differentialgleichung $y^{(n)} = G\left(x,\vec{y}\right)$
mit der Anfangsbedinung
\bq
 f^{(j)}(x_0) & = & \tilde{y}_j
 \;\;\;\;\;\;
 0 \le j < n.
\eq
\end{theorem}

\end{frame}

% page --------------------------------------------------------------------------------------------------
\begin{frame}{Existenz und Eindeutigkeit von L\"osungen}

\begin{mytheorem28}
Die L\"osung einer gew\"ohnlichen Differentialgleichung $n$-ter Ordnung
wird eindeutig durch \superalert{$n$} Anfangsbedingungen 
\bq
 f(x_0), f'(x_0), ..., f^{(n-1)}(x_0)
\eq
bestimmt.
\end{mytheorem28}

\end{frame}

%%%%%%%%%%%%%%%%%%%%%%%%%%%%%%%%%%%%%%%%%%%%%%%%%%%%%%%%%%%%%%%%%%%%%%%%%%%%%%%%%%%%%%%%%%%%%%%%%%%%%%%%%
%%%%%%%%%%%%%%%%%%%%%%%%%%%%%%%%%%%%%%%%%%%%%%%%%%%%%%%%%%%%%%%%%%%%%%%%%%%%%%%%%%%%%%%%%%%%%%%%%%%%%%%%%
%%%%%%%%%%%%%%%%%%%%%%%%%%%%%%%%%%%%%%%%%%%%%%%%%%%%%%%%%%%%%%%%%%%%%%%%%%%%%%%%%%%%%%%%%%%%%%%%%%%%%%%%%

\section{Wichtige Beispiele}

\frame{\sectionpage}

% page --------------------------------------------------------------------------------------------------
\begin{frame}{Exponentielles Wachstum / exponentieller Zerfall}

\begin{example}
Wir betrachten die Differentialgleichung
\bq
 f'(x) & = & -\lambda f(x).
\eq
Gesucht ist eine L\"osung zur Anfangsbedingung
\bq
 f(0) & = & C.
\eq
\end{example}

\end{frame}

% page --------------------------------------------------------------------------------------------------
\begin{frame}{Exponentielles Wachstum / exponentieller Zerfall}

Wir betrachten die Funktion
\begin{myemptytheorem}
\bq
 f\left(x\right) & = & C e^{-\lambda x}.
\eq
\end{myemptytheorem}
Es ist
\bq
 f'\left(x\right) & = & - \lambda C e^{-\lambda x}
 \; = \; 
 - \lambda f\left(x\right).
\eq
F\"ur die Anfangsbedingung gilt
\bq
 f\left(0\right) & = & C.
\eq

\end{frame}

% page --------------------------------------------------------------------------------------------------
\begin{frame}{Quiz}

Eine L\"osung der Differentialgleichung
\bq
 f'(x) & = & 2 f(x)
\eq
zur Anfangsbedingung $f(0)=2$ ist
\begin{description}
\item{(A)} $f(x) = 2 e^x$
\item{(B)} $f(x) = e^{2x}$
\item{(C)} $f(x) = 2 e^{2x}$
\item{(D)} $f(x) = 2 e^{-2x}$
\end{description}

\end{frame}

% page --------------------------------------------------------------------------------------------------
\begin{frame}{Exponentielles Wachstum}

\begin{example}
Annahme: Eine mit einem Virus infizierte Person steckt im Mittel pro Tag 1.3 nicht-infizierte Personen an.

Zu Beginn der Z\"ahlung seien 10 000 Personen infiziert.

Wie viele Personen sind nach 10 Tagen infiziert?
\end{example}

\end{frame}

% page --------------------------------------------------------------------------------------------------
\begin{frame}{Exponentielles Wachstum}

Es sei $f(t)$ die Anzahl der infizierten Personen am Tag $t$.

Die Anzahl der Neuinfizierten pro Tag ist proportional zur Anzahl der bereits infizierten Personen, daher haben
wir die Differentialgleichung
\bq
 f'\left(t\right)
 & = &
 \kappa f\left(t\right),
\eq
deren L\"osung durch
\bq
 f\left(t\right)
 & = & 
 C e^{\kappa t}
\eq
gegeben ist.

\end{frame}

% page --------------------------------------------------------------------------------------------------
\begin{frame}{Exponentielles Wachstum}

Wir bestimmen die Konstante $C$ aus der Anfangsbedingung:
\bq
 f\left(0\right) \; = \; 10000
 & \Rightarrow &
 C \; = \; 10000.
\eq
Wir bestimmen die Konstante $\kappa$ aus der Ver\"anderung pro Tag: 

Nach einem Tag haben wir $23000$ Infizierte 

($13000$ neu Infizierte plus $10000$ bereits Infizierte):
\bq
 f\left(1\right) \; = \; 23000
 & \Rightarrow &
 10000 e^\kappa \; = \; 23000
 \;\;\; \Rightarrow \;\;\;
 \kappa \; = \; \ln\left(2.3\right)
\eq
Somit
\bq
 f\left(t\right) 
 & = & 
 10000 \cdot \left(2.3\right)^t,
 \nonumber \\
 f\left(10\right)
 & \approx &
 41 \cdot 10^6.
\eq

\end{frame}

% page --------------------------------------------------------------------------------------------------
\begin{frame}{Exponentielles Wachstum}

Bemerkungen:
\begin{itemize}
\item Wir haben angenommen, da{\ss} eine infizierte Person ansteckend bleibt.
\item Wir haben S\"attigungseffekte vernachl\"assigt: Sind alle Personen infiziert, k\"onnen keine neuen Personen mehr infiziert werden.
\end{itemize}

\end{frame}

% page --------------------------------------------------------------------------------------------------
\begin{frame}{Der harmonische Oszillator}

\begin{example}
Wir betrachten die Differentialgleichung
\bq
 f''(t) & = & - \omega^2 f(t).
\eq
Gesucht ist eine L\"osung zu den Anfangsbedingungen
\bq
 f(0) \; = \; x_0,
 & &
 f'(0) \; = \; v_0.
\eq
\end{example}

\end{frame}

% page --------------------------------------------------------------------------------------------------
\begin{frame}{Der harmonische Oszillator}

Wir betrachten die Funktion
\begin{myemptytheorem}
\bq
 f\left(t\right) & = & A_1 \sin\left(\omega t\right) + A_2 \cos\left(\omega t\right)
\eq
\end{myemptytheorem}
Es ist
\bq
 f'\left(t\right) & = & \omega A_1 \cos\left(\omega t\right) - \omega A_2 \sin\left(\omega t\right),
 \nonumber \\
 f''\left(t\right) & = & - \omega^2 A_1 \sin\left(\omega t\right) - \omega^2 A_2 \cos\left(\omega t\right)
 \; = \; 
 - \omega^2 f(t).
\eq
F\"ur die Anfangsbedingung gilt
\bq
 x_0 & = & f\left(0\right) \; = \; A_1 \sin\left(0\right) + A_2 \cos\left(0\right) \; = \; A_2,
 \nonumber \\
 v_0 & = &  f'\left(0\right) \; = \; \omega A_1 \cos\left(0\right) - \omega A_2 \sin\left(0\right)
 \; = \; \omega A_1.
\eq

\end{frame}

% page --------------------------------------------------------------------------------------------------
\begin{frame}{Der harmonische Oszillator}

Somit lautet die L\"osung
\begin{myemptytheorem}
\bq
 f\left(t\right) & = & x_0 \cos\left(\omega t\right) +\frac{v_0}{\omega} \sin\left(\omega t\right) 
\eq
\end{myemptytheorem}

\end{frame}

% page --------------------------------------------------------------------------------------------------
\begin{frame}{Der harmonische Oszillator}

Wir k\"onnen auch die Funktion
\begin{myemptytheorem}
\bq
 f\left(t\right) & = & c_1 e^{i\omega t} + c_2 e^{-i \omega t}
\eq
\end{myemptytheorem}
betrachten.
Es ist
\bq
 f'\left(t\right) & = & i \omega c_1 e^{i\omega t} - i \omega c_2 e^{-i \omega t},
 \nonumber \\
 f''\left(t\right) & = & - \omega^2 c_1 e^{i\omega t} - \omega^2 c_2 e^{-i \omega t}
 \; = \; 
 - \omega^2 f(t).
\eq
Aufgrund von
\bq
 \cos\left(\omega t\right) = \frac{1}{2} \left( e^{i\omega t} + e^{-i\omega t} \right),
 & &
 \sin\left(\omega t\right) = \frac{1}{2i} \left( e^{i\omega t} - e^{-i\omega t} \right),
\eq
ist dies \"aquivalent zur vorherigen L\"osung.

\end{frame}

% page --------------------------------------------------------------------------------------------------
\begin{frame}{Der harmonische Oszillator}

\begin{example}
F\"ur ein Federpendel ist die r\"ucktreibende Kraft proportional zur Auslenkung:
\bq
 F & = & - D x,
\eq
wobei $D$ die Federkonstante angibt.
Das Newtonsche Gesetz lautet
\bq
 F & = & m a,
\eq
wobei $a$ die Beschleunigung angibt. 
Die Beschleunigung ist die zweite Ableitung des Ortes nach der Zeit.
\end{example}

\end{frame}

% page --------------------------------------------------------------------------------------------------
\begin{frame}{Der harmonische Oszillator}

F\"ur die Auslenkung $x(t)$ erhalten wir die Differentialgleichung
\bq
 x''\left(t\right) & = & - \frac{D}{m} x\left(t\right).
\eq
Dies ist die Differentialgleichung eines harmonischen Oszillators mit der L\"osung
\bq
 x\left(t\right) & = & x_0 \cos\left(\omega t\right) +\frac{v_0}{\omega} \sin\left(\omega t\right),
 \;\;\;\;\;\;
 \omega \; = \; \sqrt{\frac{D}{m}}.
\eq

\end{frame}

% page --------------------------------------------------------------------------------------------------
\begin{frame}{Quiz}

Eine L\"osung der Differentialgleichung
\bq
 f'(x) & = & \omega^2 f(x)
\eq
zu den Anfangsbedingungen $f(0)=2$, $f'(0)=0$ ist
\begin{description}
\item{(A)} $f(x) = 2 cos(\omega x)$
\item{(B)} $f(x) = 2 cos(\omega x) + \frac{2}{\omega} sin(\omega x)$
\item{(C)} $f(x) = cos(2 \omega x)$
\item{(D)} $f(x) = 2 cosh(\omega x)$
\end{description}

\end{frame}

%%%%%%%%%%%%%%%%%%%%%%%%%%%%%%%%%%%%%%%%%%%%%%%%%%%%%%%%%%%%%%%%%%%%%%%%%%%%%%%%%%%%%%%%%%%%%%%%%%%%%%%%%
%%%%%%%%%%%%%%%%%%%%%%%%%%%%%%%%%%%%%%%%%%%%%%%%%%%%%%%%%%%%%%%%%%%%%%%%%%%%%%%%%%%%%%%%%%%%%%%%%%%%%%%%%
%%%%%%%%%%%%%%%%%%%%%%%%%%%%%%%%%%%%%%%%%%%%%%%%%%%%%%%%%%%%%%%%%%%%%%%%%%%%%%%%%%%%%%%%%%%%%%%%%%%%%%%%%

\section{Elementare L\"osungsmethoden}

\frame{\sectionpage}

% page --------------------------------------------------------------------------------------------------
\begin{frame}{Elementare L\"osungsmethoden}

\begin{myemptytheorem}
Wir betrachten eine gew\"ohnliche Differentialgleichung erster Ordnung:
\bq
 y' & = & G(x,y)
\eq
\end{myemptytheorem}

\end{frame}

% page --------------------------------------------------------------------------------------------------
\begin{frame}{Elementare L\"osungsmethoden}

\begin{myemptytheorem}
H\"angt die Funktion $G$ nur von $x$, aber nicht von $y$ ab, so hat man
\bq
 f'(x) & = & G(x)
\eq
und man erh\"alt eine L\"osung durch Integration:
\bq
 f(x) & = & c + \int\limits_{x_0}^x G(t) \; dt.
\eq
\end{myemptytheorem}

\end{frame}

% page --------------------------------------------------------------------------------------------------
\begin{frame}{Differentialgleichungen mit separierten Variablen}

\begin{myemptytheorem}
Als n\"achstes betrachten wir den Fall, da{\ss} die Funktion $G$ faktorisiert:
\bq
 G(x,y) & = & h(x) k(y).
\eq
In diesem Fall spricht man von einer Differentialgleichungen mit separierten Variablen.
\end{myemptytheorem}

\end{frame}

% page --------------------------------------------------------------------------------------------------
\begin{frame}{Differentialgleichungen mit separierten Variablen}

Wir wollen annehmen, da{\ss} 
\bq
 h : I \rightarrow {\mathbb R},
 & &
 k : J \rightarrow {\mathbb R},
\eq
stetige Funktionen auf offenen Intervallen $I, J \subset {\mathbb R}$ sind.
Weiter sei $k(y)\neq 0$ f\"ur alle $y\in J$.
Sei nun $(x_0,y_0) \in I \times J$.
Wir setzen
\bq
 H(x) = \int\limits_{x_0}^x h(t) dt,
 & &
 K(y) = \int\limits_{y_0}^y \frac{dt}{k(t)}.
\eq
Es sei $I' \subset I$ ein Intervall mit $x_0 \in I'$ und $H(I') \subset K(J)$.
Dann exisitiert genau eine L\"osung $f:I' \rightarrow {\mathbb R}$ mit
\bq
 f(x_0) = y_0.
\eq
Diese L\"osung erf\"ullt die Beziehung
\bq
 K\left(f(x)\right) & = & H(x).
\eq

\end{frame}

% page --------------------------------------------------------------------------------------------------
\begin{frame}{Differentialgleichungen mit separierten Variablen}

\begin{example}
Wir betrachten die Differentialgleichung
\bq
 y' & = & 2 x e^{-y}
\eq
und suchen eine L\"osung zu der Anfangsbedingung $f(0)=c$.
Die Variablen sind klarerweise getrennt.
F\"ur dieses Beispiel k\"onnen wir
\bq
 h(x) = 2x, 
 & &
 k(y) = e^{-y}
\eq
setzen. 
\end{example}

\end{frame}

% page --------------------------------------------------------------------------------------------------
\begin{frame}{Differentialgleichungen mit separierten Variablen}

Wir erhalten
\bq
 H(x) & = & 2 \int\limits_0^x t \; dt = x^2,
 \nonumber \\
 K(y) & = & \int\limits_{c}^y \frac{dt}{e^{-t}} = e^y - e^c.
\eq
Somit
\bq
 e^{f(x)} - e^c & = & x^2.
\eq
Umgeformt ergibt sich
\bq
 f(x) & = & \ln\left( x^2 + e^c \right).
\eq

\end{frame}

% page --------------------------------------------------------------------------------------------------
\begin{frame}{Differentialgleichungen mit separierten Variablen}

\begin{example}
Als zweites Beispiel betrachten wir die Differentialgleichung
\bq
 y' & = & y^2.
\eq
Gesucht ist eine L\"osung zu der Anfangsbedingung $y(0)=1$.
\end{example}

\end{frame}

% page --------------------------------------------------------------------------------------------------
\begin{frame}{Differentialgleichungen mit separierten Variablen}

Wir haben
\bq
 \frac{dy}{y^2} & = & dx,
\eq
und somit liefert die Integration
\bq
 - \frac{1}{y} & = & x + c.
\eq
Durch Aufl\"osen nach $y$ erh\"alt man
\bq
 y & = & - \frac{1}{x+c}.
\eq
Die Anfangsbedingung $y(0)=1$ liefert $c=-1$, somit lautet die L\"osung
\bq
 y & = & \frac{1}{1-x}.
\eq

\end{frame}

%%%%%%%%%%%%%%%%%%%%%%%%%%%%%%%%%%%%%%%%%%%%%%%%%%%%%%%%%%%%%%%%%%%%%%%%%%%%%%%%%%%%%%%%%%%%%%%%%%%%%%%%%
%%%%%%%%%%%%%%%%%%%%%%%%%%%%%%%%%%%%%%%%%%%%%%%%%%%%%%%%%%%%%%%%%%%%%%%%%%%%%%%%%%%%%%%%%%%%%%%%%%%%%%%%%
%%%%%%%%%%%%%%%%%%%%%%%%%%%%%%%%%%%%%%%%%%%%%%%%%%%%%%%%%%%%%%%%%%%%%%%%%%%%%%%%%%%%%%%%%%%%%%%%%%%%%%%%%

% page --------------------------------------------------------------------------------------------------
\begin{frame}

\end{frame}

\end{document}



