\documentclass[german]{beamer}

\mode<presentation>
{
 \usetheme{Madrid}

% \usecolortheme{crane}
 \usecolortheme{wolverine}
}

\usepackage{hyperref}

\usepackage[german]{babel}
\usepackage{times}
\usepackage[latin1,utf8]{inputenc}
\usepackage[OT2,T1]{fontenc}
\usepackage{shuffle}

\definecolor{MyYellowOrange}   {cmyk}{0,0.5,1,0}


% Stefan's abbreveations
\newcommand{\bq}{\begin{eqnarray*}}
\newcommand{\eq}{\end{eqnarray*}}
\newcommand{\eps}{\varepsilon}

\newtheorem*{mytheorem}{}

% dedicated environments
\newtheorem*{mytheorem1}{Die {\bf Axiome von Peano} f\"ur die nat\"urlichen Zahlen:}
\newtheorem*{mytheorem2}{{\bf Definition einer Gruppe}:}
\newtheorem*{mytheorem3}{{\bf Definition eines Rings}:}
\newtheorem*{mytheorem4}{{\bf Definition eines K\"orpers}:}
\newtheorem*{mytheorem5}{{\bf Anordnungsaxiome}:}

\newtheorem*{myproof1}{{\bf Induktionsanfang}:}
\newtheorem*{myproof2}{{\bf Induktionsschritt}:}

\newcommand{\superalert}[1]{{\color{MyYellowOrange}{#1}}}

%%%%%%%%%%%%%%%%%%%%%%%%%%%%%%%%%%%%%%%%%%%%%%%%%%%%%%%%%%%%%%%%%%%%%%%%%%%%%%%%%%%%%%%%%%%%%%%%%%%%%%%%%
%%%%%%%%%%%%%%%%%%%%%%%%%%%%%%%%%%%%%%%%%%%%%%%%%%%%%%%%%%%%%%%%%%%%%%%%%%%%%%%%%%%%%%%%%%%%%%%%%%%%%%%%%
%%%%%%%%%%%%%%%%%%%%%%%%%%%%%%%%%%%%%%%%%%%%%%%%%%%%%%%%%%%%%%%%%%%%%%%%%%%%%%%%%%%%%%%%%%%%%%%%%%%%%%%%%

\title{Zahlen}

\subtitle{Mathematischer Br\"uckenkurs}

\author{Stefan Weinzierl}

\institute[Uni Mainz]{Institut f\"ur Physik, Universit\"at Mainz}%

\date[WiSe 2020/21]{Wintersemester 2020/21}

\begin{document}

%%%%%%%%%%%%%%%%%%%%%%%%%%%%%%%%%%%%%%%%%%%%%%%%%%%%%%%%%%%%%%%%%%%%%%%%%%%%%%%%%%%%%%%%%%%%%%%%%%%%%%%%%
%%%%%%%%%%%%%%%%%%%%%%%%%%%%%%%%%%%%%%%%%%%%%%%%%%%%%%%%%%%%%%%%%%%%%%%%%%%%%%%%%%%%%%%%%%%%%%%%%%%%%%%%%
%%%%%%%%%%%%%%%%%%%%%%%%%%%%%%%%%%%%%%%%%%%%%%%%%%%%%%%%%%%%%%%%%%%%%%%%%%%%%%%%%%%%%%%%%%%%%%%%%%%%%%%%%

\begin{frame}
  \titlepage
\end{frame}

%%%%%%%%%%%%%%%%%%%%%%%%%%%%%%%%%%%%%%%%%%%%%%%%%%%%%%%%%%%%%%%%%%%%%%%%%%%%%%%%%%%%%%%%%%%%%%%%%%%%%%%%%
%%%%%%%%%%%%%%%%%%%%%%%%%%%%%%%%%%%%%%%%%%%%%%%%%%%%%%%%%%%%%%%%%%%%%%%%%%%%%%%%%%%%%%%%%%%%%%%%%%%%%%%%%
%%%%%%%%%%%%%%%%%%%%%%%%%%%%%%%%%%%%%%%%%%%%%%%%%%%%%%%%%%%%%%%%%%%%%%%%%%%%%%%%%%%%%%%%%%%%%%%%%%%%%%%%%

\section{Die nat\"urlichen Zahlen}

\frame{\sectionpage}

% page --------------------------------------------------------------------------------------------------
\begin{frame}{Die nat\"urlichen Zahlen}

\begin{itemize}
\item \alert{${\mathbb N}$}: Die nat\"urlichen Zahlen ${\mathbb N} = \{1,2,3,4,....\}$.

\begin{mytheorem1}
\begin{itemize}
\item (P1) Die Zahl 1 ist eine nat\"urliche Zahl.
\item (P2) Falls $n$ eine nat\"urliche Zahl, so ist die nachfolgende Zahl $n+1$ ebenfalls eine 
nat\"urliche Zahl.
\item (P3) Die nat\"urlichen Zahlen sind die minimale Menge, welche die ersten beiden Axiome erf\"ullt.
\end{itemize}
\end{mytheorem1}

\item \alert{${\mathbb N}_0$}: Die nat\"urlichen Zahlen mit der Null ${\mathbb N}_0 = \{0,1,2,3,4,....\}$.
\end{itemize}

\end{frame}

% page --------------------------------------------------------------------------------------------------
\begin{frame}{Die nat\"urlichen Zahlen}

Sei $a, b \in {\mathbb N}$:
\begin{itemize}
\item \alert{Addition}: $a+b \in {\mathbb N}$
\item Aber: $a-b$ ist im Allgemeinen keine nat\"urliche Zahl.

Gegenbeispiel: $a=1$ und $b=3$.
\item \alert{Multiplikation}: $a \cdot b \in {\mathbb N}$
\item Aber: $a/b$ ist im Allgemeinen keine nat\"urliche Zahl.

Gegenbeispiel: $a=1$ und $b=3$.
\end{itemize}

\end{frame}

% page --------------------------------------------------------------------------------------------------
\begin{frame}{Der Induktionsbeweis}

Man ist oft in der Situation
eine Aussage der Form
\bq
 f(n) & = & g(n)
\eq
\alert{f\"ur alle $n \in {\mathbb N}$} beweisen zu m\"ussen. Hier bietet sich der Induktionsbeweis an.

\vspace*{3mm}

Der Induktionsbeweis verl\"auft in zwei Teilen: 
\begin{enumerate}
\item Induktionsanfang: Im ersten Teil zeigt man zun\"achst, da{\ss} die Behauptung f\"ur $n=1$ richtig ist.
\item Induktionsschritt: Im zweiten Teil nimmt man an, da{\ss} die Behauptung f\"ur $(n-1)$ richtig ist und zeigt, da{\ss} sie dann auch f\"ur $n$ richtig ist.
\end{enumerate}

\end{frame}

% page --------------------------------------------------------------------------------------------------
\begin{frame}{Der Induktionsbeweis}

Man sieht leicht, da{\ss} dies die allgemeine Aussage beweist:
\begin{itemize}
\item F\"ur $n=1$ wird die Aussage im ersten Teil bewiesen.
\item F\"ur $n=2$ k\"onnen wir dann verwenden, da{\ss} die Aussage f\"ur $n=1$ richtig ist. 

Somit liegt die Voraussetzung f\"ur den zweiten Teil vor und es folgt
aufgrund des zweiten Teils die Richtigkeit f\"ur $n=2$.
\item Diese Argumentation l\"a{\ss} sich nun fortsetzen: 

Da die Aussage f\"ur $n=2$ richtig ist, mu{\ss} sie aufgrund des zweiten Teils auch f\"ur $n=3$ richtig sein, usw..
\end{itemize}

\end{frame}

% page --------------------------------------------------------------------------------------------------
\begin{frame}{Der Induktionsbeweis}

\begin{example}
F\"ur jede nat\"urliche Zahl $n$ ist die folgende Behauptung zu zeigen:
\bq
 \sum\limits_{j=1}^n j & = & \frac{n(n+1)}{2}
\eq
\end{example}
\begin{myproof1}
F\"ur $n=1$ haben wir
\bq
 \mbox{linke Seite}: & & \sum\limits_{j=1}^1 j = 1. \\
 \mbox{rechte Seite}: & & \frac{1(1+1)}{2} = 1.
\eq
\end{myproof1}

\end{frame}

% page --------------------------------------------------------------------------------------------------
\begin{frame}{Der Induktionsbeweis}

\begin{myproof2}
Wir d\"urfen nun annehmen, da{\ss} die Behauptung f\"ur $n-1$ richtig ist, und m\"ussen zeigen, da{\ss} sie dann auch 
f\"ur $n$ gilt. In unserem Fall:
\bq
  \sum\limits_{j=1}^n j & = &  \left( \sum\limits_{j=1}^{n-1} j \right) + n
 = \frac{(n-1)n}{2} + n = \frac{n^2-n+2n}{2} = \frac{n(n+1)}{2}
\eq
\end{myproof2}

\end{frame}


%%%%%%%%%%%%%%%%%%%%%%%%%%%%%%%%%%%%%%%%%%%%%%%%%%%%%%%%%%%%%%%%%%%%%%%%%%%%%%%%%%%%%%%%%%%%%%%%%%%%%%%%%
%%%%%%%%%%%%%%%%%%%%%%%%%%%%%%%%%%%%%%%%%%%%%%%%%%%%%%%%%%%%%%%%%%%%%%%%%%%%%%%%%%%%%%%%%%%%%%%%%%%%%%%%%
%%%%%%%%%%%%%%%%%%%%%%%%%%%%%%%%%%%%%%%%%%%%%%%%%%%%%%%%%%%%%%%%%%%%%%%%%%%%%%%%%%%%%%%%%%%%%%%%%%%%%%%%%

\section{Die ganzen Zahlen}

\frame{\sectionpage}

% page --------------------------------------------------------------------------------------------------
\begin{frame}{Gruppen}

\begin{mytheorem2}
Sei $G$ eine nicht-leere Menge mit einer Verkn\"upfung $\circ$, d.h.
eine Abbildung $\circ : G \times G \rightarrow G$.  Das Paar $(G,\circ)$ ist eine Gruppe,
falls:
\begin{itemize}
\item (G1) $\circ$ ist assoziativ: $a \circ ( b \circ c ) = ( a \circ b ) \circ c$
\item (G2) Es gibt ein links-neutrales Element : $e \circ a = a$ f\"ur alle $a \in G$
\item (G3) Zu jedem $a \in G$ gibt es ein links-inverses Element $a^{-1}$ : $a^{-1} \circ a = e$
\end{itemize}
Eine Gruppe $(G,\circ)$ nennt man kommutativ oder Abelsch, falls 
$a \circ b = b \circ a$.
\end{mytheorem2}

In einer Gruppe ist das links-neutrale Element identisch mit dem 
recht-neutralen Element. 

Ebenso sind links- und rechts-inverses Element identisch.

\end{frame}

% page --------------------------------------------------------------------------------------------------
\begin{frame}{Die ganzen Zahlen}

\alert{${\mathbb Z}$}: Die ganzen Zahlen ${\mathbb Z}=\{....,-3,-2,-1,0,1,2,3,....\}$. 

Die ganzen Zahlen bilden bez\"uglich der Addition eine Gruppe.

\begin{itemize}
\item Assoziativgesetz: 

Beispiel: $3 + ( 5 + 7 ) = ( 3 + 5 ) + 7$
\item Die Null ist das links-neutrale Element: 

Beispiel: $0 + 7 = 7$.
\item Das links-inverse Element zu $a$ ist $(-a)$: 

Beispiel: Es ist $(-7) + 7 = 0$.

\item Die Gruppe ist kommutativ:

Beispiel: $5 + 7 = 7 + 5$.
\end{itemize}

\end{frame}

% page --------------------------------------------------------------------------------------------------
\begin{frame}{Ringe}

\begin{mytheorem3}
Ein {\bf Ring} ist eine nicht-leere Menge $R$ mit zwei Verkn\"upfungen, die \"ublicherweise als $+$ und $\cdot$ geschrieben
werden, so da{\ss}
\begin{itemize}
\item (R1) $(R,+)$ ist eine kommutative Gruppe.
\item (R2) $(R,\cdot)$ ist assoziativ: $ a \cdot ( b \cdot c ) = ( a \cdot b ) \cdot c$
\item (R3) Es gelten die Distributivgesetze:
\bq
a \cdot ( b+ c ) &=  & (a \cdot b ) + ( a \cdot c ) \\
(a + b ) \cdot c & = & (a \cdot c) + ( b \cdot c)
\eq
\end{itemize}
\end{mytheorem3}

\end{frame}

% page --------------------------------------------------------------------------------------------------
\begin{frame}{Die ganzen Zahlen}

Die ganzen Zahlen ${\mathbb Z}$ bilden einen Ring.

\begin{itemize}
\item Assoziativgesetz: 

Beispiel: $3 \cdot ( 5 \cdot 7 ) = ( 3 \cdot 5 ) \cdot 7$
\item Distributivgesetze:
\bq
3 \cdot (5+ 7) &=  & (3 \cdot 5) + (3 \cdot 7) 
 \nonumber \\
(3 + 5) \cdot 7 & = & (3 \cdot 7) + (5 \cdot 7)
\eq
\end{itemize}
\end{frame}

%%%%%%%%%%%%%%%%%%%%%%%%%%%%%%%%%%%%%%%%%%%%%%%%%%%%%%%%%%%%%%%%%%%%%%%%%%%%%%%%%%%%%%%%%%%%%%%%%%%%%%%%%
%%%%%%%%%%%%%%%%%%%%%%%%%%%%%%%%%%%%%%%%%%%%%%%%%%%%%%%%%%%%%%%%%%%%%%%%%%%%%%%%%%%%%%%%%%%%%%%%%%%%%%%%%
%%%%%%%%%%%%%%%%%%%%%%%%%%%%%%%%%%%%%%%%%%%%%%%%%%%%%%%%%%%%%%%%%%%%%%%%%%%%%%%%%%%%%%%%%%%%%%%%%%%%%%%%%

\section{Die rationalen Zahlen}

\frame{\sectionpage}

% page --------------------------------------------------------------------------------------------------
\begin{frame}{Die rationalen Zahlen}

\alert{${\mathbb Q}$}: Die rationalen Zahlen.
\bq
 {\mathbb Q } & = & \left\{ r \; \left| r = \frac{p}{q}, \right. \; p,q \in {\mathbb Z}, \; q \neq 0 \right\}.
\eq
Die rationalen Zahlen sind bez\"uglich der Division abgeschlossen.
Sie bilden einen K\"orper.

\end{frame}

% page --------------------------------------------------------------------------------------------------
\begin{frame}{K\"orper}

\begin{mytheorem4}
Eine nicht-leere Menge $K$ mit zwei Verkn\"upfungen $+$ und $\cdot$ nennt man {\bf K\"orper},
falls gilt:
\begin{itemize}
\item (K1) $(K,+)$ ist eine kommutative Gruppe.
\item (K2) $(K\setminus\{0\}, \cdot)$ ist eine kommutative Gruppe.
\item (K3) Es gelten die Distributivgesetze:
\bq
a \cdot ( b+ c ) &=  & (a \cdot b ) + ( a \cdot c ) \\
(a + b ) \cdot c & = & (a \cdot c) + ( b \cdot c)
\eq
\end{itemize}
\end{mytheorem4}

\end{frame}


% page --------------------------------------------------------------------------------------------------
\begin{frame}{Bruchrechnen}

\begin{itemize}

\item Erweitern/K\"urzen:
\bq
 \frac{c \cdot p_1}{c \cdot q_1} & = & \frac{p_1}{q_1}
\eq

\item Multiplikation:
\bq
 \frac{p_1}{q_1} \cdot \frac{p_2}{q_2} & = & \frac{p_1 \cdot p_2}{q_1 \cdot q_2}
\eq

\item Division:
\bq
 \frac{p_1}{q_1} : \frac{p_2}{q_2} & = & \frac{p_1 \cdot q_2}{q_1 \cdot p_2}
 \nonumber \\
 \frac{\hphantom{a} \frac{p_1}{q_1} \hphantom{a} }{\frac{p_2}{q_2}} & = & \frac{p_1 \cdot q_2}{q_1 \cdot p_2}
\eq

\end{itemize}

\end{frame}

% page --------------------------------------------------------------------------------------------------
\begin{frame}{Bruchrechnen}

\begin{itemize}

\item Addition:
\bq
 \frac{p_1}{q_1} + \frac{p_2}{q_2} & = & \frac{p_1 \cdot q_2 + p_2 \cdot q_1}{q_1 \cdot q_2}
\eq

\item Subtraktion:
\bq
 \frac{p_1}{q_1} - \frac{p_2}{q_2} & = & \frac{p_1 \cdot q_2 - p_2 \cdot q_1}{q_1 \cdot q_2}
\eq

\end{itemize}

\end{frame}

% page --------------------------------------------------------------------------------------------------
\begin{frame}{Beispiele}

\begin{itemize}

\item Erweitern/K\"urzen:
\bq
 \frac{15}{9} \; = \; \frac{3 \cdot 5}{3 \cdot 3} \; = \; \frac{5}{3}
\eq

\item Addition:
\bq
 \frac{3}{5} + \frac{2}{3} 
 \; = \;
 \frac{3 \cdot 3}{3 \cdot 5} + \frac{2 \cdot 5}{3 \cdot 5} 
 \; = \;
 \frac{9}{15} + \frac{10}{15} 
 \; = \;
 \frac{9+10}{15}
 \; = \;
 \frac{19}{15}
\eq

\item Division:
\bq
 \frac{\hphantom{a} \frac{2}{3} \hphantom{a} }{\frac{5}{7}} 
 \; = \; 
 \frac{2 \cdot 7}{3 \cdot 5}
 \; = \; 
 \frac{14}{15}
\eq

\end{itemize}

\end{frame}

% page --------------------------------------------------------------------------------------------------
\begin{frame}{Potenzen}

F\"ur Potenzen schreiben wir
\bq
 a^n & = & \underbrace{a \cdot a \cdot a \cdot ... \cdot a}_{n \;\;\;\mathrm{mal}}
\eq
Rechnen mit Potenzen:
\bq
 a^n \cdot b^n = \left( a \cdot b \right)^n
 & &
\;\;\;\;\;\;\;\;\;
 \frac{a^n}{b^n} = \left( \frac{a}{b} \right)^n
 \nonumber \\
 a^n \cdot a^m = a^{n+m}
 & &
\;\;\;\;\;\;\;\;\;
 \frac{a^n}{a^m} = a^{n-m}
 \nonumber \\
 \left( a^n \right)^m = a^{n\cdot m}
\eq

\end{frame}

% page --------------------------------------------------------------------------------------------------
\begin{frame}{Beispiele}

\begin{itemize}

\item Gleicher Exponent:
\bq
 2^7 \cdot 3^7 \; = \; \left(2 \cdot 3 \right)^7 \; = \; 6^7
\eq

\item Gleiche Basis:
\bq
 2^5 \cdot 2^7 \; = \; 2^{\left(5+7\right)} \; = \; 2^{12}
\eq

\item Potenz einer Potenz:
\bq
 \left(3^2\right)^5 \; = \; 3^{\left(2 \cdot 5 \right)} \; = \; 3^{10}
\eq

\end{itemize}

\end{frame}

% page --------------------------------------------------------------------------------------------------
\begin{frame}{Quiz}

\bq
 \frac{x^{-1} x^3 x^5}{x^2 x^{7}} & = & ?
\eq 
\begin{description}
\item{(A)} $0$
\item{(B)} $1$
\item{(C)} $x^2$
\item{(D)} $\frac{1}{x^2}$
\end{description}

\end{frame}

% page --------------------------------------------------------------------------------------------------
\begin{frame}{Die binomischen Formeln}

\begin{mytheorem}
\bq
 \left(a+b\right)^2 & = & a^2 + 2 a b + b^2
 \nonumber \\
 \left(a-b\right)^2 & = & a^2 - 2 a b + b^2
 \nonumber \\
 \left(a+b\right)\left(a-b\right) & = & a^2 - b^2
\eq
\end{mytheorem}

\bq
 \left(\alert{a}+\alert{b}\right)\left(\superalert{a}-\superalert{b}\right)
 & = &
 \alert{a} \cdot \superalert{a} - \alert{a} \cdot \superalert{b} + \alert{b} \cdot \superalert{a} - \alert{b} \cdot \superalert{b}
 \nonumber \\
 & = &
 a^2 - a b + a b - b^2
 \nonumber \\
 & = &
 a^2 - b^2
\eq

\end{frame}


%%%%%%%%%%%%%%%%%%%%%%%%%%%%%%%%%%%%%%%%%%%%%%%%%%%%%%%%%%%%%%%%%%%%%%%%%%%%%%%%%%%%%%%%%%%%%%%%%%%%%%%%%
%%%%%%%%%%%%%%%%%%%%%%%%%%%%%%%%%%%%%%%%%%%%%%%%%%%%%%%%%%%%%%%%%%%%%%%%%%%%%%%%%%%%%%%%%%%%%%%%%%%%%%%%%
%%%%%%%%%%%%%%%%%%%%%%%%%%%%%%%%%%%%%%%%%%%%%%%%%%%%%%%%%%%%%%%%%%%%%%%%%%%%%%%%%%%%%%%%%%%%%%%%%%%%%%%%%

\section{Die reellen Zahlen}

\frame{\sectionpage}

% page --------------------------------------------------------------------------------------------------
\begin{frame}{Die reellen Zahlen}

\alert{${\mathbb R}$}: Die reellen Zahlen.

Die reellen Zahlen bilden einen K\"orper.

\begin{itemize}

\item Alle rationalen Zahlen sind in den reellen Zahlen enthalten.

\item $\mathbb R$ enth\"alt Zahlen, die nicht rational sind. 
Diese nennt man irrational.

\begin{itemize}

\item 
$\sqrt{2}$ ist eine irrationale Zahl.
$\sqrt{2}$ ist L\"osung der Gleichung $x^2=2$. 
Zahlen, welche L\"osungen einer algebraischen Gleichung sind, nennt man algebraisch.

\item
$\mathbb R$ enth\"alt auch irrationale Zahlen, die keine L\"osung einer algebraischen Gleichung sind.
Solche Zahlen nennt man transzendental. Die Kreiszahl $\pi$ oder die Eulersche Konstante $e$ sind
transzendental.

\end{itemize}

\end{itemize}

\end{frame}

% page --------------------------------------------------------------------------------------------------
\begin{frame}{Cauchy-Folgen}

Eine Folge $(a_n)_{n\in \mathbb N}$ reeller Zahlen nennt man {\bf Cauchy-Folge}, falls es zu jedem $\eps>0$ ein
$N\in \mathbb N$ gibt, so da{\ss}
\bq
 \left| a_n - a_m \right | < \eps, \;\;\; \forall n,m \ge N.
\eq
Vollst\"andigkeitsaxiom: Jede Cauchy-Folge konvergiert.

\end{frame}

% page --------------------------------------------------------------------------------------------------
\begin{frame}{Anordnungseigenschaften}

Die reellen Zahlen sind {\bf angeordnet}:
\begin{mytheorem5}
Es sind gewisse Elemente als positiv ausgezeichnet $(x>0)$, so da{\ss} die folgenden Axiome erf\"ullt sind:
\begin{itemize}
\item (O1) Es gilt genau eine der Beziehungen $x<0$, $x=0$ oder $x>0$.
\item (O2) Aus $x>0$ und $y>0$ folgt $x+y>0$.
\item (O3) Aus $x>0$ und $y>0$ folgt $x \cdot y >0$.
\end{itemize}
\end{mytheorem5}

Man nennt eine Ordnung {\bf archimedisch}, falls zu jedem $x>0$ und $y>0$ ein nat\"urliche Zahl $n$ exisiert,
so da{\ss}
\bq
n \cdot x > y.
\eq

\end{frame}

% page --------------------------------------------------------------------------------------------------
\begin{frame}{Die reellen Zahlen}

Axiomatisch lassen sich die reellen Zahlen als ein K\"orper, der archimedisch angeordnet ist und in dem jede
Cauchy-Folge konvergiert, charakterisieren.

\end{frame}

% page --------------------------------------------------------------------------------------------------
\begin{frame}{Lineare Gleichungen}

Es seien $a \neq 0$ und $b$ gegebene reelle Zahlen und $x$ eine Unbekannte.
Man nennt
\bq
 a x + b & = & 0
\eq
eine \alert{lineare Gleichung} f\"ur $x$.

\vspace*{3mm} 
Die Gleichung hat die L\"osung
\bq
 x & = & - \frac{b}{a}
\eq

\end{frame}

% page --------------------------------------------------------------------------------------------------
\begin{frame}{Quadratische Gleichungen ($a b c$-Formel)}

Es seien $a \neq 0$, $b$ und $c$ gegebene reelle Zahlen und $x$ eine Unbekannte.
Man nennt
\bq
 a x^2 + b x + c & = & 0
\eq
eine \alert{quadratische Gleichung} f\"ur $x$.

Falls $D=b^2-4ac \ge 0$, so hat die Gleichung die L\"osungen 
\bq
 x_{1/2}
 & = &
 \frac{1}{2a} \left( -b \pm \sqrt{b^2 -4 a c} \right)
\eq

\end{frame}

% page --------------------------------------------------------------------------------------------------
\begin{frame}{Quadratische Gleichungen ($p q$-Formel)}

Da $a\neq 0$ k\"onnen wir durch $a$ teilen:
\bq
 a x^2 + b x + c & = & 0
 \nonumber \\
 x^2 + \frac{b}{a} x + \frac{c}{a} & = & 0
\eq
Setzen wir $p = b/a$ und $q=c/a$ so ergibt sich
\bq
 x^2 + p x + q & = & 0
\eq
Falls $D=p^2-4q \ge 0$, so hat die Gleichung die L\"osungen 
\bq
 x_{1/2}
 & = &
 -\frac{p}{2} \pm \sqrt{\left(\frac{p}{2}\right)^2 - q}
\eq

\end{frame}

%%%%%%%%%%%%%%%%%%%%%%%%%%%%%%%%%%%%%%%%%%%%%%%%%%%%%%%%%%%%%%%%%%%%%%%%%%%%%%%%%%%%%%%%%%%%%%%%%%%%%%%%%
%%%%%%%%%%%%%%%%%%%%%%%%%%%%%%%%%%%%%%%%%%%%%%%%%%%%%%%%%%%%%%%%%%%%%%%%%%%%%%%%%%%%%%%%%%%%%%%%%%%%%%%%%
%%%%%%%%%%%%%%%%%%%%%%%%%%%%%%%%%%%%%%%%%%%%%%%%%%%%%%%%%%%%%%%%%%%%%%%%%%%%%%%%%%%%%%%%%%%%%%%%%%%%%%%%%

\section{$\mathbb{Q}[\sqrt{3}]$}

\frame{\sectionpage}

% page --------------------------------------------------------------------------------------------------
\begin{frame}{$\mathbb{Q}[\sqrt{3}]$}

Wir betrachten Zahlen der Form
\bq
 a + b \sqrt{3},
 & & 
 a,b \; \in \; \mathbb{Q}
\eq
Wir bezeichnen die Menge dieser Zahlen als
\bq
 \mathbb{Q}[\sqrt{3}]
 & = &
 \left\{
  a + b \sqrt{3}
  \; | \; a,b \; \in \; \mathbb{Q}
 \right\}
\eq
Beispiel:
\bq
 \frac{1}{3} + \frac{7}{8} \sqrt{3}
 & \in & \mathbb{Q}[\sqrt{3}]
 \nonumber \\
 \sqrt{7} & \notin &
 \mathbb{Q}[\sqrt{3}]
\eq

\end{frame}

% page --------------------------------------------------------------------------------------------------
\begin{frame}{$\mathbb{Q}[\sqrt{3}]$}

\begin{theorem}
$\mathbb{Q}[\sqrt{3}]$ ist ein K\"orper.
\end{theorem}
\end{frame}

% page --------------------------------------------------------------------------------------------------
\begin{frame}{Abgeschlossenheit}

\begin{itemize}
\item Addition:
\bq
 \left( a_1 + b_1 \sqrt{3} \right) + \left( a_2 + b_2 \sqrt{3} \right)
 & = &
 \left( a_1 + a_2 \right) + \left( b_1 + b_2 \right) \sqrt{3}
\eq

\item Multiplikation:
\bq
\lefteqn{
 \left( a_1 + b_1 \sqrt{3} \right) \cdot \left( a_2 + b_2 \sqrt{3} \right)
 = } & & \nonumber \\
 & = &
 a_1 a_2 + a_1 b_2 \sqrt{3} + b_1 a_2 \sqrt{3} + b_1 b_2 \left(\sqrt{3}\right)^2
 \nonumber \\
 & = &
 a_1 a_2 + a_1 b_2 \sqrt{3} + b_1 a_2 \sqrt{3} + 3 b_1 b_2
 \nonumber \\
 & = &
 \left( a_1 a_2 + 3 b_1 b_2 \right) + \left( a_1 b_2 + b_1 a_2 \right) \sqrt{3} 
\eq

\end{itemize}

\end{frame}

% page --------------------------------------------------------------------------------------------------
\begin{frame}{Neutrale Elemente}

\begin{itemize}
\item Addition:
\bq
 0 + \left( a + b \sqrt{3} \right)
 \; = \;
 \left( 0 + 0 \cdot \sqrt{3} \right) + \left( a + b \sqrt{3} \right)
 \; = \;
 a + b \sqrt{3}
\eq

\item Multiplikation:
\bq
 1 \cdot \left( a + b \sqrt{3} \right)
 \; = \;
 \left( 1 + 0 \cdot \sqrt{3} \right) \cdot \left( a + b \sqrt{3} \right)
 \; = \;
 a + b \sqrt{3}
\eq

\end{itemize}

\end{frame}

% page --------------------------------------------------------------------------------------------------
\begin{frame}{Inverse Elemente}

\begin{itemize}
\item Addition:
\bq
 - \left( a + b \sqrt{3} \right)
 & = &
 \left(-a\right) + \left(-b\right) \sqrt{3}
\eq

\item Multiplikation $(a,b) \neq (0,0)$:
\bq
 \frac{1}{a + b \sqrt{3}}
 & = &
 \frac{a - b \sqrt{3}}{\left(a + b \sqrt{3}\right) \left(a - b \sqrt{3}\right)}
 \nonumber \\
 & = &
 \frac{a - b \sqrt{3}}{a^2 - 3 b^2}
 \nonumber \\
 & = &
 \frac{a}{a^2 - 3 b^2}
 - \frac{b}{a^2 - 3 b^2}  \sqrt{3}
\eq

\end{itemize}

\end{frame}

% page --------------------------------------------------------------------------------------------------
\begin{frame}{Beispiel}

Das zu $1+\sqrt{3}$ bez\"uglich der Multiplikation inverse Element ist
\bq
 \frac{1}{1 + \sqrt{3}}
 & = &
 \frac{1 - \sqrt{3}}{\left(1 + \sqrt{3}\right) \left(1 - \sqrt{3}\right)}
 \; = \;
 \frac{1 - \sqrt{3}}{1 - 3}
 \; = \;
 - \frac{1}{2} \left(1 - \sqrt{3}\right)
 \nonumber \\
 & = &
 - \frac{1}{2} + \frac{1}{2} \sqrt{3}
\eq
Probe:
\bq
 \left( - \frac{1}{2} + \frac{1}{2} \sqrt{3} \right) \left( 1+\sqrt{3} \right) 
 & = &
 - \frac{1}{2} - \frac{1}{2} \sqrt{3} + \frac{1}{2} \sqrt{3} + \frac{1}{2} \left( \sqrt{3} \right)^2 
 \nonumber \\
 & = &
 - \frac{1}{2} + \frac{3}{2}
 \; = \; 
 1
\eq

\end{frame}

% page --------------------------------------------------------------------------------------------------
\begin{frame}{Bemerkung}

Bei den Grundrechenarten mit Zahlen der Form 
$a+b\sqrt{3}$ ist der wesentliche Trick
\bq
 \left( \sqrt{3} \right)^2
 & = & 3.
\eq
Setzen wir $w=\sqrt{3}$ und betrachten Zahlen $a+bw$, so lautet der wesentliche Trick
\bq
 w^2 & = & 3.
\eq

\end{frame}

% page --------------------------------------------------------------------------------------------------
\begin{frame}{Quiz}

\bq
 i^2 & = & ?
\eq 
\begin{description}
\item{(A)} $-1$
\item{(B)} unbekannt
\end{description}

\end{frame}

%%%%%%%%%%%%%%%%%%%%%%%%%%%%%%%%%%%%%%%%%%%%%%%%%%%%%%%%%%%%%%%%%%%%%%%%%%%%%%%%%%%%%%%%%%%%%%%%%%%%%%%%%
%%%%%%%%%%%%%%%%%%%%%%%%%%%%%%%%%%%%%%%%%%%%%%%%%%%%%%%%%%%%%%%%%%%%%%%%%%%%%%%%%%%%%%%%%%%%%%%%%%%%%%%%%
%%%%%%%%%%%%%%%%%%%%%%%%%%%%%%%%%%%%%%%%%%%%%%%%%%%%%%%%%%%%%%%%%%%%%%%%%%%%%%%%%%%%%%%%%%%%%%%%%%%%%%%%%

% page --------------------------------------------------------------------------------------------------
\begin{frame}

\end{frame}

\end{document}
