\documentclass[german]{beamer}

\mode<presentation>
{
 \usetheme{Madrid}

% \usecolortheme{crane}
 \usecolortheme{wolverine}
}

\usepackage{hyperref}

\usepackage[german]{babel}
\usepackage{times}
\usepackage[latin1,utf8]{inputenc}
\usepackage[OT2,T1]{fontenc}
\usepackage{shuffle}


% Stefan's abbreveations
\newcommand{\bq}{\begin{eqnarray*}}
\newcommand{\eq}{\end{eqnarray*}}
\newcommand{\eps}{\varepsilon}

\definecolor{MyYellowOrange}{cmyk}{0,0.5,1,0}
\newcommand{\superalert}[1]{{\color{MyYellowOrange}{#1}}}
\newtheorem*{myemptytheorem}{}

% dedicated environments
\newtheorem*{mytheorem27}{Hauptsatz der Differential- und Integralrechnung:} 

%%%%%%%%%%%%%%%%%%%%%%%%%%%%%%%%%%%%%%%%%%%%%%%%%%%%%%%%%%%%%%%%%%%%%%%%%%%%%%%%%%%%%%%%%%%%%%%%%%%%%%%%%
%%%%%%%%%%%%%%%%%%%%%%%%%%%%%%%%%%%%%%%%%%%%%%%%%%%%%%%%%%%%%%%%%%%%%%%%%%%%%%%%%%%%%%%%%%%%%%%%%%%%%%%%%
%%%%%%%%%%%%%%%%%%%%%%%%%%%%%%%%%%%%%%%%%%%%%%%%%%%%%%%%%%%%%%%%%%%%%%%%%%%%%%%%%%%%%%%%%%%%%%%%%%%%%%%%%

\title{Fehlerrechnung}

\subtitle{Mathematischer Br\"uckenkurs}

\author{Stefan Weinzierl}

\institute[Uni Mainz]{Institut f\"ur Physik, Universit\"at Mainz}%

\date[WiSe 2020/21]{Wintersemester 2020/21}

\begin{document}

%%%%%%%%%%%%%%%%%%%%%%%%%%%%%%%%%%%%%%%%%%%%%%%%%%%%%%%%%%%%%%%%%%%%%%%%%%%%%%%%%%%%%%%%%%%%%%%%%%%%%%%%%
%%%%%%%%%%%%%%%%%%%%%%%%%%%%%%%%%%%%%%%%%%%%%%%%%%%%%%%%%%%%%%%%%%%%%%%%%%%%%%%%%%%%%%%%%%%%%%%%%%%%%%%%%
%%%%%%%%%%%%%%%%%%%%%%%%%%%%%%%%%%%%%%%%%%%%%%%%%%%%%%%%%%%%%%%%%%%%%%%%%%%%%%%%%%%%%%%%%%%%%%%%%%%%%%%%%

\begin{frame}
  \titlepage
\end{frame}

%%%%%%%%%%%%%%%%%%%%%%%%%%%%%%%%%%%%%%%%%%%%%%%%%%%%%%%%%%%%%%%%%%%%%%%%%%%%%%%%%%%%%%%%%%%%%%%%%%%%%%%%%
%%%%%%%%%%%%%%%%%%%%%%%%%%%%%%%%%%%%%%%%%%%%%%%%%%%%%%%%%%%%%%%%%%%%%%%%%%%%%%%%%%%%%%%%%%%%%%%%%%%%%%%%%
%%%%%%%%%%%%%%%%%%%%%%%%%%%%%%%%%%%%%%%%%%%%%%%%%%%%%%%%%%%%%%%%%%%%%%%%%%%%%%%%%%%%%%%%%%%%%%%%%%%%%%%%%

\section{Motivation}

\frame{\sectionpage}

% page --------------------------------------------------------------------------------------------------
\begin{frame}{Motivation}

Eine Person A mi{\ss}t eine bestimmte Gr\"o{\ss}e experimentell.
Sie f\"uhrt diese Messung \"ofters durch. 
Aufgrund der experimentellen Me{\ss}ungenauigkeit ergeben sich leicht unterschiedliche Werte.

\vspace*{2mm}
{\footnotesize
Me{\ss}reihe 1:
\bq
\begin{array}{l|lllllllllllll}
\mbox{Messung}  & 1   & 2   & 3   & 4   & 5   & 6   & 7   & 8   & 9   & 10  & 11  & 12  & 13  \\
\hline \\
\mbox{Ergebnis} & 2.6 & 2.3 & 2.5 & 2.3 & 2.6 & 2.4 & 2.2 & 2.3 & 2.4 & 2.5 & 2.6 & 2.8 & 2.7 \\
\end{array}
\eq
}

Wir definieren den \superalert{Mittelwert einer Me{\ss}reihe} mit $n$ Me{\ss}punkten als
\bq
 \bar{x} & = & \frac{1}{n} \sum\limits_{j=1}^n x_j.
\eq
F\"ur die obige Me{\ss}reihe ergibt sich
\bq
 \bar{x} & = & 2.48
\eq

\end{frame}

% page --------------------------------------------------------------------------------------------------
\begin{frame}{Motivation}

Eine Person B bestimmt die gleiche Gr\"o{\ss}e ebenfalls experimentell.
Person B verwendet allerdings eine schlechtere Me{\ss}apparatur und f\"uhrt weniger Messungen durch.
Person B erh\"alt die folgenden Me{\ss}werte:

\vspace*{2mm}
{\footnotesize
Me{\ss}reihe 2:
\bq
\begin{array}{l|llll}
\mbox{Messung}  & 1   & 2   & 3   & 4   \\
\hline \\
\mbox{Ergebnis} & 0.3 & 5.2 & 3.1 & 1.4 \\
\end{array}
\eq
}

Der Mittelwert ergibt sich zu
\bq
 \bar{x} & = & 2.48
\eq

\end{frame}

% page --------------------------------------------------------------------------------------------------
\begin{frame}{Motivation}

\begin{itemize}
\item Im zweiten Fall streuen die einzelnen Messungen wesentlich st\"arker als im ersten Fall.

\item Es ist daher offensichtlich, da{\ss} das Ergebnis von Person A vertauensw\"urdiger als das 
Ergebnis von Person B ist. 

\item Wir wollen nun diese Aussage quantitativ machen und suchen ein
Ma{\ss} f\"ur die Streuung der Me{\ss}punkte.
\end{itemize}

\end{frame}

%%%%%%%%%%%%%%%%%%%%%%%%%%%%%%%%%%%%%%%%%%%%%%%%%%%%%%%%%%%%%%%%%%%%%%%%%%%%%%%%%%%%%%%%%%%%%%%%%%%%%%%%%
%%%%%%%%%%%%%%%%%%%%%%%%%%%%%%%%%%%%%%%%%%%%%%%%%%%%%%%%%%%%%%%%%%%%%%%%%%%%%%%%%%%%%%%%%%%%%%%%%%%%%%%%%
%%%%%%%%%%%%%%%%%%%%%%%%%%%%%%%%%%%%%%%%%%%%%%%%%%%%%%%%%%%%%%%%%%%%%%%%%%%%%%%%%%%%%%%%%%%%%%%%%%%%%%%%%

\section{Grundlagen}

\frame{\sectionpage}

% page --------------------------------------------------------------------------------------------------
\begin{frame}{Stochastik}

\begin{definition}
\bq
\Omega & : & \mbox{Ergebnismenge eines Zufallsexperiments},
 \nonumber \\
\mbox{Zufallsfunktion} & : & \mbox{Funktion} \;\; X: \; \Omega \rightarrow {\mathbb R},
\eq
Wahrscheinlichkeitsfunktion einer Zufallsgr\"o{\ss}e $X$:
\bq
 W : x & \rightarrow & P\left( \omega | X(\omega) = x \right).
\eq
{\bf Erwartungswert einer Zufallsgr\"o{\ss}e}: Nimmt die Zufallsgr\"o{\ss}e $X$ die Werte
$x_1$, $x_2$, ..., $x_n$ an, so bezeichnet man mit
\bq
 \mu\left(X \right) & = & \sum\limits_{j=1}^n x_j W(x_j)
\eq
den Erwartungswert von $X$.
\end{definition}

\end{frame}

% page --------------------------------------------------------------------------------------------------
\begin{frame}{Stochastik}

\begin{theorem}
Entsprechen die einzelnen Messungen einzelnen unabh\"angigen Realisierungen eines
Zufallsexperiments, so ist der Mittelwert $\bar{x}$ eine Sch\"atzung f\"ur $\mu(X)$.
\end{theorem}

\end{frame}

% page --------------------------------------------------------------------------------------------------
\begin{frame}{Varianz und Standardabweichung}

\begin{definition}
Die Varianz einer Zufallsgr\"o{\ss}e ist definiert durch
\bq
 \mbox{Var}(X) & = & \sum\limits_{j=1}^n \left( x_j - \mu \right)^2 W(x_j).
\eq
\end{definition}

\begin{definition}
Die Standardabweichung einer Zufallsgr\"o{\ss}e ist definiert durch
\bq
 \sigma(X) & = & \sqrt{\mbox{Var}(X)}.
\eq
\end{definition}

\end{frame}

% page --------------------------------------------------------------------------------------------------
\begin{frame}{Sch\"atzfunktion f\"ur die Varianz}

\alert{Kennen wir den Erwartungswert $\mu$} einer Zufallsgr\"o{\ss}e und machen $n$ Messungen $x_j$, so ist
\bq
 \frac{1}{n} \sum\limits_{j=1}^n \left( x_j - \mu \right)^2
\eq
eine Sch\"atzfunktion f\"ur die Varianz.

\vspace*{3mm}
\superalert{Im Allgemeinen ist $\mu$ aber nicht bekannt} und man verwendet $\bar{x}$ als Sch\"atzung f\"ur $\mu$.
In diesem Fall ist 
\bq
 S^2 & = & 
 \frac{1}{n-1} \sum\limits_{j=1}^n \left( x_j - \bar{x} \right)^2
\eq
eine Sch\"atzfunktion f\"ur die Varianz der Zufallsgr\"o{\ss}e $X$.

\end{frame}

% page --------------------------------------------------------------------------------------------------
\begin{frame}{S\"atze \"uber die Varianz}

Sei $c \in \mathbb R$ und seien $X_1$, $X_2$, ..., $X_n$ unabh\"angige Zufallsgr\"o{\ss}en.
Dann gilt:
\begin{myemptytheorem}
\bq
 \mbox{Var}\left( c X \right) & = & c^2 \; \mbox{Var}\left(X \right),
 \nonumber \\
 \mbox{Var}\left( X_1 + X_2 + ... + X_n  \right) & = & 
  \mbox{Var}\left(X_1 \right) + \mbox{Var}\left(X_2 \right) + ...
  + \mbox{Var}\left(X_n \right).
\eq
\end{myemptytheorem}
Insbesondere gilt:
\bq
 \mbox{Var}\left( \frac{1}{n} \underbrace{\left( X + X + ... + X  \right)}_{n\;\mbox{\scriptsize mal}} \right) & = & 
 \frac{1}{n^2} \left( 
  \mbox{Var}\left(X \right) + \mbox{Var}\left(X \right) + ...
  + \mbox{Var}\left(X \right) \right)
 \nonumber \\
 & = &
 \frac{1}{n} \mbox{Var}\left(X \right).
\eq

\end{frame}

% page --------------------------------------------------------------------------------------------------
\begin{frame}{Varianz des Mittelwertes}

Es interessiert in erster Linie nicht die Varianz der einzelnen Messungen $\mbox{Var}(X)$, sondern
die Varianz des Mittelwertes $\mbox{Var}(\bar{X})$.
Bei $n$ Messungen gilt:
\bq
 \mbox{Var}(\bar{X}) & = & \frac{1}{n} \mbox{Var}(X).
\eq
Somit erh\"alt man als Sch\"atzung f\"ur die Varianz des Mittelwertes
\bq
 S_{\bar{X}}^2 & = & \frac{1}{n} S^2
 = \frac{1}{n(n-1)} \sum\limits_{j=1}^n \left( x_j - \bar{x} \right)^2.
\eq
F\"ur die Standardabweichung erh\"alt man
\bq
 \sigma_{\bar{X}} & = & 
 \sqrt{ \frac{1}{n(n-1)} \sum\limits_{j=1}^n \left( x_j - \bar{x} \right)^2}.
\eq

\end{frame}

% page --------------------------------------------------------------------------------------------------
\begin{frame}{Beispiel}

Somit findet man f\"ur die beiden oben aufgef\"uhrten Me{\ss}reihen:
\bq
 \mbox{Me{\ss}reihe}\; 1: & & \sigma_{\bar{X}} = 0.05,
 \nonumber \\
 \mbox{Me{\ss}reihe}\; 2: & & \sigma_{\bar{X}} = 1.07.
\eq
Es ist \"ublich mit dem Mittelwert auch immer die Standardabweichung anzugeben, also
\bq
 \mbox{Me{\ss}reihe}\; 1: & & x = 2.48 \pm 0.05,
 \nonumber \\
 \mbox{Me{\ss}reihe}\; 2: & & x = 2.48 \pm 1.07.
\eq

\end{frame}

% page --------------------------------------------------------------------------------------------------
\begin{frame}{Die Normalverteilung}

Zur Interpretation der Standardabweichung betrachten wir zun\"achst \alert{kontinuierliche Zufallsgr\"o{\ss}en}.
Die Wahrscheinlichkeitsdichtefunktion $p(x)$ f\"ur eine kontinuierliche Zufallsgr\"o{\ss}e
beschreibt
\bq
 P\left( a < X \le b \right)
 & = & \int\limits_a^b p(x) \; dx.
\eq
Definition: Man nennt eine kontinuierliche Zufallsgr\"o{\ss}e \superalert{normalverteilt},
falls sie die Wahrscheinlichkeitsdichtefunktion
\bq
 p(x) & = & \frac{1}{\sqrt{2\pi}} \frac{1}{\sigma} \exp\left( - \frac{1}{2} \frac{(x-\mu)^2}{\sigma^2} \right)
\eq
besitzt. Der Erwartungswert dieser normalverteilten Zufallsgr\"o{\ss}e ist $\mu$, die Standardabweichung
ist $\sigma$.

\end{frame}

% page --------------------------------------------------------------------------------------------------
\begin{frame}{Die Normalverteilung}

F\"ur eine normalverteilte Zufallsgr\"o{\ss}e gilt:
\bq
 P\left( \mu-\sigma < X \le \mu+\sigma \right) & \approx & 68.27 \%,
 \nonumber \\
 P\left( \mu-2\sigma < X \le \mu+2\sigma \right) & \approx & 95.45 \%,
 \nonumber \\
 P\left( \mu-3\sigma < X \le \mu+3\sigma \right) & \approx & 99.73 \%.
\eq

\end{frame}

% page --------------------------------------------------------------------------------------------------
\begin{frame}{Quiz}

Ein Experiment mi{\ss}t eine Gr\"o{\ss}e $O$ und berichtet
\bq
 O & = &
 5.94\pm 0.02
\eq
Dies bedeutet:

\vspace*{5mm}
(A) Der wahre Wert der Observablen $O$ ist $5.94$.

\vspace*{2mm}
(B) Der wahre Wert der Observablen $O$ liegt im Intervall $[5.92,5.96]$.

\vspace*{2mm}
(C) Die Wahrscheinlichkeit, da{\ss} der wahre Wert der Observablen $O$ im Intervall $[5.92,5.96]$ liegt, betr\"agt $99.7\%$.

\vspace*{2mm}
(D) Die Wahrscheinlichkeit, da{\ss} der wahre Wert der Observablen $O$ im Intervall $[5.92,5.96]$ liegt, betr\"agt $68.3\%$,
falls der Me{\ss}wert einer Normalverteilung folgt.

\end{frame}

%%%%%%%%%%%%%%%%%%%%%%%%%%%%%%%%%%%%%%%%%%%%%%%%%%%%%%%%%%%%%%%%%%%%%%%%%%%%%%%%%%%%%%%%%%%%%%%%%%%%%%%%%
%%%%%%%%%%%%%%%%%%%%%%%%%%%%%%%%%%%%%%%%%%%%%%%%%%%%%%%%%%%%%%%%%%%%%%%%%%%%%%%%%%%%%%%%%%%%%%%%%%%%%%%%%
%%%%%%%%%%%%%%%%%%%%%%%%%%%%%%%%%%%%%%%%%%%%%%%%%%%%%%%%%%%%%%%%%%%%%%%%%%%%%%%%%%%%%%%%%%%%%%%%%%%%%%%%%

\section{Fehlerfortpflanzung}

\frame{\sectionpage}

% page --------------------------------------------------------------------------------------------------
\begin{frame}{Problemstellung}

Gesucht wird eine Gr\"o{\ss}e $f=f(x,y)$ die von zwei weiteren Gr\"o{\ss}en
$x$ und $y$ abh\"angt. 

Die Funktion $f$ wird als bekannt vorausgesetzt, die Gr\"o{\ss}en 
$x$ und $y$ werden durch eine Messung mit Fehlern $x \pm \Delta x$ und $y \pm \Delta y$ bestimmt.

Gesucht ist nun der Fehler f\"ur die Gr\"o{\ss}e $f$.

\end{frame}

% page --------------------------------------------------------------------------------------------------
\begin{frame}{Fehlerfortpflanzung}

F\"ur die Gr\"o{\ss}e $f$ beginnen wir mit der Taylorentwicklung:
\bq
 f\left(x+\Delta x, y+ \Delta y \right) 
 & = & f(x,y) + \frac{\partial f(x,y)}{\partial x} \Delta x
              + \frac{\partial f(x,y)}{\partial y} \Delta y 
              + ...
\eq
Wir nehmen an, da{\ss} wir $n$ Messungen f\"ur die Gr\"o{\ss}en $x$ und $y$ haben, die einzelnen
Me{\ss}werte seien mit $x_j$ und $y_j$ bezeichnet.
Somit haben wir auch $n$ Ergebnisse f\"ur $f$.
F\"ur die Abweichung eines Einzelergebnisses vom Mittelwert gilt f\"ur kleine Abweichungen
\bq
 f_j - \bar{f} & = & \frac{\partial f}{\partial x} \cdot \left( x_j - \bar{x} \right)
                   + \frac{\partial f}{\partial y} \cdot \left( y_j - \bar{y} \right)
                   + ...
\eq
\end{frame}

% page --------------------------------------------------------------------------------------------------
\begin{frame}{Fehlerfortpflanzung}

\bq
 f_j - \bar{f} & = & \frac{\partial f}{\partial x} \cdot \left( x_j - \bar{x} \right)
                   + \frac{\partial f}{\partial y} \cdot \left( y_j - \bar{y} \right)
                   + ...
\eq
Somit gilt f\"ur die Varianz:
{\scriptsize
\bq
\lefteqn{
 \sigma_f^2 = 
} & &
\nonumber \\
 & = & 
 \lim\limits_{n \rightarrow \infty} \frac{1}{n} \sum\limits_{j=1}^n \left( f_j - \bar{f} \right)^2
 \nonumber \\
 & = & 
 \lim\limits_{n \rightarrow \infty} \frac{1}{n} \sum\limits_{j=1}^n 
  \left[ 
         \left( x_j - \bar{x} \right)^2 \left( \frac{\partial f}{\partial x} \right)^2
       + \left( y_j - \bar{y} \right)^2 \left( \frac{\partial f}{\partial y} \right)^2
       + 2 \left( x_j - \bar{x} \right) \left( y_j - \bar{y} \right) 
           \left( \frac{\partial f}{\partial x} \right) \left( \frac{\partial f}{\partial y} \right)
  \right]
\eq
}

\end{frame}

% page --------------------------------------------------------------------------------------------------
\begin{frame}{Kovarianz}

Wir definieren die {\bf Kovarianz} als
\bq
 \mbox{Cov}(x,y) & = & \sigma_{x y} = 
 \lim\limits_{n \rightarrow \infty} \frac{1}{n} \sum\limits_{j=1}^n 
  \left( x_j - \bar{x} \right) \left( y_j - \bar{y} \right) 
\eq
Somit haben wir
\bq
 \sigma_f^2 & = & 
         \left( \frac{\partial f}{\partial x} \right)^2 \sigma_x^2
       + \left( \frac{\partial f}{\partial y} \right)^2 \sigma_y^2
       + 2 \left( \frac{\partial f}{\partial x} \right) \left( \frac{\partial f}{\partial y} \right)
           \sigma_{xy}.
\eq

\end{frame}

% page --------------------------------------------------------------------------------------------------
\begin{frame}{Unkorrelierte Zufallsgr\"o{\ss}en}

Falls $x$ und $y$ unkorreliert sind, gilt $\sigma_{x y} = 0$ und somit
\bq
 \sigma_f^2 & = & 
         \left( \frac{\partial f}{\partial x} \right)^2 \sigma_x^2
       + \left( \frac{\partial f}{\partial y} \right)^2 \sigma_y^2,
\eq
bzw.
\begin{myemptytheorem}
\bq
 \sigma_f & = & 
         \sqrt{ \left( \frac{\partial f}{\partial x} \right)^2 \sigma_x^2
       + \left( \frac{\partial f}{\partial y} \right)^2 \sigma_y^2}.
\eq
\end{myemptytheorem}

\end{frame}

% page --------------------------------------------------------------------------------------------------
\begin{frame}{Beispiele: Addition}

\begin{enumerate}

\item $f=x+y$. In diesem Fall haben wir
\bq
 \sigma_f & = & \sqrt{\sigma_x^2 + \sigma_y^2},
\eq
man sagt, die (absoluten) Fehler addieren sich quadratisch.

\end{enumerate}

\end{frame}

% page --------------------------------------------------------------------------------------------------
\begin{frame}{Beispiele: Addition}

\begin{example}
\bq
 f \; = \; x+y,
 & & 
 \sigma_f \; = \; \sqrt{\sigma_x^2 + \sigma_y^2}.
\eq
Es sei $x=15\pm 3$ und $y=17\pm 4$.

\vspace*{5mm}
Es ist
\bq
 \bar{f} & = & \bar{x} + \bar{y} \; = \; 15 + 17 \; = \; 32,
 \nonumber \\
 \sigma_f & = & \sqrt{\sigma_x^2 + \sigma_y^2} \; = \; \sqrt{3^2 + 4^2} \; = \; 5.
\eq
Somit
\bq
 f & = & 32 \pm 5.
\eq
\end{example}

\end{frame}

% page --------------------------------------------------------------------------------------------------
\begin{frame}{Beispiele: Multiplikation}

\begin{enumerate}[2]

\item $f = x \cdot y$. In diesem Fall findet man
\bq 
\sigma_f & = & \sqrt{y^2 \sigma_x^2 + x^2 \sigma_y^2},
\eq
oder anders geschrieben
\bq
 \frac{\sigma_f}{f} & = & \sqrt{ \left(\frac{\sigma_x}{x}\right)^2 + \left(\frac{\sigma_y}{y}\right)^2 }.
\eq
Bei einem Produkt addieren sich die relativen Fehler quadratisch.

\end{enumerate}

\end{frame}

% page --------------------------------------------------------------------------------------------------
\begin{frame}{Beispiele: Multiplikation}

\begin{example}
\bq
 f \; = \; x \cdot y,
 & & 
 \frac{\sigma_f}{f} \; = \; \sqrt{ \left(\frac{\sigma_x}{x}\right)^2 + \left(\frac{\sigma_y}{y}\right)^2 }.
\eq
Es sei $x=2\pm 0.06$ und $y=5\pm 0.2$.

\vspace*{5mm}
Es ist
{\footnotesize
\bq
 \bar{f} & = & \bar{x} \cdot \bar{y} \; = \; 2 \cdot 5 \; = \; 10,
 \nonumber \\
 \sigma_f & = & f \sqrt{ \left(\frac{\sigma_x}{x}\right)^2 + \left(\frac{\sigma_y}{y}\right)^2 } 
 \; = \; 
 10 \sqrt{ \left(\frac{6}{200}\right)^2 + \left(\frac{2}{50}\right)^2 }
 \; = \;
 \frac{10}{\sqrt{400}}
 \; = \; 
 \frac{1}{2}.
\eq
}

\vspace*{-2mm}
Somit
\bq
 f & = & 10 \pm 0.5.
\eq
\end{example}

\end{frame}

% page --------------------------------------------------------------------------------------------------
\begin{frame}{Beispiele: Subtraktion}

\begin{enumerate}[3]

\item $f=x-y$. Hier findet man wie bei einer Summe
\bq
 \sigma_f & = & \sqrt{\sigma_x^2 + \sigma_y^2}.
\eq

\end{enumerate}

\end{frame}

% page --------------------------------------------------------------------------------------------------
\begin{frame}{Quiz}

Es sei $x=17 \pm 4$ und $y=15\pm3$, sowie
\bq
 f & = & x-y.
\eq
Mittelwert und Fehler f\"ur $f$ ergeben sich zu
\begin{description}
\item{(A)} $f = 2 \pm 1$
\item{(B)} $f = 2 \pm \sqrt{7}$
\item{(C)} $f = 2 \pm 4$
\item{(D)} $f = 2 \pm 5$
\end{description}

\end{frame}

% page --------------------------------------------------------------------------------------------------
\begin{frame}{Beispiele: Division}

\begin{enumerate}[4]

\item $f = \frac{x}{y}$. In diesem Fall findet man
\bq 
\sigma_f & = & \sqrt{\frac{1}{y^2} \sigma_x^2 + \frac{x^2}{y^4} \sigma_y^2}.
\eq
Schreibt man dies mit Hilfe der relativen Fehler erh\"alt man wie beim Produkt
\bq
 \frac{\sigma_f}{f} & = & \sqrt{ \left(\frac{\sigma_x}{x}\right)^2 + \left(\frac{\sigma_y}{y}\right)^2 }.
\eq

\end{enumerate}

\end{frame}

% page --------------------------------------------------------------------------------------------------
\begin{frame}{Beispiele: Division}

\begin{example}
\bq
 f \; = \; \frac{x}{y},
 & & 
 \frac{\sigma_f}{f} \; = \; \sqrt{ \left(\frac{\sigma_x}{x}\right)^2 + \left(\frac{\sigma_y}{y}\right)^2 }.
\eq
Es sei $x=2\pm 0.06$ und $y=5\pm 0.2$.

\vspace*{5mm}
Es ist
{\footnotesize
\bq
 \bar{f} & = & \frac{\bar{x}}{\bar{y}} \; = \; \frac{2}{5} \; = \; 0.4,
 \nonumber \\
 \sigma_f & = & f \sqrt{ \left(\frac{\sigma_x}{x}\right)^2 + \left(\frac{\sigma_y}{y}\right)^2 } 
 \; = \; 
 \frac{2}{5} \sqrt{ \left(\frac{6}{200}\right)^2 + \left(\frac{2}{50}\right)^2 }
 \; = \;
 \frac{2}{5 \sqrt{400}}
 \; = \; 
 \frac{1}{50}.
\eq
}

\vspace*{-2mm}
Somit
\bq
 f & = & 0.4 \pm 0.02.
\eq
\end{example}

\end{frame}

% page --------------------------------------------------------------------------------------------------
\begin{frame}{Beispiele: Potenzen}

\begin{enumerate}[5]

\item Zum Abschluss betrachten wir noch $f = x^a y^b$. Man erh\"alt
\bq
 \sigma_f & = & \sqrt{ \left( a x^{a-1} y^b \right)^2 \sigma_x^2 
                     + \left( b x^a y^{b-1} \right)^2 \sigma_y^2}
\eq
Auch hier empfiehlt es sich wieder, die Formel in relativen Fehler zu schreiben:
\bq
 \frac{\sigma_f}{f} & = & \sqrt{ a^2 \left(\frac{\sigma_x}{x}\right)^2 + b^2 \left(\frac{\sigma_y}{y}\right)^2 }.
\eq

\end{enumerate}

\end{frame}

%%%%%%%%%%%%%%%%%%%%%%%%%%%%%%%%%%%%%%%%%%%%%%%%%%%%%%%%%%%%%%%%%%%%%%%%%%%%%%%%%%%%%%%%%%%%%%%%%%%%%%%%%
%%%%%%%%%%%%%%%%%%%%%%%%%%%%%%%%%%%%%%%%%%%%%%%%%%%%%%%%%%%%%%%%%%%%%%%%%%%%%%%%%%%%%%%%%%%%%%%%%%%%%%%%%
%%%%%%%%%%%%%%%%%%%%%%%%%%%%%%%%%%%%%%%%%%%%%%%%%%%%%%%%%%%%%%%%%%%%%%%%%%%%%%%%%%%%%%%%%%%%%%%%%%%%%%%%%

\section{Kombination von Messungen}

\frame{\sectionpage}

% page --------------------------------------------------------------------------------------------------
\begin{frame}{Anwendungsbeispiel}

Wir hatten zuvor den Fall betrachtet, da{\ss} eine Gr\"o{\ss}e durch zwei Me{\ss}reihen 
experimentell bestimmt wird: 
\bq
 \mbox{Me{\ss}reihe}\; 1: & & x = 2.48 \pm 0.05,
 \nonumber \\
 \mbox{Me{\ss}reihe}\; 2: & & x = 2.48 \pm 1.07.
\eq
Es stellt sich nun die Frage, wie man diese Ergebnisse miteinander kombiniert.

\end{frame}

% page --------------------------------------------------------------------------------------------------
\begin{frame}{Kombination von Messungen}

Etwas allgemeiner seien f\"ur eine Gr\"o{\ss}e $x$ $n$ Messungen $x_j$ mit Fehlern $\sigma_j$ gegeben.

\vspace*{3mm}
Dann setzt man
\begin{myemptytheorem}
\bq
 x = \frac{\frac{1}{\sigma_1^2} x_1 + \frac{1}{\sigma_2^2} x_2 + ... + \frac{1}{\sigma_n^2} x_n}
              {\frac{1}{\sigma_1^2} + \frac{1}{\sigma_2^2} + ... + \frac{1}{\sigma_n^2} },
 & &
 \sigma = \frac{1}{\sqrt{\frac{1}{\sigma_1^2} + \frac{1}{\sigma_2^2} + ... + \frac{1}{\sigma_n^2}}}.
\eq
\end{myemptytheorem}

\end{frame}

% page --------------------------------------------------------------------------------------------------
\begin{frame}{Anwendungsbeispiel}

F\"ur das Beispiel hat man
\bq
 x_1 = 2.48, & & \sigma_1 = 0.05,
 \nonumber \\
 x_2 = 2.48, & & \sigma_2 = 1.07.
\eq
Man findet somit
\bq
 x = 2.48, & & \sigma = 0.04995.
\eq
Die zweite Me{\ss}reihe liefert keinen wesentlichen Beitrag zur Verbesserung des Fehlers.

\end{frame}

%%%%%%%%%%%%%%%%%%%%%%%%%%%%%%%%%%%%%%%%%%%%%%%%%%%%%%%%%%%%%%%%%%%%%%%%%%%%%%%%%%%%%%%%%%%%%%%%%%%%%%%%%
%%%%%%%%%%%%%%%%%%%%%%%%%%%%%%%%%%%%%%%%%%%%%%%%%%%%%%%%%%%%%%%%%%%%%%%%%%%%%%%%%%%%%%%%%%%%%%%%%%%%%%%%%
%%%%%%%%%%%%%%%%%%%%%%%%%%%%%%%%%%%%%%%%%%%%%%%%%%%%%%%%%%%%%%%%%%%%%%%%%%%%%%%%%%%%%%%%%%%%%%%%%%%%%%%%%

% page --------------------------------------------------------------------------------------------------
\begin{frame}

\end{frame}

\end{document}

