\documentclass[german]{beamer}

\mode<presentation>
{
 \usetheme{Madrid}

% \usecolortheme{crane}
 \usecolortheme{wolverine}
}

\usepackage{hyperref}

\usepackage[german]{babel}
\usepackage{times}
\usepackage[latin1,utf8]{inputenc}
\usepackage[OT2,T1]{fontenc}
\usepackage{shuffle}

\usepackage{tikz}
\usetikzlibrary{hobby}

% Stefan's abbreveations
\newcommand{\bq}{\begin{eqnarray*}}
\newcommand{\eq}{\end{eqnarray*}}
\newcommand{\eps}{\varepsilon}

\definecolor{MyYellowOrange}{cmyk}{0,0.5,1,0}
\newcommand{\superalert}[1]{{\color{MyYellowOrange}{#1}}}
\newtheorem*{myemptytheorem}{}

% dedicated environments

%%%%%%%%%%%%%%%%%%%%%%%%%%%%%%%%%%%%%%%%%%%%%%%%%%%%%%%%%%%%%%%%%%%%%%%%%%%%%%%%%%%%%%%%%%%%%%%%%%%%%%%%%
%%%%%%%%%%%%%%%%%%%%%%%%%%%%%%%%%%%%%%%%%%%%%%%%%%%%%%%%%%%%%%%%%%%%%%%%%%%%%%%%%%%%%%%%%%%%%%%%%%%%%%%%%
%%%%%%%%%%%%%%%%%%%%%%%%%%%%%%%%%%%%%%%%%%%%%%%%%%%%%%%%%%%%%%%%%%%%%%%%%%%%%%%%%%%%%%%%%%%%%%%%%%%%%%%%%

\title{Funktionen}

\subtitle{Mathematischer Br\"uckenkurs}

\author{Stefan Weinzierl}

\institute[Uni Mainz]{Institut f\"ur Physik, Universit\"at Mainz}%

\date[WiSe 2020/21]{Wintersemester 2020/21}

\begin{document}

%%%%%%%%%%%%%%%%%%%%%%%%%%%%%%%%%%%%%%%%%%%%%%%%%%%%%%%%%%%%%%%%%%%%%%%%%%%%%%%%%%%%%%%%%%%%%%%%%%%%%%%%%
%%%%%%%%%%%%%%%%%%%%%%%%%%%%%%%%%%%%%%%%%%%%%%%%%%%%%%%%%%%%%%%%%%%%%%%%%%%%%%%%%%%%%%%%%%%%%%%%%%%%%%%%%
%%%%%%%%%%%%%%%%%%%%%%%%%%%%%%%%%%%%%%%%%%%%%%%%%%%%%%%%%%%%%%%%%%%%%%%%%%%%%%%%%%%%%%%%%%%%%%%%%%%%%%%%%

\begin{frame}
  \titlepage
\end{frame}

%%%%%%%%%%%%%%%%%%%%%%%%%%%%%%%%%%%%%%%%%%%%%%%%%%%%%%%%%%%%%%%%%%%%%%%%%%%%%%%%%%%%%%%%%%%%%%%%%%%%%%%%%
%%%%%%%%%%%%%%%%%%%%%%%%%%%%%%%%%%%%%%%%%%%%%%%%%%%%%%%%%%%%%%%%%%%%%%%%%%%%%%%%%%%%%%%%%%%%%%%%%%%%%%%%%
%%%%%%%%%%%%%%%%%%%%%%%%%%%%%%%%%%%%%%%%%%%%%%%%%%%%%%%%%%%%%%%%%%%%%%%%%%%%%%%%%%%%%%%%%%%%%%%%%%%%%%%%%

\section{Grundlagen}

\frame{\sectionpage}

% page --------------------------------------------------------------------------------------------------
\begin{frame}{Funktionen}

\begin{definition}
Seien $D$ und $W$ Teilmengen von ${\mathbb R}$. Unter einer reellwertigen Funktion auf $D$
versteht man eine Abbildung
\bq
 f & : & D \rightarrow W, \nonumber \\
   & & x \rightarrow y = f(x).
\eq
Man nennt $D$ den \alert{Definitionsbereich} und $W$ den \alert{Wertebereich} der Funktion.

Eine Funktion $f$ ordnet jedem $x\in D$  ein $y\in W$ zu.
\end{definition}

\end{frame}

% page --------------------------------------------------------------------------------------------------
\begin{frame}{Umkehrfunktion}

\begin{definition}
Gibt es zu jedem $y\in W$ genau ein $x\in D$ mit $y=f(x)$,
so ist die Funktion $f$ umkehrbar. In diesem Fall bezeichnet
man mit $f^{-1}$ die Umkehrfunktion:
\bq
 f^{-1} & : & W \rightarrow D,
 \nonumber \\
 & & y \rightarrow x=f^{-1}(y).
\eq
\end{definition}

\end{frame}

% page --------------------------------------------------------------------------------------------------
\begin{frame}{Umkehrfunktion}

\begin{example}
Es sei $D=\mathbb{R}_0^+$ und $W=\mathbb{R}_0^+$ sowie
\bq
 f & : & D \rightarrow W,
 \nonumber \\
 & & x \rightarrow x^2.
\eq
Dann lautet die Umkehrfunktion
\bq
 f^{-1} & : & W \rightarrow D,
 \nonumber \\
 & & y \rightarrow \sqrt{y}.
\eq
\end{example}

\end{frame}

%%%%%%%%%%%%%%%%%%%%%%%%%%%%%%%%%%%%%%%%%%%%%%%%%%%%%%%%%%%%%%%%%%%%%%%%%%%%%%%%%%%%%%%%%%%%%%%%%%%%%%%%%
%%%%%%%%%%%%%%%%%%%%%%%%%%%%%%%%%%%%%%%%%%%%%%%%%%%%%%%%%%%%%%%%%%%%%%%%%%%%%%%%%%%%%%%%%%%%%%%%%%%%%%%%%
%%%%%%%%%%%%%%%%%%%%%%%%%%%%%%%%%%%%%%%%%%%%%%%%%%%%%%%%%%%%%%%%%%%%%%%%%%%%%%%%%%%%%%%%%%%%%%%%%%%%%%%%%

\section{Stetigkeit}

\frame{\sectionpage}

% page --------------------------------------------------------------------------------------------------
\begin{frame}{Grenzwerte von Funktionen}

\begin{definition}
Man sagt eine Funktion hat im Punkte $a$ den Grenzwert $c$,
falls es mindestens eine Folge $(x_n)\in D$ mit $\lim\limits_{n \rightarrow \infty} x_n=a$ gibt.
Gilt dann f\"ur jede Folge $(x_n)\in D$ mit $\lim\limits_{n \rightarrow \infty} x_n=a$, da{\ss}
\bq
 \lim\limits_{n \rightarrow \infty} f\left(x_n\right) & = & c,
\eq
so bezeichnet man $c$ als den Grenzwert der Funktion $f(x)$ im Punkte $a$.
\end{definition}
In diesem Fall schreibt man
\bq
 \lim\limits_{x\rightarrow a} f(x) & = & c.
\eq

\end{frame}

% page --------------------------------------------------------------------------------------------------
\begin{frame}{Grenzwerte von Funktionen}

\begin{theorem}
Die obige Bedingung ist \"aquivalent zu der Forderung, da{\ss} 
es zu jedem $\eps>0$ ein $\delta>0$ gibt, so da{\ss}
\bq
 \left| f(x) - c \right| & < & \eps, \;\;\; \forall \left|x-a\right|<\delta
 \;\;\;\mbox{und} \;\;\;x\in D.
\eq
\end{theorem}

Bemerkung: Es wird nicht vorausgesetzt, da{\ss} $a\in D$ liegt. Die Definition macht auch Sinn,
falls $D$ ein offenes Intervall ist und der Grenzwert an den Intervallgrenzen betrachtet wird.

\end{frame}

% page --------------------------------------------------------------------------------------------------
\begin{frame}{Stetigkeit}

\begin{definition}
Sei nun $a\in D$. Man bezeichnet eine Funktion als {\bf stetig} im Punkte $a$,
falls
\bq
 \lim\limits_{x \rightarrow a} f(x) & = & f(a)
\eq
gilt.
\end{definition}

\begin{definition}
Man bezeichnet eine Funktion als in einem Intervall stetig, falls sie in jedem Punkt des
Intervalls stetig ist.
\end{definition}

\end{frame}

% page --------------------------------------------------------------------------------------------------
\begin{frame}{Die Heaviside-Funktion}

\begin{example}
Wir betrachten die Heaviside-Funktion, definiert durch
\bq
 \Theta(x) & = & \left\{
 \begin{array}{ll}
 1, & x>0, \\
 0, & x \le 0.
 \end{array} \right.
\eq
F\"ur diese Funktion gilt $\Theta(0)=0$, aber
\bq 
\lim\limits_{x\rightarrow 0+} \Theta(x) & = & 1.
\eq
Die Heaviside-Funktion ist im Punkte $0$ nicht stetig.
\end{example}

\end{frame}

% page --------------------------------------------------------------------------------------------------
\begin{frame}{Stetige Funktionen}

\begin{example}
Beispiele von Funktionen, die auf ganz $\mathbb R$ stetig sind, sind
Polynomfunktionen, $\exp x$, $\sin x$, $\cos x$, $\sinh x$, $\cosh x$.
\end{example}

\end{frame}

% page --------------------------------------------------------------------------------------------------
\begin{frame}{S\"atze \"uber stetige Funktionen}

\begin{theorem}
Seien $f,g: D \rightarrow {\mathbb R}$ Funktionen, die in $a$ stetig sind 
und sei $\lambda \in {\mathbb R}$.
Dann sind auch die Funktionen
\bq
 f+g & : & D \rightarrow {\mathbb R},
 \nonumber \\
 \lambda \cdot f & : & D \rightarrow {\mathbb R},
 \nonumber \\
 f \cdot g & : & D \rightarrow {\mathbb R}
\eq
im Punkte $a$ stetig. Ist ferner $g(a)\neq0$, so ist auch die Funktion
\bq
 \frac{f}{g} & : & D' \rightarrow {\mathbb R}
\eq
in $a$ stetig, wobei $D'=\{x\in D: g(x) \neq 0 \}$.
\end{theorem}

\end{frame}

% page --------------------------------------------------------------------------------------------------
\begin{frame}{Gleichm\"a{\ss}ige Stetigkeit}

\begin{definition}
Eine Funktion $f:D\rightarrow \mathbb R$ hei{\ss}t in $D$ gleichm\"a{\ss}ig stetig,
falls es zu jedem $\eps>0$ ein $\delta>0$ gibt, so da{\ss}
\bq
 \left| f(x) - f(y) \right| < \eps
 & & \forall \; \left| x-y \right| < \delta.
\eq
\end{definition}

\begin{itemize}
\item Jede Funktion, die auf $D$ gleichm\"assig stetig ist, ist auch in jedem Punkte aus $D$ stetig im
herk\"ommlichen Sinne. Die Umkehrung gilt jedoch nicht.

\item Ist eine Funktion in jedem Punkte $x \in D$ stetig im herk\"ommlichen Sinne, 
so gen\"ugt es f\"ur ein vorgegebenes $\eps$
f\"ur jeden Punkt ein $\delta_x$ zu finden. Dieses $\delta_x$ darf mit $x$ variieren.
F\"ur die gleichm\"a{\ss}ige Stetigkeit wird dagegen gefordert, da{\ss} $\delta$ von $x$ 
unabh\"angig ist.

\end{itemize}

\end{frame}

% page --------------------------------------------------------------------------------------------------
\begin{frame}{Quiz}

Die Funktion
\bq
 f(x) & = & \left\{\begin{array}{ll} 0 & x \le 0 \\ \sin(x) & x>0 \end{array} \right.
\eq
ist im Punkte $x=0$
\begin{description}
\item{(A)} stetig
\item{(B)} nicht stetig
\end{description}

\end{frame}

% page --------------------------------------------------------------------------------------------------
\begin{frame}{Quiz}

Die Funktion
\bq
 f(x) & = & \left\{\begin{array}{ll} e^{-x} & x \le 0 \\ \cos(x) & x>0 \end{array} \right.
\eq
ist im Punkte $x=0$
\begin{description}
\item{(A)} stetig
\item{(B)} nicht stetig
\end{description}

\end{frame}

% page --------------------------------------------------------------------------------------------------
\begin{frame}{Quiz}

Die Funktion
\bq
 f(x) & = & \left\{\begin{array}{ll} \frac{1}{2} - e^{-x} & x \le 0 \\ \frac{1}{2} + e^{x} & x>0 \end{array} \right.
\eq
ist im Punkte $x=0$
\begin{description}
\item{(A)} stetig
\item{(B)} nicht stetig
\end{description}

\end{frame}

%%%%%%%%%%%%%%%%%%%%%%%%%%%%%%%%%%%%%%%%%%%%%%%%%%%%%%%%%%%%%%%%%%%%%%%%%%%%%%%%%%%%%%%%%%%%%%%%%%%%%%%%%
%%%%%%%%%%%%%%%%%%%%%%%%%%%%%%%%%%%%%%%%%%%%%%%%%%%%%%%%%%%%%%%%%%%%%%%%%%%%%%%%%%%%%%%%%%%%%%%%%%%%%%%%%
%%%%%%%%%%%%%%%%%%%%%%%%%%%%%%%%%%%%%%%%%%%%%%%%%%%%%%%%%%%%%%%%%%%%%%%%%%%%%%%%%%%%%%%%%%%%%%%%%%%%%%%%%

\section{Rationale Funktionen}

\frame{\sectionpage}

% page --------------------------------------------------------------------------------------------------
\begin{frame}{Rationale Funktionen}

\begin{definition}
Seien $p(x)$ und $q(x)$ Polynomfunktionen.
Unter einer rationalen Funktion versteht man eine Funktion
\bq
 R(x) & = & \frac{p(x)}{q(x)}.
\eq
Der Definitionsbereich einer rationalen Funktion ist gegeben durch
$D =\{ x\in {\mathbb R}, q(x) \neq 0\}$.
\end{definition}
Eine rationale Funktion ist in ihrem Definitionsbereich stetig.

\end{frame}

% page --------------------------------------------------------------------------------------------------
\begin{frame}{Partialbruchzerlegung}

Rationale Funktionen k\"onnen in {\bf Partialbr\"uche} zerlegt werden.
Ist 
{\footnotesize
\bq
 p(x) = p_n x^n + p_{n-1} x^{n-1} + ... + p_1 x + p_0,
 \nonumber \\
 q(x) = q_m x^m + q_{m-1} x^{m-1} + ... + q_1 x + q_0,
\eq
}

\vspace*{-4mm}
und ist ausserdem die Faktorisierung des Nennerpolynoms bekannt
{\footnotesize
\bq
 q(x) = c \prod\limits_{j=1}^r \left( x - x_j \right)^{\lambda_j},
\eq
}

\vspace*{-3mm}
wobei $\lambda_j$ die Multiziplit\"at der Nullstelle $x_j$ angibt, 
so l\"a{\ss}t sich die rationale Funktion
schreiben als
\bq
 R(x) & = & \frac{p(x)}{q(x)}
 \; = \;
 P(x) + \sum\limits_{j=1}^r \sum\limits_{k=1}^{\lambda_j} \frac{a_{jk}}{(x-x_j)^k},
\eq
wobei $P(x)$ ein Polynom vom Grad $\deg p(x) - \deg q(x)$ ist und $a_{jk}\in \mathbb R$.

\end{frame}

% page --------------------------------------------------------------------------------------------------
\begin{frame}{Partialbruchzerlegung}

Berechnung von $P(x)$ und der Konstanten $a_{jk}$:

\vspace*{5mm}
$P(x)$ bestimmt sich durch Polynomdivision mit Rest.

\vspace*{5mm}
Wir betrachten als Beispiel die rationale Funktion
\bq
\frac{x^4+3x^3-12x^2-3x+18}{(x-2)^2(x+2)}
\eq
F\"ur das Nennerpolynom haben wir
\bq
 (x-2)^2(x+2) & = & x^3 - 2 x^2 - 4 x + 8.
\eq

\end{frame}

% page --------------------------------------------------------------------------------------------------
\begin{frame}{Polynomdivision}

Polynomdivision mit Rest liefert
{\scriptsize
\bq
\begin{array}{rcrcl}
(x^4+3x^3-12x^2-3x+18) & : & (x^3 - 2 x^2 - 4 x + 8) & = & x + 5 + \frac{(2x^2+9x-22)}{(x^3-2x^2-4x+8)} \\
-(x^4-2x^3-4x^2+8x) & & & & \\
-------------& & & & \\
5x^3-8x^2-11x+18 & & & & \\
-(5x^3-10x^2-20x+40) & & & & \\
------------ & & & & \\
2x^2+9x-22 & & & & \\
\end{array}
\eq
}

Somit ist also $P(x)=x+5$. 


\end{frame}

% page --------------------------------------------------------------------------------------------------
\begin{frame}{Partialbruchzerlegung}

F\"ur den Rest verwendet man den Ansatz
\bq
\frac{2x^2+9x-22}{x^3-2x^2-4x+8}
& = & \frac{a_{12}}{(x-2)^2} + \frac{a_{11}}{x-2} + \frac{a_{21}}{x+2}.
\eq
Man bringt die rechte Seite auf den Hauptnenner
{\footnotesize
\bq
\frac{a_{12}}{(x-2)^2} + \frac{a_{11}}{x-2} + \frac{a_{21}}{x+2}
 & = &
 \frac{(a_{11}+a_{21})x^2 + (a_{12}-4a_{21})x+(2a_{12}-4a_{11}+4a_{21})}{x^3-2x^2-4x+8}
\eq
}

Koeffizientenvergleich liefert ein lineares Gleichungssystem:
\bq
 a_{11}+a_{21} & = & 2,
 \nonumber \\
 a_{12}-4a_{21} & = & 9,
 \nonumber \\
 2a_{12}-4a_{11}+4a_{21} & = & -22.
\eq

\end{frame}

% page --------------------------------------------------------------------------------------------------
\begin{frame}{Partialbruchzerlegung}

Durch L\"osen des linearen Gleichungssystems
\bq
 a_{11}+a_{21} & = & 2,
 \nonumber \\
 a_{12}-4a_{21} & = & 9,
 \nonumber \\
 2a_{12}-4a_{11}+4a_{21} & = & -22,
\eq
findet man
\bq
 a_{12} = 1, \;\;\; a_{11}=4, \;\;\; a_{21}=-2.
\eq
Somit erhalten wir das Ergebnis
\bq
\frac{x^4+3x^3-12x^2-3x+18}{(x-2)^2(x+2)}
 & = & 
 x + 5 + \frac{1}{(x-2)^2} + \frac{4}{x-2} - \frac{2}{x+2}.
\eq

\end{frame}

% page --------------------------------------------------------------------------------------------------
\begin{frame}{Trick}

Die Koeffizienten der Partialbr\"uche mit der \superalert{h\"ochsten Potenz einer Nullstelle} lassen
sich einfacher bestimmen, indem man im Ansatz mit $(x-x_j)^{\lambda_j}$ multipliziert und dann
$x=x_j$ setzt. 

\vspace*{4mm}
In unserem Beispiel lassen sich so $a_{12}$ und $a_{21}$ bestimmen:
{\scriptsize
\bq
 a_{12} & = & \left. \frac{2x^2+9x-22}{(x-2)^2(x+2)} (x-2)^2 \right|_{x=2}
 = \left. \frac{2x^2+9x-22}{x+2} \right|_{x=2}
 = \frac{8+18-22}{4} = 1,
 \nonumber \\
 a_{21} & = & \left. \frac{2x^2+9x-22}{(x-2)^2(x+2)} (x+2) \right|_{x=-2}
 = \left. \frac{2x^2+9x-22}{(x-2)^2} \right|_{x=-2}
 = \frac{8-18-22}{16} = -2.
\eq
}

\end{frame}

%%%%%%%%%%%%%%%%%%%%%%%%%%%%%%%%%%%%%%%%%%%%%%%%%%%%%%%%%%%%%%%%%%%%%%%%%%%%%%%%%%%%%%%%%%%%%%%%%%%%%%%%%
%%%%%%%%%%%%%%%%%%%%%%%%%%%%%%%%%%%%%%%%%%%%%%%%%%%%%%%%%%%%%%%%%%%%%%%%%%%%%%%%%%%%%%%%%%%%%%%%%%%%%%%%%
%%%%%%%%%%%%%%%%%%%%%%%%%%%%%%%%%%%%%%%%%%%%%%%%%%%%%%%%%%%%%%%%%%%%%%%%%%%%%%%%%%%%%%%%%%%%%%%%%%%%%%%%%

\section{Trigonometrische Funktionen}

\frame{\sectionpage}

% page --------------------------------------------------------------------------------------------------
\begin{frame}{Trigonometrische Funktionen}

Neben den Winkelfunktionen \alert{Sinus} und \alert{Kosinus}
\bq
 \cos x = \frac{1}{2} \left( e^{ix} + e^{-ix} \right),
 & &
 \sin x = \frac{1}{2i} \left( e^{ix} - e^{-ix} \right),
\eq
gibt es weitere trigonometrische Funktionen:
\bq
 \tan x & = & \frac{\sin x}{\cos x}, \;\;\;\alert{\mbox{Tangens}}
 \nonumber \\
 \cot x & = & \frac{\cos x}{\sin x}, \;\;\;\alert{\mbox{Kotangens}}
 \nonumber \\
 \sec x & = & \frac{1}{\cos x}, \;\;\;\alert{\mbox{Sekans}}
 \nonumber \\
 \csc x & = & \frac{1}{\sin x}, \;\;\;\alert{\mbox{Kosekans}}
\eq

\end{frame}

% page --------------------------------------------------------------------------------------------------
\begin{frame}{Umkehrfunktionenen}

Die Umkehrfunktionen werden mit $\arcsin$, $\arccos$, $\arctan$, etc. bezeichnet:
\bq
 \arcsin(x) & = & \sin^{-1}(x), \;\;\;\alert{\mbox{Arkussinus}}
 \nonumber \\
 \arccos(x) & = & \cos^{-1}(x), \;\;\;\alert{\mbox{Arkuskosinus}}
 \nonumber \\
 \arctan(x) & = & \tan^{-1}(x), \;\;\;\alert{\mbox{Arkustangens}}
\eq
Diese Umkehrfunktionen lassen sich durch den Logarithmus ausdr\"ucken:
\bq
 \arcsin(x) & = & \frac{1}{i} \ln\left( i x + \sqrt{1-x^2} \right),
 \nonumber \\
 \arccos(x) & = & \frac{1}{i} \ln\left( x + i \sqrt{1-x^2} \right),
 \nonumber \\
 \arctan(x) & = & \frac{1}{2i} \ln\left(\frac{1+ix}{1-ix}\right).
\eq

\end{frame}

% page --------------------------------------------------------------------------------------------------
\begin{frame}{Geometrie}

\begin{center}
\begin{tikzpicture}
\draw [->] (-2.0,0.0) -- (2.0,0.0);
\draw [->] (0.0,-2.0) -- (0.0,2.0);
\draw (0.0,0.0) circle (1.5);
\draw (0.0,0.0) -- (1.299,0.75);
\draw (0.3,0.0) arc [radius=0.3, start angle=0, end angle=30];
\node[left] at (0.7,0.1) {\tiny $\varphi$};
\uncover<1>{
\node[above] at (0.6,0.3) {\footnotesize $1$};
}
\uncover<2>{
\draw [red,thick] (1.299,0.0) -- (1.299,0.75);
\node[right] at (1.6,0.375) {\footnotesize $\sin(\varphi)$};
}
\uncover<3>{
\draw (1.299,0.0) -- (1.299,0.75);
\draw [red,thick] (0.0,0.0) -- (1.299,0.0);
\node[right] at (1.6,0.375) {\footnotesize $\cos(\varphi)$};
}
\uncover<4>{
\draw (0.0,0.0) -- (1.5,0.866);
\draw [red,thick] (1.5,0.0) -- (1.5,0.866);
\node[right] at (1.6,0.375) {\footnotesize $\tan(\varphi)$};
}
\uncover<5>{
\draw (0.0,0.0) -- (2.598,1.5);
\draw [red,thick] (0.0,1.5) -- (2.598,1.5);
\node[right] at (1.6,0.375) {\footnotesize $\cot(\varphi)$};
}
\uncover<6>{
\draw [red,thick] (0.0,0.0) -- (1.5,0.866);
\draw (1.5,0.0) -- (1.5,0.866);
\node[right] at (1.6,0.375) {\footnotesize $\sec(\varphi)$};
}
\uncover<7>{
\draw [red,thick] (0.0,0.0) -- (2.598,1.5);
\draw (0.0,1.5) -- (2.598,1.5);
\node[right] at (1.6,0.375) {\footnotesize $\csc(\varphi)$};
}
\end{tikzpicture}
\end{center}

\end{frame}

%%%%%%%%%%%%%%%%%%%%%%%%%%%%%%%%%%%%%%%%%%%%%%%%%%%%%%%%%%%%%%%%%%%%%%%%%%%%%%%%%%%%%%%%%%%%%%%%%%%%%%%%%
%%%%%%%%%%%%%%%%%%%%%%%%%%%%%%%%%%%%%%%%%%%%%%%%%%%%%%%%%%%%%%%%%%%%%%%%%%%%%%%%%%%%%%%%%%%%%%%%%%%%%%%%%
%%%%%%%%%%%%%%%%%%%%%%%%%%%%%%%%%%%%%%%%%%%%%%%%%%%%%%%%%%%%%%%%%%%%%%%%%%%%%%%%%%%%%%%%%%%%%%%%%%%%%%%%%

\section{Hyperbolische Funktionen}

\frame{\sectionpage}

% page --------------------------------------------------------------------------------------------------
\begin{frame}{Hyperbolische Funktionen}

Neben den bereits eingef\"uhrten hyperbolischen Funktionen
\bq
 \cosh x = \frac{1}{2} \left( e^{x} + e^{-x} \right),
 & &
 \sinh x = \frac{1}{2} \left( e^{x} - e^{-x} \right),
\eq
definiert man auch
\bq
 \tanh x & = & \frac{\sinh x}{\cosh x}.
\eq
Bemerkung: F\"ur $\sinh$ und $\cosh$ gilt
\bq
 \cosh^2 x - \sinh^2 x & = & 1.
\eq

\end{frame}

% page --------------------------------------------------------------------------------------------------
\begin{frame}{Umkehrfunktionenen}

Die inversen Funktionen werden als Areafunktionen bezeichnet:
\bq
 \mbox{arsinh}(x) & = & \sinh^{-1}(x), \;\;\;\alert{\mbox{Areasinus Hyperbolicus}}
 \nonumber \\
 \mbox{arcosh}(x) & = & \cosh^{-1}(x), \;\;\;\alert{\mbox{Areakosinus Hyperbolicus}}
 \nonumber \\
 \mbox{artanh}(x) & = & \tanh^{-1}(x), \;\;\;\alert{\mbox{Areatangens Hyperbolicus}}
\eq
Diese Umkehrfunktionen lassen sich ebenfalls durch den Logarithmus ausdr\"ucken:
\bq
 \mbox{arsinh}(x) & = & \ln\left( x + \sqrt{x^2+1} \right),
 \nonumber \\
 \mbox{arcosh}(x) & = & \ln\left( x + \sqrt{x^2-1} \right),
 \nonumber \\
 \mbox{artanh}(x) & = & \frac{1}{2} \ln\left(\frac{1+x}{1-x}\right).
\eq

\end{frame}

% page --------------------------------------------------------------------------------------------------
\begin{frame}{Zusammenhang zwischen trigonometrischen Funktionen und hyperbolischen Funktionen}

\bq
 \sin x & = & \frac{1}{i} \sinh\left(i x \right),
 \nonumber \\
 \cos x & = & \cosh\left( i x \right),
 \nonumber \\
 \tan x & = & \frac{1}{i} \tanh\left(i x \right),
 \nonumber \\
 \arcsin(x) & = & \frac{1}{i} \mbox{arsinh}\left(i x \right),
 \nonumber \\
 \arccos(x) & = & \frac{1}{i} \mbox{arcosh}\left(x \right),
 \nonumber \\
 \arctan(x) & = & \frac{1}{i} \mbox{artanh}\left(i x \right).
\eq

\end{frame}

% page --------------------------------------------------------------------------------------------------
\begin{frame}{Quiz}

Die Fl\"acheninhalt der schraffierten Fl\"ache ist
\begin{center}
\begin{tikzpicture}
\path [fill=lightgray] (0.0,0.0) -- (0.866,-0.5) to [curve through={(1.0,0.0)}] (0.866,0.5) -- (0.0,0.0);
\draw [->] (-1.5,0.0) -- (1.5,0.0);
\draw [->] (0.0,-1.5) -- (0.0,1.5);
\draw (0.0,0.0) circle (1.0);
\draw (0.0,0.0) -- (0.866,0.5);
\draw (0.0,0.0) -- (0.866,-0.5);
\draw (0.3,0.0) arc [radius=0.3, start angle=0, end angle=30];
\node[above] at (0.5,0.25) {\tiny $1$};
\node[left] at (0.7,0.1) {\tiny $\varphi$};
\end{tikzpicture}
\end{center}
\begin{columns}[b]
\begin{column}{5cm}
\begin{description}
\item{(A)} $\frac{1}{6}$
\item{(C)} $2 \varphi$
\end{description}
\end{column}
\begin{column}{5cm}
\begin{description}
\item{(B)} $\varphi$
\item{(D)} $\sin(\varphi)\cos(\varphi)$
\end{description}
\end{column}
\end{columns}

\end{frame}

% page --------------------------------------------------------------------------------------------------
\begin{frame}{Hyperbolische Geometrie}

\begin{center}
\begin{tikzpicture}
\node[below] at (-1.1,2.3) {\footnotesize $x^2-y^2=1$};
\draw [->] (-2.0,0.0) -- (2.0,0.0);
\draw [->] (0.0,-2.0) -- (0.0,2.0);
\draw (-2.236,-2.0) to [curve through={(-2.059,-1.8)(-1.803,-1.5)(-1.72,-1.4)(-1.64,-1.3)(-1.0,0.0)(-1.64,1.3)(-1.72,1.4)(-1.803,1.5)(-2.059,1.8)}] (-2.236,2.0);
\draw (2.236,-2.0) to [curve through={(2.059,-1.8)(1.803,-1.5)(1.72,-1.4)(1.64,-1.3)(1.0,0.0)(1.64,1.3)(1.72,1.4)(1.803,1.5)(2.059,1.8)}] (2.236,2.0);
\draw (0.0,0.0) -- (1.803,1.5);
\draw (0.0,0.0) -- (1.803,-1.5);
\path [fill=lightgray] (0.0,0.0) -- (1.803,-1.5) to [curve through={(1.72,-1.4)(1.64,-1.3)(1.0,0.0)(1.64,1.3)(1.72,1.4)}] (1.803,1.5) -- (0.0,0.0);
\node[above] at (0.7,0.1) {\tiny $\varphi$};
\uncover<1>{
}
\uncover<2>{
\draw [red,thick] (1.803,0.0) -- (1.803,1.5);
\node[right] at (1.9,0.75) {\footnotesize $\sinh(\varphi)$};
}
\uncover<3>{
\draw [red,thick] (0.0,1.5) -- (1.803,1.5);
\node[right] at (1.9,0.75) {\footnotesize $\cosh(\varphi)$};
}
\uncover<4>{
\draw [red,thick] (1.0,0.0) -- (1.0,0.832);
\node[right] at (1.9,0.75) {\footnotesize $\tanh(\varphi)$};
}
\end{tikzpicture}
\end{center}

\end{frame}

%%%%%%%%%%%%%%%%%%%%%%%%%%%%%%%%%%%%%%%%%%%%%%%%%%%%%%%%%%%%%%%%%%%%%%%%%%%%%%%%%%%%%%%%%%%%%%%%%%%%%%%%%
%%%%%%%%%%%%%%%%%%%%%%%%%%%%%%%%%%%%%%%%%%%%%%%%%%%%%%%%%%%%%%%%%%%%%%%%%%%%%%%%%%%%%%%%%%%%%%%%%%%%%%%%%
%%%%%%%%%%%%%%%%%%%%%%%%%%%%%%%%%%%%%%%%%%%%%%%%%%%%%%%%%%%%%%%%%%%%%%%%%%%%%%%%%%%%%%%%%%%%%%%%%%%%%%%%%

% page --------------------------------------------------------------------------------------------------
\begin{frame}

\end{frame}

\end{document}

