\documentclass[german]{beamer}

\mode<presentation>
{
 \usetheme{Madrid}

% \usecolortheme{crane}
 \usecolortheme{wolverine}
}

\usepackage{hyperref}

\usepackage[german]{babel}
\usepackage{times}
\usepackage[latin1,utf8]{inputenc}
\usepackage[OT2,T1]{fontenc}
\usepackage{shuffle}

\usepackage{tikz}
\usetikzlibrary{hobby}

% Stefan's abbreveations
\newcommand{\bq}{\begin{eqnarray*}}
\newcommand{\eq}{\end{eqnarray*}}
\newcommand{\eps}{\varepsilon}

\definecolor{MyYellowOrange}{cmyk}{0,0.5,1,0}
\newcommand{\superalert}[1]{{\color{MyYellowOrange}{#1}}}
\newtheorem*{myemptytheorem}{}

% dedicated environments
\newtheorem*{mytheorem27}{Hauptsatz der Differential- und Integralrechnung:} 


%%%%%%%%%%%%%%%%%%%%%%%%%%%%%%%%%%%%%%%%%%%%%%%%%%%%%%%%%%%%%%%%%%%%%%%%%%%%%%%%%%%%%%%%%%%%%%%%%%%%%%%%%
%%%%%%%%%%%%%%%%%%%%%%%%%%%%%%%%%%%%%%%%%%%%%%%%%%%%%%%%%%%%%%%%%%%%%%%%%%%%%%%%%%%%%%%%%%%%%%%%%%%%%%%%%
%%%%%%%%%%%%%%%%%%%%%%%%%%%%%%%%%%%%%%%%%%%%%%%%%%%%%%%%%%%%%%%%%%%%%%%%%%%%%%%%%%%%%%%%%%%%%%%%%%%%%%%%%

\title{Integralrechnung}

\subtitle{Mathematischer Br\"uckenkurs}

\author{Stefan Weinzierl}

\institute[Uni Mainz]{Institut f\"ur Physik, Universit\"at Mainz}%

\date[WiSe 2020/21]{Wintersemester 2020/21}

\begin{document}

%%%%%%%%%%%%%%%%%%%%%%%%%%%%%%%%%%%%%%%%%%%%%%%%%%%%%%%%%%%%%%%%%%%%%%%%%%%%%%%%%%%%%%%%%%%%%%%%%%%%%%%%%
%%%%%%%%%%%%%%%%%%%%%%%%%%%%%%%%%%%%%%%%%%%%%%%%%%%%%%%%%%%%%%%%%%%%%%%%%%%%%%%%%%%%%%%%%%%%%%%%%%%%%%%%%
%%%%%%%%%%%%%%%%%%%%%%%%%%%%%%%%%%%%%%%%%%%%%%%%%%%%%%%%%%%%%%%%%%%%%%%%%%%%%%%%%%%%%%%%%%%%%%%%%%%%%%%%%

\begin{frame}
  \titlepage
\end{frame}

%%%%%%%%%%%%%%%%%%%%%%%%%%%%%%%%%%%%%%%%%%%%%%%%%%%%%%%%%%%%%%%%%%%%%%%%%%%%%%%%%%%%%%%%%%%%%%%%%%%%%%%%%
%%%%%%%%%%%%%%%%%%%%%%%%%%%%%%%%%%%%%%%%%%%%%%%%%%%%%%%%%%%%%%%%%%%%%%%%%%%%%%%%%%%%%%%%%%%%%%%%%%%%%%%%%
%%%%%%%%%%%%%%%%%%%%%%%%%%%%%%%%%%%%%%%%%%%%%%%%%%%%%%%%%%%%%%%%%%%%%%%%%%%%%%%%%%%%%%%%%%%%%%%%%%%%%%%%%

\section{Integrale}

\frame{\sectionpage}

% page --------------------------------------------------------------------------------------------------
\begin{frame}{Treppenfunktionen}

\begin{definition}
Man nennt $t : [a,b] \rightarrow \mathbb R$ {\bf Treppenfunktion},
falls es eine Unterteilung
\bq
 a = x_0 < x_1 < ... < x_n = b
\eq
gibt, so da{\ss} $t$ auf jedem offenen Intervall $]x_{j-1},x_j[$ konstant ist.
Der Wert auf diesem Intervall sei mit $c_j$ bezeichnet.
\end{definition}

\begin{definition}
Das {\bf Integral einer Treppenfunktion} wird definiert als
\bq
 \int\limits_a^b t(x) \; dx 
 & = &
 \sum\limits_{j=1}^n c_j \left( x_j - x_{j-1} \right).
\eq
\end{definition}

\end{frame}

% page --------------------------------------------------------------------------------------------------
\begin{frame}{Treppenfunktionen}

\begin{center}
\begin{tikzpicture}
\draw [->] (-0.5,0.0) -- (5.0,0.0);
\draw [->] (0.0,-0.5) -- (0.0,4.0);
\draw (0.5,-0.2) -- (0.5,0.2);
\draw (4.5,-0.2) -- (4.5,0.2);
\node[below] at (0.5,-0.2) {\footnotesize $a$};
\node[below] at (4.5,-0.2) {\footnotesize $b$};
\node[right] at (5.0,0.0) {\footnotesize $x$};
\node[above] at (0.0,4.0) {\footnotesize $y$};
\uncover<1>{
\draw (0.5,1.0) -- (1.5,1.0);
\draw (1.5,2.0) -- (2.5,2.0);
\draw (2.5,3.0) -- (3.5,3.0);
\draw (3.5,2.5) -- (4.5,2.5);
\draw [fill] (0.5,1.0) circle (0.03);
\draw [fill] (1.5,2.0) circle (0.03);
\draw [fill] (2.5,3.0) circle (0.03);
\draw [fill] (3.5,2.5) circle (0.03);
}
\uncover<2>{
\path [fill=red] (0.5,0.0) -- (0.5,1.0) -- (1.5,1.0) -- (1.5,2.0) -- (2.5,2.0) -- (2.5,3.0) -- (3.5,3.0) -- (3.5,2.5) -- (4.5,2.5) -- (4.5,0.0) -- (0.5,0.0);
\draw (0.5,1.0) -- (1.5,1.0);
\draw (1.5,2.0) -- (2.5,2.0);
\draw (2.5,3.0) -- (3.5,3.0);
\draw (3.5,2.5) -- (4.5,2.5);
\draw [fill] (0.5,1.0) circle (0.03);
\draw [fill] (1.5,2.0) circle (0.03);
\draw [fill] (2.5,3.0) circle (0.03);
\draw [fill] (3.5,2.5) circle (0.03);
\draw [dashed] (0.5,0.0) -- (0.5,1.0);
\draw [dashed] (1.5,0.0) -- (1.5,2.0);
\draw [dashed] (2.5,0.0) -- (2.5,3.0);
\draw [dashed] (3.5,0.0) -- (3.5,3.0);
\draw [dashed] (4.5,0.0) -- (4.5,2.5);
}
\end{tikzpicture}
\end{center}

\end{frame}

% page --------------------------------------------------------------------------------------------------
\begin{frame}{Treppenfunktionen}

\begin{itemize}
\item Die Menge aller Treppenfunktionen auf dem Intervall $[a,b]$ bilden einen Vektorraum.

\item Wir bezeichnen diesen Vektorraum mit $T[a,b]$.

\item Sei $f : [a,b] \rightarrow \mathbb R$ eine beliebige beschr\"ankte Funktion
und $t\in T[a,b]$. 

Man schreibt $f\ge t$ falls $f(x) \ge t(x)$ f\"ur alle $x\in[a,b]$ gilt.
\end{itemize}

\end{frame}

% page --------------------------------------------------------------------------------------------------
\begin{frame}{Ober- und Unterintegrale}

\begin{definition}
Wir definieren nun das Ober- und Unterintegral f\"ur $f$:
\bq
 {\int\limits_a^b}^\ast f(x) \; dx 
 & = &
 \mbox{inf} \left\{ \int\limits_a^b t(x) \; dx; \; t \in T[a,b], \; t \ge f \right\},
 \nonumber \\
 {\int\limits_a^b}_\ast f(x) \; dx 
 & = &
 \mbox{sup} \left\{ \int\limits_a^b t(x) \; dx; \; t \in T[a,b], \; t \le f \right\}.
\eq
\end{definition}

\end{frame}

% page --------------------------------------------------------------------------------------------------
\begin{frame}{Das Riemann-Integral}

\begin{definition}
Eine beschr\"ankte Funktion $f : [a,b] \rightarrow \mathbb R$
hei{\ss}t {\bf Riemann-integrierbar}, falls
\bq
 {\int\limits_a^b}^\ast f(x) \; dx 
 & = &
 {\int\limits_a^b}_\ast f(x) \; dx.
\eq
In diesem Fall setzt man
\bq
 {\int\limits_a^b} f(x) \; dx 
 & = & 
 {\int\limits_a^b}^\ast f(x) \; dx.
\eq
\end{definition}

\end{frame}

% page --------------------------------------------------------------------------------------------------
\begin{frame}{Approximation durch Treppenfunktionen}

\begin{center}
\begin{tikzpicture}
\draw [->] (-0.5,0.0) -- (5.0,0.0);
\draw [->] (0.0,-0.5) -- (0.0,4.0);
\draw (0.5,-0.2) -- (0.5,0.2);
\draw (4.5,-0.2) -- (4.5,0.2);
\node[below] at (0.5,-0.2) {\footnotesize $a$};
\node[below] at (4.5,-0.2) {\footnotesize $b$};
\node[right] at (5.0,0.0) {\footnotesize $x$};
\node[above] at (0.0,4.0) {\footnotesize $y$};
\uncover<1>{
\draw (0.5,1.0) to [curve through={(1.5,2.0)(2.5,3.0)(3.5,2.0)}] (4.5,2.5);
}
\uncover<2>{
\path [fill=yellow] (0.5,0.0) -- (0.5,2.0) -- (1.5,2.0) -- (1.5,3.0) -- (3.5,3.0) -- (3.5,2.5) -- (4.5,2.5) -- (4.5,0.0) -- (0.5,0.0);
%
\draw [dashed] (0.5,2.0) -- (1.5,2.0);
\draw [dashed] (1.5,3.0) -- (2.5,3.0);
\draw [dashed] (2.5,3.0) -- (3.5,3.0);
\draw [dashed] (3.5,2.5) -- (4.5,2.5);
%
\draw [dashed] (0.5,0.0) -- (0.5,2.0);
\draw [dashed] (1.5,0.0) -- (1.5,3.0);
\draw [dashed] (2.5,0.0) -- (2.5,3.0);
\draw [dashed] (3.5,0.0) -- (3.5,3.0);
\draw [dashed] (4.5,0.0) -- (4.5,2.5);
%
\draw (0.5,1.0) to [curve through={(1.5,2.0)(2.5,3.0)(3.5,2.0)}] (4.5,2.5);
}
\uncover<3>{
\path [fill=orange] (0.5,0.0) 
              -- (0.5,1.38) -- (1.0,1.38) 
              -- (1.0,2.0) -- (1.5,2.0) 
              -- (1.5,2.85) -- (2.0,2.85) 
              -- (2.0,3.0) -- (2.5,3.0) 
              -- (2.5,3.0) -- (3.0,3.0) 
              -- (3.0,2.6) -- (3.5,2.6) 
              -- (3.5,2.0) -- (4.0,2.0) 
              -- (4.0,2.5) -- (4.5,2.5) 
 -- (4.5,0.0) -- (0.5,0.0);
%
\draw [dashed] (0.5,1.38) -- (1.0,1.38);
\draw [dashed] (1.0,2.0) -- (1.5,2.0);
\draw [dashed] (1.5,2.85) -- (2.0,2.85);
\draw [dashed] (2.0,3.0) -- (2.5,3.0);
\draw [dashed] (2.5,3.0) -- (3.0,3.0);
\draw [dashed] (3.0,2.6) -- (3.5,2.6);
\draw [dashed] (3.5,2.0) -- (4.0,2.0);
\draw [dashed] (4.0,2.5) -- (4.5,2.5);
%
\draw [dashed] (0.5,0.0) -- (0.5,1.38);
\draw [dashed] (1.0,0.0) -- (1.0,2.0);
\draw [dashed] (1.5,0.0) -- (1.5,2.85);
\draw [dashed] (2.0,0.0) -- (2.0,3.0);
\draw [dashed] (2.5,0.0) -- (2.5,3.0);
\draw [dashed] (3.0,0.0) -- (3.0,3.0);
\draw [dashed] (3.5,0.0) -- (3.5,2.6);
\draw [dashed] (4.0,0.0) -- (4.0,2.5);
\draw [dashed] (4.5,0.0) -- (4.5,2.5);
%
\draw (0.5,1.0) to [curve through={(1.5,2.0)(2.5,3.0)(3.5,2.0)}] (4.5,2.5);
}
\uncover<4>{
\path [fill=red] (0.5,0.0) 
              -- (0.5,1.17) -- (0.75,1.17) 
              -- (0.75,1.38) -- (1.0,1.38) 
              -- (1.0,1.64) -- (1.25,1.64) 
              -- (1.25,2.0) -- (1.5,2.0) 
              -- (1.5,2.5) -- (1.75,2.5) 
              -- (1.75,2.85) -- (2.0,2.85) 
              -- (2.0,3.0) -- (2.25,3.0) 
              -- (2.25,3.0) -- (2.5,3.0) 
              -- (2.5,3.0) -- (2.75,3.0) 
              -- (2.75,2.9) -- (3.0,2.9) 
              -- (3.0,2.6) -- (3.25,2.6) 
              -- (3.25,2.25) -- (3.5,2.25) 
              -- (3.5,2.0) -- (3.75,2.0) 
              -- (3.75,1.95) -- (4.0,1.95) 
              -- (4.0,2.1) -- (4.25,2.1) 
              -- (4.25,2.5) -- (4.5,2.5) 
 -- (4.5,0.0) -- (0.5,0.0);
%
\draw [dashed] (0.5,1.17) -- (0.75,1.17);
\draw [dashed] (0.75,1.38) -- (1.0,1.38);
\draw [dashed] (1.0,1.64) -- (1.25,1.64);
\draw [dashed] (1.25,2.0) -- (1.5,2.0);
\draw [dashed] (1.5,2.5) -- (1.75,2.5);
\draw [dashed] (1.75,2.85) -- (2.0,2.85);
\draw [dashed] (2.0,3.0) -- (2.25,3.0);
\draw [dashed] (2.25,3.0) -- (2.5,3.0);
\draw [dashed] (2.5,3.0) -- (2.75,3.0);
\draw [dashed] (2.75,2.9) -- (3.0,2.9);
\draw [dashed] (3.0,2.6) -- (3.25,2.6);
\draw [dashed] (3.25,2.25) -- (3.5,2.25);
\draw [dashed] (3.5,2.0) -- (3.75,2.0);
\draw [dashed] (3.75,1.95) -- (4.0,1.95);
\draw [dashed] (4.0,2.1) -- (4.25,2.1);
\draw [dashed] (4.25,2.5) -- (4.5,2.5);
%
\draw [dashed] (0.5,0.0) -- (0.5,1.17);
\draw [dashed] (0.75,0.0) -- (0.75,1.38);
\draw [dashed] (1.0,0.0) -- (1.0,1.64);
\draw [dashed] (1.25,0.0) -- (1.25,2.0);
\draw [dashed] (1.5,0.0) -- (1.5,2.5);
\draw [dashed] (1.75,0.0) -- (1.75,2.85);
\draw [dashed] (2.0,0.0) -- (2.0,3.0);
\draw [dashed] (2.25,0.0) -- (2.25,3.0);
\draw [dashed] (2.5,0.0) -- (2.5,3.0);
\draw [dashed] (2.75,0.0) -- (2.75,3.0);
\draw [dashed] (3.0,0.0) -- (3.0,2.9);
\draw [dashed] (3.25,0.0) -- (3.25,2.6);
\draw [dashed] (3.5,0.0) -- (3.5,2.25);
\draw [dashed] (3.75,0.0) -- (3.75,2.0);
\draw [dashed] (4.0,0.0) -- (4.0,2.1);
\draw [dashed] (4.25,0.0) -- (4.25,2.5);
\draw [dashed] (4.5,0.0) -- (4.5,2.5);
%
\draw (0.5,1.0) to [curve through={(1.5,2.0)(2.5,3.0)(3.5,2.0)}] (4.5,2.5);
}
\uncover<5>{
\path [fill=red] (0.5,0.0) 
              -- (0.5,1.17) -- (0.75,1.17) 
              -- (0.75,1.38) -- (1.0,1.38) 
              -- (1.0,1.64) -- (1.25,1.64) 
              -- (1.25,2.0) -- (1.5,2.0) 
              -- (1.5,2.5) -- (1.75,2.5) 
              -- (1.75,2.85) -- (2.0,2.85) 
              -- (2.0,3.0) -- (2.25,3.0) 
              -- (2.25,3.0) -- (2.5,3.0) 
              -- (2.5,3.0) -- (2.75,3.0) 
              -- (2.75,2.9) -- (3.0,2.9) 
              -- (3.0,2.6) -- (3.25,2.6) 
              -- (3.25,2.25) -- (3.5,2.25) 
              -- (3.5,2.0) -- (3.75,2.0) 
              -- (3.75,1.95) -- (4.0,1.95) 
              -- (4.0,2.1) -- (4.25,2.1) 
              -- (4.25,2.5) -- (4.5,2.5) 
 -- (4.5,0.0) -- (0.5,0.0);
%
\draw [dashed] (0.5,1.17) -- (0.75,1.17);
\draw [dashed] (0.75,1.38) -- (1.0,1.38);
\draw [dashed] (1.0,1.64) -- (1.25,1.64);
\draw [dashed] (1.25,2.0) -- (1.5,2.0);
\draw [dashed] (1.5,2.5) -- (1.75,2.5);
\draw [dashed] (1.75,2.85) -- (2.0,2.85);
\draw [dashed] (2.0,3.0) -- (2.25,3.0);
\draw [dashed] (2.25,3.0) -- (2.5,3.0);
\draw [dashed] (2.5,3.0) -- (2.75,3.0);
\draw [dashed] (2.75,2.9) -- (3.0,2.9);
\draw [dashed] (3.0,2.6) -- (3.25,2.6);
\draw [dashed] (3.25,2.25) -- (3.5,2.25);
\draw [dashed] (3.5,2.0) -- (3.75,2.0);
\draw [dashed] (3.75,1.95) -- (4.0,1.95);
\draw [dashed] (4.0,2.1) -- (4.25,2.1);
\draw [dashed] (4.25,2.5) -- (4.5,2.5);
%
\draw [dashed] (0.5,0.0) -- (0.5,1.17);
\draw [dashed] (0.75,0.0) -- (0.75,1.38);
\draw [dashed] (1.0,0.0) -- (1.0,1.64);
\draw [dashed] (1.25,0.0) -- (1.25,2.0);
\draw [dashed] (1.5,0.0) -- (1.5,2.5);
\draw [dashed] (1.75,0.0) -- (1.75,2.85);
\draw [dashed] (2.0,0.0) -- (2.0,3.0);
\draw [dashed] (2.25,0.0) -- (2.25,3.0);
\draw [dashed] (2.5,0.0) -- (2.5,3.0);
\draw [dashed] (2.75,0.0) -- (2.75,3.0);
\draw [dashed] (3.0,0.0) -- (3.0,2.9);
\draw [dashed] (3.25,0.0) -- (3.25,2.6);
\draw [dashed] (3.5,0.0) -- (3.5,2.25);
\draw [dashed] (3.75,0.0) -- (3.75,2.0);
\draw [dashed] (4.0,0.0) -- (4.0,2.1);
\draw [dashed] (4.25,0.0) -- (4.25,2.5);
\draw [dashed] (4.5,0.0) -- (4.5,2.5);
%
\path [fill=lime] (0.5,0.0) -- (0.5,1.0) -- (1.5,1.0) -- (1.5,2.0) -- (3.5,2.0) -- (3.5,1.91) -- (4.5,1.91) -- (4.5,0.0) -- (0.5,0.0);
%
\draw [dashed] (0.5,1.0) -- (1.5,1.0);
\draw [dashed] (1.5,2.0) -- (2.5,2.0);
\draw [dashed] (2.5,2.0) -- (3.5,2.0);
\draw [dashed] (3.5,1.91) -- (4.5,1.91);
%
\draw [dashed] (0.5,0.0) -- (0.5,1.0);
\draw [dashed] (1.5,0.0) -- (1.5,2.0);
\draw [dashed] (2.5,0.0) -- (2.5,2.0);
\draw [dashed] (3.5,0.0) -- (3.5,2.0);
\draw [dashed] (4.5,0.0) -- (4.5,1.91);
%
\draw (0.5,1.0) to [curve through={(1.5,2.0)(2.5,3.0)(3.5,2.0)}] (4.5,2.5);
}
\uncover<6>{
\path [fill=red] (0.5,0.0) 
              -- (0.5,1.17) -- (0.75,1.17) 
              -- (0.75,1.38) -- (1.0,1.38) 
              -- (1.0,1.64) -- (1.25,1.64) 
              -- (1.25,2.0) -- (1.5,2.0) 
              -- (1.5,2.5) -- (1.75,2.5) 
              -- (1.75,2.85) -- (2.0,2.85) 
              -- (2.0,3.0) -- (2.25,3.0) 
              -- (2.25,3.0) -- (2.5,3.0) 
              -- (2.5,3.0) -- (2.75,3.0) 
              -- (2.75,2.9) -- (3.0,2.9) 
              -- (3.0,2.6) -- (3.25,2.6) 
              -- (3.25,2.25) -- (3.5,2.25) 
              -- (3.5,2.0) -- (3.75,2.0) 
              -- (3.75,1.95) -- (4.0,1.95) 
              -- (4.0,2.1) -- (4.25,2.1) 
              -- (4.25,2.5) -- (4.5,2.5) 
 -- (4.5,0.0) -- (0.5,0.0);
%
\draw [dashed] (0.5,1.17) -- (0.75,1.17);
\draw [dashed] (0.75,1.38) -- (1.0,1.38);
\draw [dashed] (1.0,1.64) -- (1.25,1.64);
\draw [dashed] (1.25,2.0) -- (1.5,2.0);
\draw [dashed] (1.5,2.5) -- (1.75,2.5);
\draw [dashed] (1.75,2.85) -- (2.0,2.85);
\draw [dashed] (2.0,3.0) -- (2.25,3.0);
\draw [dashed] (2.25,3.0) -- (2.5,3.0);
\draw [dashed] (2.5,3.0) -- (2.75,3.0);
\draw [dashed] (2.75,2.9) -- (3.0,2.9);
\draw [dashed] (3.0,2.6) -- (3.25,2.6);
\draw [dashed] (3.25,2.25) -- (3.5,2.25);
\draw [dashed] (3.5,2.0) -- (3.75,2.0);
\draw [dashed] (3.75,1.95) -- (4.0,1.95);
\draw [dashed] (4.0,2.1) -- (4.25,2.1);
\draw [dashed] (4.25,2.5) -- (4.5,2.5);
%
\draw [dashed] (0.5,0.0) -- (0.5,1.17);
\draw [dashed] (0.75,0.0) -- (0.75,1.38);
\draw [dashed] (1.0,0.0) -- (1.0,1.64);
\draw [dashed] (1.25,0.0) -- (1.25,2.0);
\draw [dashed] (1.5,0.0) -- (1.5,2.5);
\draw [dashed] (1.75,0.0) -- (1.75,2.85);
\draw [dashed] (2.0,0.0) -- (2.0,3.0);
\draw [dashed] (2.25,0.0) -- (2.25,3.0);
\draw [dashed] (2.5,0.0) -- (2.5,3.0);
\draw [dashed] (2.75,0.0) -- (2.75,3.0);
\draw [dashed] (3.0,0.0) -- (3.0,2.9);
\draw [dashed] (3.25,0.0) -- (3.25,2.6);
\draw [dashed] (3.5,0.0) -- (3.5,2.25);
\draw [dashed] (3.75,0.0) -- (3.75,2.0);
\draw [dashed] (4.0,0.0) -- (4.0,2.1);
\draw [dashed] (4.25,0.0) -- (4.25,2.5);
\draw [dashed] (4.5,0.0) -- (4.5,2.5);
%
\path [fill=green] (0.5,0.0) 
              -- (0.5,1.0) -- (1.0,1.0) 
              -- (1.0,1.32) -- (1.5,1.32) 
              -- (1.5,2.0) -- (2.0,2.0) 
              -- (2.0,2.81) -- (2.5,2.81) 
              -- (2.5,2.63) -- (3.0,2.63) 
              -- (3.0,2.0) -- (3.5,2.0) 
              -- (3.5,1.91) -- (4.0,1.91) 
              -- (4.0,1.94) -- (4.5,1.94) 
 -- (4.5,0.0) -- (0.5,0.0);
%
\draw [dashed] (0.5,1.0) -- (1.0,1.0);
\draw [dashed] (1.0,1.32) -- (1.5,1.32);
\draw [dashed] (1.5,2.0) -- (2.0,2.0);
\draw [dashed] (2.0,2.81) -- (2.5,2.81);
\draw [dashed] (2.5,2.63) -- (3.0,2.63);
\draw [dashed] (3.0,2.0) -- (3.5,2.0);
\draw [dashed] (3.5,1.91) -- (4.0,1.91);
\draw [dashed] (4.0,1.94) -- (4.5,1.94);
%
\draw [dashed] (0.5,0.0) -- (0.5,1.0);
\draw [dashed] (1.0,0.0) -- (1.0,1.32);
\draw [dashed] (1.5,0.0) -- (1.5,2.0);
\draw [dashed] (2.0,0.0) -- (2.0,2.81);
\draw [dashed] (2.5,0.0) -- (2.5,2.81);
\draw [dashed] (3.0,0.0) -- (3.0,2.63);
\draw [dashed] (3.5,0.0) -- (3.5,2.0);
\draw [dashed] (4.0,0.0) -- (4.0,1.94);
\draw [dashed] (4.5,0.0) -- (4.5,1.94);
%
\draw (0.5,1.0) to [curve through={(1.5,2.0)(2.5,3.0)(3.5,2.0)}] (4.5,2.5);
}
\uncover<7>{
\path [fill=red] (0.5,0.0) 
              -- (0.5,1.17) -- (0.75,1.17) 
              -- (0.75,1.38) -- (1.0,1.38) 
              -- (1.0,1.64) -- (1.25,1.64) 
              -- (1.25,2.0) -- (1.5,2.0) 
              -- (1.5,2.5) -- (1.75,2.5) 
              -- (1.75,2.85) -- (2.0,2.85) 
              -- (2.0,3.0) -- (2.25,3.0) 
              -- (2.25,3.0) -- (2.5,3.0) 
              -- (2.5,3.0) -- (2.75,3.0) 
              -- (2.75,2.9) -- (3.0,2.9) 
              -- (3.0,2.6) -- (3.25,2.6) 
              -- (3.25,2.25) -- (3.5,2.25) 
              -- (3.5,2.0) -- (3.75,2.0) 
              -- (3.75,1.95) -- (4.0,1.95) 
              -- (4.0,2.1) -- (4.25,2.1) 
              -- (4.25,2.5) -- (4.5,2.5) 
 -- (4.5,0.0) -- (0.5,0.0);
%
\draw [dashed] (0.5,1.17) -- (0.75,1.17);
\draw [dashed] (0.75,1.38) -- (1.0,1.38);
\draw [dashed] (1.0,1.64) -- (1.25,1.64);
\draw [dashed] (1.25,2.0) -- (1.5,2.0);
\draw [dashed] (1.5,2.5) -- (1.75,2.5);
\draw [dashed] (1.75,2.85) -- (2.0,2.85);
\draw [dashed] (2.0,3.0) -- (2.25,3.0);
\draw [dashed] (2.25,3.0) -- (2.5,3.0);
\draw [dashed] (2.5,3.0) -- (2.75,3.0);
\draw [dashed] (2.75,2.9) -- (3.0,2.9);
\draw [dashed] (3.0,2.6) -- (3.25,2.6);
\draw [dashed] (3.25,2.25) -- (3.5,2.25);
\draw [dashed] (3.5,2.0) -- (3.75,2.0);
\draw [dashed] (3.75,1.95) -- (4.0,1.95);
\draw [dashed] (4.0,2.1) -- (4.25,2.1);
\draw [dashed] (4.25,2.5) -- (4.5,2.5);
%
\draw [dashed] (0.5,0.0) -- (0.5,1.17);
\draw [dashed] (0.75,0.0) -- (0.75,1.38);
\draw [dashed] (1.0,0.0) -- (1.0,1.64);
\draw [dashed] (1.25,0.0) -- (1.25,2.0);
\draw [dashed] (1.5,0.0) -- (1.5,2.5);
\draw [dashed] (1.75,0.0) -- (1.75,2.85);
\draw [dashed] (2.0,0.0) -- (2.0,3.0);
\draw [dashed] (2.25,0.0) -- (2.25,3.0);
\draw [dashed] (2.5,0.0) -- (2.5,3.0);
\draw [dashed] (2.75,0.0) -- (2.75,3.0);
\draw [dashed] (3.0,0.0) -- (3.0,2.9);
\draw [dashed] (3.25,0.0) -- (3.25,2.6);
\draw [dashed] (3.5,0.0) -- (3.5,2.25);
\draw [dashed] (3.75,0.0) -- (3.75,2.0);
\draw [dashed] (4.0,0.0) -- (4.0,2.1);
\draw [dashed] (4.25,0.0) -- (4.25,2.5);
\draw [dashed] (4.5,0.0) -- (4.5,2.5);
%
\path [fill=cyan] (0.5,0.0) 
              -- (0.5,1.0) -- (0.75,1.0) 
              -- (0.75,1.15) -- (1.0,1.15) 
              -- (1.0,1.32) -- (1.25,1.32) 
              -- (1.25,1.65) -- (1.5,1.65) 
              -- (1.5,2.0) -- (1.75,2.0) 
              -- (1.75,2.55) -- (2.0,2.53) 
              -- (2.0,2.81) -- (2.25,2.81) 
              -- (2.25,2.97) -- (2.5,2.97) 
              -- (2.5,2.9) -- (2.75,2.9) 
              -- (2.75,2.62) -- (3.0,2.62) 
              -- (3.0,2.25) -- (3.25,2.25) 
              -- (3.25,2.0) -- (3.5,2.0) 
              -- (3.5,1.91) -- (3.75,1.91) 
              -- (3.75,1.9) -- (4.0,1.9) 
              -- (4.0,1.93) -- (4.25,1.93) 
              -- (4.25,2.05) -- (4.5,2.05) 
 -- (4.5,0.0) -- (0.5,0.0);
%
\draw [dashed] (0.5,1.0) -- (0.75,1.0);
\draw [dashed] (0.75,1.15) -- (1.0,1.15);
\draw [dashed] (1.0,1.32) -- (1.25,1.32);
\draw [dashed] (1.25,1.65) -- (1.5,1.65);
\draw [dashed] (1.5,2.0) -- (1.75,2.0);
\draw [dashed] (1.75,2.53) -- (2.0,2.53);
\draw [dashed] (2.0,2.81) -- (2.25,2.81);
\draw [dashed] (2.25,2.97) -- (2.5,2.97);
\draw [dashed] (2.5,2.9) -- (2.75,2.9);
\draw [dashed] (2.75,2.62) -- (3.0,2.62);
\draw [dashed] (3.0,2.25) -- (3.25,2.25);
\draw [dashed] (3.25,2.0) -- (3.5,2.0);
\draw [dashed] (3.5,1.91) -- (3.75,1.91);
\draw [dashed] (3.75,1.9) -- (4.0,1.9);
\draw [dashed] (4.0,1.93) -- (4.25,1.93);
\draw [dashed] (4.25,2.05) -- (4.5,2.05);
%
\draw [dashed] (0.5,0.0) -- (0.5,1.0);
\draw [dashed] (0.75,0.0) -- (0.75,1.15);
\draw [dashed] (1.0,0.0) -- (1.0,1.32);
\draw [dashed] (1.25,0.0) -- (1.25,1.65);
\draw [dashed] (1.5,0.0) -- (1.5,2.0);
\draw [dashed] (1.75,0.0) -- (1.75,2.53);
\draw [dashed] (2.0,0.0) -- (2.0,2.81);
\draw [dashed] (2.25,0.0) -- (2.25,2.97);
\draw [dashed] (2.5,0.0) -- (2.5,2.97);
\draw [dashed] (2.75,0.0) -- (2.75,2.9);
\draw [dashed] (3.0,0.0) -- (3.0,2.62);
\draw [dashed] (3.25,0.0) -- (3.25,2.25);
\draw [dashed] (3.5,0.0) -- (3.5,2.0);
\draw [dashed] (3.75,0.0) -- (3.75,1.91);
\draw [dashed] (4.0,0.0) -- (4.0,1.93);
\draw [dashed] (4.25,0.0) -- (4.25,2.05);
\draw [dashed] (4.5,0.0) -- (4.5,2.05);
%
\draw (0.5,1.0) to [curve through={(1.5,2.0)(2.5,3.0)(3.5,2.0)}] (4.5,2.5);
}
\end{tikzpicture}
\end{center}

\end{frame}

% page --------------------------------------------------------------------------------------------------
\begin{frame}{Geometrische Interpretation des Integrals}

\begin{center}
\begin{tikzpicture}
\draw [->] (-0.5,0.0) -- (5.0,0.0);
\draw [->] (0.0,-0.5) -- (0.0,4.0);
\draw (0.5,-0.2) -- (0.5,0.2);
\draw (4.5,-0.2) -- (4.5,0.2);
\node[below] at (0.5,-0.2) {\footnotesize $a$};
\node[below] at (4.5,-0.2) {\footnotesize $b$};
\node[right] at (5.0,0.0) {\footnotesize $x$};
\node[above] at (0.0,4.0) {\footnotesize $y$};
\uncover<1>{
\draw (0.5,1.0) to [curve through={(1.5,2.0)(2.5,3.0)(3.5,2.0)}] (4.5,2.5);
}
\uncover<2>{
\path [fill=red] (0.5,0.0) -- (0.5,1.0) to [curve through={(1.5,2.0)(2.5,3.0)(3.5,2.0)}] (4.5,2.5) -- (4.5,0.0) -- (0.5,0.0);
\draw (0.5,1.0) to [curve through={(1.5,2.0)(2.5,3.0)(3.5,2.0)}] (4.5,2.5);
}
\end{tikzpicture}
\end{center}

\end{frame}

% page --------------------------------------------------------------------------------------------------
\begin{frame}{Das Riemann-Integral}

\begin{example}
\bq
 \int\limits_0^1 f\left(x\right) dx,
 & &
 f\left(x\right)
 = 
 \left\{ \begin{array}{ll}
  1 & x \in \mathbb{Q} \\
  0 & x \notin \mathbb{Q} \\
 \end{array} \right.
\eq
Diese Funktion ist nicht Riemann-integrierbar: 
Alle Obersummen sind stets $1$, alle Untersummen sind stets $0$.
\end{example}

\vspace*{15mm}
(Diese Funktion ist Lebesgue-integrierbar.)

\end{frame}

% page --------------------------------------------------------------------------------------------------
\begin{frame}{S\"atze \"uber integrierbare Funktionen}

\begin{theorem}
Jede stetige Funktion $f : [a,b] \rightarrow \mathbb R$ ist integrierbar.
\end{theorem}

\begin{theorem}
Jede monotone Funktion $f : [a,b] \rightarrow \mathbb R$ ist integrierbar.
\end{theorem}

\begin{theorem}
Seien $f, g : [a,b] \rightarrow \mathbb R$ integrierbare Funktionen und $\lambda \in \mathbb R$.
Dann sind auch die Funktionen $f+g$ und $\lambda \cdot f$ integrierbar und es gilt
{\footnotesize
\bq
 \int\limits_a^b (f+g)(x) \; dx & = & \int\limits_a^b f(x) \; dx + \int\limits_a^b g(x) \; dx,
 \nonumber \\
 \int\limits_a^b (\lambda f)(x) \; dx & = & \lambda \int\limits_a^b f(x) \; dx.
\eq
}
\end{theorem}

\end{frame}

% page --------------------------------------------------------------------------------------------------
\begin{frame}{S\"atze \"uber integrierbare Funktionen}

\begin{theorem}
Seien $f, g : [a,b] \rightarrow \mathbb R$ integrierbare Funktionen.
Dann ist auch die Funktion $f \cdot g$ integrierbar.
\end{theorem}

Im Allgemeinen ist allerdings
\bq
 \int\limits_a^b (f \cdot g)(x) \; dx & \neq & \left( \int\limits_a^b f(x) \; dx \right)
 \cdot \left(  \int\limits_a^b g(x) \; dx \right).
\eq

\end{frame}

% page --------------------------------------------------------------------------------------------------
\begin{frame}{Stammfunktionen}

\begin{definition}
Eine differenzierbare Funktion $F : I \rightarrow \mathbb R$ hei{\ss}t {\bf Stammfunktion} einer
Funktion $f : I \rightarrow \mathbb R$, falls $F'(x)=f(x)$.
\end{definition}

Eine weitere Funktion $G : I \rightarrow \mathbb R$ ist genau dann ebenfalls eine Stammfunktion, falls
$F-G$ eine Konstante ist.

\vspace*{4mm}
Man schreibt auch
\bq
 F(x) & = & \int f(x)\; dx.
\eq
Der Ausdruck auf der rechten Seite wird auch als {\bf unbestimmtes Integral} bezeichnet.

\end{frame}

% page --------------------------------------------------------------------------------------------------
\begin{frame}{Hauptsatz der Differential- und Integralrechnung}

\begin{mytheorem27}
Sei $F(x)$ eine Stammfunktion von $f(x)$.
\bq
 \int\limits_a^b f(x) \; dx & = & F(b) - F(a).
\eq
\end{mytheorem27}

Man schreibt auch
\bq
 \left. F(x) \right|_a^b & = & F(b) - F(a).
\eq

\end{frame}

% page --------------------------------------------------------------------------------------------------
\begin{frame}{Stammfunktionen}

Stammfunktionen einiger Grundfunktionen:
\bq
 f(x) = x^n & \Rightarrow & F(x) = \frac{x^{n+1}}{n+1}, \;\;\; n \neq -1,
 \nonumber \\
 f(x) = e^x & \Rightarrow & F(x) = e^x, 
 \nonumber \\
 f(x) = \frac{1}{x} & \Rightarrow & F(x) = \ln\left| x \right|,
 \nonumber \\
 f(x) = \sin(x) & \Rightarrow & F(x) = -\cos(x),
 \nonumber \\
 f(x) = \cos(x) & \Rightarrow & F(x) = \sin(x),
 \nonumber \\
 f(x) = \frac{1}{1+x^2} & \Rightarrow & F(x) = \arctan(x).
\eq

\end{frame}

% page --------------------------------------------------------------------------------------------------
\begin{frame}{Quiz}

Gesucht ist eine Stammfunktion zu
\bq
 3 x^2 - 4 x + 5
\eq
\begin{description}
\item{(A)} $6 x - 4$
\item{(B)} $3 x^3 - 4 x^2 + 5 x$
\item{(C)} $\frac{3}{2}x^3-4x^2 + 5 x$
\item{(D)} $x^3 -2 x^2 + 5 x + 42$
\end{description}

\end{frame}

% page --------------------------------------------------------------------------------------------------
\begin{frame}{Substitutionsregel}

\begin{theorem}
Sei $f : [a,b] \rightarrow W_1$ eine stetig differenzierbare Funktion und 
$g : D_2 \rightarrow W_2$ eine stetige Funktion mit $W_1 \subset D_2$.
Dann gilt
\bq
 \int\limits_a^b g\left( f(x) \right) f'(x) \; dx & = & 
 \int\limits_{f(a)}^{f(b)} g(x) \; dx.
\eq
\end{theorem}

\end{frame}

% page --------------------------------------------------------------------------------------------------
\begin{frame}{Substitutionsregel}

\begin{example}
Wir betrachten das Integral
\bq
 I & = &
 \int\limits_0^\pi d\theta \; \sin \theta \left( 5 \cos^2 \theta + 3 \cos \theta + 1 \right).
\eq
F\"ur die Substitution $u = - \cos \theta$ gilt
\bq
 \frac{du}{d\theta} & = & \sin \theta
\eq
und daher ergibt sich mit Hilfe der Substitutionsregel
\bq
 I & = &
 \int\limits_{-1}^1 du \left( 5 u^2 - 3 u + 1 \right)
 \;\; = \;\;
 \left. \left( \frac{5}{3} u^3 - \frac{3}{2} u^2 + u \right) \right|_{-1}^1
 \;\; = \;\;
 \frac{16}{3}. 
\eq
\end{example}

\end{frame}

% page --------------------------------------------------------------------------------------------------
\begin{frame}{Partielle Integration}

\begin{theorem}
Seien $f,g: [a,b] \rightarrow \mathbb R$ zwei stetig differenzierbare
Funktionen. Dann gilt
\bq
 \int\limits_a^b f(x) g'(x) \; dx
 & = & 
 \left. f(x) g(x) \right|_a^b - \int\limits_a^b f'(x) g(x) \; dx.
\eq
\end{theorem}

\end{frame}

% page --------------------------------------------------------------------------------------------------
\begin{frame}{Partielle Integration}

\begin{example}
Wir betrachten das Integral
\bq
 I & = &
 \int\limits_0^1 dx \; x \; e^x.
\eq
Setzen wir $f(x)=x$ und $g'(x)=\exp(x)$, so l\"a{\ss}t sich die partielle Integration anwenden,
falls wir eine Stammfunktion zu $g'(x)$ kennen. In diesem Beispiel ist dies besonders einfach,
es ist $g(x)=\exp(x)$. Somit erhalten wir
\bq
 I & = &
 \left. x \; e^x \right|_0^1
 - 
 \int\limits_0^1 dx \; e^x
 \;\; = \;\;
 \left. \left( x - 1 \right) \; e^x \right|_0^1
 \;\; = \;\;
 1.
\eq
\end{example}

\end{frame}

%%%%%%%%%%%%%%%%%%%%%%%%%%%%%%%%%%%%%%%%%%%%%%%%%%%%%%%%%%%%%%%%%%%%%%%%%%%%%%%%%%%%%%%%%%%%%%%%%%%%%%%%%
%%%%%%%%%%%%%%%%%%%%%%%%%%%%%%%%%%%%%%%%%%%%%%%%%%%%%%%%%%%%%%%%%%%%%%%%%%%%%%%%%%%%%%%%%%%%%%%%%%%%%%%%%
%%%%%%%%%%%%%%%%%%%%%%%%%%%%%%%%%%%%%%%%%%%%%%%%%%%%%%%%%%%%%%%%%%%%%%%%%%%%%%%%%%%%%%%%%%%%%%%%%%%%%%%%%

\section{Integrale \"uber rationale Funktionen}

\frame{\sectionpage}

% page --------------------------------------------------------------------------------------------------
\begin{frame}{Integrale \"uber rationale Funktionen}

Wir betrachten als Beispiel
\bq
 I & = & \int\limits_0^1 \frac{x^4+3x^3-12x^2-3x+18}{(x-2)^2(x+2)} \; dx.
\eq
Im ersten Schritt zerlegt man den Integranden mit Hilfe der Partialbruchzerlegung:
\bq
\frac{x^4+3x^3-12x^2-3x+18}{(x-2)^2(x+2)}
 & = & 
 x + 5 + \frac{1}{(x-2)^2} + \frac{4}{x-2} - \frac{2}{x+2}.
\eq
Somit ist
\bq
 I & = & 
 \int\limits_0^1 \left( x + 5 \right) \; dx 
 + \int\limits_0^1 \frac{dx}{(x-2)^2} 
 + 4 \int\limits_0^1 \frac{dx}{x-2} 
 - 2 \int\limits_0^1 \frac{dx}{x+2}.
\eq

\end{frame}

% page --------------------------------------------------------------------------------------------------
\begin{frame}{Integrale \"uber rationale Funktionen}

{\footnotesize
\bq
 I & = & 
 \int\limits_0^1 \left( x + 5 \right) \; dx 
 + \int\limits_0^1 \frac{dx}{(x-2)^2} 
 + 4 \int\limits_0^1 \frac{dx}{x-2} 
 - 2 \int\limits_0^1 \frac{dx}{x+2}.
\eq
}

Wir berechnen nun die einzelnen Integrale:
{\footnotesize
\bq
 \int\limits_0^1 \left( x + 5 \right) \; dx & = & 
 \left. \frac{1}{2} x^2 + 5 x \right|_0^1 = \frac{11}{2},
 \nonumber \\
 \int\limits_0^1 \frac{dx}{(x-2)^2} & = & \left. - \frac{1}{x-2} \right|_0^1 = 1 - \frac{1}{2} = \frac{1}{2},
 \nonumber \\
 \int\limits_0^1 \frac{dx}{x-2} & = & \left. \ln\left(|x-2|\right) \right|_0^1 = \ln 1 - \ln 2 = -\ln 2,
 \nonumber \\
 \int\limits_0^1 \frac{dx}{x+2} & = & \left. \ln\left(|x+2|\right) \right|_0^1 = \ln 3 -\ln 2,
\eq
}

\end{frame}

% page --------------------------------------------------------------------------------------------------
\begin{frame}{Integrale \"uber rationale Funktionen}

Somit erhalten wir
\bq
 I & = & \int\limits_0^1 \frac{x^4+3x^3-12x^2-3x+18}{(x-2)^2(x+2)} \; dx
 \nonumber \\
 & = & 
 \frac{11}{2} + \frac{1}{2} + 4 \left( - \ln 2 \right) - 2 \left( \ln 3 - \ln 2 \right)
 \nonumber \\
 & = & 
 6 - 2 \ln 2 - 2 \ln 3 
 \nonumber \\
 & = & 
 6 -2 \ln 6.
\eq
\end{frame}

%%%%%%%%%%%%%%%%%%%%%%%%%%%%%%%%%%%%%%%%%%%%%%%%%%%%%%%%%%%%%%%%%%%%%%%%%%%%%%%%%%%%%%%%%%%%%%%%%%%%%%%%%
%%%%%%%%%%%%%%%%%%%%%%%%%%%%%%%%%%%%%%%%%%%%%%%%%%%%%%%%%%%%%%%%%%%%%%%%%%%%%%%%%%%%%%%%%%%%%%%%%%%%%%%%%
%%%%%%%%%%%%%%%%%%%%%%%%%%%%%%%%%%%%%%%%%%%%%%%%%%%%%%%%%%%%%%%%%%%%%%%%%%%%%%%%%%%%%%%%%%%%%%%%%%%%%%%%%

\section{Uneigentliche Integrale}

\frame{\sectionpage}

% page --------------------------------------------------------------------------------------------------
\begin{frame}{Uneigentliche Integrale}

\begin{definition}
Unter einem uneigentlichen Integral versteht man ein Integral, bei dem eine Integrationsgrenze unendlich ist oder
bei dem der Integrand an einer Integrationsgrenze nicht definiert ist.
Es kann auch eine Kombination der beiden F\"alle auftreten.
\end{definition}

\end{frame}

% page --------------------------------------------------------------------------------------------------
\begin{frame}{Uneigentliche Integrale}

Wir betrachten zun\"achst den Fall, da{\ss} eine Integrationsgrenze unendlich ist.
Sei $f: [a,\infty[ \rightarrow \mathbb R$ eine Funktion, die \"uber jedem Intervall $[a,\Lambda]$ mit
$a<\Lambda <\infty$ Riemann-integrierbar ist. Falls der Grenzwert
\bq
 \lim\limits_{\Lambda \rightarrow \infty} \int\limits_a^\Lambda f(x) \; dx
\eq
existiert, nennt man das Integral von $a$  bis Unendlich konvergent und man setzt
\bq
 \int\limits_a^\infty f(x) \; dx
 & = &
 \lim\limits_{\Lambda \rightarrow \infty} \int\limits_a^\Lambda f(x) \; dx.
\eq
Analog definiert man das Integral f\"ur das Intervall $]-\infty,b]$.

\end{frame}

% page --------------------------------------------------------------------------------------------------
\begin{frame}{Uneigentliche Integrale}

\begin{example}
\bq
\int\limits_1^\infty \frac{dx}{x^2}
 & = & 
 \lim\limits_{\Lambda \rightarrow \infty} \int\limits_1^\Lambda \frac{dx}{x^2}
 =
 - \lim\limits_{\Lambda \rightarrow \infty} \left. \frac{1}{x} \right|^\Lambda_1
 = 
 1 - \lim\limits_{\Lambda \rightarrow \infty} \frac{1}{\Lambda}
 = 
 1.
\eq
\end{example}

\end{frame}

% page --------------------------------------------------------------------------------------------------
\begin{frame}{Uneigentliche Integrale}

Wir betrachten nun den Fall, da{\ss} der Integrand an einer Intervallgrenze nicht definiert ist.
Sei $f:]a,b] \rightarrow \mathbb R$ eine Funktion, die \"uber jedem Teilintervall $[a+\eps,b]$ mit
$0<\eps<(b-a)$ Riemann-integrierbar ist. Falls der Grenzwert
\bq
 \lim\limits_{\eps\rightarrow 0+} \int\limits_{a+\eps}^b f(x) \; dx
\eq
existiert, nennt man das Integral \"uber $[a,b]$ konvergent und man setzt
\bq
 \int\limits_{a}^b f(x) \; dx
 & = & 
 \lim\limits_{\eps\rightarrow 0+} \int\limits_{a+\eps}^b f(x) \; dx
\eq
Analog definiert man das Integral f\"ur den Fall in der die Funktion an der oberen Intevallgrenze nicht
definiert ist.

\end{frame}

% page --------------------------------------------------------------------------------------------------
\begin{frame}{Uneigentliche Integrale}

\begin{example}
\bq
 \int\limits_0^1 \frac{dx}{\sqrt{x}}
 & = &
 \lim\limits_{\eps\rightarrow 0+} \int\limits_\eps^1 \frac{dx}{\sqrt{x}}
 =
 2 \lim\limits_{\eps\rightarrow 0+} \left. \sqrt{x} \right|_\eps^1
 =
 2 - 2 \lim\limits_{\eps\rightarrow 0+} \sqrt{\eps}
 = 
 2.
\eq
\end{example}

\end{frame}

% page --------------------------------------------------------------------------------------------------
\begin{frame}{Uneigentliche Integrale: Allgemeiner Fall}

Sei $f: ]a,b[ \rightarrow \mathbb R$, 
$a\in {\mathbb R} \cup \{-\infty\}$,
$b\in {\mathbb R} \cup \{\infty\}$,
eine Funktion, die \"uber jedem Teilintervall $[\alpha,\beta] \subset ]a,b[$ Riemann-integrierbar ist und sei
$c\in ]a,b[$ beliebig.
Falls die beiden uneigentlichen Integrale
\bq
 \int\limits_a^c f(x)\;dx = \lim\limits_{\alpha\rightarrow a+} \int\limits_\alpha^c f(x)\;dx,
 & &
 \int\limits_c^b f(x)\;dx = \lim\limits_{\beta\rightarrow b-} \int\limits_c^\beta f(x)\;dx
\eq
existieren, nennt man das Integral \"uber $]a,b[$ konvergent und man setzt
\bq
 \int\limits_a^b f(x)\;dx
 & = & \int\limits_a^c f(x)\;dx + \int\limits_c^b f(x)\;dx.
\eq

\end{frame}

% page --------------------------------------------------------------------------------------------------
\begin{frame}{Quiz}

\bq
 \int\limits_0^1 \frac{dx}{x^{\frac{9}{10}}}
 & = & ?
\eq
\begin{description}
\item{(A)} $0$
\item{(B)} $1$
\item{(C)} $9$
\item{(D)} $10$
\end{description}

\end{frame}

%%%%%%%%%%%%%%%%%%%%%%%%%%%%%%%%%%%%%%%%%%%%%%%%%%%%%%%%%%%%%%%%%%%%%%%%%%%%%%%%%%%%%%%%%%%%%%%%%%%%%%%%%
%%%%%%%%%%%%%%%%%%%%%%%%%%%%%%%%%%%%%%%%%%%%%%%%%%%%%%%%%%%%%%%%%%%%%%%%%%%%%%%%%%%%%%%%%%%%%%%%%%%%%%%%%
%%%%%%%%%%%%%%%%%%%%%%%%%%%%%%%%%%%%%%%%%%%%%%%%%%%%%%%%%%%%%%%%%%%%%%%%%%%%%%%%%%%%%%%%%%%%%%%%%%%%%%%%%

% page --------------------------------------------------------------------------------------------------
\begin{frame}{Organisation der n\"achsten Woche}

Aufgrund von diversen Einf\"uhrungsveranstaltungen:
\begin{itemize}
\item Mittwoch, 28.10.2020: Plenum um 13:00h 

(11:00h Erstsemesterbegr\"u{\ss}ung des Pr\"asidenten)
\item Freitag, 30.10.2020: keine Vorlesung, kein neues \"Ubungsblatt.
Plenum und \"Ubungsgruppen finden wie gewohnt statt.
\end{itemize}

\end{frame}

%%%%%%%%%%%%%%%%%%%%%%%%%%%%%%%%%%%%%%%%%%%%%%%%%%%%%%%%%%%%%%%%%%%%%%%%%%%%%%%%%%%%%%%%%%%%%%%%%%%%%%%%%
%%%%%%%%%%%%%%%%%%%%%%%%%%%%%%%%%%%%%%%%%%%%%%%%%%%%%%%%%%%%%%%%%%%%%%%%%%%%%%%%%%%%%%%%%%%%%%%%%%%%%%%%%
%%%%%%%%%%%%%%%%%%%%%%%%%%%%%%%%%%%%%%%%%%%%%%%%%%%%%%%%%%%%%%%%%%%%%%%%%%%%%%%%%%%%%%%%%%%%%%%%%%%%%%%%%

% page --------------------------------------------------------------------------------------------------
\begin{frame}

\end{frame}

\end{document}

