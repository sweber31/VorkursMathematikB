\documentclass[german]{beamer}

\mode<presentation>
{
 \usetheme{Madrid}

% \usecolortheme{crane}
 \usecolortheme{wolverine}
}

\usepackage{hyperref}

\usepackage[german]{babel}
\usepackage{times}
\usepackage[latin1,utf8]{inputenc}
\usepackage[OT2,T1]{fontenc}
\usepackage{shuffle}

\usepackage{tikz}
\usetikzlibrary{hobby}

% Stefan's abbreveations
\newcommand{\bq}{\begin{eqnarray*}}
\newcommand{\eq}{\end{eqnarray*}}
\newcommand{\eps}{\varepsilon}

\definecolor{MyYellowOrange}{cmyk}{0,0.5,1,0}
\newcommand{\superalert}[1]{{\color{MyYellowOrange}{#1}}}
\newtheorem*{myemptytheorem}{}

% dedicated environments
\newtheorem*{mytheorem23}{Produktregel:} 
\newtheorem*{mytheorem24}{Quotientenregel:} 
\newtheorem*{mytheorem25}{Kettenregel:} 
\newtheorem*{mytheorem26}{Ableitung der Umkehrfunktion:} 

%%%%%%%%%%%%%%%%%%%%%%%%%%%%%%%%%%%%%%%%%%%%%%%%%%%%%%%%%%%%%%%%%%%%%%%%%%%%%%%%%%%%%%%%%%%%%%%%%%%%%%%%%
%%%%%%%%%%%%%%%%%%%%%%%%%%%%%%%%%%%%%%%%%%%%%%%%%%%%%%%%%%%%%%%%%%%%%%%%%%%%%%%%%%%%%%%%%%%%%%%%%%%%%%%%%
%%%%%%%%%%%%%%%%%%%%%%%%%%%%%%%%%%%%%%%%%%%%%%%%%%%%%%%%%%%%%%%%%%%%%%%%%%%%%%%%%%%%%%%%%%%%%%%%%%%%%%%%%

\title{Differentialrechnung}

\subtitle{Mathematischer Br\"uckenkurs}

\author{Stefan Weinzierl}

\institute[Uni Mainz]{Institut f\"ur Physik, Universit\"at Mainz}%

\date[WiSe 2020/21]{Wintersemester 2020/21}

\begin{document}

%%%%%%%%%%%%%%%%%%%%%%%%%%%%%%%%%%%%%%%%%%%%%%%%%%%%%%%%%%%%%%%%%%%%%%%%%%%%%%%%%%%%%%%%%%%%%%%%%%%%%%%%%
%%%%%%%%%%%%%%%%%%%%%%%%%%%%%%%%%%%%%%%%%%%%%%%%%%%%%%%%%%%%%%%%%%%%%%%%%%%%%%%%%%%%%%%%%%%%%%%%%%%%%%%%%
%%%%%%%%%%%%%%%%%%%%%%%%%%%%%%%%%%%%%%%%%%%%%%%%%%%%%%%%%%%%%%%%%%%%%%%%%%%%%%%%%%%%%%%%%%%%%%%%%%%%%%%%%

\begin{frame}
  \titlepage
\end{frame}

%%%%%%%%%%%%%%%%%%%%%%%%%%%%%%%%%%%%%%%%%%%%%%%%%%%%%%%%%%%%%%%%%%%%%%%%%%%%%%%%%%%%%%%%%%%%%%%%%%%%%%%%%
%%%%%%%%%%%%%%%%%%%%%%%%%%%%%%%%%%%%%%%%%%%%%%%%%%%%%%%%%%%%%%%%%%%%%%%%%%%%%%%%%%%%%%%%%%%%%%%%%%%%%%%%%
%%%%%%%%%%%%%%%%%%%%%%%%%%%%%%%%%%%%%%%%%%%%%%%%%%%%%%%%%%%%%%%%%%%%%%%%%%%%%%%%%%%%%%%%%%%%%%%%%%%%%%%%%

\section{Die Ableitung}

\frame{\sectionpage}

% page --------------------------------------------------------------------------------------------------
\begin{frame}{Die Ableitung}

\begin{definition}
Sei $D \subset \mathbb R$ und $f:D \rightarrow \mathbb R$ eine Funktion.
$f$ nennt man im Punkte $x\in D$ {\bf differenzierbar}, falls es mindestens eine Folge
$(\xi_n) \in D\backslash x$ mit $\lim\limits_{n \rightarrow \infty} \xi_n = x$ gibt
und f\"ur jede solche Folge der Grenzwert
\bq
 f'(x) & = & 
 \lim\limits_{n \rightarrow \infty} \frac{f(\xi_n)-f(x)}{\xi_n-x}
 \;\; = \;\;
 \lim\limits_{\xi \rightarrow x} \frac{f(\xi)-f(x)}{\xi-x}
\eq
existiert.
\end{definition}

Man schreibt auch
\bq
 f'(x) & = & \frac{df(x)}{dx} = \frac{d}{dx} f(x) = \lim\limits_{h\rightarrow 0} \frac{f(x+h)-f(x)}{(x+h)-x}.
\eq

\end{frame}

% page --------------------------------------------------------------------------------------------------
\begin{frame}{Geometrische Bedeutung der Ableitung}

Die Ableitung $f'(x_0)$ gibt die \alert{Steigung der Tangente} im Punkte $x_0$ an:
\vspace*{10mm}
\begin{center}
\begin{tikzpicture}
\draw [->] (-2.0,0.0) -- (2.0,0.0);
\draw [->] (0.0,-1.0) -- (0.0,2.0);
\draw (-1.5,2.25) to [curve through={(-1.4,1.96)(-1.0,1.0)(0.0,0.0)(1.0,1.0)(1.4,1.96)}] (1.5,2.25);
\draw (0.3,-0.4) -- (1.5,2.0);
\draw[dashed] (1.0,0.0) -- (1.0,1.0);
\draw [fill] (1.0,1.0) circle (0.03);
\node [below] at (1.0,-0.1) {\footnotesize $x_0$};
\node [left] at (-1.5,1.8) {\footnotesize $f(x)$};
\node [right] at (1.6,1.8) {\footnotesize $t(x)=f(x_0)+f'(x_0) \cdot (x-x_0)$};
\end{tikzpicture}
\end{center}

\end{frame}

% page --------------------------------------------------------------------------------------------------
\begin{frame}{Quiz}

Die Funktion
\bq
 f(x) & = & \left\{\begin{array}{ll} 0 & x \le 0 \\ \sin(x) & x>0 \end{array} \right.
\eq
ist im Punkte $x=0$
\begin{description}
\item{(A)} differenzierbar
\item{(B)} nicht differenzierbar
\end{description}

\end{frame}

% page --------------------------------------------------------------------------------------------------
\begin{frame}{Quiz}

Die Funktion
\bq
 f(x) & = & \left\{\begin{array}{ll} 1 & x \le 0 \\ \cos(x) & x>0 \end{array} \right.
\eq
ist im Punkte $x=0$
\begin{description}
\item{(A)} differenzierbar
\item{(B)} nicht differenzierbar
\end{description}

\end{frame}

% page --------------------------------------------------------------------------------------------------
\begin{frame}{S\"atze \"uber Ableitungen}

\begin{theorem}
Seien $f,g : D \rightarrow \mathbb R$ in $x \in D$ differenzierbare Funktionen und 
$\lambda \in \mathbb R$.
Dann sind auch die Funktionen $f+g$, und $\lambda f$ in $x$ differenzierbar
und es gilt
\bq
 \left(f+g\right)'(x) & = & f'(x) + g'(x),
 \nonumber \\
 \left( \lambda f \right)'(x) & = & \lambda f'(x).
\eq
\end{theorem}

\begin{mytheorem23}
Mit den Voraussetzungen wie oben ist auch die Funktion $f\cdot g$ in $x$ differenzierbar
und es gilt
\bq
 \left( f \cdot g \right)'(x) & = & f'(x) g(x) + f(x) g'(x).
\eq
\end{mytheorem23}

\end{frame}

% page --------------------------------------------------------------------------------------------------
\begin{frame}{Beweis der Produktregel}

{\small
\bq
\lefteqn{
\left( f \cdot g \right)'(x) = 
} & &
 \nonumber \\
 & = &
 \lim\limits_{h\rightarrow 0} \frac{f(x+h)g(x+h)-f(x)g(x)}{h}
 \nonumber \\
 & = &
 \lim\limits_{h\rightarrow 0} \frac{1}{h} \left[ f(x+h)g(x+h)-f(x+h)g(x)+f(x+h)g(x)-f(x)g(x) \right]
 \nonumber \\
 & = &
 \lim\limits_{h\rightarrow 0} f(x+h) \frac{g(x+h)-g(x)}{h} 
 + \lim\limits_{h\rightarrow 0}\frac{f(x+h)-f(x)}{h}g(x)
 \nonumber \\
 & = &
 f(x) g'(x) + f'(x) g(x).
\eq
}

\end{frame}

% page --------------------------------------------------------------------------------------------------
\begin{frame}{Die Produktregel}

\begin{example}
\bq
 f\left(x\right) & = & \underbrace{\phantom{\left(x\right.}x^2\phantom{\left.x\right)} }_{f_1\left(x\right)} \underbrace{\sin\left(x\right)}_{f_2\left(x\right)}
 \nonumber \\
 f'\left(x\right) & = & 
 \underbrace{\phantom{\left(x\right.}2 x\phantom{\left.x\right)} }_{f_1'\left(x\right)} \underbrace{\sin\left(x\right)}_{f_2\left(x\right)} 
 + \underbrace{\phantom{\left(x\right.}x^2\phantom{\left.x\right)} }_{f_1\left(x\right)} \underbrace{\cos\left(x\right)}_{f_2'\left(x\right)}
\eq
\end{example}

\end{frame}

% page --------------------------------------------------------------------------------------------------
\begin{frame}{S\"atze \"uber Ableitungen}

\begin{mytheorem24}
{\bf Quotientenregel}: 
Ist weiter $g(x)\neq 0$ f\"ur alle $x\in D$, so ist auch die Funktion $f/g$ in $x$ differenzierbar und es gilt
\bq
 \left( \frac{f}{g} \right)'(x) & = &
 \frac{f'(x) g(x) - f(x) g'(x)}{g(x)^2}.
\eq
\end{mytheorem24}

\end{frame}

% page --------------------------------------------------------------------------------------------------
\begin{frame}{Die Quotientenregel}

\begin{example}
\bq
 f\left(x\right) & = & \frac{2x-3}{x+1}
 \nonumber \\
 f'\left(x\right) & = & 
 \frac{2 \cdot \left(x+1\right)-\left(2x-3\right) \cdot 1}{\left(x+1\right)^2}
 \; = \;
 \frac{5}{\left(x+1\right)^2}
\eq
\end{example}

\end{frame}

% page --------------------------------------------------------------------------------------------------
\begin{frame}{S\"atze \"uber Ableitungen}

\begin{mytheorem25}
{\bf Kettenregel}: Seien $f : D_1 \rightarrow W_1$ und $g : D_2 \rightarrow W_2$ Funktionen mit
$W_1 \subset D_2$. Falls $f$ im Punkte $x\in D_1$ differenzierbar ist und $g$ im Punkte $y=f(x)\in D_2$
differenzierbar ist, so ist die zusammengesetzte Funktion $g \circ f: D_1 \rightarrow W_2$
in $x$ differenzierbar und es gilt
\bq
 \left( g \circ f \right)'(x) & = & g'\left(f(x)\right) f'(x).
\eq
\end{mytheorem25}

\end{frame}

% page --------------------------------------------------------------------------------------------------
\begin{frame}{Die Kettenregel}

\begin{example}
\bq
 f\left(x\right) & = & \sin\left(3x^2+4x+5\right)
 \nonumber \\
 f'\left(x\right) & = & 
 \left(6x+4\right) \cdot \cos\left(3x^2+4x+5\right)
\eq
\end{example}

\end{frame}

% page --------------------------------------------------------------------------------------------------
\begin{frame}{S\"atze \"uber Ableitungen}

\begin{mytheorem26}
{\bf Ableitung der Umkehrfunktion}:
Sei $D \subset \mathbb R$ ein abgeschlossenes Intervall, $f : D \rightarrow W$ eine stetige, steng monotone
Funktion und $f^{-1} : W \rightarrow D$ die Umkehrfunktion.
Ist $f$ im Punkt $x\in D$ differenzierbar und ist $f'(x) \neq 0$, so ist $f^{-1}$ im Punkt $y=f(x)$
differenzierbar und es gilt
\bq
 \left( f^{-1} \right)'(y) & = & \frac{1}{f'\left(x\right)} = \frac{1}{f'\left(f^{-1}(y)\right)}.
\eq
\end{mytheorem26}

\end{frame}

% page --------------------------------------------------------------------------------------------------
\begin{frame}{Die Ableitung der Umkehrfunktion}

\begin{example}
Die Ableitung des \alert{Logarithmus} erh\"alt man mit Hilfe der Regel \"uber die Umkehrfunktion:

Wir beginnen mit der Exponentialfunktion: $f(x)=e^x$, $f'(x)=e^x$.

Die Umkehrfunktion ist der Logarithmus: 
\bq
 f^{-1}(y)=\ln y
\eq
Nun ist
\bq
 (f^{-1})'(y) & = & \frac{1}{f'(f^{-1}(y))} \; = \; \frac{1}{\exp(\ln y)} \; = \; \frac{1}{y},
\eq 
also
\bq
 f(x) = \ln x, & & f'(x) = \frac{1}{x}.
\eq
\end{example}

\end{frame}

% page --------------------------------------------------------------------------------------------------
\begin{frame}{Ableitungen}

\begin{example}
Die Ableitungen von \alert{Sinus} und \alert{Kosinus} erh\"alt man aus der Darstellung
\bq
 \sin(x) \; = \; \frac{1}{2i} \left(e^{ix}-e^{-ix}\right), & & \cos(x) \; = \; \frac{1}{2}\left(e^{ix}+e^{-ix}\right)
\eq 
zu
\bq
 f(x) = \sin(x) & \Rightarrow & f'(x) = \cos(x),
 \nonumber \\
 f(x) = \cos(x) & \Rightarrow & f'(x) = -\sin(x).
\eq
\end{example}
Rechnung:
\bq
 \frac{d}{dx} \sin\left(x\right)
 & = &
 \frac{d}{dx} \left[ \frac{1}{2i} \left(e^{ix}-e^{-ix}\right) \right]
 \; = \;
 \frac{1}{2i} \left(\frac{d}{dx}e^{ix}-\frac{d}{dx}e^{-ix}\right)
 \nonumber \\
 & = &
 \frac{1}{2i} \left(ie^{ix}+ie^{-ix}\right)
 \; = \;
 \frac{1}{2} \left(e^{ix}+e^{-ix}\right)
 \; = \;
 \cos\left(x\right)
\eq

\end{frame}

% page --------------------------------------------------------------------------------------------------
\begin{frame}{Wichtige Ableitungen}

Ableitungen einiger \alert{Grundfunktionen}:
\bq
 f(x) = x^n & \Rightarrow & f'(x) = n x^{n-1},
 \nonumber \\
 f(x) = e^x & \Rightarrow & f'(x) = e^x.
\eq
Die Ableitung des \alert{Logarithmus} erh\"alt man mit Hilfe der Regel \"uber die Umkehrfunktion:
\bq
 f(x) = \ln x & \Rightarrow & f'(x) = \frac{1}{x}.
\eq
Die Ableitungen von \alert{Sinus} und \alert{Kosinus} erh\"alt man aus der Darstellung
mittels der Exponentialfunktion:
\bq
 f(x) = \sin(x) & \Rightarrow & f'(x) = \cos(x),
 \nonumber \\
 f(x) = \cos(x) & \Rightarrow & f'(x) = -\sin(x).
\eq

\end{frame}

% page --------------------------------------------------------------------------------------------------
\begin{frame}{Weitere Ableitungen}

Die Ableitung aller weiteren trigonometrischen und hyperbolischen Funktionen lassen sich ebenfalls mit den
obigen Regeln bestimmen:
{\small
\bq
 f(x) = \tan(x) & \Rightarrow & f'(x) = \frac{1}{\cos^2(x)},
 \nonumber \\
 f(x) = \arcsin(x) & \Rightarrow & f'(x) = \frac{1}{\sqrt{1-x^2}},
 \nonumber \\
 f(x) = \arctan(x) & \Rightarrow & f'(x) = \frac{1}{1+x^2},
 \nonumber \\
 f(x) = \sinh(x) & \Rightarrow & f'(x) = \cosh(x),
 \nonumber \\
 f(x) = \cosh(x) & \Rightarrow & f'(x) = \sinh(x),
 \nonumber \\
 f(x) = \tanh(x) & \Rightarrow & f'(x) = \frac{1}{\cosh^2(x)},
 \nonumber \\
 f(x) = \mbox{arsinh}(x) & \Rightarrow & f'(x) = \frac{1}{\sqrt{1+x^2}},
 \nonumber \\
 f(x) = \mbox{artanh}(x) & \Rightarrow & f'(x) = \frac{1}{1-x^2}.
\eq
}

\end{frame}

% page --------------------------------------------------------------------------------------------------
\begin{frame}{Quiz}

\bq
 f\left(x\right) & = & 3 x^3 -4
 \nonumber \\
 f'\left(x\right) & = & ?
\eq
\begin{description}
\item{(A)} $\frac{3}{4}x^4-4x$
\item{(B)} $9 x^2 - 4$
\item{(C)} $9x^2$
\item{(D)} $3x^2-4$
\end{description}

\end{frame}

% page --------------------------------------------------------------------------------------------------
\begin{frame}{Quiz}

\bq
 f\left(x\right) & = & \sin\left(\cos\left(2x\right)\right)
 \nonumber \\
 f'\left(x\right) & = & ?
\eq
\begin{description}
\item{(A)} $2 \cos\left(\cos\left(2x\right)\right)$
\item{(B)} $2 \sin\left(2x\right) \cdot \cos\left(\cos\left(2x\right)\right)$
\item{(C)} $- 2 \cos\left(2x\right) \cdot \cos\left(\sin\left(2x\right)\right)$
\item{(D)} $- 2 \sin\left(2x\right) \cdot \cos\left(\cos\left(2x\right)\right)$
\end{description}

\end{frame}

% page --------------------------------------------------------------------------------------------------
\begin{frame}{H\"ohere Ableitungen}

Sei $f : D \rightarrow {\mathbb R}$ eine differenzierbare Funktion.
Ist $f' : D \rightarrow {\mathbb R}$ ebenfalls wieder differenzierbar, so bezeichnet man
mit
\bq
 f''(x) & = &
 \frac{d^2 f(x)}{dx^2}
 \;\; = \;\;
 (f')'(x)
\eq
die \superalert{zweite Ableitung}. Ist auch $f''(x)$ wieder differenzierbar, so erh\"alt man durch Ableiten
die dritte Ableitung $f'''(x)$.
Allgemein schreiben wir f\"ur die \superalert{$n$-te Ableitung}
\bq
 f^{(n)}(x) & = &
 \frac{d^nf(x)}{dx^n}.
\eq
Unter der \alert{$0$-ten Ableitung} einer Funktion versteht man die Funktion selbst.

\end{frame}

% page --------------------------------------------------------------------------------------------------
\begin{frame}{H\"ohere Ableitungen}

\begin{example}
\bq
 f\left(x\right) & = & 3 x^5 + 7 x^4 + 2 x^3 + x^2 - x + 5
 \nonumber \\
 f'\left(x\right) & = & 15 x^4 + 28 x^3 + 6 x^2 + 2 x - 1
 \nonumber \\
 f''\left(x\right) & = & 60 x^3 + 84 x^2 + 12 x + 2
 \nonumber \\
 f'''\left(x\right) & = & 180 x^2 + 168 x + 12
 \nonumber \\
 f^{(4)}\left(x\right) & = & 360 x + 168
 \nonumber \\
 f^{(5)}\left(x\right) & = & 360
 \nonumber \\
 f^{(6)}\left(x\right) & = & 0
\eq
\end{example}

\end{frame}

% page --------------------------------------------------------------------------------------------------
\begin{frame}{H\"ohere Ableitungen}

\begin{example}
\bq
 f\left(x\right) & = & \sin\left(x\right)
 \nonumber \\
 f'\left(x\right) & = & \cos\left(x\right)
 \nonumber \\
 f''\left(x\right) & = & -\sin\left(x\right)
 \nonumber \\
 f'''\left(x\right) & = & -\cos\left(x\right)
 \nonumber \\
 f^{(4)}\left(x\right) & = & \sin\left(x\right)
 \nonumber \\
 f^{(5)}\left(x\right) & = & \cos\left(x\right)
\eq
\end{example}

\end{frame}

% page --------------------------------------------------------------------------------------------------
\begin{frame}{Quiz}

\bq
 f\left(x\right) & = & e^{2x}
 \nonumber \\
 f^{(4)}\left(x\right) & = & ?
\eq
\begin{description}
\item{(A)} $e^{2x}$
\item{(B)} $2 e^{2x}$
\item{(C)} $16 e^{2x}$
\item{(D)} $24 e^{2x}$
\end{description}

\end{frame}

% page --------------------------------------------------------------------------------------------------
\begin{frame}{Stetige Differenzierbarkeit}

\begin{definition}
Eine Funktion $f(x)$ nennt man \superalert{stetig differenzierbar}, falls sie differenzierbar ist und die Ableitung $f'(x)$ stetig ist.
\end{definition}
\begin{definition}
Eine Funktion $f(x)$ nennt man \superalert{$n$-mal stetig differenzierbar}, falls sie $n$-mal differenzierbar ist und die $n$-te 
Ableitung $f^{(n)}(x)$ stetig ist.
\end{definition}

\end{frame}

% page --------------------------------------------------------------------------------------------------
\begin{frame}{Stetige Differenzierbarkeit}

\begin{example}
\bq
 f\left(x\right)
 & = &
 \left\{ \begin{array}{ll}
 x^2 \sin\left(\frac{1}{x}\right) & x \neq 0 \\
 0 & x=0 \\
 \end{array} \right.
\eq
$f$ ist differenzierbar im Punkt $x=0$:
\bq
 f'\left(0\right) 
 & = &
 \lim\limits_{h \rightarrow 0} \frac{h^2 \sin\left(\frac{1}{h}\right) - 0}{h} \; = \; 0.
\eq
Somit
\bq
 f'\left(x\right)
 & = &
 \left\{ \begin{array}{ll}
 2 x \sin\left(\frac{1}{x}\right) - \cos\left(\frac{1}{x}\right) & x \neq 0 \\
 0 & x=0 \\
 \end{array} \right.
\eq
$f'$ ist nicht stetig im Punkt $x=0$.
\end{example}

\end{frame}

%%%%%%%%%%%%%%%%%%%%%%%%%%%%%%%%%%%%%%%%%%%%%%%%%%%%%%%%%%%%%%%%%%%%%%%%%%%%%%%%%%%%%%%%%%%%%%%%%%%%%%%%%
%%%%%%%%%%%%%%%%%%%%%%%%%%%%%%%%%%%%%%%%%%%%%%%%%%%%%%%%%%%%%%%%%%%%%%%%%%%%%%%%%%%%%%%%%%%%%%%%%%%%%%%%%
%%%%%%%%%%%%%%%%%%%%%%%%%%%%%%%%%%%%%%%%%%%%%%%%%%%%%%%%%%%%%%%%%%%%%%%%%%%%%%%%%%%%%%%%%%%%%%%%%%%%%%%%%

\section{Taylorreihen}

\frame{\sectionpage}

% page --------------------------------------------------------------------------------------------------
\begin{frame}{Taylorreihen}

Motivation:
\begin{itemize}
\item Wir haben bereits die \alert{Reihendarstellung} einiger Funktionen, wie zum Beispiel der Exponentialfunktion,
Sinus oder Kosinus kennengelernt.
\item In diesem Abschnitt geht es um die \superalert{systematische Entwicklung} von Funktionen \superalert{in Potenzreihen}.
\end{itemize}

\end{frame}

% page --------------------------------------------------------------------------------------------------
\begin{frame}{Taylorentwicklung}

\begin{theorem}[Taylorsche Formel]
Sei $I \subset \mathbb R$ und $f : I \rightarrow \mathbb R$ eine
$(n+1)$-mal stetig differenzierbare Funktion. Dann gilt f\"ur $a \in I$ und $x\in I$
{\footnotesize
\bq
 f(x) & = &
  f(a) + \frac{f'(a)}{1!} \cdot (x-a) + \frac{f''(a)}{2!} \cdot (x-a)^2 
  + ... + \frac{f^{(n)}(a)}{n!} \cdot (x-a)^n + R_{n+1}(x).
\eq
}

\vspace*{-2mm}
F\"ur das Restglied gilt:
Es gibt ein $\xi$ zwischen $a$ und $x$ 

(d.h. $\xi \in [a,x]$ f\"ur $x>a$ bzw. $\xi \in [x,a]$ f\"ur $x<a$), so da{\ss}
{\footnotesize
\bq
 R_{n+1}(x) & = & \frac{f^{(n+1)}(\xi)}{(n+1)!} \cdot (x-a)^{n+1}.
\eq
}

\vspace*{-2mm}
Bemerkung: Dies ist eine Existenzaussage, $\xi$ ist im allgemeinen schwer zu bestimmen.
\end{theorem}

\end{frame}

% page --------------------------------------------------------------------------------------------------
\begin{frame}{Taylorentwicklung}

\begin{itemize}
\item In der Praxis verwendet man die ersten $n$ Terme der Taylorentwicklung, um eine Funktion zu approximieren
und vernachl\"assigt das Restglied. 

\item Das vernachl\"assigte Restglied liefert den Fehler dieser Absch\"atzung.
\end{itemize}

\end{frame}

% page --------------------------------------------------------------------------------------------------
\begin{frame}{Taylorentwicklung}

\begin{example}
\bq
 f\left(x\right) & = & \cos\left( x \cdot e^x \right) + \sin\left( x^2 \cdot e^{-x} \right)
\eq
Taylorentwicklung um $x_0=0$:
\bq
 f\left(x\right) & = & 
 1 + \frac{1}{2} x^2 - 2 x^3 - \frac{11}{24} x^4 - \frac{2}{3} x^5 + \mathcal{O}\left(x^6\right)
\eq
\end{example}

\end{frame}

% page --------------------------------------------------------------------------------------------------
\begin{frame}{Taylorreihen}

\begin{definition}[Taylorreihe]
Sei nun $f : I \rightarrow \mathbb R$ eine beliebig of differenzierbare Funktion und $a \in I$.
Wir definieren die Taylorreihe einer Funktion $f$ um den Entwicklungspunkt $a$:
\bq
 T_f(x) & = & \sum\limits_{n=0}^\infty \frac{f^{(n)}(a)}{n!} \cdot (x-a)^n
\eq
\end{definition}

\end{frame}

% page --------------------------------------------------------------------------------------------------
\begin{frame}{Taylorreihen}

Bemerkungen:
\begin{enumerate}
\item Der Konvergenzradius der Taylorreihe ist nicht notwendig $>0$.

\item Falls die Taylorreihe von $f$ konvergiert, konvergiert sie nicht notwendigerweise gegen $f$.

\item Die Taylorreihe konvergiert genau f\"ur diejenigen $x\in I$ gegen $f(x)$, f\"ur die das Restglied gegen
Null konvergiert.
\end{enumerate}

\end{frame}

% page --------------------------------------------------------------------------------------------------
\begin{frame}{Taylorreihen}

\begin{example}
Wir geben ein Gegenbeispiel zu Punkt 2 an: Wir betrachten die Taylorreihe der Funktion
\bq
 f(x) & = & \left\{ \begin{array}{ll}
                     \exp\left(-\frac{1}{x^2}\right), & x \neq 0, \\
                     0                                & x=0, \\
                    \end{array} \right.
\eq
im Punkte $a=0$. $f$ ist beliebig oft differenzierbar und es gilt
\bq
 f^{(n)}(0) & = & 0.
\eq
Die Taylorreihe von $f$ um den Nullpunkt ist also identisch Null.
\end{example}

\end{frame}

% page --------------------------------------------------------------------------------------------------
\begin{frame}{Taylorreihen}

\begin{center}
\begin{tikzpicture}
\draw [->] (-4.0,0.0) -- (4.0,0.0);
\draw [->] (0.0,-1.0) -- (0.0,2.0);
\draw (-4.0,0.939) to [curve through={(-3.0,0.895)(-2.0,0.7788)(-1.0,0.368)(-0.5,0.018)(-0.4,0.0019)(0.0,0.0)(0.4,0.0019)(0.5,0.018)(1.0,0.368)(2.0,0.7788)(3.0,0.895)}] (4.0,0.939);
\node [left] at (-1.5,1.8) {\footnotesize $f(x)=\exp\left(-\frac{1}{x^2}\right)$};
\node [right] at (4.0,0.0) {\footnotesize $x$};
\node [above] at (0.0,2.0) {\footnotesize $y$};
\end{tikzpicture}
\end{center}

\end{frame}

%%%%%%%%%%%%%%%%%%%%%%%%%%%%%%%%%%%%%%%%%%%%%%%%%%%%%%%%%%%%%%%%%%%%%%%%%%%%%%%%%%%%%%%%%%%%%%%%%%%%%%%%%
%%%%%%%%%%%%%%%%%%%%%%%%%%%%%%%%%%%%%%%%%%%%%%%%%%%%%%%%%%%%%%%%%%%%%%%%%%%%%%%%%%%%%%%%%%%%%%%%%%%%%%%%%
%%%%%%%%%%%%%%%%%%%%%%%%%%%%%%%%%%%%%%%%%%%%%%%%%%%%%%%%%%%%%%%%%%%%%%%%%%%%%%%%%%%%%%%%%%%%%%%%%%%%%%%%%

\section{Die Regeln von l'Hospital}

\frame{\sectionpage}

% page --------------------------------------------------------------------------------------------------
\begin{frame}{Die erste Regel von l'Hospital}

\begin{theorem}
Seien $f,g : D \rightarrow \mathbb R$ zwei in $x_0 \in D$ stetige Funktionen mit
$f(x_0)=g(x_0)=0$. Weiter seien $f$ und $g$ in einer Umgebung von $x_0$ differenzierbar.
Existiert $\lim\limits_{x\rightarrow x_0} f'(x)/g'(x)$, so gilt:
\bq
 \lim\limits_{x\rightarrow x_0} \frac{f(x)}{g(x)} & = & \lim\limits_{x\rightarrow x_0} \frac{f'(x)}{g'(x)}.
\eq
\end{theorem}

\end{frame}

% page --------------------------------------------------------------------------------------------------
\begin{frame}{Die erste Regel von l'Hospital}

\begin{example}
\bq
 \lim\limits_{x\rightarrow 0} \frac{1-\cos x}{x^2}
 & = & 
 \lim\limits_{x\rightarrow 0} \frac{\sin x}{2 x}
 =
 \frac{1}{2} \lim\limits_{x\rightarrow 0} \frac{\cos x}{1}
 = \frac{1}{2}.
\eq
\end{example}

\end{frame}

% page --------------------------------------------------------------------------------------------------
\begin{frame}{Die zweite Regel von l'Hospital}

\begin{theorem}
Ist $\lim\limits_{x \rightarrow x_0} |f(x)| = \infty$ und $\lim\limits_{x \rightarrow x_0} |g(x)| = \infty$
und exisitiert
$\lim\limits_{x \rightarrow x_0} f'(x)/g'(x)$, so gilt ebenfalls
\bq
 \lim\limits_{x\rightarrow x_0} \frac{f(x)}{g(x)} & = & \lim\limits_{x\rightarrow x_0} \frac{f'(x)}{g'(x)}.
\eq
\end{theorem}

\end{frame}

% page --------------------------------------------------------------------------------------------------
\begin{frame}{Die zweite Regel von l'Hospital}

\begin{example}
\bq
 \lim\limits_{x\rightarrow 0} \frac{\ln x}{\frac{1}{x}}
 & = & 
 \lim\limits_{x\rightarrow 0} \frac{\frac{1}{x}}{-\frac{1}{x^2}}
 =
 \lim\limits_{x\rightarrow 0} \left( - x \right) = 0.
\eq
\end{example}

\vspace*{10mm}
Bemerkung: Die l'Hospitalschen Regeln gelten auch f\"ur $x_0 \rightarrow \pm \infty$.

\end{frame}

% page --------------------------------------------------------------------------------------------------
\begin{frame}{Quiz}

\bq
 \lim\limits_{x \rightarrow \infty} \frac{3x-4}{x^2+6x+5} 
 & = & ?
\eq
\begin{description}
\item{(A)} $0$
\item{(B)} $\frac{1}{2}$
\item{(C)} $\frac{3}{2}$
\item{(D)} $3$
\end{description}

\end{frame}

% page --------------------------------------------------------------------------------------------------
\begin{frame}{Quiz}

\bq
 \lim\limits_{x \rightarrow \infty} \frac{7x^3+5x^2-3x-1}{x^3-20x^2+x+10} 
 & = & ?
\eq
\begin{description}
\item{(A)} $0$
\item{(B)} $-\frac{1}{10}$
\item{(C)} $7$
\item{(D)} $21$
\end{description}

\end{frame}

% page --------------------------------------------------------------------------------------------------
\begin{frame}{Mentoring-Programm für Studienanf\"angerinnen und -anf\"anger im Fach Physik}

Was ist das Mentoring-Programm?
\begin{itemize}
\item Studierende („Mentees“) werden im ersten Bachelorsemester durch einen Professor bzw. eine Professorin aus der Physik unterst\"utzt.
\item Der Mentor/die Mentorin ist Anlaufstelle für Fragen zum Studium.
\item Die Teilnahme ist freiwillig und kostet Sie nichts.
\end{itemize}

So k\"onnen Sie teilnehmen:
\begin{itemize}
\item Alle Physik-Erstsemester erhalten am Ende der Einf\"uhrungswoche,
Freitag, 30. Oktober 2020, eine E-Mail mit n\"aheren Anmeldeinfos.
\end{itemize}

\end{frame}

%%%%%%%%%%%%%%%%%%%%%%%%%%%%%%%%%%%%%%%%%%%%%%%%%%%%%%%%%%%%%%%%%%%%%%%%%%%%%%%%%%%%%%%%%%%%%%%%%%%%%%%%%
%%%%%%%%%%%%%%%%%%%%%%%%%%%%%%%%%%%%%%%%%%%%%%%%%%%%%%%%%%%%%%%%%%%%%%%%%%%%%%%%%%%%%%%%%%%%%%%%%%%%%%%%%
%%%%%%%%%%%%%%%%%%%%%%%%%%%%%%%%%%%%%%%%%%%%%%%%%%%%%%%%%%%%%%%%%%%%%%%%%%%%%%%%%%%%%%%%%%%%%%%%%%%%%%%%%

% page --------------------------------------------------------------------------------------------------
\begin{frame}

\end{frame}

\end{document}

