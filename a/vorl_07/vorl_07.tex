\documentclass[german]{beamer}

\mode<presentation>
{
 \usetheme{Madrid}

% \usecolortheme{crane}
 \usecolortheme{wolverine}
}

\usepackage{hyperref}

\usepackage[german]{babel}
\usepackage{times}
\usepackage[latin1,utf8]{inputenc}
\usepackage[OT2,T1]{fontenc}
\usepackage{shuffle}


% Stefan's abbreveations
\newcommand{\bq}{\begin{eqnarray*}}
\newcommand{\eq}{\end{eqnarray*}}
\newcommand{\eps}{\varepsilon}

\definecolor{MyYellowOrange}{cmyk}{0,0.5,1,0}
\newcommand{\superalert}[1]{{\color{MyYellowOrange}{#1}}}
\newtheorem*{myemptytheorem}{}

% dedicated environments
\newtheorem*{mytheorem18}{Vollst\"andigkeitsaxiom:}
\newtheorem*{mytheorem19}{Konvergenzkriterium von Cauchy:} 
\newtheorem*{mytheorem20}{Konvergenzkriterium von Leibniz f\"ur alternierende Reihen:} 
\newtheorem*{mytheorem21}{Majorantenkriterium:} 
\newtheorem*{mytheorem22}{Quotientenkriterium:} 

%%%%%%%%%%%%%%%%%%%%%%%%%%%%%%%%%%%%%%%%%%%%%%%%%%%%%%%%%%%%%%%%%%%%%%%%%%%%%%%%%%%%%%%%%%%%%%%%%%%%%%%%%
%%%%%%%%%%%%%%%%%%%%%%%%%%%%%%%%%%%%%%%%%%%%%%%%%%%%%%%%%%%%%%%%%%%%%%%%%%%%%%%%%%%%%%%%%%%%%%%%%%%%%%%%%
%%%%%%%%%%%%%%%%%%%%%%%%%%%%%%%%%%%%%%%%%%%%%%%%%%%%%%%%%%%%%%%%%%%%%%%%%%%%%%%%%%%%%%%%%%%%%%%%%%%%%%%%%

\title{Folgen und Reihen}

\subtitle{Mathematischer Br\"uckenkurs}

\author{Stefan Weinzierl}

\institute[Uni Mainz]{Institut f\"ur Physik, Universit\"at Mainz}%

\date[WiSe 2020/21]{Wintersemester 2020/21}

\begin{document}

%%%%%%%%%%%%%%%%%%%%%%%%%%%%%%%%%%%%%%%%%%%%%%%%%%%%%%%%%%%%%%%%%%%%%%%%%%%%%%%%%%%%%%%%%%%%%%%%%%%%%%%%%
%%%%%%%%%%%%%%%%%%%%%%%%%%%%%%%%%%%%%%%%%%%%%%%%%%%%%%%%%%%%%%%%%%%%%%%%%%%%%%%%%%%%%%%%%%%%%%%%%%%%%%%%%
%%%%%%%%%%%%%%%%%%%%%%%%%%%%%%%%%%%%%%%%%%%%%%%%%%%%%%%%%%%%%%%%%%%%%%%%%%%%%%%%%%%%%%%%%%%%%%%%%%%%%%%%%

\begin{frame}
  \titlepage
\end{frame}

%%%%%%%%%%%%%%%%%%%%%%%%%%%%%%%%%%%%%%%%%%%%%%%%%%%%%%%%%%%%%%%%%%%%%%%%%%%%%%%%%%%%%%%%%%%%%%%%%%%%%%%%%
%%%%%%%%%%%%%%%%%%%%%%%%%%%%%%%%%%%%%%%%%%%%%%%%%%%%%%%%%%%%%%%%%%%%%%%%%%%%%%%%%%%%%%%%%%%%%%%%%%%%%%%%%
%%%%%%%%%%%%%%%%%%%%%%%%%%%%%%%%%%%%%%%%%%%%%%%%%%%%%%%%%%%%%%%%%%%%%%%%%%%%%%%%%%%%%%%%%%%%%%%%%%%%%%%%%

\section{Folgen}

\frame{\sectionpage}

% page --------------------------------------------------------------------------------------------------
\begin{frame}{Folgen}

\begin{definition}
Unter einer Folge $(a_n)$ reeller Zahlen versteht man eine Abbildung ${\mathbb N} \rightarrow {\mathbb R}$.

\vspace*{2mm}

Jedem $n \in \mathbb N$ wird also ein $a_n \in \mathbb R$ zugeordnet.
\end{definition}

\begin{example}
\bq
 a_n & = & \frac{1}{n^2}
\eq
definiert eine Folge.

\vspace*{2mm}
Explizit: $a_1=1$, $a_2=\frac{1}{4}$, $a_3=\frac{1}{9}$, $a_4=\frac{1}{16}$, $a_5=\frac{1}{25}$, ...
\end{example}

\end{frame}

% page --------------------------------------------------------------------------------------------------
\begin{frame}{Konvergente Folgen}

\begin{definition}
Eine Folge $(a_n)$ hei{\ss}t {\bf konvergent} gegen $a\in \mathbb R$, falls es zu jedem
$\eps > 0$ eine nat\"urliche Zahl $N$ gibt, so da{\ss}
\bq
 \left| a_n - a \right | & < & \eps,
 \;\;\;
 \forall n \ge N.
\eq
In diesem Fall schreibt man
\bq
 \lim\limits_{n\rightarrow \infty} a_n & = & a.
\eq
\end{definition}

In anderen Worten liegen f\"ur eine konvergente Folge ab einem bestimmten $N$ alle Folgenglieder
im Intervall $]a-\eps,a+\eps[$.

\end{frame}

% page --------------------------------------------------------------------------------------------------
\begin{frame}{Divergente Folgen}

\begin{definition}
Eine Folge nennt man {\bf divergent}, wenn sie gegen keine reelle Zahl konvergiert.
\end{definition}

\begin{example}
Beispiele f\"ur divergente Folgen sind
\bq
 a_n & = & n,
 \nonumber \\
 b_n & = & \left\{\begin{array}{rl}
                   1 & \mbox{$n$ gerade} \\
                  -1 & \mbox{$n$ ungerade} \\
                  \end{array}\right.
\eq
\end{example}

\end{frame}

% page --------------------------------------------------------------------------------------------------
\begin{frame}{Beschr\"ankte Folgen}

\begin{definition}
Eine Folge hei{\ss}t nach oben (bzw. unten) beschr\"ankt, falls es ein $c\in \mathbb R$ gibt, so da{\ss}
$a_n \le c$ (bzw. $a_n \ge c$) f\"ur alle $n \in \mathbb N$.
Die Folge hei{\ss}t beschr\"ankt, wenn sie nach oben und unten beschr\"ankt ist.
\end{definition}

\begin{itemize}
\item Jede konvergente Folge beschr\"ankt.

\item Die Umkehrung gilt nicht, eine beschr\"ankte Folge ist nicht notwendiger Weise konvergent,
siehe obiges Beispiel mit der Folge $(b_n)$.

\end{itemize}

\end{frame}

% page --------------------------------------------------------------------------------------------------
\begin{frame}{Rechenregeln f\"ur konvergente Folgen}

Seien $(a_n)$ und $(b_n)$ zwei konvergente Folgen mit den Grenzwerten
\bq
 \lim\limits_{n \rightarrow \infty} a_n = a,
 & &
 \lim\limits_{n \rightarrow \infty} b_n = b.
\eq
Dann sind auch die Folgen $(a_n+b_n)$, $(a_n-b_n)$, $(\lambda a_n)$, $(a_n b_n)$ konvergent
($\lambda \in \mathbb R$)
und es gilt
\begin{myemptytheorem}
\bq
 \lim\limits_{n \rightarrow \infty} \left(a_n+b_n\right) & = & a + b,
\nonumber \\
 \lim\limits_{n \rightarrow \infty} \left(a_n-b_n\right) & = & a - b,
\nonumber \\
 \lim\limits_{n \rightarrow \infty} \left(\lambda a_n\right) & = & \lambda a,
\nonumber \\
 \lim\limits_{n \rightarrow \infty} \left(a_n\cdot b_n\right) & = & a \cdot b.
\eq
\end{myemptytheorem}

\end{frame}

% page --------------------------------------------------------------------------------------------------
\begin{frame}{Rechenregeln f\"ur konvergente Folgen}

Ist weiter $b \neq 0$, so gibt es ein $N \in \mathbb N$, so da{\ss} $b_n\neq 0$ f\"ur alle
$n\ge N$ und wir k\"onnen die Folge
\bq
 \left( \frac{a_n}{b_n} \right)_{n\ge N}
\eq
betrachten. Es gilt
\begin{myemptytheorem}
\bq
 \lim\limits_{n \rightarrow \infty} \left( \frac{a_n}{b_n} \right) & = & \frac{a}{b}.
\eq
\end{myemptytheorem}

\end{frame}

% page --------------------------------------------------------------------------------------------------
\begin{frame}{Rechenregeln f\"ur konvergente Folgen}

\begin{theorem}
Seien $(a_n)$ und $(b_n)$ zwei konvergente Folgen mit $a_n \le b_n$ f\"ur alle $n$.
Dann gilt auch
\bq
\lim\limits_{n \rightarrow \infty} a_n & \le & \lim\limits_{n \rightarrow \infty} b_n.
\eq
\end{theorem}

\vspace*{6mm}
\alert{Bemerkung}: Aus $a_n < b_n$ \superalert{folgt nicht} 
\bq
\lim\limits_{n \rightarrow \infty} a_n & < & \lim\limits_{n \rightarrow \infty} b_n,
\eq
wie das Beispiel
$a_n=0$ und $b_n=1/n$ zeigt.

\end{frame}

% page --------------------------------------------------------------------------------------------------
\begin{frame}{Cauchy-Folgen}

Wir hatten bereits bei der axiomatischen Charakterisierung der reellen Zahlen
den Begriff einer Cauchy-Folge eingef\"uhrt, den wir uns nochmal in Erinnerung rufen:
\begin{definition}
Eine Folge $(a_n)$ reeller Zahlen nennt man {\bf Cauchy-Folge}, falls es zu jedem $\eps > 0$ 
ein $N \in {\mathbb N}$ gibt, so da{\ss}
\bq
 \left| a_n - a_m \right| & < & \eps,
 \;\;\;\;\;\;
 \forall n,m \ge N.
\eq
\end{definition}
\begin{theorem}
Ist eine Folge $(a_n)$ reeller Zahlen konvergent, so ist sie auch eine Cauchy-Folge.
\end{theorem}

\end{frame}

% page --------------------------------------------------------------------------------------------------
\begin{frame}{Vollst\"andigkeitsaxiom}

Die Umkehrung dieses Satzes postuliert nennt man als Axiom:
\begin{mytheorem18}
In ${\mathbb R}$ ist jede Cauchy-Folge konvergent.
\end{mytheorem18}

Wir hatten dies bei der axiomatischen Charakterisierung der reellen Zahlen bereits erw\"ahnt.

Somit gilt in ${\mathbb R}$, da{\ss} eine Folge $(a_n)$ reeller Zahlen genau dann konvergent ist,
falls sie eine Cauchy-Folge ist.

\vspace*{3mm}

Der Vorteil der Definition einer Cauchy-Folge gegen\"uber der Definition des Begriffes Konvergenz
besteht darin, da{\ss} sich erstere nur auf einzelne Folgenglieder bezieht und keinen
Bezug auf einen (eventuellen) Grenzwert nimmt.

\end{frame}

% page --------------------------------------------------------------------------------------------------
\begin{frame}{Quiz}

Die Folge
\bq
 a_n & = & \cos\left(\frac{1}{n^2}\right)
\eq
ist
\begin{description}
\item{(A)} divergent
\item{(B)} konvergent mit Grenzwert $0$
\item{(C)} konvergent mit Grenzwert $\frac{1}{2} \sqrt{2}$
\item{(D)} konvergent mit Grenzwert $1$
\end{description}

\end{frame}

%%%%%%%%%%%%%%%%%%%%%%%%%%%%%%%%%%%%%%%%%%%%%%%%%%%%%%%%%%%%%%%%%%%%%%%%%%%%%%%%%%%%%%%%%%%%%%%%%%%%%%%%%
%%%%%%%%%%%%%%%%%%%%%%%%%%%%%%%%%%%%%%%%%%%%%%%%%%%%%%%%%%%%%%%%%%%%%%%%%%%%%%%%%%%%%%%%%%%%%%%%%%%%%%%%%
%%%%%%%%%%%%%%%%%%%%%%%%%%%%%%%%%%%%%%%%%%%%%%%%%%%%%%%%%%%%%%%%%%%%%%%%%%%%%%%%%%%%%%%%%%%%%%%%%%%%%%%%%

\section{Reihen}

\frame{\sectionpage}

% page --------------------------------------------------------------------------------------------------
\begin{frame}{Reihen}

Sei $(a_n)$ eine Folge reeller Zahlen. Man betrachtet nun die Folge $(s_n)$ der Partialsummen
\bq
 s_n & = & \sum\limits_{j=1}^n a_j.
\eq
Als unendliche Reihe bezeichnet man nun die Folge dieser Partialsummen.
Man schreibt
\bq
 \sum\limits_{j=1}^\infty a_j & = & \lim\limits_{n \rightarrow \infty} s_n \; = \; \lim\limits_{n \rightarrow \infty} \sum\limits_{j=1}^n a_j.
\eq
Eine unendliche Reihe hei{\ss}t konvergent, wenn die Folge der Partialsummen konvergiert.

\end{frame}

% page --------------------------------------------------------------------------------------------------
\begin{frame}{Absolute Konvergenz}

\begin{definition}
Eine unendliche Reihe hei{\ss}t {\bf absolut konvergent}, falls die Reihe
\bq
 \sum\limits_{j=1}^\infty \left| a_j \right|
\eq
konvergent ist.
\end{definition}

\begin{theorem}
Eine absolut konvergente Reihe konvergiert auch im gew\"ohnlichen Sinne.
\end{theorem}

\end{frame}

% page --------------------------------------------------------------------------------------------------
\begin{frame}{Konvergenzkriterien}

\begin{theorem}
Eine notwendige (aber nicht hinreichende) Bedingung f\"ur die Konvergenz einer 
Reihe ist, da{\ss}
\bq
 \lim\limits_{n \rightarrow \infty} a_n & = & 0.
\eq
\end{theorem}

\begin{mytheorem19}
Die Reihe $\sum\limits_{j=1}^\infty a_j$ konvergiert genau dann,
wenn zu jedem $\eps > 0$ ein $N \in \mathbb N$ existiert, so da{\ss}
\bq
 \left|  \sum\limits_{j=m}^n a_j \right| & < & \eps,
 \;\;\; \forall n \ge m \ge N.
\eq
\end{mytheorem19}

\end{frame}

% page --------------------------------------------------------------------------------------------------
\begin{frame}{Konvergenzkriterien}

\begin{theorem}
Eine Reihe $\sum\limits_{j=1}^\infty a_j$ mit \superalert{$a_j \ge 0$} konvergiert genau dann,
wenn die Folge der Partialsummen beschr\"ankt ist.
\end{theorem}

\begin{mytheorem20}
Sei $(a_n)$ eine \superalert{monoton fallende Folge nicht-negativer Zahlen mit $\lim\limits_{n\rightarrow \infty} a_n=0$}.
Dann konvergiert die Reihe
\bq
 \sum\limits_{j=1}^\infty (-1)^j a_j.
\eq
\end{mytheorem20}


\end{frame}

% page --------------------------------------------------------------------------------------------------
\begin{frame}{Konvergenzkriterien}

\begin{mytheorem21}
Sei $\sum\limits_{j=1}^\infty c_j$ eine konvergente Reihe mit
lauter nicht-negativen Gliedern und $(a_n)$ eine Folge mit
\superalert{$|a_n|\le c_n$}. Dann konvergiert die Reihe
\bq
 \sum\limits_{j=1}^\infty a_j
\eq
absolut.
Man nennt $\sum\limits_{j=1}^\infty c_j$ eine Majorante von $\sum\limits_{j=1}^\infty a_j$.
\end{mytheorem21}

\end{frame}

% page --------------------------------------------------------------------------------------------------
\begin{frame}{Konvergenzkriterien}

\begin{mytheorem22}
Sei $\sum\limits_{j=1}^\infty a_j$ eine Reihe mit $a_n \neq 0$
f\"ur alle $n$ und $x$ eine reelle Zahl \superalert{$0<x<1$}, so da{\ss}
\bq
 \left| \frac{a_{n+1}}{a_n} \right| & \le & x,
\;\;\;\; \forall n\ge N.
\eq
Dann konvergiert die Reihe absolut.
\end{mytheorem22}

\end{frame}

% page --------------------------------------------------------------------------------------------------
\begin{frame}{Reihen}

\begin{example}
\bq
 \sum\limits_{j=1}^\infty \frac{x^j}{j!}
\eq
Es ist
\bq
 \left| \frac{a_{n+1}}{a_n} \right|
 & = & 
 \left| \frac{\frac{x^{n+1}}{(n+1)!}}{\frac{x^n}{n!}} \right|
 \; = \;
 \frac{\left|x\right|}{n+1} 
 \; < \; 1
 \;\;\;\;\;\; \mbox{f\"ur} \; n > |x|
\eq
Die Reihe ist nach dem Quotientenkriterium konvergent.
\bq
 \sum\limits_{j=1}^\infty \frac{x^j}{j!}
 \; = \; \exp\left(x\right) - 1,
 & \mbox{bzw.} &
 \sum\limits_{j=0}^\infty \frac{x^j}{j!}
 \; = \;\exp\left(x\right).
\eq

\end{example}

\end{frame}

% page --------------------------------------------------------------------------------------------------
\begin{frame}{Reihen}

\begin{example}
Es sei $|x|<1$.
\bq
 \sum\limits_{j=1}^\infty \frac{x^j}{j}
\eq
Es ist
\bq
 \left| \frac{a_{n+1}}{a_n} \right|
 & = & 
 \left| \frac{\frac{x^{n+1}}{(n+1)}}{\frac{x^n}{n}} \right|
 \; = \;
 \frac{n}{n+1} \left|x\right|
 \; \le \;  \left|x\right|
 \; < \; 1
\eq
Die Reihe ist nach dem Quotientenkriterium konvergent.
\bq
 \sum\limits_{j=1}^\infty \frac{x^j}{j}
 \; = \; -\ln\left(1-x\right).
\eq
\end{example}

\end{frame}

% page --------------------------------------------------------------------------------------------------
\begin{frame}{Reihen}

\begin{example}
\bq
 \sum\limits_{j=1}^\infty \frac{\left(-1\right)^j}{j}
\eq
Die Reihe ist alternierend und $a_n=1/n$ ist eine monoton fallende Folge nicht-negativer Zahlen.

Die Reihe ist nach dem Leibnizkriterium f\"ur alternierende Reihen konvergent.
\bq
 \sum\limits_{j=1}^\infty \frac{\left(-1\right)^j}{j}
 \; = \; -\ln\left(2\right).
\eq
\end{example}

\end{frame}

% page --------------------------------------------------------------------------------------------------
\begin{frame}{Reihen}

\begin{example}
\bq
 \sum\limits_{j=1}^\infty \frac{1}{j}
\eq
Diese Reihe wird als harmonische Reihe bezeichnet.
Diese Reihe ist divergent.
F\"ur die Partialsummen gilt
\begin{alignat*}{10}
 & S_n 
 & 
 = & 1 
 & 
 & \hspace*{2mm} &
 + & \frac{1}{2} 
 & 
 & \hspace*{4mm} &
 + & \frac{1}{3} + \frac{1}{4} 
 & 
 & \hspace*{4mm} &
 + & \frac{1}{5} + \frac{1}{6} + \frac{1}{7} + \frac{1}{8} 
 & 
 & \hspace*{4mm} &
 + & \cdots + \frac{1}{n} 
 \nonumber \\ 
 & 
 & 
 \ge & 1 
 & 
 & &
 + & \frac{1}{2} 
 & 
 & &
 + & \frac{1}{4} + \frac{1}{4} 
 & 
 & &
 + & \frac{1}{8} + \frac{1}{8} + \frac{1}{8} + \frac{1}{8} 
 & 
 & &
 + & \cdots + \frac{1}{n} 
 \nonumber \\ 
 & 
 & 
 = & 1 
 & 
 & &
 + & \frac{1}{2} 
 & 
 & &
 + & \frac{1}{2}
 & 
 & &
 + & \frac{1}{2} 
 & 
 & &
 + & \cdots + \frac{1}{n} 
\end{alignat*}
\end{example}

\end{frame}

% page --------------------------------------------------------------------------------------------------
\begin{frame}{Cauchy-Produkt von Reihen}

Seien $\sum\limits_{j=1}^\infty a_j$ und $\sum\limits_{j=1}^\infty b_j$ zwei 
\superalert{absolut konvergente} Reihen. F\"ur $n\in \mathbb N$ setzen wir
{\footnotesize
\bq
 c_n & = & \sum\limits_{j=1}^{n-1} a_j b_{n-j}.
\eq
}

Dann ist auch die Reihe $\sum\limits_{j=1}^\infty c_j$ absolut konvergent und es gilt
{\footnotesize
\bq
 \sum\limits_{j=1}^\infty c_j & = & 
 \left( \sum\limits_{j=1}^\infty a_j \right)
 \left( \sum\limits_{j=1}^\infty b_j \right).
\eq
}

Bemerkung: Die absolute Konvergenz ist wesentlich f\"ur die G\"ultigkeit des Satzes!
\superalert{Im Allgemeinen gilt, da{\ss} Umordnungen innerhalb einer Reihe nur erlaubt sind, falls die Reihe
absolut konvergiert}.

\end{frame}

%%%%%%%%%%%%%%%%%%%%%%%%%%%%%%%%%%%%%%%%%%%%%%%%%%%%%%%%%%%%%%%%%%%%%%%%%%%%%%%%%%%%%%%%%%%%%%%%%%%%%%%%%
%%%%%%%%%%%%%%%%%%%%%%%%%%%%%%%%%%%%%%%%%%%%%%%%%%%%%%%%%%%%%%%%%%%%%%%%%%%%%%%%%%%%%%%%%%%%%%%%%%%%%%%%%
%%%%%%%%%%%%%%%%%%%%%%%%%%%%%%%%%%%%%%%%%%%%%%%%%%%%%%%%%%%%%%%%%%%%%%%%%%%%%%%%%%%%%%%%%%%%%%%%%%%%%%%%%

\section{Beispiele und Anwendungen}

\frame{\sectionpage}

% page --------------------------------------------------------------------------------------------------
\begin{frame}{Wichtige Reihen}

{\small
\bq
\exp x & = & \sum\limits_{j=0}^\infty \frac{x^j}{j!},
\nonumber \\
 -\ln\left(1-x\right) & = & \sum\limits_{j=1}^\infty \frac{x^j}{j}, \;\;\;\;\;\;\;\;\;\;\;\;\;\;\; \superalert{|x|<1},
 \nonumber \\
\sin x & = & \sum\limits_{j=0}^\infty (-1)^{j} \frac{x^{2j+1}}{(2j+1)!},
\nonumber \\
\cos x & = & \sum\limits_{j=0}^\infty (-1)^{j} \frac{x^{2j}}{(2j)!},
\nonumber \\
\sinh x & = & \sum\limits_{j=0}^\infty \frac{x^{2j+1}}{(2j+1)!},
\nonumber \\
\cosh x & = & \sum\limits_{j=0}^\infty \frac{x^{2j}}{(2j)!}.
\eq
}

\end{frame}

% page --------------------------------------------------------------------------------------------------
\begin{frame}{Sinus und Kosinus}

{\small
\bq
\sin x & = & \sum\limits_{j=0}^\infty (-1)^{j} \frac{x^{2j+1}}{(2j+1)!}
\nonumber \\
 & = & 
 \frac{x}{1!} -\frac{x^3}{3!} + \frac{x^5}{5!} - \frac{x^7}{7!} + \dots
\nonumber \\
 & = & 
 x -\frac{x^3}{6} + \frac{x^5}{120} - \frac{x^7}{5040} + \dots
 \nonumber \\
\cos x & = & \sum\limits_{j=0}^\infty (-1)^{j} \frac{x^{2j}}{(2j)!}
\nonumber \\
 & = & 
 \frac{x^0}{0!} - \frac{x^2}{2!} + \frac{x^4}{4!} - \frac{x^6}{6!} + \dots
\nonumber \\
 & = & 
 1 - \frac{x^2}{2} + \frac{x^4}{24} - \frac{x^6}{720} + \dots
\eq
}

\end{frame}

% page --------------------------------------------------------------------------------------------------
\begin{frame}{Bemerkungen}

\begin{itemize}

\item $\sinh$ und $\cosh$ bezeichnet man als Sinus Hyperbolicus bzw. Kosinus Hyperbolicus.

\item Mit Ausnahme der Reihe f\"ur $\ln(1-x)$ konvergieren alle Reihen absolut f\"ur alle Werte von $x$.
Man sagt die Reihen haben einen unendlichen {\bf Konvergenzradius}.

\item Die Reihe f\"ur $\ln(1-x)$ konvergiert absolut f\"ur $|x|<1$. Somit hat diese Reihe den Konvergenzradius
$1$.

\item Man spricht von einem Konvergenzradius, da die obigen Reihen auch definiert sind, wenn man die reelle
Variable $x$ durch eine komplexe Variable $z$ ersetzt.

\end{itemize}

\end{frame}

% page --------------------------------------------------------------------------------------------------
\begin{frame}{Quiz}

\bq
 \sum\limits_{j=0}^\infty \left(-\frac{1}{2}\right)^{j} \frac{x^{2j}}{(2j)!} & = & ?
\eq
\begin{description}
\item{(A)} $\frac{1}{2}\cos\left(x\right)$
\item{(B)} $\cos\left(\frac{x}{2}\right)$
\item{(C)} $\cosh\left(\frac{x}{2}\right)$
\item{(D)} $\cos\left(\frac{1}{2}\sqrt{2} x\right)$
\end{description}

\end{frame}

% page --------------------------------------------------------------------------------------------------
\begin{frame}{Die Formel von Euler}

Wir betrachten $\exp(i x)$:
{\footnotesize
\bq
 \exp\left(i x\right) & = & \sum\limits_{j=0}^\infty \frac{i^j x^j}{j!}
 = 
 \sum\limits_{j=0}^\infty i^{2j} \frac{x^{2j}}{(2j)!}
 + \sum\limits_{j=0}^\infty i^{2j+1} \frac{x^{2j+1}}{(2j+1)!}
 \nonumber \\
 & = & 
 \sum\limits_{j=0}^\infty (-1)^{j} \frac{x^{2j}}{(2j)!}
 + i \sum\limits_{j=0}^\infty (-1)^{j} \frac{x^{2j+1}}{(2j+1)!}
 = \cos x + i \sin x.
\eq
}

Die Reihendarstellung liefert also einen einfachen Beweis der Formel:
\bq
 \exp(i x) & = & \cos x + i \sin x.
\eq
Ebenso findet man
\bq
 \exp x & = & \cosh x + \sinh x.
\eq
Man beachte, da{\ss} f\"ur die Umordnung der Reihen die absolute Konvergenz notwendig ist.

\end{frame}

% page --------------------------------------------------------------------------------------------------
\begin{frame}{Trigometrische und hyperbolische Funktionen}

Man kann die trigometrischen und die hyperbolischen Funktionen auch durch die Exponentialfunktion ausdr\"ucken:
\begin{myemptytheorem}
\bq
 \cos x = \frac{1}{2} \left( e^{ix} + e^{-ix} \right),
 & &
 \sin x = \frac{1}{2i} \left( e^{ix} - e^{-ix} \right),
 \nonumber \\
 \cosh x = \frac{1}{2} \left( e^{x} + e^{-x} \right),
 & &
 \sinh x = \frac{1}{2} \left( e^{x} - e^{-x} \right).
\eq
\end{myemptytheorem}

\end{frame}

% page --------------------------------------------------------------------------------------------------
\begin{frame}{Additionstheoreme}

\begin{myemptytheorem}
\bq
 \sin\left(\alpha+\beta\right) 
 & = &
 \sin\left(\alpha\right) \cos\left(\beta\right) + \cos\left(\alpha\right) \sin\left(\beta\right)
 \nonumber \\
 \cos\left(\alpha+\beta\right) 
 & = &
 \cos\left(\alpha\right) \cos\left(\beta\right) - \sin\left(\alpha\right) \sin\left(\beta\right)
\eq
\end{myemptytheorem}
Leichter zu merken:
\begin{myemptytheorem}
\bq
 e^{i \left(\alpha+\beta\right)} 
 & = &
 e^{i \alpha} 
 e^{i \beta} 
\eq
\end{myemptytheorem}

\end{frame}

% page --------------------------------------------------------------------------------------------------
\begin{frame}{Additionstheoreme}

Es ist
{\small
\bq
 e^{i \left(\alpha+\beta\right)} 
 & = &
 \cos\left(\alpha+\beta\right) + i \sin\left(\alpha+\beta\right)
 \nonumber \\
 e^{i \alpha} 
 & = &
 \cos\left(\alpha\right) + i \sin\left(\alpha\right)
 \nonumber \\
 e^{i \beta} 
 & = &
 \cos\left(\beta\right) + i \sin\left(\beta\right)
\eq
}

\vspace*{1mm} und
{\small
\bq
 e^{i \alpha} 
 e^{i \beta} 
 & = & 
 \left[ \cos\left(\alpha\right) + i \sin\left(\alpha\right) \right]
 \left[ \cos\left(\beta\right) + i \sin\left(\beta\right) \right]
 \nonumber \\
 & = &
 \cos\left(\alpha\right) \cos\left(\beta\right)
 + i \cos\left(\alpha\right) \sin\left(\beta\right)
 + i \sin\left(\alpha\right) \cos\left(\beta\right)
 + i^2 \sin\left(\alpha\right) \sin\left(\beta\right)
 \nonumber \\
 & = &
 \left[ \cos\left(\alpha\right) \cos\left(\beta\right)
 - \sin\left(\alpha\right) \sin\left(\beta\right) \right]
 + i \left[ \cos\left(\alpha\right) \sin\left(\beta\right)
 + \sin\left(\alpha\right) \cos\left(\beta\right) \right]
\eq
}

\end{frame}

% page --------------------------------------------------------------------------------------------------
\begin{frame}{Additionstheoreme}

Somit folgt aus $e^{i \left(\alpha+\beta\right)} = e^{i \alpha} e^{i \beta}$
\begin{myemptytheorem}
\bq
\lefteqn{
 \alert{\cos\left(\alpha+\beta\right)} + i \superalert{\sin\left(\alpha+\beta\right)}
 = } & &
 \nonumber \\
 & &
 \left[ \alert{\cos\left(\alpha\right) \cos\left(\beta\right) - \sin\left(\alpha\right) \sin\left(\beta\right)} \right]
 + i \left[ \superalert{\cos\left(\alpha\right) \sin\left(\beta\right) + \sin\left(\alpha\right) \cos\left(\beta\right)} \right]
\eq
\end{myemptytheorem}
Nimmt man nun den Real- bzw. Imagin\"arteil dieser Gleichung, so erh\"alt man die Additionstheoreme f\"ur Kosinus und Sinus.

\end{frame}

%%%%%%%%%%%%%%%%%%%%%%%%%%%%%%%%%%%%%%%%%%%%%%%%%%%%%%%%%%%%%%%%%%%%%%%%%%%%%%%%%%%%%%%%%%%%%%%%%%%%%%%%%
%%%%%%%%%%%%%%%%%%%%%%%%%%%%%%%%%%%%%%%%%%%%%%%%%%%%%%%%%%%%%%%%%%%%%%%%%%%%%%%%%%%%%%%%%%%%%%%%%%%%%%%%%
%%%%%%%%%%%%%%%%%%%%%%%%%%%%%%%%%%%%%%%%%%%%%%%%%%%%%%%%%%%%%%%%%%%%%%%%%%%%%%%%%%%%%%%%%%%%%%%%%%%%%%%%%

% page --------------------------------------------------------------------------------------------------
\begin{frame}

\end{frame}

\end{document}


